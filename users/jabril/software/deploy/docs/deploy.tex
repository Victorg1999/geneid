% -*- mode: Noweb; noweb-code-mode: perl-mode; tab-width: 4 -*-
\documentclass[11pt]{article}
%
%2345678901234567890123456789012345678901234567890123456789012345678901234567890
%        1         2         3         4         5         6         7         8
%
% $Id: deploy.tex,v 1.1 2001-08-18 19:21:15 jabril Exp $
%
\usepackage{noweb}
\usepackage[a4paper,offset={0pt,0pt},hmargin={2cm,2cm},vmargin={1cm,1cm}]{geometry}
\usepackage{graphics}
\usepackage[dvips]{graphicx}
%% pstricks
\usepackage[dvips]{pstcol}
\usepackage{pstricks}
%\usepackage{pst-node}
%\usepackage{pst-char}
%\usepackage{pst-grad}
%% bibliography
\usepackage{natbib}
%% latex2html
\usepackage{url}
\usepackage{html}     
\usepackage{htmllist} 
%% tables    
%\usepackage{colortbl}
%\usepackage{multirow}
%\usepackage{hhline}
%\usepackage{tabularx}
\usepackage{dcolumn}
%% seminar
%\usepackage{semcolor,semlayer,semrot,semhelv,sem-page,slidesec}
%% draft watermark
%\usepackage[all,dvips]{draftcopy}
%\draftcopySetGrey{0.9}
%\draftcopyName{CONFIDENTIAL}{100}
%% layout
\usepackage{fancyhdr} % Do not use \usepackage{fancybox} -> TOCs disappear
%\usepackage{lscape}
%\usepackage{rotating}
%\usepackage{multicol}
%% fonts
\usepackage{times}\fontfamily{ptm}\selectfont
\usepackage{t1enc}

% noweb options
\noweboptions{smallcode}
\def\nwendcode{\endtrivlist \endgroup} % relax page breaking scheme
\let\nwdocspar=\par                    %
 
% Colors for gff2ps
\input ColorDefs.tex
% New Commands are defined here
\newcommand{\sctn}[1]{\section{#1}}
\newcommand{\subsctn}[1]{\subsection{#1}}
\newcommand{\subsubsctn}[1]{\subsubsection{#1}}
\newcommand{\desc}[1]{\item[#1] \ \\}

% PSTRICKs definitions
\pslongbox{ExFrame}{\psframebox}
\newcommand{\cln}[1]{\fcolorbox{black}{#1}{\textcolor{#1}{\rule[-.3ex]{1cm}{1ex}}}}
\newpsobject{showgrid}{psgrid}{subgriddiv=0,griddots=1,gridlabels=6pt}
% \pscharpath[fillstyle=solid, fillcolor=verydarkcyan, linecolor=black, linewidth=1pt]{\sffamily\scshape\bfseries\veryHuge #1 }

%%%%% global urls
% \newcommand{\getpsf}[1]{\html{(\htmladdnormallink{Get PostScript file}{./Psfiles/#1})}}   

% Setting text for footers and headers

% \def\tit{\textsc{{\tt{}deploy.pl} }}
\fancyhead{} % clear all fields
\fancyfoot{} % clear all fields
\fancyhead[RO,LE]{\thepage}
\fancyhead[LO,RE]{\rightmark}
\fancyfoot[LO,LE]{\small\textsl{Abril, J.F.}}
\fancyfoot[RO,RE]{\small\textbf{\today}}
\renewcommand{\headrulewidth}{1pt}
\renewcommand{\footrulewidth}{1pt}

%%%%%%%%%%%%%%%%%%%%%%%%%%%%%%%%%%%%%%%%%%%%%%%%%%%%%%%%%%%%%%%%%%%%%%%%%%%

\def\progname{deploy.pl}
\def\mtjabril{
 \htmladdnormallink{\texttt{jabril@imim.es}}
                   {MAILTO:jabril@imim.es?subject=[deploy.pl]}
 } % def mtjabril
\def\authorslist{
 Josep F. Abril {\small (\mtjabril) }\\
 Other authors here...\\
 } % def authorslist
\def\license{GNU General Public License (GNU-GPL)}
\def\description{
This perl script creates all the files we need to start a project report or to describe a program implementation under the {\tt{}noweb} Literate Programming tool. It will produce the main {\tt{}noweb} file, from which we can tangle the sources or weabe the \LaTeX documentation.
 } % def description 

%%%%%%%%%%%%%%%%%%%%%%%%%%%%%%%%%%%%%%%%%%%%%%%%%%%%%%%%%%%%%%%%%%%%%%%%%%%

\begin{document}
\nwfilename{/home/ug/jabril/development/jabril/software/deploy/deploy.nw}\nwbegindocs{1}\nwdocspar
\thispagestyle{empty}

\begin{titlepage}

\ \vfill
\begin{center}
\begin{bfseries}
\begin{large}
\newlength{\lttbl}\setlength{\lttbl}{0.25\linewidth}
\newlength{\rttbl}\setlength{\rttbl}{0.70\linewidth}
%\fbox{
\rule[0.5ex]{0.95\linewidth}{0.5ex}
%\vskip 2ex
\begin{tabular}{>{\scshape}r@{\quad}l}
\rule{\lttbl}{0pt} & \rule{\rttbl}{0pt} \\[2ex]
Program Name: & {\Huge\progname}                       \\[2ex]
      Author: & {\Large
                 \begin{minipage}[t]{0.95\rttbl}
                 \authorslist
                 \end{minipage}}                       \\[2ex]
     License: & {\license}                             \\[2ex]
 Last Update: & {\today}                               \\[2ex]
 Description: & {\large\mdseries
                 \begin{minipage}[t]{0.95\rttbl}
                 \description
                 \end{minipage}}                       \\[2ex]
\\
\end{tabular}
\vskip 2ex
\rule[0.5ex]{0.95\linewidth}{0.5ex}
%} % fbox
\end{large}
\end{bfseries}

\vfill

\begin{raggedleft}
\scalebox{0.9 1}{\Large\textsl{\textbf{Genome Informatics Research Lab}}}\\
Grup de Recerca en Infom\`atica Biom\`edica\\
Institut Municipal d'Investigaci\'o M\`edica\\
Universitat Pompeu Fabra\\[2ex]
\end{raggedleft}
\end{center}

\end{titlepage} %'

%%%%%%%%%%%%%%%%%%%% FRONTMATTER

\newpage
\pagenumbering{roman}
\setcounter{page}{1}
\pagestyle{fancy}
% Marks redefinition must go here because pagestyle 
% resets the values to the default ones.
\renewcommand{\sectionmark}[1]{\markboth{}{\thesection.\ #1}}
\renewcommand{\subsectionmark}[1]{\markboth{}{\thesubsection.\ \textsl{#1}}}

\tableofcontents
\listoftables
\listoffigures

\vfill
\begin{center}
{\small$<$ \verb$Id: deploy.tex,v 1.1 2001-08-18 19:21:15 jabril Exp $$>$ }
\end{center}

%%%%%%%%%%%%%%%%%%%% MAINMATTER

\newpage
\pagenumbering{arabic}
\setcounter{page}{1}

\sctn{empty section}

\subsctn{empty subsection}

\nwenddocs{}\nwbegincode{2}\sublabel{NWdepy-ProF-1}\nwmargintag{{\nwtagstyle{}\subpageref{NWdepy-ProF-1}}}\moddef{Program Outline~{\nwtagstyle{}\subpageref{NWdepy-ProF-1}}}\endmoddef
#!/usr/bin/perl -w
# This is perl, version 5.005_03 built for i386-linux
#
# $Id: deploy.tex,v 1.1 2001-08-18 19:21:15 jabril Exp $
#

use strict;

#
# MODULES
#
# CONSTANTS
my $USAGE = "\\nUSAGE:\\n\\tdeploy.pl <projectname>\\n".
            "(It asumes that you are in the right directory)\\n\\n";
my @working_dirs = qw(
                       RCS
                       bin  bin/param
                       data
                       docs docs/psfigures docs/tables docs/html
                       tests
                       );
#
# VARIABLES
my $PROJECT;
# my $HOME = $ENV\{HOME\};
my $CWD  = `pwd`;
chomp($CWD);
my $PATH = $CWD;
# $PATH =~ s%^$HOME/%%o;
#
# MAIN

@ARGV > 0 || do \{
    print STDERR $USAGE;
    exit(1);
\};
$PROJECT = shift @ARGV;

print STDERR "###\\n### RUNNING deploy.pl..........\\n###\\n".
             "### Current Working Directory: $CWD\\n".
             "### Setting PATH to: $PATH\\n".
             "### Project NAME: $PROJECT\\n###\\n";

&make_dirs;
&new_noweb_doc;
&extract_files;

print STDERR "###\\n### RUNNING deploy.pl............ DONE\\n###\\n";

exit(0);

#
# SUBS
sub make_dirs() \{
    print STDERR "###\\n### Creating Project Subdirectories...\\n###\\n";
    foreach my $d (@working_dirs) \{
        print STDERR "### ... $d\\n";
        system("mkdir $d") unless (-e $d && -d _);
    \};
    print STDERR "###\\n### Project Subdirectories............ DONE\\n###\\n";
\} # make_dirs
sub new_noweb_doc() \{
    my $file = "$PROJECT.nw";
    (-e $file && -f _) && do \{
         print STDERR "###\\n### Project file \\"$file\\" does exist...\\n".
                      "### EXITING PROGRAM !!!\\n";
         exit(1);   
    \};
    print STDERR "###\\n### Writing Project NOWEB file: $file\\n###\\n";
    open(NOWEB,"> $file");
    while (<DATA>) \{
        my ($FINDPATH,$FINDPROJECT) = ('@@@PATH@@@','@@@PROJECT@@@');
        my $l = $_;
        $l =~ /$FINDPATH/o && do \{
            $l =~ s/$FINDPATH/$PATH/o;
        \}; 
        $l =~ /$FINDPROJECT/o && do \{
            $l =~ s/$FINDPROJECT/$PROJECT/o;
        \};
        print NOWEB $l;
    \}; 
    close(NOWEB);
    print STDERR "###\\n### NOWEB file........................ DONE\\n###\\n";
\} # new_noweb_doc
sub extract_files() \{
    print STDERR "###\\n### Extracting Files from NOWEB file...\\n###\\n";
    # my $WORK = '$HOME/'.$PATH;
    my $WORK = $PATH;
    my $nwfile = "$PROJECT.nw";
    system << "+++EOS+++" ;
notangle -R\\'BASH Environment Variables\\' $WORK/$nwfile > $WORK/.bash_VARS ; 
notangle -R\\'CSH Environment Variables\\'  $WORK/$nwfile > $WORK/.csh_VARS ; 
notangle -Rweaving  $WORK/$nwfile > $WORK/nw2tex ;
notangle -RLaTeXing $WORK/$nwfile > $WORK/ltx ;
chmod a+x $WORK/nw2tex ;
chmod a+x $WORK/ltx ;
ci -l -i0.1 -t-\\'\\t\\t$nwfile: NOWEB file for $PROJECT\\' \\\\
   -m'BASIC TEMPLATE for THIS PROJECT' $nwfile ;
emacs $nwfile \\&
$WORK/nw2tex ;
$WORK/ltx ;
/usr/X11R6/bin/ghostview -color -title -magstep -1 \\\\
                         -portrait -a4 $WORK/docs/$PROJECT.ps \\&
+++EOS+++
    print STDERR "###\\n### File Extraction................. DONE\\n###\\n";
\} # extract_files

__DATA__
\nwnotused{Program\ Outline}\nwendcode{}\nwbegindocs{3}\nwdocspar

%%%%%%%%%%%%%%%%%%%%%%%%%%%%%%%%%%%%%%%%%%%%%%%%%%%%%%%%%%%%% INNER LATEX DOC
%%%%%%%%%%%%%%%%%%%%%%%%%%%%%%%%%%%%%%%%%%%%%%%%%%%%%%%%%%%%% INNER LATEX DOC
%%%%%%%%%%%%%%%%%%%%%%%%%%%%%%%%%%%%%%%%%%%%%%%%%%%%%%%%%%%%% INNER LATEX DOC
%%%%%%%%%%%%%%%%%%%%%%%%%%%%%%%%%%%%%%%%%%%%%%%%%%%%%%%%%%%%% INNER LATEX DOC
\nwenddocs{}\nwbegincode{4}\sublabel{NWdepy-PRON-1}\nwmargintag{{\nwtagstyle{}\subpageref{NWdepy-PRON-1}}}\moddef{PROJECT REPORT Template~{\nwtagstyle{}\subpageref{NWdepy-PRON-1}}}\endmoddef
% -*- mode: Noweb; noweb-code-mode: perl-mode; tab-width: 4 -*-
\\documentclass[11pt]\{article\}
%
%2345678901234567890123456789012345678901234567890123456789012345678901234567890
%        1         2         3         4         5         6         7         8
%
% $Id: deploy.tex,v 1.1 2001-08-18 19:21:15 jabril Exp $
%
\\usepackage\{noweb\}
\\usepackage[a4paper,offset=\{0pt,0pt\},hmargin=\{2cm,2cm\},vmargin=\{1cm,1cm\}]\{geometry\}
\\usepackage\{graphics\}
\\usepackage[dvips]\{graphicx\}
%% pstricks
\\usepackage[dvips]\{pstcol\}
\\usepackage\{pstricks\}
%\\usepackage\{pst-node\}
%\\usepackage\{pst-char\}
%\\usepackage\{pst-grad\}
%% bibliography
\\usepackage\{natbib\}
%% latex2html
\\usepackage\{url\}
\\usepackage\{html\}     
\\usepackage\{htmllist\} 
%% tables    
%\\usepackage\{colortbl\}
%\\usepackage\{multirow\}
%\\usepackage\{hhline\}
%\\usepackage\{tabularx\}
%\\usepackage\{dcolumn\}
%% seminar
%\\usepackage\{semcolor,semlayer,semrot,semhelv,sem-page,slidesec\}
%% draft watermark
%\\usepackage[all,dvips]\{draftcopy\}
%\\draftcopySetGrey\{0.9\}
%\\draftcopyName\{CONFIDENTIAL\}\{100\}
%% layout
\\usepackage\{fancyhdr\} % Do not use \\usepackage\{fancybox\} -> TOCs disappear
%\\usepackage\{lscape\}
%\\usepackage\{rotating\}
%\\usepackage\{multicol\}
%% fonts
\\usepackage\{times\}\\fontfamily\{ptm\}\\selectfont
\\usepackage\{t1enc\}

% noweb options
\\noweboptions\{smallcode\}
\\def\\nwendcode\{\\endtrivlist \\endgroup\} % relax page breaking scheme
\\let\\nwdocspar=\\par                    %
 
% Colors for gff2ps
\\input ColorDefs.tex
% New Commands are defined here
\\newcommand\{\\sctn\}[1]\{\\section\{#1\}\}
\\newcommand\{\\subsctn\}[1]\{\\subsection\{#1\}\}
\\newcommand\{\\subsubsctn\}[1]\{\\subsubsection\{#1\}\}
\\newcommand\{\\desc\}[1]\{\\item[#1] \\ \\\\\}

% PSTRICKs definitions
\\pslongbox\{ExFrame\}\{\\psframebox\}
\\newcommand\{\\cln\}[1]\{\\fcolorbox\{black\}\{#1\}\{\\textcolor\{#1\}\{\\rule[-.3ex]\{1cm\}\{1ex\}\}\}\}
\\newpsobject\{showgrid\}\{psgrid\}\{subgriddiv=0,griddots=1,gridlabels=6pt\}
% \\pscharpath[fillstyle=solid, fillcolor=verydarkcyan, linecolor=black, linewidth=1pt]\{\\sffamily\\scshape\\bfseries\\veryHuge #1 \}

%%%%% global urls
% \\newcommand\{\\getpsf\}[1]\{\\html\{(\\htmladdnormallink\{Get PostScript file\}\{./Psfiles/#1\})\}\}   
\\def\\mtjabril\{\\htmladdnormallink\{\\textbf\{jabril@imim.es\}\}\{MAILTO:jabril@imim.es\}\}

% defs
\\def\\drome\{\\textit\{Drosophila melanogaster\}\}
\\def\\dro\{\\textit\{Drosophila\}\}
\\def\\dme\{\\textit\{D. melanogaster\}\}
\\def\\seq\{\\texttt\{\\textbf\{X62937\}\}\}
\\def\\nowf\{[[DromeRepeats.nw]]\}
\\def\\bn\{\\textsc\{blastn\}\}
\\def\\ps\{\\textsc\{PostScript\}\}

% Setting text for footers and headers

\\def\\tit\{\\textsc\{Project Title Here.- \}\}
\\fancyhead\{\} % clear all fields
\\fancyfoot\{\} % clear all fields
\\fancyhead[RO,LE]\{\\thepage\}
\\fancyhead[LO,RE]\{\\rightmark\}
\\fancyfoot[LO,LE]\{\\small\\textsl\{Authors List Here\}\}
\\fancyfoot[RO,RE]\{\\small\\textbf\{\\today\}\}
\\renewcommand\{\\headrulewidth\}\{1pt\}
\\renewcommand\{\\footrulewidth\}\{1pt\}

%%%%%%%%%%%%%%%%%%%%%%%%%%%%%%%%%%%%%%%%%%%%%%%%%%%%%%%%%%%%%%%%%%%%%%%%%%%

\\begin\{document\}
@ 
\\thispagestyle\{empty\}

\\begin\{titlepage\}

\\ \\vfill
\\begin\{center\}
\\textbf\{\\Huge Project Title Here\}\\\\[5ex]

\\textbf\{\\Large Authors List Here\}\\\\[1ex]
\\textbf\{\\Large Josep F. Abril\}\\\\[5ex] % \\raisebox\{0.85ex\}\{\\footnotesize$\\,\\dag$\}\\\\[0.5ex]

\\textbf\{\\large --- \\today ---\}\\\\[10ex]

\\begin\{abstract\}
\\begin\{center\}
\\parbox\{0.75\\linewidth\}\{
\} % parbox
\\end\{center\}
\\end\{abstract\}

\\vfill

\\begin\{raggedleft\}
\\scalebox\{0.9 1\}\{\\Large\\textsl\{\\textbf\{Genome Informatics Research Lab\}\}\}\\\\
Grup de Recerca en Infom\\`atica Biom\\`edica\\\\
Institut Municipal d'Investigaci\\'o M\\`edica\\\\
Universitat Pompeu Fabra\\\\[2ex]
\\raisebox\{0.85ex\}\{\\footnotesize$\\dag\\,$\}\{\\large e-mail: \\mtjabril\}\\\\
\\end\{raggedleft\}
\\end\{center\}

\\end\{titlepage\} %'

%%%%%%%%%%%%%%%%%%%% FRONTMATTER

\\newpage
\\pagenumbering\{roman\}
\\setcounter\{page\}\{1\}
\\pagestyle\{fancy\}
% Marks redefinition must go here because pagestyle 
% resets the values to the default ones.
\\renewcommand\{\\sectionmark\}[1]\{\\markboth\{\}\{\\thesection.\\ #1\}\}
\\renewcommand\{\\subsectionmark\}[1]\{\\markboth\{\}\{\\thesubsection.\\ \\textsl\{#1\}\}\}

\\tableofcontents
\\listoftables
\\listoffigures

\\vfill
\\begin\{center\}
\{\\small$<$ \\verb$Id: deploy.tex,v 1.1 2001-08-18 19:21:15 jabril Exp $$>$ \}
\\end\{center\}

%%%%%%%%%%%%%%%%%%%% MAINMATTER

\\newpage
\\pagenumbering\{arabic\}
\\setcounter\{page\}\{1\}

\\sctn\{empty section\}

\\subsctn\{empty subsection\}

%%%%%%%%%%%%%%%%%%%%%%%%%%%%%%%%%%%
\\begin\{comment\}
\\end\{comment\}
%%%%%%%%%%%%%%%%%%%%%%%%%%%%%%%%%%%

%%%%%%%%%%%%%%%%%%%% BACKMATTER

% \\newpage
% 
% \\bibliographystyle\{apalike\}
% \\bibliography\{/home1/rguigo/docs/biblio/References\}

\\newpage
\\appendix

\\sctn\{empty appendix section\}

\\subsctn\{empty appendix subsection\}

%%%%%%%%%%%%%%%%%%%%%%%%%%%%%%%%%%%
\\begin\{comment\}
\\end\{comment\}
%%%%%%%%%%%%%%%%%%%%%%%%%%%%%%%%%%%

\\newpage

\\sctn\{Common code blocks\}

\\subsctn\{PERL scripts\}

<<PERL shebang>>=
#!/usr/bin/perl -w
# This is perl, version 5.005_03 built for i386-linux
<<Version Control Id Tag>>
#
use strict;
@

<<Global Constants - Boolean>>=
my ($T,$F) = (1,0); # for 'T'rue and 'F'alse
@ %def $T $F

We also set here the date when the script is running and who is the user running it.

<<Global Vars - User and Date>>=
my $DATE = localtime;
my $USER = $ENV\{USER\};
@ %def $DATE $USER


\\subsubsctn\{Timing our scripts\}

The '[[Benchmark]]' module encapsulates a number of routines to help to figure out how long it takes to execute a piece of code and the whole script.

<<Use Modules - Benchmark>>=
use Benchmark;
  <<Timer ON>>
@ 

See '[[man Benchmark]]' for further info about this package. 
We set an array to keep record of timing for each section.

<<Timer ON>>=
my @Timer = (new Benchmark);
@ 

<<Common PERL subs - Benchmark>>=
sub timing() \{
    push @Timer, (new Benchmark);
    # partial time 
    $_[0] || 
        (return timestr(timediff($Timer[$#Timer],$Timer[($#Timer - 1)])));
    # total time
    return timestr(timediff($Timer[$#Timer],$Timer[0]));
\} # timing
@ 


\\subsubsctn\{Printing complex Data Structures\}

With '[[Data::Dumper]]' we are able to pretty print complex data structures for debugging them.


<<Use Modules - Dumper>>=
use Data::Dumper;
local $Data::Dumper::Purity = 0;
local $Data::Dumper::Deepcopy = 1;
@ 


\\subsubsctn\{Common functions\}

<<Skip comments and empty records>>=
next if /^\\#/o;
next if /^\\s*$/o;
chomp;
@

<<Common PERL subs - Min Max>>=
#
sub max() \{
    my $z = shift @_;
    foreach my $l (@_) \{ $z = $l if $l > $z \};
    return $z;
\} # max
sub min() \{
    my $z = shift @_;
    foreach my $l (@_) \{ $z = $l if $l < $z \};
    return $z;
\} # min
@

<<Common PERL subs - Text fill>>=
#
sub fill_right() \{ $_[0].($_[2] x ($_[1] - length($_[0]))) \}
sub fill_left()  \{ ($_[2] x ($_[1] - length($_[0]))).$_[0] \}
sub fill_mid()   \{ 
    my $l = length($_[0]);
    my $k = int(($_[1] - $l)/2);
    ($_[2] x $k).$_[0].($_[2] x ($_[1] - ($l+$k)));
\} # fill_mid
@

These functions are used to report to STDERR a single char for each record processed (useful for reporting parsed records).

<<Common PERL subs - Counter>>=
#
sub counter \{ # $_[0]~current_pos++ $_[1]~char
    print STDERR "$_[1]";
    (($_[0] % 50) == 0) && (print STDERR "[".&fill_left($_[0],6,"0")."]\\n");
\} # counter
#
sub counter_end \{ # $_[0]~current_pos   $_[1]~char
    (($_[0] % 50) != 0) && (print STDERR "[".&fill_left($_[0],6,"0")."]\\n");
\} # counter_end
@

<<Global Vars - Counter>>=
my ($n,$c); # counter and char (for &counter function)
@ %def $n $c


\\subsubsctn\{Common functions for reporting program processes\}
\\label\{sec:messagerpt\}

Function '[[report]]' requires that a hash variable '[[%MessageList]]' has been set, such hash contains the strings for each report message we will need. The first parameter for '[[report]]' is a key for that hash, in order to retrieve the message string, the other parameters passed are processed by the [[sprintf]] function on that string.

<<Common PERL subs - STDERR>>=
sub report() \{ print STDERR sprintf($MessageList\{ shift @_ \},@_) \}
@

The same happens to '[[warn]]' function which also requires a hash variable '[[%ErrorList]]' containing the error messages.

<<Common PERL subs - STDERR>>=
sub warn() \{ print STDERR sprintf($ErrorList\{ shift @_ \}, @_) \}
@

\\subsctn\{AWK scripts\}

<<GAWK shebang>>=
#!/usr/bin/gawk -f
# GNU Awk 3.0.4
<<Version Control Id Tag>>
@

\\subsctn\{BASH scripts\}

<<BASH shebang>>=
#!/usr/bin/bash
# GNU bash, version 2.03.6(1)-release (i386-redhat-linux-gnu)
<<Version Control Id Tag>>
#
SECONDS=0 # Reset Timing
# Which script are we running...
L="####################"
\{ echo "$L$L$L$L";
  echo "### RUNNING [$0]";
  echo "### Current date:`date`";
  echo "###"; \} 1>&2;
@

<<BASH script end>>=
\{ echo "###"; echo "### Execution time for [$0] : $SECONDS secs";
  echo "$L$L$L$L";
  echo ""; \} 1>&2;
#
exit 0
@

\\subsctn\{Version control tags\}

This document is under Revision Control System (RCS). The version you are currently reading is the following:

<<Version Control Id Tag>>=
# $Id: deploy.tex,v 1.1 2001-08-18 19:21:15 jabril Exp $
@ 

\\newpage

\\sctn\{Extracting code blocks from this document\}

From this file we can obtain both the code and the
documentation. The following instructions are needed:

\\subsctn\{Extracts Script code chunks from the [[noweb]] file\} % \\\\[-0.5ex]

Remember when tangling that '-L' option allows you to include program line-numbering relative to original [[noweb]] file. Then the first line of the executable files is a comment, not a shebang, and must be removed to make scripts runnable.

<<tangling>>=
# showing line numbering comments in program
notangle -L -R"root" $WORK/$nwfile.nw | \\
    perl -ne '$.>1 && print' > $BIN/root_file ;
# program without line numbering comments
notangle -t4 -R"root" $WORK/$nwfile.nw \\
    > $BIN/root_file ;
# making them runnable
chmod a+x $BIN/root_file ;
@ 

\\subsctn\{Extracting different Config Files\} % \\\\[-0.5ex]

<<tangling>>=
notangle -R"root" $WORK/$nwfile.nw \\
    > $DATA/root_config ;
@ %$

\\subsctn\{Extracting documentation and \\LaTeX\{\}'ing it\} % \\\\[-0.5ex] %'

<<tangling>>=
notangle -Rweaving  $WORK/$nwfile.nw > $WORK/nw2tex ;
notangle -RLaTeXing $WORK/$nwfile.nw > $WORK/ltx ;
chmod a+x $WORK/nw2tex $WORK/ltx;
@ 

<<weaving>>=
<<BASH shebang>>
# weaving and LaTeXing
<<BASH Environment Variables>>
noweave -t4 -delay -index $WORK/$nwfile.nw > $DOCS/$nwfile.tex 
pushd $DOCS/ ;
latex $nwfile.tex ;
dvips $nwfile.dvi -o $nwfile.ps -t a4 ;
popd;
<<BASH script end>>
@ 

<<LaTeXing>>=
<<BASH shebang>>
# only LaTeXing
<<BASH Environment Variables>>
pushd $DOCS/ ;
latex $nwfile.tex ; 
latex $nwfile.tex ; 
latex $nwfile.tex ;
dvips $nwfile.dvi -o $nwfile.ps -t a4 ;
popd ;
<<BASH script end>>
@ %$

\\subsctn\{Defining working shell variables for the current project\} % \\\\[-0.5ex]

<<BASH Environment Variables>>=
# Global Variables
export WORK="@@@PATH@@@" ;
export BIN="$WORK/bin" ;
export DOCS="$WORK/docs" ;
export DATA="$WORK/data" ;
export nwfile="@@@PROJECT@@@" ;
@ 

<<CSH Environment Variables>>=
# Global Variables
setenv WORK "@@@PATH@@@" ;
setenv BIN  "$WORK/bin" ;
setenv DOCS "$WORK/docs" ;
setenv DATA "$WORK/data" ;
setenv nwfile "@@@PROJECT@@@" ;
@ 

<<tangling>>=
# TO DO: add a test to check which shell is running
# BASH shell
notangle -R'BASH Environment Variables' $WORK/$nwfile.nw \\
         > $WORK/.bash_VARS ; 
# CSH shell
notangle -R'CSH Environment Variables'  $WORK/$nwfile.nw \\
         > $WORK/.csh_VARS ; 
# sourcing
source $WORK/.bash_VARS ;
source $WORK/.csh_VARS ;
@

\\end\{document\}
\nwidentuses{\\{{{\char36}DATE}{:doDATE}}\\{{{\char36}F}{:doF}}\\{{{\char36}T}{:doT}}\\{{{\char36}USER}{:doUSER}}\\{{{\char36}c}{:doc}}\\{{{\char36}n}{:don}}}\nwindexuse{{\char36}DATE}{:doDATE}{NWdepy-PRON-1}\nwindexuse{{\char36}F}{:doF}{NWdepy-PRON-1}\nwindexuse{{\char36}T}{:doT}{NWdepy-PRON-1}\nwindexuse{{\char36}USER}{:doUSER}{NWdepy-PRON-1}\nwindexuse{{\char36}c}{:doc}{NWdepy-PRON-1}\nwindexuse{{\char36}n}{:don}{NWdepy-PRON-1}\nwnotused{PROJECT\ REPORT\ Template}\nwendcode{}\nwbegindocs{5}\nwdocspar
%%%%%%%%%%%%%%%%%%%%%%%%%%%%%%%%%%%%%%%%%%%%%%%%%%%%%%%%%%%%% INNER LATEX DOC
%%%%%%%%%%%%%%%%%%%%%%%%%%%%%%%%%%%%%%%%%%%%%%%%%%%%%%%%%%%%% INNER LATEX DOC
%%%%%%%%%%%%%%%%%%%%%%%%%%%%%%%%%%%%%%%%%%%%%%%%%%%%%%%%%%%%% INNER LATEX DOC
%%%%%%%%%%%%%%%%%%%%%%%%%%%%%%%%%%%%%%%%%%%%%%%%%%%%%%%%%%%%% INNER LATEX DOC

%%%%%%%%%%%%%%%%%%%%%%%%%%%%%%%%%%%
\begin{comment}
\end{comment}
%%%%%%%%%%%%%%%%%%%%%%%%%%%%%%%%%%%

%%%%%%%%%%%%%%%%%%%% BACKMATTER

% \newpage
% 
% \bibliographystyle{apalike}
% \bibliography{/home1/rguigo/docs/biblio/References}

\newpage
\appendix

\sctn{empty appendix section}

\subsctn{empty appendix subsection}

%%%%%%%%%%%%%%%%%%%%%%%%%%%%%%%%%%%
\begin{comment}
\end{comment}
%%%%%%%%%%%%%%%%%%%%%%%%%%%%%%%%%%%

\newpage

\sctn{Common code blocks}

\subsctn{PERL scripts}

\nwenddocs{}\nwbegincode{6}\sublabel{NWdepy-PERC-1}\nwmargintag{{\nwtagstyle{}\subpageref{NWdepy-PERC-1}}}\moddef{PERL shebang~{\nwtagstyle{}\subpageref{NWdepy-PERC-1}}}\endmoddef
#!/usr/bin/perl -w
# This is perl, version 5.005_03 built for i386-linux
\LA{}Version Control Id Tag~{\nwtagstyle{}\subpageref{NWdepy-VerM-1}}\RA{}
#
use strict;
\nwnotused{PERL\ shebang}\nwendcode{}\nwbegindocs{7}\nwdocspar

\nwenddocs{}\nwbegincode{8}\sublabel{NWdepy-GloQ-1}\nwmargintag{{\nwtagstyle{}\subpageref{NWdepy-GloQ-1}}}\moddef{Global Constants - Boolean~{\nwtagstyle{}\subpageref{NWdepy-GloQ-1}}}\endmoddef
my ($T,$F) = (1,0); # for 'T'rue and 'F'alse
\nwindexdefn{{\char36}T}{:doT}{NWdepy-GloQ-1}\nwindexdefn{{\char36}F}{:doF}{NWdepy-GloQ-1}\eatline
\nwidentdefs{\\{{{\char36}F}{:doF}}\\{{{\char36}T}{:doT}}}\nwnotused{Global\ Constants\ -\ Boolean}\nwendcode{}\nwbegindocs{9}\nwdocspar
We also set here the date when the script is running and who is the user running it.

\nwenddocs{}\nwbegincode{10}\sublabel{NWdepy-GloR-1}\nwmargintag{{\nwtagstyle{}\subpageref{NWdepy-GloR-1}}}\moddef{Global Vars - User and Date~{\nwtagstyle{}\subpageref{NWdepy-GloR-1}}}\endmoddef
my $DATE = localtime;
my $USER = $ENV\{USER\};
\nwindexdefn{{\char36}DATE}{:doDATE}{NWdepy-GloR-1}\nwindexdefn{{\char36}USER}{:doUSER}{NWdepy-GloR-1}\eatline
\nwidentdefs{\\{{{\char36}DATE}{:doDATE}}\\{{{\char36}USER}{:doUSER}}}\nwnotused{Global\ Vars\ -\ User\ and\ Date}\nwendcode{}\nwbegindocs{11}\nwdocspar

\subsubsctn{Timing our scripts}

The '{\tt{}Benchmark}' module encapsulates a number of routines to help to figure out how long it takes to execute a piece of code and the whole script.

\nwenddocs{}\nwbegincode{12}\sublabel{NWdepy-UseN-1}\nwmargintag{{\nwtagstyle{}\subpageref{NWdepy-UseN-1}}}\moddef{Use Modules - Benchmark~{\nwtagstyle{}\subpageref{NWdepy-UseN-1}}}\endmoddef
use Benchmark;
  \LA{}Timer ON~{\nwtagstyle{}\subpageref{NWdepy-Tim8-1}}\RA{}
\nwnotused{Use\ Modules\ -\ Benchmark}\nwendcode{}\nwbegindocs{13}\nwdocspar

See '{\tt{}man\ Benchmark}' for further info about this package. 
We set an array to keep record of timing for each section.

\nwenddocs{}\nwbegincode{14}\sublabel{NWdepy-Tim8-1}\nwmargintag{{\nwtagstyle{}\subpageref{NWdepy-Tim8-1}}}\moddef{Timer ON~{\nwtagstyle{}\subpageref{NWdepy-Tim8-1}}}\endmoddef
my @Timer = (new Benchmark);
\nwused{\\{NWdepy-UseN-1}}\nwendcode{}\nwbegindocs{15}\nwdocspar

\nwenddocs{}\nwbegincode{16}\sublabel{NWdepy-ComS-1}\nwmargintag{{\nwtagstyle{}\subpageref{NWdepy-ComS-1}}}\moddef{Common PERL subs - Benchmark~{\nwtagstyle{}\subpageref{NWdepy-ComS-1}}}\endmoddef
sub timing() \{
    push @Timer, (new Benchmark);
    # partial time 
    $_[0] || 
        (return timestr(timediff($Timer[$#Timer],$Timer[($#Timer - 1)])));
    # total time
    return timestr(timediff($Timer[$#Timer],$Timer[0]));
\} # timing
\nwnotused{Common\ PERL\ subs\ -\ Benchmark}\nwendcode{}\nwbegindocs{17}\nwdocspar


\subsubsctn{Printing complex Data Structures}

With '{\tt{}Data::Dumper}' we are able to pretty print complex data structures for debugging them.


\nwenddocs{}\nwbegincode{18}\sublabel{NWdepy-UseK-1}\nwmargintag{{\nwtagstyle{}\subpageref{NWdepy-UseK-1}}}\moddef{Use Modules - Dumper~{\nwtagstyle{}\subpageref{NWdepy-UseK-1}}}\endmoddef
use Data::Dumper;
local $Data::Dumper::Purity = 0;
local $Data::Dumper::Deepcopy = 1;
\nwnotused{Use\ Modules\ -\ Dumper}\nwendcode{}\nwbegindocs{19}\nwdocspar


\subsubsctn{Common functions}

\nwenddocs{}\nwbegincode{20}\sublabel{NWdepy-SkiV-1}\nwmargintag{{\nwtagstyle{}\subpageref{NWdepy-SkiV-1}}}\moddef{Skip comments and empty records~{\nwtagstyle{}\subpageref{NWdepy-SkiV-1}}}\endmoddef
next if /^\\#/o;
next if /^\\s*$/o;
chomp;
\nwnotused{Skip\ comments\ and\ empty\ records}\nwendcode{}\nwbegindocs{21}\nwdocspar

\nwenddocs{}\nwbegincode{22}\sublabel{NWdepy-ComQ-1}\nwmargintag{{\nwtagstyle{}\subpageref{NWdepy-ComQ-1}}}\moddef{Common PERL subs - Min Max~{\nwtagstyle{}\subpageref{NWdepy-ComQ-1}}}\endmoddef
#
sub max() \{
    my $z = shift @_;
    foreach my $l (@_) \{ $z = $l if $l > $z \};
    return $z;
\} # max
sub min() \{
    my $z = shift @_;
    foreach my $l (@_) \{ $z = $l if $l < $z \};
    return $z;
\} # min
\nwnotused{Common\ PERL\ subs\ -\ Min\ Max}\nwendcode{}\nwbegindocs{23}\nwdocspar

\nwenddocs{}\nwbegincode{24}\sublabel{NWdepy-ComS.2-1}\nwmargintag{{\nwtagstyle{}\subpageref{NWdepy-ComS.2-1}}}\moddef{Common PERL subs - Text fill~{\nwtagstyle{}\subpageref{NWdepy-ComS.2-1}}}\endmoddef
#
sub fill_right() \{ $_[0].($_[2] x ($_[1] - length($_[0]))) \}
sub fill_left()  \{ ($_[2] x ($_[1] - length($_[0]))).$_[0] \}
sub fill_mid()   \{ 
    my $l = length($_[0]);
    my $k = int(($_[1] - $l)/2);
    ($_[2] x $k).$_[0].($_[2] x ($_[1] - ($l+$k)));
\} # fill_mid
\nwnotused{Common\ PERL\ subs\ -\ Text\ fill}\nwendcode{}\nwbegindocs{25}\nwdocspar

These functions are used to report to STDERR a single char for each record processed (useful for reporting parsed records).

\nwenddocs{}\nwbegincode{26}\sublabel{NWdepy-ComQ.2-1}\nwmargintag{{\nwtagstyle{}\subpageref{NWdepy-ComQ.2-1}}}\moddef{Common PERL subs - Counter~{\nwtagstyle{}\subpageref{NWdepy-ComQ.2-1}}}\endmoddef
#
sub counter \{ # $_[0]~current_pos++ $_[1]~char
    print STDERR "$_[1]";
    (($_[0] % 50) == 0) && (print STDERR "[".&fill_left($_[0],6,"0")."]\\n");
\} # counter
#
sub counter_end \{ # $_[0]~current_pos   $_[1]~char
    (($_[0] % 50) != 0) && (print STDERR "[".&fill_left($_[0],6,"0")."]\\n");
\} # counter_end
\nwnotused{Common\ PERL\ subs\ -\ Counter}\nwendcode{}\nwbegindocs{27}\nwdocspar

\nwenddocs{}\nwbegincode{28}\sublabel{NWdepy-GloL-1}\nwmargintag{{\nwtagstyle{}\subpageref{NWdepy-GloL-1}}}\moddef{Global Vars - Counter~{\nwtagstyle{}\subpageref{NWdepy-GloL-1}}}\endmoddef
my ($n,$c); # counter and char (for &counter function)
\nwindexdefn{{\char36}n}{:don}{NWdepy-GloL-1}\nwindexdefn{{\char36}c}{:doc}{NWdepy-GloL-1}\eatline
\nwidentdefs{\\{{{\char36}c}{:doc}}\\{{{\char36}n}{:don}}}\nwnotused{Global\ Vars\ -\ Counter}\nwendcode{}\nwbegindocs{29}\nwdocspar

\subsubsctn{Common functions for reporting program processes}
\label{sec:messagerpt}

Function '{\tt{}report}' requires that a hash variable '{\tt{}{\char37}MessageList}' has been set, such hash contains the strings for each report message we will need. The first parameter for '{\tt{}report}' is a key for that hash, in order to retrieve the message string, the other parameters passed are processed by the {\tt{}sprintf} function on that string.

\nwenddocs{}\nwbegincode{30}\sublabel{NWdepy-ComP-1}\nwmargintag{{\nwtagstyle{}\subpageref{NWdepy-ComP-1}}}\moddef{Common PERL subs - STDERR~{\nwtagstyle{}\subpageref{NWdepy-ComP-1}}}\endmoddef
sub report() \{ print STDERR sprintf($MessageList\{ shift @_ \},@_) \}
\nwalsodefined{\\{NWdepy-ComP-2}}\nwnotused{Common\ PERL\ subs\ -\ STDERR}\nwendcode{}\nwbegindocs{31}\nwdocspar

The same happens to '{\tt{}warn}' function which also requires a hash variable '{\tt{}{\char37}ErrorList}' containing the error messages.

\nwenddocs{}\nwbegincode{32}\sublabel{NWdepy-ComP-2}\nwmargintag{{\nwtagstyle{}\subpageref{NWdepy-ComP-2}}}\moddef{Common PERL subs - STDERR~{\nwtagstyle{}\subpageref{NWdepy-ComP-1}}}\plusendmoddef
sub warn() \{ print STDERR sprintf($ErrorList\{ shift @_ \}, @_) \}
\nwendcode{}\nwbegindocs{33}\nwdocspar

\subsctn{AWK scripts}

\nwenddocs{}\nwbegincode{34}\sublabel{NWdepy-GAWC-1}\nwmargintag{{\nwtagstyle{}\subpageref{NWdepy-GAWC-1}}}\moddef{GAWK shebang~{\nwtagstyle{}\subpageref{NWdepy-GAWC-1}}}\endmoddef
#!/usr/bin/gawk -f
# GNU Awk 3.0.4
\LA{}Version Control Id Tag~{\nwtagstyle{}\subpageref{NWdepy-VerM-1}}\RA{}
\nwnotused{GAWK\ shebang}\nwendcode{}\nwbegindocs{35}\nwdocspar

\subsctn{BASH scripts}

\nwenddocs{}\nwbegincode{36}\sublabel{NWdepy-BASC-1}\nwmargintag{{\nwtagstyle{}\subpageref{NWdepy-BASC-1}}}\moddef{BASH shebang~{\nwtagstyle{}\subpageref{NWdepy-BASC-1}}}\endmoddef
#!/usr/bin/bash
# GNU bash, version 2.03.6(1)-release (i386-redhat-linux-gnu)
\LA{}Version Control Id Tag~{\nwtagstyle{}\subpageref{NWdepy-VerM-1}}\RA{}
#
SECONDS=0 # Reset Timing
# Which script are we running...
L="####################"
\{ echo "$L$L$L$L";
  echo "### RUNNING [$0]";
  echo "### Current date:`date`";
  echo "###"; \} 1>&2;
\nwused{\\{NWdepy-wea7-1}\\{NWdepy-LaT8-1}}\nwendcode{}\nwbegindocs{37}\nwdocspar

\nwenddocs{}\nwbegincode{38}\sublabel{NWdepy-BASF-1}\nwmargintag{{\nwtagstyle{}\subpageref{NWdepy-BASF-1}}}\moddef{BASH script end~{\nwtagstyle{}\subpageref{NWdepy-BASF-1}}}\endmoddef
\{ echo "###"; echo "### Execution time for [$0] : $SECONDS secs";
  echo "$L$L$L$L";
  echo ""; \} 1>&2;
#
exit 0
\nwused{\\{NWdepy-wea7-1}\\{NWdepy-LaT8-1}}\nwendcode{}\nwbegindocs{39}\nwdocspar

\subsctn{Version control tags}

This document is under Revision Control System (RCS). The version you are currently reading is the following:

\nwenddocs{}\nwbegincode{40}\sublabel{NWdepy-VerM-1}\nwmargintag{{\nwtagstyle{}\subpageref{NWdepy-VerM-1}}}\moddef{Version Control Id Tag~{\nwtagstyle{}\subpageref{NWdepy-VerM-1}}}\endmoddef
# $Id: deploy.tex,v 1.1 2001-08-18 19:21:15 jabril Exp $
\nwused{\\{NWdepy-PERC-1}\\{NWdepy-GAWC-1}\\{NWdepy-BASC-1}}\nwendcode{}\nwbegindocs{41}\nwdocspar

\newpage

\sctn{Extracting code blocks from this document}

From this file we can obtain both the code and the
documentation. The following instructions are needed:

\subsctn{Extracts Script code chunks from the {\tt{}noweb} file} % \\[-0.5ex]

Remember when tangling that '-L' option allows you to include program line-numbering relative to original {\tt{}noweb} file. Then the first line of the executable files is a comment, not a shebang, and must be removed to make scripts runnable.

\nwenddocs{}\nwbegincode{42}\sublabel{NWdepy-tan8-1}\nwmargintag{{\nwtagstyle{}\subpageref{NWdepy-tan8-1}}}\moddef{tangling~{\nwtagstyle{}\subpageref{NWdepy-tan8-1}}}\endmoddef
# showing line numbering comments in program
notangle -L -R"root" $WORK/$nwfile.nw | \\
    perl -ne '$.>1 && print' > $BIN/root_file ;
# program without line numbering comments
notangle -t4 -R"root" $WORK/$nwfile.nw \\
    > $BIN/root_file ;
# making them runnable
chmod a+x $BIN/root_file ;
\nwalsodefined{\\{NWdepy-tan8-2}\\{NWdepy-tan8-3}\\{NWdepy-tan8-4}}\nwnotused{tangling}\nwendcode{}\nwbegindocs{43}\nwdocspar

\subsctn{Extracting different Config Files} % \\[-0.5ex]

\nwenddocs{}\nwbegincode{44}\sublabel{NWdepy-tan8-2}\nwmargintag{{\nwtagstyle{}\subpageref{NWdepy-tan8-2}}}\moddef{tangling~{\nwtagstyle{}\subpageref{NWdepy-tan8-1}}}\plusendmoddef
notangle -R"root" $WORK/$nwfile.nw \\
    > $DATA/root_config ;
\nwendcode{}\nwbegindocs{45}%$

\subsctn{Extracting documentation and \LaTeX{}'ing it} % \\[-0.5ex] %'

\nwenddocs{}\nwbegincode{46}\sublabel{NWdepy-tan8-3}\nwmargintag{{\nwtagstyle{}\subpageref{NWdepy-tan8-3}}}\moddef{tangling~{\nwtagstyle{}\subpageref{NWdepy-tan8-1}}}\plusendmoddef
notangle -Rweaving  $WORK/$nwfile.nw > $WORK/nw2tex ;
notangle -RLaTeXing $WORK/$nwfile.nw > $WORK/ltx ;
chmod a+x $WORK/nw2tex $WORK/ltx;
\nwendcode{}\nwbegindocs{47}\nwdocspar

\nwenddocs{}\nwbegincode{48}\sublabel{NWdepy-wea7-1}\nwmargintag{{\nwtagstyle{}\subpageref{NWdepy-wea7-1}}}\moddef{weaving~{\nwtagstyle{}\subpageref{NWdepy-wea7-1}}}\endmoddef
\LA{}BASH shebang~{\nwtagstyle{}\subpageref{NWdepy-BASC-1}}\RA{}
# weaving and LaTeXing
\LA{}BASH Environment Variables~{\nwtagstyle{}\subpageref{NWdepy-BASQ-1}}\RA{}
noweave -t4 -delay -index $WORK/$nwfile.nw > $DOCS/$nwfile.tex 
pushd $DOCS/ ;
latex $nwfile.tex ;
dvips $nwfile.dvi -o $nwfile.ps -t a4 ;
popd;
\LA{}BASH script end~{\nwtagstyle{}\subpageref{NWdepy-BASF-1}}\RA{}
\nwnotused{weaving}\nwendcode{}\nwbegindocs{49}\nwdocspar

\nwenddocs{}\nwbegincode{50}\sublabel{NWdepy-LaT8-1}\nwmargintag{{\nwtagstyle{}\subpageref{NWdepy-LaT8-1}}}\moddef{LaTeXing~{\nwtagstyle{}\subpageref{NWdepy-LaT8-1}}}\endmoddef
\LA{}BASH shebang~{\nwtagstyle{}\subpageref{NWdepy-BASC-1}}\RA{}
# only LaTeXing
\LA{}BASH Environment Variables~{\nwtagstyle{}\subpageref{NWdepy-BASQ-1}}\RA{}
pushd $DOCS/ ;
latex $nwfile.tex ; 
latex $nwfile.tex ; 
latex $nwfile.tex ;
dvips $nwfile.dvi -o $nwfile.ps -t a4 ;
popd ;
\LA{}BASH script end~{\nwtagstyle{}\subpageref{NWdepy-BASF-1}}\RA{}
\nwnotused{LaTeXing}\nwendcode{}\nwbegindocs{51}%$

\subsctn{Defining working shell variables for the current project} % \\[-0.5ex]

\nwenddocs{}\nwbegincode{52}\sublabel{NWdepy-BASQ-1}\nwmargintag{{\nwtagstyle{}\subpageref{NWdepy-BASQ-1}}}\moddef{BASH Environment Variables~{\nwtagstyle{}\subpageref{NWdepy-BASQ-1}}}\endmoddef
# Global Variables
export WORK="/home/ug/jabril/development/jabril/software/deploy" ;
export BIN="$WORK/bin" ;
export DOCS="$WORK/docs" ;
export DATA="$WORK/data" ;
export nwfile="deploy" ;
\nwused{\\{NWdepy-wea7-1}\\{NWdepy-LaT8-1}}\nwendcode{}\nwbegindocs{53}\nwdocspar

\nwenddocs{}\nwbegincode{54}\sublabel{NWdepy-CSHP-1}\nwmargintag{{\nwtagstyle{}\subpageref{NWdepy-CSHP-1}}}\moddef{CSH Environment Variables~{\nwtagstyle{}\subpageref{NWdepy-CSHP-1}}}\endmoddef
# Global Variables
setenv WORK "/home/ug/jabril/development/jabril/software/deploy" ;
setenv BIN  "$WORK/bin" ;
setenv DOCS "$WORK/docs" ;
setenv DATA "$WORK/data" ;
setenv nwfile "deploy" ;
\nwnotused{CSH\ Environment\ Variables}\nwendcode{}\nwbegindocs{55}\nwdocspar

\nwenddocs{}\nwbegincode{56}\sublabel{NWdepy-tan8-4}\nwmargintag{{\nwtagstyle{}\subpageref{NWdepy-tan8-4}}}\moddef{tangling~{\nwtagstyle{}\subpageref{NWdepy-tan8-1}}}\plusendmoddef
# TO DO: add a test to check which shell is running
# BASH shell
notangle -R'BASH Environment Variables' $WORK/$nwfile.nw \\
         > $WORK/.bash_VARS ; 
# CSH shell
notangle -R'CSH Environment Variables'  $WORK/$nwfile.nw \\
         > $WORK/.csh_VARS ; 
# sourcing
source $WORK/.bash_VARS ;
source $WORK/.csh_VARS ;
\nwendcode{}

\nwixlogsorted{c}{{BASH Environment Variables}{NWdepy-BASQ-1}{\nwixu{NWdepy-wea7-1}\nwixu{NWdepy-LaT8-1}\nwixd{NWdepy-BASQ-1}}}%
\nwixlogsorted{c}{{BASH script end}{NWdepy-BASF-1}{\nwixd{NWdepy-BASF-1}\nwixu{NWdepy-wea7-1}\nwixu{NWdepy-LaT8-1}}}%
\nwixlogsorted{c}{{BASH shebang}{NWdepy-BASC-1}{\nwixd{NWdepy-BASC-1}\nwixu{NWdepy-wea7-1}\nwixu{NWdepy-LaT8-1}}}%
\nwixlogsorted{c}{{CSH Environment Variables}{NWdepy-CSHP-1}{\nwixd{NWdepy-CSHP-1}}}%
\nwixlogsorted{c}{{Common PERL subs - Benchmark}{NWdepy-ComS-1}{\nwixd{NWdepy-ComS-1}}}%
\nwixlogsorted{c}{{Common PERL subs - Counter}{NWdepy-ComQ.2-1}{\nwixd{NWdepy-ComQ.2-1}}}%
\nwixlogsorted{c}{{Common PERL subs - Min Max}{NWdepy-ComQ-1}{\nwixd{NWdepy-ComQ-1}}}%
\nwixlogsorted{c}{{Common PERL subs - STDERR}{NWdepy-ComP-1}{\nwixd{NWdepy-ComP-1}\nwixd{NWdepy-ComP-2}}}%
\nwixlogsorted{c}{{Common PERL subs - Text fill}{NWdepy-ComS.2-1}{\nwixd{NWdepy-ComS.2-1}}}%
\nwixlogsorted{c}{{GAWK shebang}{NWdepy-GAWC-1}{\nwixd{NWdepy-GAWC-1}}}%
\nwixlogsorted{c}{{Global Constants - Boolean}{NWdepy-GloQ-1}{\nwixd{NWdepy-GloQ-1}}}%
\nwixlogsorted{c}{{Global Vars - Counter}{NWdepy-GloL-1}{\nwixd{NWdepy-GloL-1}}}%
\nwixlogsorted{c}{{Global Vars - User and Date}{NWdepy-GloR-1}{\nwixd{NWdepy-GloR-1}}}%
\nwixlogsorted{c}{{LaTeXing}{NWdepy-LaT8-1}{\nwixd{NWdepy-LaT8-1}}}%
\nwixlogsorted{c}{{PERL shebang}{NWdepy-PERC-1}{\nwixd{NWdepy-PERC-1}}}%
\nwixlogsorted{c}{{PROJECT REPORT Template}{NWdepy-PRON-1}{\nwixd{NWdepy-PRON-1}}}%
\nwixlogsorted{c}{{Program Outline}{NWdepy-ProF-1}{\nwixd{NWdepy-ProF-1}}}%
\nwixlogsorted{c}{{Skip comments and empty records}{NWdepy-SkiV-1}{\nwixd{NWdepy-SkiV-1}}}%
\nwixlogsorted{c}{{Timer ON}{NWdepy-Tim8-1}{\nwixu{NWdepy-UseN-1}\nwixd{NWdepy-Tim8-1}}}%
\nwixlogsorted{c}{{Use Modules - Benchmark}{NWdepy-UseN-1}{\nwixd{NWdepy-UseN-1}}}%
\nwixlogsorted{c}{{Use Modules - Dumper}{NWdepy-UseK-1}{\nwixd{NWdepy-UseK-1}}}%
\nwixlogsorted{c}{{Version Control Id Tag}{NWdepy-VerM-1}{\nwixu{NWdepy-PERC-1}\nwixu{NWdepy-GAWC-1}\nwixu{NWdepy-BASC-1}\nwixd{NWdepy-VerM-1}}}%
\nwixlogsorted{c}{{tangling}{NWdepy-tan8-1}{\nwixd{NWdepy-tan8-1}\nwixd{NWdepy-tan8-2}\nwixd{NWdepy-tan8-3}\nwixd{NWdepy-tan8-4}}}%
\nwixlogsorted{c}{{weaving}{NWdepy-wea7-1}{\nwixd{NWdepy-wea7-1}}}%
\nwixlogsorted{i}{{{\char36}DATE}{:doDATE}}%
\nwixlogsorted{i}{{{\char36}F}{:doF}}%
\nwixlogsorted{i}{{{\char36}T}{:doT}}%
\nwixlogsorted{i}{{{\char36}USER}{:doUSER}}%
\nwixlogsorted{i}{{{\char36}c}{:doc}}%
\nwixlogsorted{i}{{{\char36}n}{:don}}%
\nwbegindocs{57}\nwdocspar

\end{document}
\nwenddocs{}
