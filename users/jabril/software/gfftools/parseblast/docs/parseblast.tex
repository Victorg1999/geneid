% -*- mode: Noweb; noweb-code-mode: perl-mode; tab-width: 4 -*-
\documentclass[11pt]{article}
%
%2345678901234567890123456789012345678901234567890123456789012345678901234567890
%        1         2         3         4         5         6         7         8
%
% # $Id: parseblast.tex,v 1.1 2001-09-06 18:26:37 jabril Exp $ 
%
\usepackage{noweb}
\usepackage[a4paper,offset={0pt,0pt},hmargin={2cm,2cm},vmargin={1cm,1cm}]{geometry}
\usepackage{graphics}
\usepackage[dvips]{graphicx}
%% pstricks
\usepackage[dvips]{pstcol}
\usepackage{pstricks}
%\usepackage{pst-node}
%\usepackage{pst-char}
%\usepackage{pst-grad}
%% bibliography
\usepackage{natbib}
%% latex2html
\usepackage{url}
\usepackage{html}     
\usepackage{htmllist} 
%% tables    
\usepackage{dcolumn}
%\usepackage{colortbl}
%\usepackage{multirow}
%\usepackage{hhline}
%\usepackage{tabularx}
%% seminar
%\usepackage{semcolor,semlayer,semrot,semhelv,sem-page,slidesec}
%% draft watermark
%\usepackage[all,dvips]{draftcopy}
%\draftcopySetGrey{0.9}
%\draftcopyName{CONFIDENTIAL}{100}
%% layout
\usepackage{fancyhdr} % Do not use \usepackage{fancybox} -> TOCs disappear
%\usepackage{lscape}
%\usepackage{rotating}
%\usepackage{multicol}
%% fonts
\usepackage{times}\fontfamily{ptm}\selectfont
\usepackage{t1enc}

% noweb options
\noweboptions{smallcode}
\def\nwendcode{\endtrivlist \endgroup} % relax page breaking scheme
\let\nwdocspar=\par                    %

\input defs.tex % from <LaTeX new definitions> chunk

%
%%%%%%%%%%%%%%%%%%%%%%%%%%%%%%%%%%%%%%%%%%%%%%%%%%%%%%%%%%%%%%%%%%%%%%%%%%%%%%%%
%
\begin{document}
%
\nwfilename{/home/ug/jabril/development/softjabril/parseblast/parseblast.nw}%
%
%
%
%
%
%
%
%
%
%
%
%
%
%
%
%
%
%
%
%
%
%
%
%
%
%
%
%
%
%
%
%
%
%
%
%
%
%
%
%
%
%
%
%
%
%
%
%
%
%
%
%
%
%
%
%
%
%
%
%
%
%
%
%
%
%
%
%
%
%
%
%
%
%
%
%
%
%
%
%
%
%
%
\nwbegindocs{2}\nwdocspar

\nwenddocs{}%
%
%
%
%
%
%
%
%
%
\nwbegindocs{4}\nwdocspar

\nwenddocs{}%
\nwbegindocs{6}\nwdocspar
\nwenddocs{}%
\nwbegindocs{8}\nwdocspar
\nwenddocs{}%
\nwbegindocs{10}\nwdocspar

\thispagestyle{empty}

\begin{titlepage}

\ \vfill
\begin{center}
\begin{bfseries}
\begin{large}
\newlength{\lttbl}\setlength{\lttbl}{0.25\linewidth}
\newlength{\rttbl}\setlength{\rttbl}{0.70\linewidth}
%\fbox{
%\vskip 2ex
\begin{tabular}{>{\scshape}r@{\quad}l}
\rule{\lttbl}{0pt} & \rule{\rttbl}{0pt} \\[2ex]
\multicolumn{2}{c}{\shortstack{\rule[0ex]{0.95\linewidth}{2pt}\\[0ex]
                               \rule[1ex]{0.95\linewidth}{2pt}}}\\[2ex]
Program Name: & {\Huge\progname}                       \\[3ex]
\multicolumn{2}{c}{\rule[0.5ex]{0.95\linewidth}{2pt}}\\[2ex]
      Author: & {\Large
                 \begin{minipage}[t]{0.95\rttbl}
                 \authorslist
                 \end{minipage}}                       \\[2ex]
     License: & {\license}                             \\[2ex]
 Last Update: & {\today}                               \\[2ex]
 Description: & {\large\mdseries
                 \begin{minipage}[t]{0.95\rttbl}
                 \description
                 \end{minipage}}                       \\[2ex]
\\
\multicolumn{2}{c}{\shortstack{\rule[0ex]{0.95\linewidth}{2pt}\\[0ex]
                               \rule[1ex]{0.95\linewidth}{2pt}}}\\[2ex]
\end{tabular}
%} % fbox
\end{large}
\end{bfseries}
\end{center}

\vfill

\begin{raggedleft}
\showaffiliation
\end{raggedleft}

\end{titlepage} %'

%
%%%%%%%%%%%%%%%%%%%% FRONTMATTER

\newpage %%%%%%%%%%%%%%%%%%%%%%%%%%%%%%%%%%%%%%%%%%%%%%%%%
\pagenumbering{roman}
\setcounter{page}{1}
\pagestyle{fancy}
% Marks redefinition must go here because pagestyle 
% resets the values to the default ones.
\renewcommand{\sectionmark}[1]{\markboth{}{\thesection.\ #1}}
\renewcommand{\subsectionmark}[1]{\markboth{}{\thesubsection.\ \textsl{#1}}}

\tableofcontents
\listoftables
\listoffigures

\vfill
\begin{center}
{\small$<$ \verb$Id: parseblast.tex,v 1.1 2001-09-06 18:26:37 jabril Exp $$>$ }
\end{center}

%%%%%%%%%%%%%%%%%%%% MAINMATTER

\newpage %%%%%%%%%%%%%%%%%%%%%%%%%%%%%%%%%%%%%%%%%%%%%%%%%

\pagenumbering{arabic}
\setcounter{page}{1}

\sctn{Introduction}

\subsctn{Description}

\subsctn{Input}

\subsctn{Output}

% \subsctn{Comments}

\subsctn{To Do}

\begin{itemize}
 \input todo.tex
\end{itemize}

\newpage %%%%%%%%%%%%%%%%%%%%%%%%%%%%%%%%%%%%%%%%%%%%%%%%%

\sctn{Implementation}

\subsctn{Program outline}

\nwenddocs{}\nwbegincode{11}\sublabel{NWpar11-parA-1}\nwmargintag{{\nwtagstyle{}\subpageref{NWpar11-parA-1}}}\moddef{parseblast~{\nwtagstyle{}\subpageref{NWpar11-parA-1}}}\endmoddef
\LA{}PERL shebang~{\nwtagstyle{}\subpageref{NWpar11-PERC-1}}\RA{}
#
# MODULES
#
\LA{}Use Modules~{\nwtagstyle{}\subpageref{NWpar11-UseB-1}}\RA{}
#
# VARIABLES
#
\LA{}Global Vars~{\nwtagstyle{}\subpageref{NWpar11-GloB-1}}\RA{}
#
# MAIN LOOP
#
\LA{}Main Loop~{\nwtagstyle{}\subpageref{NWpar11-Mai9-1}}\RA{}
#
# FUNCTIONS
#
\LA{}Functions~{\nwtagstyle{}\subpageref{NWpar11-Fun9-1}}\RA{}
#
# TEMPORARY
# 
\LA{}whole script~{\nwtagstyle{}\subpageref{NWpar11-whoC-1}}\RA{}
\nwnotused{parseblast}\nwendcode{}\nwbegindocs{12}\nwdocspar

\nwenddocs{}\nwbegincode{13}\sublabel{NWpar11-UseB-1}\nwmargintag{{\nwtagstyle{}\subpageref{NWpar11-UseB-1}}}\moddef{Use Modules~{\nwtagstyle{}\subpageref{NWpar11-UseB-1}}}\endmoddef
\nwused{\\{NWpar11-parA-1}}\nwendcode{}\nwbegindocs{14}\nwdocspar

\nwenddocs{}\nwbegincode{15}\sublabel{NWpar11-GloB-1}\nwmargintag{{\nwtagstyle{}\subpageref{NWpar11-GloB-1}}}\moddef{Global Vars~{\nwtagstyle{}\subpageref{NWpar11-GloB-1}}}\endmoddef
\nwused{\\{NWpar11-parA-1}}\nwendcode{}\nwbegindocs{16}\nwdocspar

\nwenddocs{}\nwbegincode{17}\sublabel{NWpar11-Mai9-1}\nwmargintag{{\nwtagstyle{}\subpageref{NWpar11-Mai9-1}}}\moddef{Main Loop~{\nwtagstyle{}\subpageref{NWpar11-Mai9-1}}}\endmoddef

exit(0);
\nwused{\\{NWpar11-parA-1}}\nwendcode{}\nwbegindocs{18}\nwdocspar

\nwenddocs{}\nwbegincode{19}\sublabel{NWpar11-Fun9-1}\nwmargintag{{\nwtagstyle{}\subpageref{NWpar11-Fun9-1}}}\moddef{Functions~{\nwtagstyle{}\subpageref{NWpar11-Fun9-1}}}\endmoddef
sub \{
\} # 
\nwused{\\{NWpar11-parA-1}}\nwendcode{}\nwbegindocs{20}\nwdocspar

\label{todo:AAA}
\nwenddocs{}%
%
\nwbegindocs{22}\nwdocspar
\nwenddocs{}%
%
\nwbegindocs{24}\nwdocspar
\todo{ \item \todoAAA } % todo

\nwenddocs{}\nwbegincode{25}\sublabel{NWpar11-whoC-1}\nwmargintag{{\nwtagstyle{}\subpageref{NWpar11-whoC-1}}}\moddef{whole script~{\nwtagstyle{}\subpageref{NWpar11-whoC-1}}}\endmoddef
my $PROGRAM = "parseblast";
my $VERSION = "Version 1.0";
my $Start = time;

use strict;
use Getopt::Long;


##############################################################################
##                       Getting Comand-line Options                        ##
##############################################################################

Getopt::Long::Configure("bundling");

my ($hsp_flg, $gff_flg, $fullgff_flg, $aplot_flg, $nogff_flg, $subject_flg, $sequence_flg,
    $full_scores, $comment_flg, $nocmmnt_flg, $split_flg, $help_flg, $err_flg,
    $expanded_flg, $pairwise_flg, $msf_flg, $aln_flg, $bit_flg, $ids_flg);
my ($prt, $main, $seqflg, $hsp, $fragment, $param, $aln_split, $prt_pos_flg
    ) = (0, 0, 0, 0, 0, 0, 0, 0);

GetOptions( "G|gff"            => \\$gff_flg      ,
            "F|fullgff"        => \\$fullgff_flg  ,
            "A|aplot"          => \\$aplot_flg    ,
            "S|subject"        => \\$subject_flg  ,
            "Q|sequence"       => \\$sequence_flg ,
            "X|extended"       => \\$expanded_flg ,
            "P|pairwise"       => \\$pairwise_flg ,
            "M|msf"            => \\$msf_flg      ,
            "N|aln"            => \\$aln_flg      ,
            "W|show-coords"    => \\$prt_pos_flg  ,
            "b|bit-score"      => \\$bit_flg      ,
            "i|identity-score" => \\$ids_flg      ,
            "s|full-scores"    => \\$full_scores  ,
            "c|comments"       => \\$comment_flg  ,
            "n|no-comments"    => \\$nocmmnt_flg  ,
            "v|verbose"        => \\$err_flg      ,
            "h|help|\\?"        => \\$help_flg      );

($help_flg) && &prt_help;

\{   # first choose disables any other command-line option.
    $aln_flg
        && ($expanded_flg=$msf_flg=$pairwise_flg=$hsp_flg=$comment_flg=0,
            $nogff_flg=1, last);
    $msf_flg
        && ($expanded_flg=$aln_flg=$pairwise_flg=$hsp_flg=$comment_flg=0,
            $nogff_flg=1, last);
    $pairwise_flg
        && ($expanded_flg=$aln_flg=$msf_flg=$hsp_flg=$comment_flg=0,
            $nogff_flg=1, last);
    $gff_flg
        && ($expanded_flg=$aln_flg=$msf_flg=$pairwise_flg=$fullgff_flg=$aplot_flg=$hsp_flg=0,
            $nogff_flg=0, last);
    $fullgff_flg
        && ($expanded_flg=$aln_flg=$msf_flg=$pairwise_flg=$gff_flg=$aplot_flg=$hsp_flg=0,
            $nogff_flg=0, last);
    $aplot_flg
        && ($expanded_flg=$aln_flg=$msf_flg=$pairwise_flg=$gff_flg=$fullgff_flg=$hsp_flg=0,
            $nogff_flg=0, last);
    $expanded_flg
        && ($aln_flg=$msf_flg=$pairwise_flg=$hsp_flg=0,
            $nogff_flg=1, last);
    $hsp_flg = 1; $nogff_flg = 1;
\};

\{   # first choose disables any other command-line option.  
    $bit_flg && ($ids_flg=0, last);
    $ids_flg && ($bit_flg=0, last);
\};

$nocmmnt_flg && ($comment_flg = 0);


##############################################################################
##                          Global Variables                                ##
##############################################################################

my ($prog_params, $program, $version, $seqname);
my ($scQ, $scS);   # reSCale_Query/Subject : 1 to re-scalate lengths.
my ($query_name, $db_name, $score, $descr,
    $ori, $end, $seq, $txt, $tt, $pt, $ht);
my (@seqlist, %prgseq, %dbase, %query, %cnt,
    %desc, %sco, %hsp_start, %hsp_end, %hsp_seq);
my ($qm, $sm, $x, $y, $ml, $a, $b, $aq, $as, $sql, $lq, $ls, $sq);
my $foe = 0;
my $index = 0; 
my $chars_per_line = 50; # chars of sequences to show per line (in alignment modes)


##############################################################################
##                          Global Subroutines                              ##
##############################################################################

sub prt_progress \{
    $err_flg && do \{
        print STDERR ".";
        (($_[0] % 50) == 0) && print STDERR "[".&fill_left($_[0],6,"0")."]\\n";
    \};
\}
sub prt_foeprg \{
    $err_flg && ((($_[0] % 50) != 0) && print STDERR "[".&fill_left($_[0],6,"0")."]\\n" );
\}

sub fill_right \{ $_[0].($_[2] x ($_[1] - length($_[0]))) \}
sub fill_left  \{ ($_[2] x ($_[1] - length($_[0]))).$_[0] \}

sub max \{ my ($z) = shift @_; my $l; foreach $l (@_) \{ $z = $l if $l > $z ; \}; $z; \}

#
# Timing.

sub get_exec_time \{
    $err_flg && do \{
        my $End = $_[0];
        my ($c,$s,$m,$h,$r);
        $r = $End - $Start;
        $s = $r % 60;
        $r = ($r - $s) / 60;
        $m = $r % 60;
        $r = ($r - $m) / 60;
        $h = $r % 24;
        ($s,$m,$h) = (&fill_left($s,2,"0"),&fill_left($m,2,"0"),&fill_left($h,2,"0"));
print STDERR <<EOF;
##########################################################
## \\"$PROGRAM\\"  Execution Time:  $h:$m:$s
##########################################################
EOF
    \};
\} # END_SUB: get_exec_time
    
#
# Print help to STDERR.

sub prt_help \{
    open(HELP, "| more") ;
    print HELP <<EndOfHelp;
PROGRAM:
        $PROGRAM
        $VERSION

USAGE:  parseblast.pl [options] <results.from.blast>

COMMAND-LINE OPTIONS:

    "$PROGRAM" prints output in "HSP" format by default (see below).
  It takes input from <STDIN> or single/multiple files, and writes
  its output to <STDOUT>, so user can redirect to a file but
  he also could use the program as a filter within a pipe. 
    "-N", "-M", "-P", "-G", "-F", "-A" and "-X" options (also the long
  name versions for each one) are mutually exclusive, and their
  precedence order is shown above.

  GFF OPTIONS:

    -G, --gff            : prints output in GFF format.
    -F, --fullgff        : prints output in GFF "alignment" format.
    -A, --aplot          : prints output in APLOT "GFF" format.
    -S, --subject        : projecting GFF output by SUBJECT (default by QUERY).
    -Q, --sequence       : append query and subject sequences to GFF record.
    -b, --bit-score      : set <score> field to Bits (default Alignment Score).
    -i, --identity-score : set <score> field to Identities (default Alignment).
    -s, --full-scores    : include all scores for each HSP in each GFF record.

  ALIGNMENT OPTIONS:

    -P, --pairwise       : prints pairwise alignment for each HSP in TBL format.
    -M, --msf            : prints pairwise alignment for each HSP in MSF format.
    -N, --aln            : prints pairwise alignment for each HSP in ALN format.
    -W, --show-coords    : adds start/end positions to alignment output.

  GENERAL OPTIONS:

    -X, --expanded       : expanded output (producing multiline output records).
    -c, --comments       : include parameters from blast program as comments.
    -n, --no-comments    : do not print "#" lines (raw output without comments).
    -v, --verbose        : warnings sent to <STDERR>.
    -h, --help           : shows this help pages.

OUTPUT FORMATS:

    "S_" stands for "Subject_Sequence" and "Q_" for "Query_Sequence". <Program>
  name is taken from input blast file. <Strands> are calculated from <start> and
  <end> positions on original blast file. <Frame> is obtained from the blast 
  file if is present else is set to ".". <SCORE> is set to Alignment Score by 
  default, you can change it with "-b" and "-i".
    If "-S" or "--subject" options are given, then QUERY fields are referred to
  SUBJECT and SUBJECT fields are relative to QUERY (this only available for GFF
  output records).
    Dots ("...") mean that record description continues in the following line,
  but such record is printed as a single line by "$PROGRAM".

[HSP]  <- (This is the DEFAULT OUTPUT FORMAT)
 <Program> <DataBase> : ...
   ... <IdentityMatches> <Min_Length> <IdentityScore> ...
   ... <AlignmentScore> <BitScore> <E_Value> <P_Sum> : ...
   ... <Q_Name> <Q_Start> <Q_End> <Q_Strand> <Q_Frame> : ...
   ... <S_Name> <S_Start> <S_End> <S_Strand> <S_Frame> : <S_FullDescription>

[GFF]
 <Q_Name> <Program> hsp <Q_Start> <Q_End> <SCORE> <Q_Strand> <Q_Frame> <S_Name>

[FULL GFF]  <- (GFF showing alignment data)
 <Q_Name> <Program> hsp <Q_Start> <Q_End> <SCORE> <Q_Strand> <Q_Frame> ...
   ... Target "<S_Name>" <S_Start> <S_End> ...
   ... E_value <E_Value> Strand <S_Strand> Frame <S_Frame>

[APLOT]  <- (GFF format enhanced for APLOT program)
 <Q_Name>:<S_Name> <Program> hsp <Q_Start>:<S_Start> <Q_End>:<S_End> <SCORE> ...
   ... <Q_Strand>:<S_Strand> <Q_Frame>:<S_Frame> <BitScore>:<HSP_Number> ...
   ... \\# E_value <E_Value>

[EXPANDED]
 MATCH(<HSP_Number>): <Q_Name> x <S_Name>
 SCORE(<HSP_Number>): <AlignmentScore>
 BITSC(<HSP_Number>): <BitScore>
 EXPEC(<HSP_Number>): <E_Value> Psum(<P_Sum>)
 IDENT(<HSP_Number>): <IdentityMatches>/<Min_Length> : <IdentityScore> \\%
 T_GAP(<HSP_Number>): <TotalGaps(BothSeqs)>
 FRAME(<HSP_Number>): <Q_Frame>/<S_Frame>
 STRND(<HSP_Number>): <Q_Strand>/<S_Strand>
 MXLEN(<HSP_Number>): <Max_Length>
 QUERY(<HSP_Number>): length <Q_Length> : gaps <Q_TotalGaps> : ...
   ... <Q_Start> <Q_End> : <Q_Strand> : <Q_Frame> : <Q_FullSequence>
 SBJCT(<HSP_Number>): length <S_Length> : gaps <S_TotalGaps> : ...
   ... <S_Start> <S_End> : <S_Strand> : <S_Frame> : <S_FullSequence>

BUGS:    Report any problem to: abril\\@imim.es

AUTHOR:  $PROGRAM is under GNU-GPL (C) 2000 - Josep F. Abril

EndOfHelp
    close(HELP);
exit(1);
# print STDOUT "HSP:$hsp_flg GFF:$gff_flg COM:$comment_flg SPL:$split_flg HLP:$help_flg\\n";
\}

#
# Get new lines while empty line is not found and append to last line.

sub get_lines \{
    my $spc = $_[0];
    my $tmp;
    # local($tmp);
    while (<>) \{
        last if /^\\s*$/;
        # print STDERR "$_";
        chop;
        s/^\\s*/$spc/;
        $tmp .= $_;
    \};
    $tmp;
\}

#
# Getting scores from scoring vector extracted from HSP record.

sub get_scores \{
    my $t = $_[0]; 
    my ($sc, $bt, $ex, $pv, $id);
    my ($qfr, $sfr) = ('.', '.');
    (($t =~ /Score[^\\s]*\\s+=\\s+\\b(\\d+)\\b\\s+\\([^,]*,/) || ($t =~ /Score\\s+=\\s+[^,]*\\s+\\((\\d+)\\)[^,]*,/o)) &&
        ($sc = $1);
    ($t =~ /Score[^\\s]*\\s+=.*[\\s\\(]([+-]?(\\d+\\.?\\d*|\\.\\d+))\\b \\bbits[^,]*,/o) && 
        ($bt = $1);
    ($t =~ /Expect[^\\s]*\\s+=\\s+([+-]?([Ee][+-]?\\d+|(\\d+\\.?\\d*|\\.\\d+)([Ee][+-]?\\d+)?))\\s*,/o) && 
        ($ex = $1);
    ($ex =~ /^[Ee]/o) && ($ex = "1".$ex);
    ($t =~ /Sum[^\\s]*\\s+\\bP[^\\s]*\\s+=\\s+([+-]?([Ee][+-]?\\d+|(\\d+\\.?\\d*|\\.\\d+)([Ee][+-]?\\d+)?))\\s*,/o) ?
        ($pv = $1) : ($pv = "."); # $pv not defined then $pv="."
    ($t =~ /Identities[^\\s]*\\s+=\\s+(\\d+)\\/(\\d+)\\s+/o) && ($id = $1);
    ( $scQ && !$scS) && do \{   # BLASTX (translated nucleotides vs protein)
        ($t =~ /Frame[^\\s]*\\s+=\\s+(\\+|\\-)(\\d)\\b/o) && ($qfr = $2);
    \};
    (!$scQ &&  $scS) && do \{   # TBLASTN (protein vs translated nucleotides)
        ($t =~ /Frame[^\\s]*\\s+=\\s+(\\+|\\-)(\\d)\\b/o) && ($sfr = $2);
    \};
    ( $scQ &&  $scS) && do \{   # TBLASTX (translated nucleotides vs translated nucleotides)
        ($t =~ /Frame[^\\s]*\\s+=\\s+(\\+|\\-)(\\d)\\s*\\/\\s*(\\+|\\-)(\\d)\\b/o) && ($qfr = $2, $sfr = $4);
    \};
    $sc, $bt, $ex, $pv, $id, $qfr, $sfr;
\}

sub chk_strand \{
    my ($first, $last) = @_ ;
    my $st = "+";
    ($first >= $last) && (($first, $last) = ($last, $first), $st = "-");
    $first, $last, $st;
\}

#
# Formatting output as plain, HSPs, GFF, APLOT, ALN.
my ($n, $nm, $wnm, $tq, $ts,       # couNter, NaMe, TagQuery, TagSubject
    $sc, $bt, $ex, $pv, $id,       # SCore, BiTscore, EvalueX, IDentityscore
    $scores,                       # showing all scores
    $frq, $frs,                    # QueryFRame, SubjectFRame
    $stq, $sts,                    # QuerySTrand, SubjectSTrand
    $gsc, $prg,                    # GroupSCore, PRoGram
    $hsq, $hss, $heq, $hes,        # HspStartQuery, HspStartSubject, HspEndQuery, HspEndSubject
    $lnq, $lns, $lnmin, $lnmax,    # LeNgthQuery, LeNgthSubject, LeNgthMINqueyxsubject, LeNgthMAXqueyxsubject
    $lnmx, $gpq, $gps, $gpt        # LeNgthMaXhspseq, GaPQuery, GaPSubject, GaPTotal
    );

sub prt_hsp \{
print STDOUT <<"EndOfHSPs";
$prg $dbase\{$nm\} : $id $lnmin $gsc $sc $bt $ex $pv : $query\{$nm\} $hsq $heq $stq $frq : $wnm $hss $hes $sts $frs : $desc\{$nm\}
EndOfHSPs
last PRINT;
\}
sub prt_ext \{
print STDOUT <<"EndOfPlain";
MATCH($n): $query\{$nm\} x $wnm
SCORE($n): $sc\\nBITSC($n): $bt\\nEXPEC($n): $ex Psum($pv)
IDENT($n): $id/$lnmin : $gsc \\%
T_GAP($n): $gpt\\nFRAME($n): $frq/$frs\\nSTRND($n): $stq/$sts\\nMXLEN($n): $lnmx
QUERY($n): length $lnq : gaps $gpq : $hsq $heq : $stq : $frq : $hsp_seq\{$tq\}
SBJCT($n): length $lns : gaps $gps : $hss $hes : $sts : $frs : $hsp_seq\{$ts\}
\\#\\#
EndOfPlain
last PRINT;
\}
sub prt_Q_gff \{
print STDOUT <<"EndOfGFF";
$query\{$nm\}\\t$prg\\thsp\\t$hsq\\t$heq\\t$gsc\\t$stq\\t$frq\\t$wnm\\t\\# E_value $ex$scores$sq
EndOfGFF
last PRINT;
\}
sub prt_S_gff \{
print STDOUT <<"EndOfGFF";
$wnm\\t$prg\\thsp\\t$hss\\t$hes\\t$gsc\\t$sts\\t$frs\\t$query\{$nm\}\\t\\# E_value $ex$scores$sq
EndOfGFF
last PRINT;
\}
sub prt_Q_fullgff \{
print STDOUT <<"EndOfFullGFF";
$query\{$nm\}\\t$prg\\thsp\\t$hsq\\t$heq\\t$gsc\\t$stq\\t$frq\\tTarget \\"$wnm\\"\\t$hss\\t$hes ;\\tStrand $sts ;\\tFrame $frs ;\\tE_value $ex$scores$sq
EndOfFullGFF
last PRINT;
\}
sub prt_S_fullgff \{
print STDOUT <<"EndOfFullGFF";
$wnm\\t$prg\\thsp\\t$hss\\t$hes\\t$gsc\\t$sts\\t$frs\\tTarget \\"$query\{$nm\}\\"\\t$hsq\\t$heq ;\\tStrand $sts ;\\tFrame $frs ;\\tE_value $ex$scores$sq
EndOfFullGFF
last PRINT;
\}
sub prt_Q_aplot \{
print STDOUT <<"EndOfAPLOT";
$query\{$nm\}:$wnm\\t$prg\\thsp\\t$hsq:$hss\\t$heq:$hes\\t$gsc\\t$stq:$sts\\t$frq:$frs\\t$bt:$n\\t\\# E_value $ex$scores$sq
EndOfAPLOT
last PRINT;
\}
sub prt_S_aplot \{
print STDOUT <<"EndOfAPLOT";
$wnm:$query\{$nm\}\\t$prg\\thsp\\t$hss:$hsq\\t$hes:$heq\\t$gsc\\t$sts:$stq\\t$frs:$frq\\t$bt:$n\\t\\# E_value $ex$scores$sq
EndOfAPLOT
last PRINT;
\}
sub prt_pairwise \{
    $prt_pos_flg && do \{
        $ml = &max(length($hsq),length($heq),length($hss),length($hes));
        ($x,$y) = (" ".&fill_left($hsq,$ml," ")." ".&fill_left($heq,$ml," ")." $stq $frq ",
                   " ".&fill_left($hss,$ml," ")." ".&fill_left($hes,$ml," ")." $sts $frs ");
    \};
    print "#\\n" if $aln_split;
    $aln_split = 1;
print STDOUT <<"EndOfALIGN";
$a$x$hsp_seq\{$tq\}
$b$y$hsp_seq\{$ts\}
EndOfALIGN
last PRINT;
\}
sub prt_cmn_aln \{
print STDOUT <<"EndOfALN" if !$nocmmnt_flg;
################################################################################
##
##  $query\{$nm\} x $wnm #$n
##
##    $prg $dbase\{$nm\}
##    Identity: $gsc  Score: $sc  Bits: $bt  E_value: $ex  P_value: $pv
##    DESCR: $desc\{$nm\}
##
EndOfALN
\}
sub prt_msf \{
    my ($i,$jq,$js,$vs,$ve,$chw);
    ($aq,$as,$sql) = ($hsp_seq\{$tq\},$hsp_seq\{$ts\},length($hsp_seq\{$tq\}));
    $aq =~ s/\\.//og;
    $as =~ s/\\.//og;
    ($lq,$ls) = (length($aq),length($as));
    &prt_cmn_aln;
print STDOUT <<"EndOfMSF";
\\n
$query\{$nm\}\\_x_$wnm\\_\\#$n.msf  MSF: $sql  Type: P  May 4th, 2000  Check: 0  ..\\n
Name: $a  Len: $lq  Check: 0  Weight: 1.00
Name: $b  Len: $ls  Check: 0  Weight: 1.00
\\n//
EndOfMSF
    for ($i=0; $i<=($sql-1); $i+=$chars_per_line) \{
        ($jq, $js) = (substr($hsp_seq\{$tq\},$i,50), substr($hsp_seq\{$ts\},$i,50));
        print STDOUT "\\n";
        $prt_pos_flg && do \{
            $chw = length($jq);
            ($vs, $ve) = ($i+1, $i+$chw);
            $chw = $chw-length($vs.$ve);            
            print STDOUT " "x($ml+1).$vs." "x($chw>1 ? $chw : 1).$ve."\\n";
        \};
print STDOUT <<"EndOfALIGN";
$a $jq
$b $js
EndOfALIGN
    \};
    print STDOUT "\\n\\n";
    last PRINT;
\}
sub prt_aln \{
    my ($i,$jq,$js,$vs,$ve,$chw);
    ($aq,$as,$sql) = ($hsp_seq\{$tq\},$hsp_seq\{$ts\},length($hsp_seq\{$tq\}));
    $aq =~ s/\\./-/og;
    $as =~ s/\\./-/og;
    &prt_cmn_aln;
    for ($i=0; $i<=($sql-1); $i+=$chars_per_line) \{
        ($jq, $js) = (substr($aq,$i,50), substr($as,$i,50));
        print STDOUT "\\n";
        $prt_pos_flg && do \{
            $chw = length($jq);
            ($vs, $ve) = ($i+1, $i+$chw);
            $chw = $chw-length($vs.$ve);            
            print STDOUT " "x($ml+1).$vs." "x($chw>1 ? $chw : 1).$ve."\\n";
        \};
print STDOUT <<"EndOfALIGN";
$a $jq
$b $js
EndOfALIGN
    \};
    print STDOUT "\\n\\n";
    last PRINT;
\}

sub prt_out \{
    $prt = 0;
    $err_flg && print STDERR ("#"x58)."\\n## WRITING OUTPUT TO STDOUT ".("#"x30)."\\n".("#"x58)."\\n";
    while (@seqlist) \{
        $nm = shift(@seqlist);
        ($wnm = $nm) =~ s/\\_\\d+$//o; 
        (!$hsp_flg && $comment_flg) && (print STDOUT "#\\n# $prgseq\{$nm\} :: DB $dbase\{$nm\} :: $cnt\{$nm\} HSPs for $query\{$nm\}x$wnm \\n# DESCR: $desc\{$nm\}\\n#\\n");
        ($cnt\{$nm\}>0) && do \{
            for ($n = 1; $n <= $cnt\{$nm\}; $n++) \{
                &prt_progress(++$ht);
                $tq = $nm."query".$n;
                $ts = $nm."sbjct".$n;
                ($sc, $bt, $ex, $pv, $id, $frq, $frs) = &get_scores($sco\{$nm.$n\});
                ($hsq, $heq, $stq) = &chk_strand($hsp_start\{$tq\}, $hsp_end\{$tq\});
                ($hss, $hes, $sts) = &chk_strand($hsp_start\{$ts\}, $hsp_end\{$ts\});
                $lnq = $heq - $hsq + 1 ;
                $scQ && ($lnq = $lnq / 3) ;
                $lns = $hes - $hss + 1 ; 
                $scS && ($lns = $lns / 3) ;
                $lnmin = ($lnq>$lns) ? $lns : $lnq;
                $lnmax = ($lnq<$lns) ? $lns : $lnq;
                $lnmx = length($hsp_seq\{$tq\});
                \{
                    my $hh = $hsp_seq\{$tq\};
                    $gpq = ($hh =~ s/-/ /og) || 0; 
                    $hh = $hsp_seq\{$ts\};
                    $gps = ($hh =~ s/-/ /og) || 0;
                \};
                $gpt = $gpq + $gps;
                \{
                    ($ids_flg || $expanded_flg || $hsp_flg) && # score is Identities divided by minlength
                        (($gsc) = eval(($id/$lnmin)*100) =~ /^(\\d+(\\.\\d\{0,3\})?)/o, last);
                    $bit_flg && ($gsc = $bt, last);
                    $gsc = $sc;
                \};
                ($prg) = $prgseq\{$nm\} =~ /^([^\\s]+)\\s/o;
                # GFF format
              PRINT: \{
                  &prt_hsp       if $hsp_flg; # default output
                  #
                  do \{
                      $sq = $scores = "";
                      $scores = " ;\\tP_sum $pv ;\\tAln_Score $sc ;\\tBit_Score $bt ;\\tIdn_Score ".
                          sprintf("%6.2f",($id/$lnmin) * 100)." ($id/$lnmin) ;\\tGaps ".
                          sprintf("%4.2f",($gpt/$lnmx) * 100).
                          " (Q:$gpq|S:$gps) ;\\tLengths Q:$lnq|S:$lns|T:$lnmx"
                          if $full_scores;
                      $subject_flg && do \{
                          $sq = " #-S: $hsp_seq\{$ts\} #-Q: $hsp_seq\{$tq\}" if $sequence_flg;
                          &prt_S_gff     if $gff_flg;
                          &prt_S_fullgff if $fullgff_flg;
                          &prt_S_aplot   if $aplot_flg;
                      \}; # $subject_flg
                      $sq = " #-Q: $hsp_seq\{$tq\} #-S: $hsp_seq\{$ts\}" if $sequence_flg;
                      &prt_Q_gff     if $gff_flg;
                      &prt_Q_fullgff if $fullgff_flg;
                      &prt_Q_aplot   if $aplot_flg;
                  \} unless $nogff_flg;
                  #
                  &prt_ext       if $expanded_flg;
                  #
                  # ($qm, $sm, $x, $y) = ("$query\{$nm\}\\_\\#$n", "$nm\\_\\#$n", " ", " ");
                  ($qm, $sm, $x, $y) = ("$query\{$nm\}", "$wnm", " ", " ");
                  $ml = &max(length($qm),length($sm));
                  ($a,$b) = (&fill_right($qm,$ml," "),&fill_right($sm,$ml," "));
                  &prt_pairwise  if $pairwise_flg;
                  &prt_msf       if $msf_flg;
                  &prt_aln       if $aln_flg;
              \} # PRINT
            \} # for $cnt\{$nm\}
        \} # do if $cnt\{$nm\}>0
    \} # foreach
    &prt_foeprg($ht);
\}


###################################################
## Main Loop
###################################################

$err_flg && print STDERR ("#"x58)."\\n## PARSING STDIN FROM BLAST ".("#"x30)."\\n".("#"x58)."\\n";
while (<>) \{
    # s/\\r\\n$/\\n/; # if your input records finish with "\\r\\n" (like EMBL).
    next if /^\\s*$/;  # /^\\s*$/ is similar to AWK /^[ \\t]*$/
    &prt_progress(++$pt);
    my $tmpinput = $_;
    chop;
    # print STDOUT "$. : $_ \\n"; # "$." is record number && "$_" is whole record
  CHECK: \{
      /^\\s*T?BLAST[PNX]?/o        && do \{ # Starts with "T?BLAST[PNX]?" ?
          # print STDERR "$_\\n";
          $prt && &prt_out;
          ($program, $version) = split;
          # typeQ/typeS: 0 for proteins - 1 for nucleic acids.
          ($program =~ /^BLASTP$/o ) && ( $scQ = 0, $scS = 0); # Amino Acids vs Amino Acids
          ($program =~ /^BLASTN$/o ) && ( $scQ = 0, $scS = 0); # Nucleotides vs Nucleotides
          ($program =~ /^BLASTX$/o ) && ( $scQ = 1, $scS = 0); # Nucleotides vs Amino Acids 
          ($program =~ /^TBLASTN$/o) && ( $scQ = 0, $scS = 1); # Amino Acids vs Nucleotides
          ($program =~ /^TBLASTX$/o) && ( $scQ = 1, $scS = 1); # Nucleotides translated vs Nucleotides translated
          $prog_params = "#\\n# $program $version\\n#";
          $query_name = $db_name = '';
          $main = 1;
          $seqflg = $hsp = $fragment = $param = 0;
          last CHECK; 
      \}; # /^\\s*T?BLAST[PNX]?/
      /^>/o                    && do \{ # Starts with ">" ?: sequences.
          # print STDERR "$_\\n";
          ($seqname, $descr) = split(/\\s+/, $_, 2);
          $seqname =~ s/^\\s*>//o;
          $seqname =~ s/\\s|:|\\|/_/og;
          $seqname .= "_".($index++);
          $prgseq\{$seqname\} = "$program ($version)";
          $query\{$seqname\} = $query_name;
          ($db_name =~ /([^\\/]+)$/o) && ($dbase\{$seqname\} = $1);
          push(@seqlist,$seqname);
          $desc\{$seqname\} = join(' ', $descr, &get_lines(' '));
#         $cnt\{$seqname\} = 0;
          $seqflg = 1;
          $main = $hsp = $fragment = 0;
          last CHECK;
      \}; # /^\\s*>/
      (/^\\s*Score/o && $seqflg) && do \{ # Starts with "Score" ?: HSPs.
          # print STDERR "$_\\n";
          ( $score = $_ ) =~ s/^\\s*//o;
          $cnt\{$seqname\}++;
          $sco\{$seqname.$cnt\{$seqname\}\} = join('', $score, &get_lines(', '));
          # print STDOUT $sco\{$seqname.$cnt\{$seqname\}\}."\\n";
          $hsp = 1;
          $fragment = 0;
          last CHECK;
      \}; # (/^\\s*Score/ && ($seqflg))
      (/^Query:/o && $hsp)    && do \{ # Starts with "Query" ?: Fragments.
          # print STDERR "$_\\n";
          $fragment = 1;          
          last CHECK;
      \}; # (/^\\s*Query/ && ($hsp))
      (/^\\s*(?:Database|Parameters):/o && !$main) && do \{ # Parameters Section.
          # print STDERR "$_\\n";
          $prt = $param = 1;
          $seqflg = $hsp = $fragment = 0;
          last CHECK;
      \}; # (/^\\s*(?:Database|Parameters)/ && ($hsp))
  \} # CHECK Block
    # print STDOUT "$. : MAIN=$main SEQFLG=$seqflg HSP=$hsp FRAGMENT=$fragment => SEQNAME=$seqname\\n";
  LOAD: \{
      $fragment && do \{ # We are within a fragment.
          $txt = '';
          $tt = $cnt\{$seqname\};
          ($txt,$ori,$seq,$end) = split;
          if ($txt =~ /^Query:/o) \{
              # print STDERR "$_\\n";
              ($hsp_start\{$seqname."query".$tt\}) || ($hsp_start\{$seqname."query".$tt\} = $ori);
              $hsp_end\{$seqname."query".$tt\} = $end;
              $hsp_seq\{$seqname."query".$tt\} .= $seq;
          \} # if ($txt =~ /Query/)
          elsif ($txt =~ /^Sbjct:/o) \{
              # print STDERR "$_\\n";
              ($hsp_start\{$seqname."sbjct".$tt\}) || ($hsp_start\{$seqname."sbjct".$tt\} = $ori);
              $hsp_end\{$seqname."sbjct".$tt\} = $end;
              $hsp_seq\{$seqname."sbjct".$tt\} .= $seq;
              $fragment = 0;
          \} # elsif ($txt =~ /Sbjct/)
          else \{ last LOAD; \};
      \}; # ($fragment)
      $main && do \{ # We are within the blast file header.
          /^\\s*Query= +(.*)\\s*$/o && do \{
              # print STDERR "$_\\n";
              ($query_name = $1) =~ s/\\s|:|\\|/_/g ;
          \};
          /^\\s*Database: +(.*)\\s*$/o && do \{ 
              # print STDERR "$_\\n";
              $db_name = $1;
              while (<>) \{
                  last if /^(?:.*\\bletter.*|\\s*)$/o;
                  # print STDERR "$_\\n";
                  chop;
                  s/^\\s*//o;
                  s/\\s*$//o;
                  $db_name .= $_;
              \}; # while getline
          \}; # /^\\s*Database: +(.*)\\s*$/
          last LOAD;
      \}; # ($main)
      $param && do \{ # We are within the blast file trailer.
          /^\\s*Query= +(.*)\\s*$/o && do \{
              # print STDERR "$_\\n";
              &prt_foeprg($pt);
              $prt && &prt_out;
              $err_flg && print STDERR ("#"x58)."\\n## PARSING STDIN FROM BLAST (New BLAST Results) ".("#"x9)."\\n".("#"x58)."\\n";
              @seqlist = ();
              ($query_name = $1) =~ s/\\s|:|\\|/_/g ;
              $main = 1;
              $seqflg = $hsp = $fragment = $param = 0;
              last LOAD;
          \};
          if (/^\\s*[^\\[\\<\\-]/o) \{
              # print STDERR "$_\\n";
              chop;
              s/^/\\n\\# /o;
              $prog_params = join('', $prog_params, $_);
          \} else \{ $param = 0; \}
          last LOAD;
      \}; # ($param)
  \} # LOAD Block
    close(ARGV) if (eof);
\} # while
&prt_foeprg($pt);

# $prt && &prt_out;
&prt_out;

&get_exec_time(time);

exit(0);

###################################################
## TESTING how to compile into a binary file.... ##
#
# C libraries at: /usr/lib/perl5/5.00503/i386-linux/CORE/
#
#   perl -MO=C ../parseblast.pl > parseblast.c
#   gcc parseblast.c -E -I /usr/lib/perl5/5.00503/i386-linux/CORE/ -o ./parseblast.i
#   gcc parseblast.i -o parseblast
#
## STILL NOT WORKING...
###################################################
#           "a|align-score "   => \\$aln_flg     ,
#           "s|split-output"   => \\$split_flg   ,
#   -a, --align-score    : set <score> field to Alignment Score.
#   -s, --split-output : output each sequence match in a separate file in the current directory.
\nwused{\\{NWpar11-parA-1}}\nwendcode{}\nwbegindocs{26}\nwdocspar

%%%%%%%%%%%%%%%%%%%%%%%%%%%%%%%%%%%
\begin{comment}
\end{comment}
%%%%%%%%%%%%%%%%%%%%%%%%%%%%%%%%%%%

%%%%%%%%%%%%%%%%%%%% BACKMATTER

% \newpage %%%%%%%%%%%%%%%%%%%%%%%%%%%%%%%%%%%%%%%%%%%%%%%%%
% 
% \bibliographystyle{apalike}
% \bibliography{/home1/rguigo/docs/biblio/References}

\newpage %%%%%%%%%%%%%%%%%%%%%%%%%%%%%%%%%%%%%%%%%%%%%%%%%
\appendix

\sctn{empty appendix section}

\subsctn{empty appendix subsection}

%%%%%%%%%%%%%%%%%%%%%%%%%%%%%%%%%%%
\begin{comment}
\end{comment}
%%%%%%%%%%%%%%%%%%%%%%%%%%%%%%%%%%%

%
\newpage %%%%%%%%%%%%%%%%%%%%%%%%%%%%%%%%%%%%%%%%%%%%%%%%%

\sctn{Common code blocks}

\subsctn{PERL scripts}

\nwenddocs{}\nwbegincode{27}\sublabel{NWpar11-PERC-1}\nwmargintag{{\nwtagstyle{}\subpageref{NWpar11-PERC-1}}}\moddef{PERL shebang~{\nwtagstyle{}\subpageref{NWpar11-PERC-1}}}\endmoddef
#!/usr/bin/perl -w
# This is perl, version 5.005_03 built for i386-linux
\LA{}GNU License~{\nwtagstyle{}\subpageref{NWpar11-GNUB-1}}\RA{}
\LA{}Version Control Id Tag~{\nwtagstyle{}\subpageref{NWpar11-VerM-1}}\RA{}
#
use strict;
\nwused{\\{NWpar11-parA-1}}\nwendcode{}\nwbegindocs{28}\nwdocspar

\nwenddocs{}\nwbegincode{29}\sublabel{NWpar11-GloQ-1}\nwmargintag{{\nwtagstyle{}\subpageref{NWpar11-GloQ-1}}}\moddef{Global Constants - Boolean~{\nwtagstyle{}\subpageref{NWpar11-GloQ-1}}}\endmoddef
my ($T,$F) = (1,0); # for 'T'rue and 'F'alse
\eatline
\nwnotused{Global\ Constants\ -\ Boolean}\nwendcode{}\nwbegindocs{30}\nwdocspar
We also set here the date when the script is running and who is the user running it.

\nwenddocs{}\nwbegincode{31}\sublabel{NWpar11-GloR-1}\nwmargintag{{\nwtagstyle{}\subpageref{NWpar11-GloR-1}}}\moddef{Global Vars - User and Date~{\nwtagstyle{}\subpageref{NWpar11-GloR-1}}}\endmoddef
my $DATE = localtime;
my $USER = $ENV\{USER\};
\eatline
\nwnotused{Global\ Vars\ -\ User\ and\ Date}\nwendcode{}\nwbegindocs{32}\nwdocspar

\subsubsctn{Timing our scripts}

The '{\tt{}Benchmark}' module encapsulates a number of routines to help to figure out how long it takes to execute a piece of code and the whole script.

\nwenddocs{}\nwbegincode{33}\sublabel{NWpar11-UseN-1}\nwmargintag{{\nwtagstyle{}\subpageref{NWpar11-UseN-1}}}\moddef{Use Modules - Benchmark~{\nwtagstyle{}\subpageref{NWpar11-UseN-1}}}\endmoddef
use Benchmark;
  \LA{}Timer ON~{\nwtagstyle{}\subpageref{NWpar11-Tim8-1}}\RA{}
\nwnotused{Use\ Modules\ -\ Benchmark}\nwendcode{}\nwbegindocs{34}\nwdocspar

See '{\tt{}man\ Benchmark}' for further info about this package. 
We set an array to keep record of timing for each section.

\nwenddocs{}\nwbegincode{35}\sublabel{NWpar11-Tim8-1}\nwmargintag{{\nwtagstyle{}\subpageref{NWpar11-Tim8-1}}}\moddef{Timer ON~{\nwtagstyle{}\subpageref{NWpar11-Tim8-1}}}\endmoddef
my @Timer = (new Benchmark);
\nwused{\\{NWpar11-UseN-1}}\nwendcode{}\nwbegindocs{36}\nwdocspar

\nwenddocs{}\nwbegincode{37}\sublabel{NWpar11-ComS-1}\nwmargintag{{\nwtagstyle{}\subpageref{NWpar11-ComS-1}}}\moddef{Common PERL subs - Benchmark~{\nwtagstyle{}\subpageref{NWpar11-ComS-1}}}\endmoddef
sub timing() \{
    push @Timer, (new Benchmark);
    # partial time 
    $_[0] || 
        (return timestr(timediff($Timer[$#Timer],$Timer[($#Timer - 1)])));
    # total time
    return timestr(timediff($Timer[$#Timer],$Timer[0]));
\} # timing
\nwnotused{Common\ PERL\ subs\ -\ Benchmark}\nwendcode{}\nwbegindocs{38}\nwdocspar


\subsubsctn{Printing complex Data Structures}

With '{\tt{}Data::Dumper}' we are able to pretty print complex data structures for debugging them.


\nwenddocs{}\nwbegincode{39}\sublabel{NWpar11-UseK-1}\nwmargintag{{\nwtagstyle{}\subpageref{NWpar11-UseK-1}}}\moddef{Use Modules - Dumper~{\nwtagstyle{}\subpageref{NWpar11-UseK-1}}}\endmoddef
use Data::Dumper;
local $Data::Dumper::Purity = 0;
local $Data::Dumper::Deepcopy = 1;
\nwnotused{Use\ Modules\ -\ Dumper}\nwendcode{}\nwbegindocs{40}\nwdocspar


\subsubsctn{Common functions}

\nwenddocs{}\nwbegincode{41}\sublabel{NWpar11-SkiV-1}\nwmargintag{{\nwtagstyle{}\subpageref{NWpar11-SkiV-1}}}\moddef{Skip comments and empty records~{\nwtagstyle{}\subpageref{NWpar11-SkiV-1}}}\endmoddef
next if /^\\#/o;
next if /^\\s*$/o;
chomp;
\nwnotused{Skip\ comments\ and\ empty\ records}\nwendcode{}\nwbegindocs{42}\nwdocspar

\nwenddocs{}\nwbegincode{43}\sublabel{NWpar11-ComQ-1}\nwmargintag{{\nwtagstyle{}\subpageref{NWpar11-ComQ-1}}}\moddef{Common PERL subs - Min Max~{\nwtagstyle{}\subpageref{NWpar11-ComQ-1}}}\endmoddef
#
sub max() \{
    my $z = shift @_;
    foreach my $l (@_) \{ $z = $l if $l > $z \};
    return $z;
\} # max
sub min() \{
    my $z = shift @_;
    foreach my $l (@_) \{ $z = $l if $l < $z \};
    return $z;
\} # min
\nwnotused{Common\ PERL\ subs\ -\ Min\ Max}\nwendcode{}\nwbegindocs{44}\nwdocspar

\nwenddocs{}\nwbegincode{45}\sublabel{NWpar11-ComS.2-1}\nwmargintag{{\nwtagstyle{}\subpageref{NWpar11-ComS.2-1}}}\moddef{Common PERL subs - Text fill~{\nwtagstyle{}\subpageref{NWpar11-ComS.2-1}}}\endmoddef
#
sub fill_right() \{ $_[0].($_[2] x ($_[1] - length($_[0]))) \}
sub fill_left()  \{ ($_[2] x ($_[1] - length($_[0]))).$_[0] \}
sub fill_mid()   \{ 
    my $l = length($_[0]);
    my $k = int(($_[1] - $l)/2);
    ($_[2] x $k).$_[0].($_[2] x ($_[1] - ($l+$k)));
\} # fill_mid
\nwnotused{Common\ PERL\ subs\ -\ Text\ fill}\nwendcode{}\nwbegindocs{46}\nwdocspar

These functions are used to report to STDERR a single char for each record processed (useful for reporting parsed records).

\nwenddocs{}\nwbegincode{47}\sublabel{NWpar11-ComQ.2-1}\nwmargintag{{\nwtagstyle{}\subpageref{NWpar11-ComQ.2-1}}}\moddef{Common PERL subs - Counter~{\nwtagstyle{}\subpageref{NWpar11-ComQ.2-1}}}\endmoddef
#
sub counter \{ # $_[0]~current_pos++ $_[1]~char
    print STDERR "$_[1]";
    (($_[0] % 50) == 0) && (print STDERR "[".&fill_left($_[0],6,"0")."]\\n");
\} # counter
#
sub counter_end \{ # $_[0]~current_pos   $_[1]~char
    (($_[0] % 50) != 0) && (print STDERR "[".&fill_left($_[0],6,"0")."]\\n");
\} # counter_end
\nwnotused{Common\ PERL\ subs\ -\ Counter}\nwendcode{}\nwbegindocs{48}\nwdocspar

\nwenddocs{}\nwbegincode{49}\sublabel{NWpar11-GloL-1}\nwmargintag{{\nwtagstyle{}\subpageref{NWpar11-GloL-1}}}\moddef{Global Vars - Counter~{\nwtagstyle{}\subpageref{NWpar11-GloL-1}}}\endmoddef
my ($n,$c); # counter and char (for &counter function)
\eatline
\nwnotused{Global\ Vars\ -\ Counter}\nwendcode{}\nwbegindocs{50}\nwdocspar

\subsubsctn{Common functions for reporting program processes}
\label{sec:messagerpt}

Function '{\tt{}report}' requires that a hash variable '{\tt{}{\char37}MessageList}' has been set, such hash contains the strings for each report message we will need. The first parameter for '{\tt{}report}' is a key for that hash, in order to retrieve the message string, the other parameters passed are processed by the {\tt{}sprintf} function on that string.

\nwenddocs{}\nwbegincode{51}\sublabel{NWpar11-ComP-1}\nwmargintag{{\nwtagstyle{}\subpageref{NWpar11-ComP-1}}}\moddef{Common PERL subs - STDERR~{\nwtagstyle{}\subpageref{NWpar11-ComP-1}}}\endmoddef
sub report() \{ print STDERR sprintf($MessageList\{ shift @_ \},@_) \}
\nwalsodefined{\\{NWpar11-ComP-2}}\nwnotused{Common\ PERL\ subs\ -\ STDERR}\nwendcode{}\nwbegindocs{52}\nwdocspar

The same happens to '{\tt{}warn}' function which also requires a hash variable '{\tt{}{\char37}ErrorList}' containing the error messages.

\nwenddocs{}\nwbegincode{53}\sublabel{NWpar11-ComP-2}\nwmargintag{{\nwtagstyle{}\subpageref{NWpar11-ComP-2}}}\moddef{Common PERL subs - STDERR~{\nwtagstyle{}\subpageref{NWpar11-ComP-1}}}\plusendmoddef
sub warn() \{ print STDERR sprintf($ErrorList\{ shift @_ \}, @_) \}
\nwendcode{}\nwbegindocs{54}\nwdocspar

\subsctn{BASH scripts}

\nwenddocs{}\nwbegincode{55}\sublabel{NWpar11-BASC-1}\nwmargintag{{\nwtagstyle{}\subpageref{NWpar11-BASC-1}}}\moddef{BASH shebang~{\nwtagstyle{}\subpageref{NWpar11-BASC-1}}}\endmoddef
#!/usr/bin/bash
# GNU bash, version 2.03.6(1)-release (i386-redhat-linux-gnu)
\LA{}Version Control Id Tag~{\nwtagstyle{}\subpageref{NWpar11-VerM-1}}\RA{}
#
SECONDS=0 # Reset Timing
# Which script are we running...
L="####################"
\{ echo "$L$L$L$L";
  echo "### RUNNING [$0]";
  echo "### Current date:`date`";
  echo "###"; \} 1>&2;
\nwused{\\{NWpar11-wea7-1}\\{NWpar11-LaT8-1}}\nwendcode{}\nwbegindocs{56}\nwdocspar

\nwenddocs{}\nwbegincode{57}\sublabel{NWpar11-BASF-1}\nwmargintag{{\nwtagstyle{}\subpageref{NWpar11-BASF-1}}}\moddef{BASH script end~{\nwtagstyle{}\subpageref{NWpar11-BASF-1}}}\endmoddef
\{ echo "###"; echo "### Execution time for [$0] : $SECONDS secs";
  echo "$L$L$L$L";
  echo ""; \} 1>&2;
#
exit 0
\nwused{\\{NWpar11-wea7-1}\\{NWpar11-LaT8-1}}\nwendcode{}\nwbegindocs{58}\nwdocspar

\subsctn{Version control tags}

This document is under Revision Control System (RCS). The version you are currently reading is the following:

\nwenddocs{}\nwbegincode{59}\sublabel{NWpar11-VerM-1}\nwmargintag{{\nwtagstyle{}\subpageref{NWpar11-VerM-1}}}\moddef{Version Control Id Tag~{\nwtagstyle{}\subpageref{NWpar11-VerM-1}}}\endmoddef
# $Id: parseblast.tex,v 1.1 2001-09-06 18:26:37 jabril Exp $
\nwused{\\{NWpar11-PERC-1}\\{NWpar11-BASC-1}}\nwendcode{}\nwbegindocs{60}\nwdocspar

\subsctn{GNU General Public License}

\nwenddocs{}\nwbegincode{61}\sublabel{NWpar11-GNUB-1}\nwmargintag{{\nwtagstyle{}\subpageref{NWpar11-GNUB-1}}}\moddef{GNU License~{\nwtagstyle{}\subpageref{NWpar11-GNUB-1}}}\endmoddef
# #----------------------------------------------------------------#
# #                          parseblast                            #
# #----------------------------------------------------------------#
# 
#                Extracting HSPs from blast output.
# 
#     Copyright (C) 2001 - Josep Francesc ABRIL FERRANDO  
#
# This program is free software; you can redistribute it and/or modify
# it under the terms of the GNU General Public License as published by
# the Free Software Foundation; either version 2 of the License, or
# (at your option) any later version.
# 
# This program is distributed in the hope that it will be useful,
# but WITHOUT ANY WARRANTY; without even the implied warranty of
# MERCHANTABILITY or FITNESS FOR A PARTICULAR PURPOSE.  See the
# GNU General Public License for more details.
# 
# You should have received a copy of the GNU General Public License
# along with this program; if not, write to the Free Software
# Foundation, Inc., 675 Mass Ave, Cambridge, MA 02139, USA.
# 
# #----------------------------------------------------------------#
\nwused{\\{NWpar11-PERC-1}}\nwendcode{}\nwbegindocs{62}\nwdocspar

\newpage %%%%%%%%%%%%%%%%%%%%%%%%%%%%%%%%%%%%%%%%%%%%%%%%%

\sctn{Extracting code blocks from this document}

From this file we can obtain both the code and the
documentation. The following instructions are needed:

\subsctn{Extracts Script code chunks from the {\noweb} file} % \\[-0.5ex]

Remember when tangling that '-L' option allows you to include program line-numbering relative to original {\noweb} file. Then the first line of the executable files is a comment, not a shebang, and must be removed to make scripts runnable.

\nwenddocs{}\nwbegincode{63}\sublabel{NWpar11-tan8-1}\nwmargintag{{\nwtagstyle{}\subpageref{NWpar11-tan8-1}}}\moddef{tangling~{\nwtagstyle{}\subpageref{NWpar11-tan8-1}}}\endmoddef
# showing line numbering comments in program
notangle -L -R"parseblast" $WORK/$nwfile.nw | \\
    perl -ne '$.>1 && print' | cpif $BIN/parseblast.pl ;
# making them runnable
chmod a+x $BIN/parseblast.pl ;
\nwalsodefined{\\{NWpar11-tan8-2}\\{NWpar11-tan8-3}\\{NWpar11-tan8-4}\\{NWpar11-tan8-5}}\nwnotused{tangling}\nwendcode{}\nwbegindocs{64}\nwdocspar

\nwenddocs{}\nwbegincode{65}\sublabel{NWpar11-tan8-2}\nwmargintag{{\nwtagstyle{}\subpageref{NWpar11-tan8-2}}}\moddef{tangling~{\nwtagstyle{}\subpageref{NWpar11-tan8-1}}}\plusendmoddef
# program without line numbering comments
notangle -t4 -R"parseblast" $WORK/$nwfile.nw | \\
    cpif $BIN/parseblast.pl ;
# reformating program with perltidy
notangle -R"parseblast" $WORK/$nwfile.nw | \\
    perltidy - | cpif $BIN/parseblast.pl ;
# pretty-printing program with perltidy
notangle -R"parseblast" $WORK/$nwfile.nw | \\
    perltidy -html - | cpif $DOCS/html/parseblast.html ;
#
\nwendcode{}\nwbegindocs{66}\nwdocspar

\subsctn{Extracting different Config Files} % \\[-0.5ex]

\nwenddocs{}\nwbegincode{67}\sublabel{NWpar11-tan8-3}\nwmargintag{{\nwtagstyle{}\subpageref{NWpar11-tan8-3}}}\moddef{tangling~{\nwtagstyle{}\subpageref{NWpar11-tan8-1}}}\plusendmoddef
notangle -R"root" $WORK/$nwfile.nw | \\
    cpif $DATA/root_config ;
\nwendcode{}\nwbegindocs{68}%$

\subsctn{Extracting documentation and \LaTeX{}'ing it} % \\[-0.5ex] %'

\nwenddocs{}\nwbegincode{69}\sublabel{NWpar11-tan8-4}\nwmargintag{{\nwtagstyle{}\subpageref{NWpar11-tan8-4}}}\moddef{tangling~{\nwtagstyle{}\subpageref{NWpar11-tan8-1}}}\plusendmoddef
notangle -Rweaving  $WORK/$nwfile.nw | cpif $WORK/nw2tex ;
notangle -RLaTeXing $WORK/$nwfile.nw | cpif $WORK/ltx ;
chmod a+x $WORK/nw2tex $WORK/ltx;
\nwendcode{}\nwbegindocs{70}\nwdocspar

\nwenddocs{}\nwbegincode{71}\sublabel{NWpar11-tanY-1}\nwmargintag{{\nwtagstyle{}\subpageref{NWpar11-tanY-1}}}\moddef{tangling complementary LaTeX files~{\nwtagstyle{}\subpageref{NWpar11-tanY-1}}}\endmoddef
notangle -R"HIDE: LaTeX new definitions" $WORK/$nwfile.nw | cpif $DOCS/defs.tex ;
notangle -R"HIDE: TODO" $WORK/$nwfile.nw | cpif $DOCS/todo.tex ; 
\nwused{\\{NWpar11-wea7-1}}\nwendcode{}\nwbegindocs{72}\nwdocspar

\nwenddocs{}\nwbegincode{73}\sublabel{NWpar11-wea7-1}\nwmargintag{{\nwtagstyle{}\subpageref{NWpar11-wea7-1}}}\moddef{weaving~{\nwtagstyle{}\subpageref{NWpar11-wea7-1}}}\endmoddef
\LA{}BASH shebang~{\nwtagstyle{}\subpageref{NWpar11-BASC-1}}\RA{}
# weaving and LaTeXing
\LA{}BASH Environment Variables~{\nwtagstyle{}\subpageref{NWpar11-BASQ-1}}\RA{}
\LA{}tangling complementary LaTeX files~{\nwtagstyle{}\subpageref{NWpar11-tanY-1}}\RA{}
noweave -v -t4 -delay -x -filter 'elide "HIDE: *"' \\
        $WORK/$nwfile.nw | cpif $DOCS/$nwfile.tex ;
# noweave -t4 -delay -index $WORK/$nwfile.nw > $DOCS/$nwfile.tex 
pushd $DOCS/ ;
#
latex $nwfile.tex ;
dvips $nwfile.dvi -o $nwfile.ps -t a4 ;
#
popd;
\LA{}BASH script end~{\nwtagstyle{}\subpageref{NWpar11-BASF-1}}\RA{}
\nwnotused{weaving}\nwendcode{}\nwbegindocs{74}\nwdocspar

\nwenddocs{}\nwbegincode{75}\sublabel{NWpar11-LaT8-1}\nwmargintag{{\nwtagstyle{}\subpageref{NWpar11-LaT8-1}}}\moddef{LaTeXing~{\nwtagstyle{}\subpageref{NWpar11-LaT8-1}}}\endmoddef
\LA{}BASH shebang~{\nwtagstyle{}\subpageref{NWpar11-BASC-1}}\RA{}
# only LaTeXing
\LA{}BASH Environment Variables~{\nwtagstyle{}\subpageref{NWpar11-BASQ-1}}\RA{}
pushd $DOCS/ ;
#
echo "### RUNNING LaTeX on $nwfile.tex" 1>&2 ;
latex $nwfile.tex ; 
latex $nwfile.tex ; 
latex $nwfile.tex ;
dvips $nwfile.dvi -o $nwfile.ps -t a4 ;
#
# pdflatex $nwfile.tex ;
echo "### CONVERTING PS to PDF: $nwfile" 1>&2 ;
ps2pdf $nwfile.ps $nwfile.pdf ;
#
popd ;
\LA{}BASH script end~{\nwtagstyle{}\subpageref{NWpar11-BASF-1}}\RA{}
\nwnotused{LaTeXing}\nwendcode{}\nwbegindocs{76}%$

\subsctn{Defining working shell variables for the current project} % \\[-0.5ex]

\nwenddocs{}\nwbegincode{77}\sublabel{NWpar11-BASQ-1}\nwmargintag{{\nwtagstyle{}\subpageref{NWpar11-BASQ-1}}}\moddef{BASH Environment Variables~{\nwtagstyle{}\subpageref{NWpar11-BASQ-1}}}\endmoddef
#
# Setting Global Variables
WORK="$HOME/development/softjabril/parseblast" ;
BIN="$WORK/bin" ;
PARAM="$BIN/param" ;
DOCS="$WORK/docs" ;
DATA="$WORK/data" ;
nwfile="parseblast" ;
export WORK BIN PARAM DOCS DATA nwfile ;
#
\nwused{\\{NWpar11-wea7-1}\\{NWpar11-LaT8-1}}\nwendcode{}\nwbegindocs{78}\nwdocspar

\nwenddocs{}\nwbegincode{79}\sublabel{NWpar11-tan8-5}\nwmargintag{{\nwtagstyle{}\subpageref{NWpar11-tan8-5}}}\moddef{tangling~{\nwtagstyle{}\subpageref{NWpar11-tan8-1}}}\plusendmoddef
# 
# BASH Environment Variables
notangle -R'BASH Environment Variables' $WORK/$nwfile.nw | \\
         cpif $WORK/.bash_VARS ; 
source $WORK/.bash_VARS ;
#
\nwendcode{}

\nwixlogsorted{c}{{BASH Environment Variables}{NWpar11-BASQ-1}{\nwixu{NWpar11-wea7-1}\nwixu{NWpar11-LaT8-1}\nwixd{NWpar11-BASQ-1}}}%
\nwixlogsorted{c}{{BASH script end}{NWpar11-BASF-1}{\nwixd{NWpar11-BASF-1}\nwixu{NWpar11-wea7-1}\nwixu{NWpar11-LaT8-1}}}%
\nwixlogsorted{c}{{BASH shebang}{NWpar11-BASC-1}{\nwixd{NWpar11-BASC-1}\nwixu{NWpar11-wea7-1}\nwixu{NWpar11-LaT8-1}}}%
\nwixlogsorted{c}{{Common PERL subs - Benchmark}{NWpar11-ComS-1}{\nwixd{NWpar11-ComS-1}}}%
\nwixlogsorted{c}{{Common PERL subs - Counter}{NWpar11-ComQ.2-1}{\nwixd{NWpar11-ComQ.2-1}}}%
\nwixlogsorted{c}{{Common PERL subs - Min Max}{NWpar11-ComQ-1}{\nwixd{NWpar11-ComQ-1}}}%
\nwixlogsorted{c}{{Common PERL subs - STDERR}{NWpar11-ComP-1}{\nwixd{NWpar11-ComP-1}\nwixd{NWpar11-ComP-2}}}%
\nwixlogsorted{c}{{Common PERL subs - Text fill}{NWpar11-ComS.2-1}{\nwixd{NWpar11-ComS.2-1}}}%
\nwixlogsorted{c}{{Functions}{NWpar11-Fun9-1}{\nwixu{NWpar11-parA-1}\nwixd{NWpar11-Fun9-1}}}%
\nwixlogsorted{c}{{GNU License}{NWpar11-GNUB-1}{\nwixu{NWpar11-PERC-1}\nwixd{NWpar11-GNUB-1}}}%
\nwixlogsorted{c}{{Global Constants - Boolean}{NWpar11-GloQ-1}{\nwixd{NWpar11-GloQ-1}}}%
\nwixlogsorted{c}{{Global Vars}{NWpar11-GloB-1}{\nwixu{NWpar11-parA-1}\nwixd{NWpar11-GloB-1}}}%
\nwixlogsorted{c}{{Global Vars - Counter}{NWpar11-GloL-1}{\nwixd{NWpar11-GloL-1}}}%
\nwixlogsorted{c}{{Global Vars - User and Date}{NWpar11-GloR-1}{\nwixd{NWpar11-GloR-1}}}%
\nwixlogsorted{c}{{HIDE: LaTeX new definitions}{NWpar11-HIDR-1}{\nwixd{NWpar11-HIDR-1}}}%
\nwixlogsorted{c}{{HIDE: TODO}{NWpar11-HIDA-1}{\nwixd{NWpar11-HIDA-1}}}%
\nwixlogsorted{c}{{HIDE: new LaTeX commands}{NWpar11-HIDO-1}{\nwixu{NWpar11-HIDR-1}\nwixd{NWpar11-HIDO-1}}}%
\nwixlogsorted{c}{{HIDE: new LaTeX definitions}{NWpar11-HIDR.2-1}{\nwixu{NWpar11-HIDR-1}\nwixd{NWpar11-HIDR.2-1}}}%
\nwixlogsorted{c}{{HIDE: new LaTeX pstricks}{NWpar11-HIDO.2-1}{\nwixu{NWpar11-HIDR-1}\nwixd{NWpar11-HIDO.2-1}}}%
\nwixlogsorted{c}{{HIDE: new LaTeX urls}{NWpar11-HIDK-1}{\nwixu{NWpar11-HIDR-1}\nwixd{NWpar11-HIDK-1}}}%
\nwixlogsorted{c}{{HIDE: new defs TODO}{NWpar11-HIDJ-1}{\nwixu{NWpar11-HIDR-1}\nwixd{NWpar11-HIDJ-1}}}%
\nwixlogsorted{c}{{LaTeXing}{NWpar11-LaT8-1}{\nwixd{NWpar11-LaT8-1}}}%
\nwixlogsorted{c}{{Main Loop}{NWpar11-Mai9-1}{\nwixu{NWpar11-parA-1}\nwixd{NWpar11-Mai9-1}}}%
\nwixlogsorted{c}{{PERL shebang}{NWpar11-PERC-1}{\nwixu{NWpar11-parA-1}\nwixd{NWpar11-PERC-1}}}%
\nwixlogsorted{c}{{Skip comments and empty records}{NWpar11-SkiV-1}{\nwixd{NWpar11-SkiV-1}}}%
\nwixlogsorted{c}{{Timer ON}{NWpar11-Tim8-1}{\nwixu{NWpar11-UseN-1}\nwixd{NWpar11-Tim8-1}}}%
\nwixlogsorted{c}{{Use Modules}{NWpar11-UseB-1}{\nwixu{NWpar11-parA-1}\nwixd{NWpar11-UseB-1}}}%
\nwixlogsorted{c}{{Use Modules - Benchmark}{NWpar11-UseN-1}{\nwixd{NWpar11-UseN-1}}}%
\nwixlogsorted{c}{{Use Modules - Dumper}{NWpar11-UseK-1}{\nwixd{NWpar11-UseK-1}}}%
\nwixlogsorted{c}{{Version Control Id Tag}{NWpar11-VerM-1}{\nwixu{NWpar11-PERC-1}\nwixu{NWpar11-BASC-1}\nwixd{NWpar11-VerM-1}}}%
\nwixlogsorted{c}{{parseblast}{NWpar11-parA-1}{\nwixd{NWpar11-parA-1}}}%
\nwixlogsorted{c}{{tangling}{NWpar11-tan8-1}{\nwixd{NWpar11-tan8-1}\nwixd{NWpar11-tan8-2}\nwixd{NWpar11-tan8-3}\nwixd{NWpar11-tan8-4}\nwixd{NWpar11-tan8-5}}}%
\nwixlogsorted{c}{{tangling complementary LaTeX files}{NWpar11-tanY-1}{\nwixd{NWpar11-tanY-1}\nwixu{NWpar11-wea7-1}}}%
\nwixlogsorted{c}{{weaving}{NWpar11-wea7-1}{\nwixd{NWpar11-wea7-1}}}%
\nwixlogsorted{c}{{whole script}{NWpar11-whoC-1}{\nwixu{NWpar11-parA-1}\nwixd{NWpar11-whoC-1}}}%
\nwbegindocs{80}\nwdocspar

%
\end{document}
%
%%%%%%%%%%%%%%%%%%%%%%%%%%%%%%%%%%%%%%%%%%%%%%%%%%%%%%%%%%%%%%%%%%%%%%%%%%%%%%%%
\nwenddocs{}
