% -*- mode: Noweb; noweb-code-mode: perl-mode; tab-width: 4 -*-
\documentclass[11pt]{article}
%
%2345678901234567890123456789012345678901234567890123456789012345678901234567890
%        1         2         3         4         5         6         7         8
%
% $Id: gff2aplot.tex,v 1.2 2001-09-06 18:33:36 jabril Exp $
%
\usepackage{noweb,multicol}
\usepackage[a4paper,offset={0pt,0pt},hmargin={2cm,2cm},vmargin={1cm,1cm}]{geometry}
\usepackage{graphics}
\usepackage[dvips]{graphicx}
%% pstricks
\usepackage[dvips]{pstcol}
\usepackage{pstricks}
%\usepackage{pst-node}
%\usepackage{pst-char}
%\usepackage{pst-grad}
%% bibliography
\usepackage{natbib}
%% latex2html
\usepackage{url}
\usepackage{html}     
\usepackage{htmllist} 
%% tables    
%\usepackage{colortbl}
\usepackage{multirow}
%\usepackage{hhline}
%\usepackage{tabularx}
\usepackage{dcolumn}
%% seminar
%\usepackage{semcolor,semlayer,semrot,semhelv,sem-page,slidesec}
%% draft watermark
%\usepackage[all,dvips]{draftcopy}
%\draftcopySetGrey{0.9}
%\draftcopyName{CONFIDENTIAL}{100}
%% layout
%\usepackage{fancyheadings}
%\usepackage{fancybox}
\usepackage{fancyhdr} % Do not use \usepackage{fancybox} -> TOCs disappear
%\usepackage{lscape}
%\usepackage{rotating}
%\usepackage{multicol}
\usepackage{verbatim}
%\usepackage{version}
%% fonts
\usepackage{times}\fontfamily{ptm}\selectfont
\usepackage{t1enc}

% noweb options
\noweboptions{smallcode}
\def\nwendcode{\endtrivlist \endgroup} % relax page breaking scheme
\let\nwdocspar=\par                    %
 
\input defs.tex

%%%%%%%%%%%%%%%%%%%%%%%%%%%%%%%%%%%%%%%%%%%%%%%%%%%%%%%%%%%%%%%%%%%%%%%%%%%
%%\ \vfill
%%\begin{center}
%%\textbf{\Huge A perl development for {\prog}}\\[5ex]
%%
%%\textbf{\Large Josep F. Abril}\raisebox{0.85ex}{\footnotesize$\,\dag$}\\[5ex]
%%
%%\textbf{\large --- \today ---}\\[10ex]
%%
%%\begin{abstract}
%%\begin{center}
%%\parbox{0.75\linewidth}{
%%I am trying to improve some aspects of the {\prog} utility that may be hard to solve in its old {\tt{}GNUawk} format. Taking advantage of some capabilities of {\tt{}Perl}, such implemented quicksort sorting algorithm and the posibility of building complex data structures, will make the program faster and easier to maintain. Using {\tt{}noweb} I pretend to improve the overall design and to open the code to other developers.
%%} % parbox
%%\end{center}
%%\end{abstract}\vskip 5ex
%%{\large$<$ \verb$Id: gff2aplot.tex,v 1.2 2001-09-06 18:33:36 jabril Exp $$>$ }
%%
%%\vfill
%%
%%\begin{flushright}
%%\scalebox{0.9 1}{\Large\textsl{\textbf{Genome Informatics Research Lab}}}\\
%%Grup de Recerca en Infom\`atica Biom\`edica\\
%%Institut Municipal d'Investigaci\'o M\`edica\\
%%Universitat Pompeu Fabra\\[2ex]
%%\raisebox{0.85ex}{\footnotesize$\dag\,$}{\large e-mail: \mtjabril}\\
%%\end{flushright}
%%\end{center}
%%%%%%%%%%%%%%%%%%%%%%%%%%%%%%%%%%%%%%%%%%%%%%%%%%%%%%%%%%%%%%%%%%%%%%%%%%%

\begin{document}

\nwfilename{/home/ug/jabril/development/softjabril/gfftools/gff2aplot/gff2aplot.nw}%
%
%
%
%
%
%
%
%
%
%
%
%
%
%
%
%
%
%
%
%
%
%
%
%
%
%
%
%
%
%
%
%
%
%
%
%
%
%
%
%
%
%
%
%
%
%
%
%
%
%
%
%
%
%
%
%
%
%
%
%
%
%
%
%
%
%
%
%
%
%
%
%
%
%
%
%
%
%
%
%
%
%
%
%
%
%
%
%
%
%
%
%
\nwbegindocs{2}\nwdocspar

\nwenddocs{}%
\nwbegindocs{4}\nwdocspar
\nwenddocs{}%
\nwbegindocs{6}\nwdocspar

\thispagestyle{empty}

\begin{titlepage}

\ \vfill
\begin{center}
\begin{bfseries}
\begin{large}
\newlength{\lttbl}\setlength{\lttbl}{0.25\linewidth}
\newlength{\rttbl}\setlength{\rttbl}{0.70\linewidth}
%\fbox{
%\vskip 2ex
\begin{tabular}{>{\scshape}r@{\quad}l}
\rule{\lttbl}{0pt} & \rule{\rttbl}{0pt} \\[2ex]
\multicolumn{2}{c}{\shortstack{\rule[0ex]{0.95\linewidth}{2pt}\\[0ex]
                               \rule[1ex]{0.95\linewidth}{2pt}}}\\[2ex]
Program Name: & {\Huge\progname}                       \\[3ex]
\multicolumn{2}{c}{\rule[0.5ex]{0.95\linewidth}{2pt}}\\[2ex]
      Author: & {\Large
                 \begin{minipage}[t]{0.95\rttbl}
                 \authorslist
                 \end{minipage}}                       \\[2ex]
     License: & {\license}                             \\[2ex]
 Last Update: & {\today}                               \\[2ex]
 Description: & {\large\mdseries
                 \begin{minipage}[t]{0.95\rttbl}
                 \description
                 \end{minipage}}                       \\[2ex]
\\
\multicolumn{2}{c}{\shortstack{\rule[0ex]{0.95\linewidth}{2pt}\\[0ex]
                               \rule[1ex]{0.95\linewidth}{2pt}}}\\[2ex]
\end{tabular}
%} % fbox
\end{large}
\end{bfseries}
\end{center}

\vfill

\begin{raggedleft}
\showaffiliation
\end{raggedleft}

\end{titlepage}

%'%%%%%%%%%%%%%%%%%%% FRONTMATTER

\newpage

\thispagestyle{empty}

\ % \vfill
% \hrulefill

\newpage
\pagenumbering{roman}
\setcounter{page}{1}
\pagestyle{fancy}
% Marks redefinition must go here because pagestyle 
% resets the values to the default ones.
\renewcommand{\sectionmark}[1]{\markboth{}{\textbf{\prog}\hspace{4ex}\thesection.\ #1}} 
\renewcommand{\subsectionmark}[1]{\markboth{}{\textbf{\prog}\hspace{4ex}\thesubsection.\ \textsl{#1}}}

\tableofcontents

\clearpage

\listoftables

\listoffigures

%%%%%%%%%%%%%%%%%%%% MAINMATTER

\newpage
\pagenumbering{arabic}
\setcounter{page}{1}

\sctn{Introduction}

*** Cite here Vista, PiP-maker, Dotter and other dotplot tools

\nwenddocs{}%
\nwbegindocs{8}\nwdocspar

\subsctn{Project description}

At the first stage of the new {\prog} implementation, we are going to define a simple main loop that performs a serial process, such in the old {\tt{}GNUawk} version. So that, the program will read the input files and parameters defined by user and perform the plots, finishing the job. Once the new version will be operative, we will try to implement an iterative mode and/or a GUI in {\tt{}perlTK}.  

\label{todo:AAA}
\nwenddocs{}%
%
\nwbegindocs{10}\nwdocspar
\nwenddocs{}%
%
\nwbegindocs{12}\nwdocspar
\todo{ \item \todoAAA } % todo
\begin{figure}[!ht]
\begin{center}
\fbox{\parbox[c][8cm][c]{\linewidth}{\hfill}}
% \includegraphics[]{}
\caption[``Filter'' mode of {\prog}.]{\label{fig:filtermode} Input, output and main code blocks are represented in this flowchart for the ``Filter'' mode of {\prog}.}
\end{center}
\end{figure}

\subsctn{Basic goals}

\begin{itemize}
\item I am trying to improve some aspects of the {\prog} utility that may be hard to solve in its old \texttt{GNUawk} format. Taking advantage of some capabilities of \texttt{Perl}, such implemented quicksort sorting algorithm and the posibility of building complex data structures, will make the program faster and easier to maintain, it will be more portable too. 

\item Using {\noweb} I pretend to improve the overall design and to open the code to other developers.

\item \ldots 
\end{itemize}

\subsctn{TO DO: unfinished stuff and projects}

\begin{itemize}
 \input todo.tex
\end{itemize}

\label{todo:AAB}
\nwenddocs{}%
%
\nwbegindocs{14}\nwdocspar
\nwenddocs{}%
%
\nwbegindocs{16}\nwdocspar
\todo{ \item \todoAAB } % todo

\begin{figure}[!ht]
\begin{center}
\fbox{\parbox[c][8cm][c]{\linewidth}{\hfill}}
% \includegraphics[]{}
\caption[Future modes of {\prog}.]{\label{fig:othermodes} Input, output and main code blocks are represented in this flowchart for the ``Interactive'', ``Graphic'' and ``Web/CGI'' versions of {\prog}.}
\end{center}
\end{figure}


\newpage

\sctn{Main program definition} %%%%%%%%%%%%%%%%%%%%%%%%%%%%%%%%%%%%%%%%%%%%%%%%%

This is the {\prog} program outline:

\nwenddocs{}\nwbegincode{17}\sublabel{NWgff18-APL5-1}\nwmargintag{{\nwtagstyle{}\subpageref{NWgff18-APL5-1}}}\moddef{APLOT~{\nwtagstyle{}\subpageref{NWgff18-APL5-1}}}\endmoddef
\LA{}PERL shebang~{\nwtagstyle{}\subpageref{NWgff18-PERC-1}}\RA{}
\LA{}Program Info~{\nwtagstyle{}\subpageref{NWgff18-ProC-1}}\RA{}
#
# MODULES
#
\LA{}Use Modules~{\nwtagstyle{}\subpageref{NWgff18-UseB-1}}\RA{}
#
# CONSTANTS
#
\LA{}Global Constants~{\nwtagstyle{}\subpageref{NWgff18-GloG-1}}\RA{}
#
# VARIABLES
#
\LA{}Global Vars~{\nwtagstyle{}\subpageref{NWgff18-GloB-1}}\RA{}
\LA{}UNDEFINED VARS~{\nwtagstyle{}\subpageref{NWgff18-UNDE-1}}\RA{}
#
# MAIN PROGRAM LOOP
#
\LA{}Main Loop~{\nwtagstyle{}\subpageref{NWgff18-Mai9-1}}\RA{}
#
# MAIN FUNCTIONS
#
\LA{}Main Routines~{\nwtagstyle{}\subpageref{NWgff18-MaiD-1}}\RA{}
#
# GENERAL FUNCTIONS
#
\LA{}Common Routines~{\nwtagstyle{}\subpageref{NWgff18-ComF-1}}\RA{}
#
# POSTSCRIPT CODE
#
\LA{}PostScript CODE Chunks~{\nwtagstyle{}\subpageref{NWgff18-PosM-1}}\RA{}
\nwnotused{APLOT}\nwendcode{}\nwbegindocs{18}\nwdocspar

\subsctn{The main loop}

This is the main loop for the {\prog} ``Filter'' mode: 

\nwenddocs{}\nwbegincode{19}\sublabel{NWgff18-Mai9-1}\nwmargintag{{\nwtagstyle{}\subpageref{NWgff18-Mai9-1}}}\moddef{Main Loop~{\nwtagstyle{}\subpageref{NWgff18-Mai9-1}}}\endmoddef

  # &set_default_vars;

  %CmdLineVars = ();            # Reseting Command-Line OPTIONS
  &parse_command_line;

  \LA{}Basic common block for gfftools~{\nwtagstyle{}\subpageref{NWgff18-BasV-1}}\RA{}
  &make_plot;

  $total_time = &timing($T);
  &header("$PROGRAM HAS FINISHED","Timing: $total_time secs");
  
  &close_logfile();
  exit(0);
\nwused{\\{NWgff18-APL5-1}}\nwendcode{}\nwbegindocs{20}\nwdocspar

{\gft}, say here {\prog} and {\gps}, share many of the basic procedures, both must parse same format input files, must map the read fields to the customization variables and sort groups and their elements by acceptor site coord. That is the main reason for developing a module containing the definitions of such common functions.

\nwenddocs{}\nwbegincode{21}\sublabel{NWgff18-BasV-1}\nwmargintag{{\nwtagstyle{}\subpageref{NWgff18-BasV-1}}}\moddef{Basic common block for gfftools~{\nwtagstyle{}\subpageref{NWgff18-BasV-1}}}\endmoddef
%CustomVars = ();             # Reseting Customization OPTIONS
&parse_custom_files;

%Vars = ();
&merge_custom_vars;

%GFF_DATA = %ALN_DATA = (); # Reseting DATA
&parse_GFF_files;

&map_vars_data;

&sort_elements;
\nwused{\\{NWgff18-Mai9-1}}\nwendcode{}\nwbegindocs{22}\nwdocspar

We force that all the variables must be declared before using them and restrict unsafe constructs with'{\tt{}use\ strict}'. After that we switch on signal trapping.

\nwenddocs{}\nwbegincode{23}\sublabel{NWgff18-UseB-1}\nwmargintag{{\nwtagstyle{}\subpageref{NWgff18-UseB-1}}}\moddef{Use Modules~{\nwtagstyle{}\subpageref{NWgff18-UseB-1}}}\endmoddef
use strict;
#
\LA{}Trapping signals~{\nwtagstyle{}\subpageref{NWgff18-TraG-1}}\RA{}
#
\nwalsodefined{\\{NWgff18-UseB-2}\\{NWgff18-UseB-3}\\{NWgff18-UseB-4}}\nwused{\\{NWgff18-APL5-1}}\nwendcode{}\nwbegindocs{24}\nwdocspar

\nwenddocs{}\nwbegincode{25}\sublabel{NWgff18-GloG-1}\nwmargintag{{\nwtagstyle{}\subpageref{NWgff18-GloG-1}}}\moddef{Global Constants~{\nwtagstyle{}\subpageref{NWgff18-GloG-1}}}\endmoddef
my ($T,$F) = (1,0); # for 'T'rue and 'F'alse
\nwalsodefined{\\{NWgff18-GloG-2}\\{NWgff18-GloG-3}\\{NWgff18-GloG-4}\\{NWgff18-GloG-5}\\{NWgff18-GloG-6}}\nwused{\\{NWgff18-APL5-1}}\nwendcode{}\nwbegindocs{26}\nwdocspar

Those are the main variables containing customization parameters and input data from GFF records. We split customization variables on three input classes: {\tt{}{\char37}DefaultVars} to set default values and their type (what makes easy to check if a variable exists or not and if its given value has the expected format), {\tt{}{\char37}CmdLineVars} for those parameters passed on command-line and {\tt{}{\char37}CustomVars} which collects vaiables customization from external parameter files. Two temporary variables ({\tt{}{\char37}Defaults} to initialize element variables and {\tt{}{\char37}Vars} to provide the customization from command-line and parameter files to those element variables) allow us to merge all the information from these three variables and transfer to each GFF element in the GFF data variables ({\tt{}{\char37}GFF{\char95}DATA} for annotations and {\tt{}{\char37}ALN{\char95}DATA} for alignments).

\nwenddocs{}\nwbegincode{27}\sublabel{NWgff18-GloB-1}\nwmargintag{{\nwtagstyle{}\subpageref{NWgff18-GloB-1}}}\moddef{Global Vars~{\nwtagstyle{}\subpageref{NWgff18-GloB-1}}}\endmoddef
my (%DefaultVars,%CmdLineVars,%CustomVars,
    %Defaults,%Vars,
    %GFF_DATA,%ALN_DATA);
\nwalsodefined{\\{NWgff18-GloB-2}\\{NWgff18-GloB-3}\\{NWgff18-GloB-4}\\{NWgff18-GloB-5}\\{NWgff18-GloB-6}\\{NWgff18-GloB-7}\\{NWgff18-GloB-8}\\{NWgff18-GloB-9}\\{NWgff18-GloB-A}\\{NWgff18-GloB-B}\\{NWgff18-GloB-C}\\{NWgff18-GloB-D}\\{NWgff18-GloB-E}\\{NWgff18-GloB-F}\\{NWgff18-GloB-G}\\{NWgff18-GloB-H}}\nwused{\\{NWgff18-APL5-1}}\nwendcode{}\nwbegindocs{28}\nwdocspar

Some info about the program itself.

\nwenddocs{}\nwbegincode{29}\sublabel{NWgff18-GloG-2}\nwmargintag{{\nwtagstyle{}\subpageref{NWgff18-GloG-2}}}\moddef{Global Constants~{\nwtagstyle{}\subpageref{NWgff18-GloG-1}}}\plusendmoddef
my ($PROGRAM,$VERSION,$REVISION,$REVISION_DATE,$LAST_UPDATE) = 
   ( 'gff2aplot','v2.0',
     '$Revision: 1.2 $', #'
     '$Date: 2001-09-06 18:33:36 $', #'
     '$Id: gff2aplot.tex,v 1.2 2001-09-06 18:33:36 jabril Exp $', #'
    );
$REVISION =~ s/\\$//og;
$REVISION_DATE =~ s/\\$//og;
\nwendcode{}\nwbegindocs{30} %$

\nwenddocs{}\nwbegincode{31}\sublabel{NWgff18-UNDE-1}\nwmargintag{{\nwtagstyle{}\subpageref{NWgff18-UNDE-1}}}\moddef{UNDEFINED VARS~{\nwtagstyle{}\subpageref{NWgff18-UNDE-1}}}\endmoddef
\nwused{\\{NWgff18-APL5-1}}\nwendcode{}\nwbegindocs{32}\nwdocspar

\subsctn{Main function sections}

Main function calls share a similar internal structure: a '{\tt{}header}' reporting which function is running, the '{\tt{}code}' for that function and maybe a '{\tt{}closing\ report}' summarizing what the function did. After each main function definition, you should find those constants, variables and routines being used by that function.  

\nwenddocs{}\nwbegincode{33}\sublabel{NWgff18-MaiD-1}\nwmargintag{{\nwtagstyle{}\subpageref{NWgff18-MaiD-1}}}\moddef{Main Routines~{\nwtagstyle{}\subpageref{NWgff18-MaiD-1}}}\endmoddef
\LA{}Setting Defaults~{\nwtagstyle{}\subpageref{NWgff18-SetG-1}}\RA{}
\LA{}Parsing Command-Line Options~{\nwtagstyle{}\subpageref{NWgff18-ParS-1}}\RA{}
\LA{}Parsing Custom Files~{\nwtagstyle{}\subpageref{NWgff18-ParK-1}}\RA{}
\LA{}Parsing Input Data~{\nwtagstyle{}\subpageref{NWgff18-ParI-1}}\RA{}
\LA{}Sorting Features~{\nwtagstyle{}\subpageref{NWgff18-SorG-1}}\RA{}
\LA{}Features Setting~{\nwtagstyle{}\subpageref{NWgff18-FeaG-1}}\RA{}
\LA{}Making PS Figures~{\nwtagstyle{}\subpageref{NWgff18-MakH-1}}\RA{}
\nwalsodefined{\\{NWgff18-MaiD-2}}\nwused{\\{NWgff18-APL5-1}}\nwendcode{}\nwbegindocs{34}\nwdocspar

\subsctn{Printing Help}

Here is shown the basic outline being diplayed when user choose the '{\tt{}help}' command-line option. The '{\tt{}REQUIRES}' section summarizes those perl modules used in this program by joining all the comments in this document under '{\tt{}\LA{}requires help~{\nwtagstyle{}\subpageref{nw@notdef}}\RA{}}' tag. Similar happens to '{\tt{}COMMAND-LINE\ OPTIONS}', in which the single descriptions for each option (that follows each option '{\tt{}Getopts}' definition) are collected by the '{\tt{}\LA{}command-line help~{\nwtagstyle{}\subpageref{NWgff18-comH-1}}\RA{}}' tag.

\nwenddocs{}\nwbegincode{35}\sublabel{NWgff18-MaiD-2}\nwmargintag{{\nwtagstyle{}\subpageref{NWgff18-MaiD-2}}}\moddef{Main Routines~{\nwtagstyle{}\subpageref{NWgff18-MaiD-1}}}\plusendmoddef
sub prt_help() \{
    open(HELP, "| more") ;
    print HELP <<"+++EndOfHelp+++";
PROGRAM:
                        $PROGRAM - $VERSION

    Converting GFF files for pairwise alignments to PostScript.

USAGE:        $PROGRAM [options] <GFF_files|STDIN>

DESCRIPTION:

    This program draws color-filled alignment plots from GFF
    files for that alignment and two sequences annotations.

REQUIRES:

    \LA{}perl requires help~{\nwtagstyle{}\subpageref{NWgff18-perI-1}}\RA{}

ENVIRONMENT VARIABLES:

    \LA{}environment vars help~{\nwtagstyle{}\subpageref{NWgff18-envL-1}}\RA{}

COMMAND-LINE OPTIONS:

    \LA{}command-line help~{\nwtagstyle{}\subpageref{NWgff18-comH-1}}\RA{}

    \LA{}colors help~{\nwtagstyle{}\subpageref{NWgff18-colB-1}}\RA{}

    \LA{}pages help~{\nwtagstyle{}\subpageref{NWgff18-pagA-1}}\RA{}

BUGS:    Report any problem to 'jabril\\@imim.es'.

AUTHOR:  $PROGRAM is under GNU-GPL (C) 2000 - Josep F. Abril

+++EndOfHelp+++
    close(HELP);
    exit(1);
\} # prt_help
\nwendcode{}\nwbegindocs{36}%$

\nwenddocs{}\nwbegincode{37}\sublabel{NWgff18-perI-1}\nwmargintag{{\nwtagstyle{}\subpageref{NWgff18-perI-1}}}\moddef{perl requires help~{\nwtagstyle{}\subpageref{NWgff18-perI-1}}}\endmoddef
$PROGRAM needs the following Perl modules installed in 
your system, we used those available from the standard 
Perl distribution. Those that are not in the standard 
distribution are marked with an '(*)', in such cases 
make sure that you already have downloaded them from 
CPAN (http://www.perl.com/CPAN) and installed.

\nwalsodefined{\\{NWgff18-perI-2}\\{NWgff18-perI-3}\\{NWgff18-perI-4}}\nwused{\\{NWgff18-MaiD-2}}\nwendcode{}\nwbegindocs{38}%$

\nwenddocs{}\nwbegincode{39}\sublabel{NWgff18-comH-1}\nwmargintag{{\nwtagstyle{}\subpageref{NWgff18-comH-1}}}\moddef{command-line help~{\nwtagstyle{}\subpageref{NWgff18-comH-1}}}\endmoddef
A double dash on itself "--" signals end of the options
and start of file names (if present). You can use a single
dash "-" as STDIN placeholder. Available options and a
short description are listed here:

\nwalsodefined{\\{NWgff18-comH-2}\\{NWgff18-comH-3}\\{NWgff18-comH-4}\\{NWgff18-comH-5}\\{NWgff18-comH-6}\\{NWgff18-comH-7}\\{NWgff18-comH-8}\\{NWgff18-comH-9}\\{NWgff18-comH-A}\\{NWgff18-comH-B}\\{NWgff18-comH-C}\\{NWgff18-comH-D}\\{NWgff18-comH-E}\\{NWgff18-comH-F}\\{NWgff18-comH-G}\\{NWgff18-comH-H}\\{NWgff18-comH-I}\\{NWgff18-comH-J}\\{NWgff18-comH-K}\\{NWgff18-comH-L}\\{NWgff18-comH-M}\\{NWgff18-comH-N}\\{NWgff18-comH-O}\\{NWgff18-comH-P}\\{NWgff18-comH-Q}\\{NWgff18-comH-R}\\{NWgff18-comH-S}\\{NWgff18-comH-T}\\{NWgff18-comH-U}\\{NWgff18-comH-V}\\{NWgff18-comH-W}\\{NWgff18-comH-X}\\{NWgff18-comH-Y}\\{NWgff18-comH-Z}}\nwused{\\{NWgff18-MaiD-2}}\nwendcode{}\nwbegindocs{40}\nwdocspar


\newpage

\sctn{Variable Definition and Customization} %%%%%%%%%%%%%%%%%%%%%%%%%%%%%%%%%%%

The functions described in this section process each GFF-record, each custom definition or each command-line options, not only loading the internal variables, also checking for the correctness of the user input in each case.

\label{todo:BBB}
\nwenddocs{}%
%
\nwbegindocs{42}\nwdocspar
\nwenddocs{}%
%
\nwbegindocs{44}\nwdocspar
\todo{ \item \todoBBB } % todo
\begin{figure}[!ht]
\begin{center}
\fbox{\parbox[c][8cm][c]{\linewidth}{\hfill}}
% \includegraphics[]{}
\caption[Structure of main customization variables]{\label{fig:customvarstruc} Structure of main customization variables: {\tt{}{\char37}DefaultVars}, {\tt{}{\char37}CmdLineVars}, {\tt{}{\char37}CustomVars}, {\tt{}{\char37}Defaults}, {\tt{}{\char37}Vars}. Relationships among them and their interaction with {\tt{}{\char37}GFF{\char95}DATA} and {\tt{}{\char37}ALN{\char95}DATA} variables which contain data are also shown.}
\end{center}
\end{figure}

\subsctn{Setting customization parameters from defaults} 

The main difference from older versions, besides a cleaner hash structure than the GAWK one, is that we include some customization variables at sequence and strand levels. Almost all the variables are described in section~\ref{sec:customvardesc}, page~\pageref{sec:customvardesc}.

\nwenddocs{}\nwbegincode{45}\sublabel{NWgff18-SetG-1}\nwmargintag{{\nwtagstyle{}\subpageref{NWgff18-SetG-1}}}\moddef{Setting Defaults~{\nwtagstyle{}\subpageref{NWgff18-SetG-1}}}\endmoddef
sub set_default_vars() \{
    %DefaultVars = (
        LAYOUT   => \{           ## '# L #'
            \LA{}default layout vars values~{\nwtagstyle{}\subpageref{NWgff18-defQ-1}}\RA{}
        \},                    
        SEQUENCE => \{           ## '# Q #'
            \LA{}default sequence vars values~{\nwtagstyle{}\subpageref{NWgff18-defS-1}}\RA{}
        \},                    
        SOURCE   => \{           ## '# S #'
            \LA{}default source vars values~{\nwtagstyle{}\subpageref{NWgff18-defQ.2-1}}\RA{}
        \},                    
        STRAND   => \{           ## '# T #'
            \LA{}default strand vars values~{\nwtagstyle{}\subpageref{NWgff18-defQ.3-1}}\RA{}
        \},                    
        GROUP    => \{           ## '# G #'
            \LA{}default group vars values~{\nwtagstyle{}\subpageref{NWgff18-defP-1}}\RA{}
        \},                    
        FEATURE  => \{           ## '# F #'
            \LA{}default feature vars values~{\nwtagstyle{}\subpageref{NWgff18-defR-1}}\RA{}
        \},
    ); # %DefaultVars
    \LA{}map default vars to defaults working copy~{\nwtagstyle{}\subpageref{NWgff18-mapf-1}}\RA{}
    print LOGFILE '>>> \\%DefaultVars : '.(Dumper(\\%DefaultVars)) if ($LogFile && $Debug);
\} # set_default_vars
\nwused{\\{NWgff18-MaiD-1}}\nwendcode{}\nwbegindocs{46}\nwdocspar

We also build from this structure, making easier variable maintenance because default value and variable type are edited together in {\tt{}{\char37}DefaultVars} (as {\tt{}TYPE}/{\tt{}VALUE} pairs), an auxiliary variable which allows us to load default values into GFF elements.

\label{sec:tmpdefaults}
\nwenddocs{}\nwbegincode{47}\sublabel{NWgff18-mapf-1}\nwmargintag{{\nwtagstyle{}\subpageref{NWgff18-mapf-1}}}\moddef{map default vars to defaults working copy~{\nwtagstyle{}\subpageref{NWgff18-mapf-1}}}\endmoddef
%Defaults = ();
foreach my $sct (keys %DefaultVars) \{
    foreach my $vnm (keys %\{$DefaultVars\{$sct\}\}) \{
        $Defaults\{$sct\}\{$vnm\} = $DefaultVars\{$sct\}\{$vnm\}\{'VALUE'\};
    \}; # foreach $vnm
\}; # foreach $sct
\nwused{\\{NWgff18-SetG-1}}\nwendcode{}\nwbegindocs{48}\nwdocspar

\subsubsctn{Default customization files}

\nwenddocs{}\nwbegincode{49}\sublabel{NWgff18-GloB-2}\nwmargintag{{\nwtagstyle{}\subpageref{NWgff18-GloB-2}}}\moddef{Global Vars~{\nwtagstyle{}\subpageref{NWgff18-GloB-1}}}\plusendmoddef
my $Custom_path = $ENV\{GFF2APLOT_CUSTOMPATH\};
my $Custom_file = $ENV\{GFF2APLOT_CUSTOMFILE\};
defined($Custom_path) || ($Custom_path = '.');
defined($Custom_file) || ($Custom_file = '.gff2aplotrc');
$Custom_path =~ s\{/$\}\{\}o; 
\nwendcode{}\nwbegindocs{50}\nwdocspar

We remove any trailing '{\tt{}/}' from '{\tt{}{\char36}Custom{\char95}path}' that user could have introduced to avoid any problem with full custom filenames later.

\nwenddocs{}\nwbegincode{51}\sublabel{NWgff18-envL-1}\nwmargintag{{\nwtagstyle{}\subpageref{NWgff18-envL-1}}}\moddef{environment vars help~{\nwtagstyle{}\subpageref{NWgff18-envL-1}}}\endmoddef
    There are two environmental variables that can be set by 
users to their preferences:
 + You can specify the path where $PROGRAM can find the default
  files with the shell variable \\"GFF2APLOT_CUSTOMPATH\\". Default
  value is the path where you are running $PROGRAM.
 + You can also define the default custom filename you will like
  with the variable \\"GFF2APLOT_CUSTOMFILE\\", program default
  filename for custom file is \\".gff2aplotrc\\".
 + Now $PROGRAM does not need to write any temporary file, 
  so that previous versions default temporary directory path
  variable (\\"GFF2APLOT_TMP\\") is no longer used.
 + Setting those vars in Bourne-shell and C-shell:
   o Using a Bourne-Shell (e.g. bash):
        export GFF2APLOT_CUSTOMPATH=\\"path\\"
        export GFF2APLOT_CUSTOMFILE=\\"file_name\\"
   o Using a C-Shell:
        setenv GFF2APLOT_CUSTOMPATH \\"path\\"
        setenv GFF2APLOT_CUSTOMFILE \\"file_name\\"
\nwused{\\{NWgff18-MaiD-2}}\nwendcode{}\nwbegindocs{52}%$


\subsctn{Setting customization parameters from command-line} 

'{\tt{}GetOptions}' loads command-line options as key-value pairs in the hash variable '{\tt{}{\char37}CmdLineVars}' 
that is reset in '{\tt{}\LA{}Main Loop~{\nwtagstyle{}\subpageref{NWgff18-Mai9-1}}\RA{}}'. Once parsed all the command-line options what remains in '{\tt{}@ARGV}' is just input filenames (which have to be in GFF format), or '{\tt{}STDIN}' if there is none.

\nwenddocs{}\nwbegincode{53}\sublabel{NWgff18-ParS-1}\nwmargintag{{\nwtagstyle{}\subpageref{NWgff18-ParS-1}}}\moddef{Parsing Command-Line Options~{\nwtagstyle{}\subpageref{NWgff18-ParS-1}}}\endmoddef
sub parse_command_line() \{
    \LA{}looking for STDIN~{\nwtagstyle{}\subpageref{NWgff18-looH-1}}\RA{}

    $SIG\{__WARN__\} = sub \{ &warn('UNKNOWN_CL_OPTION',$T,$_[0]) \};
    GetOptions(
               \LA{}command-line options~{\nwtagstyle{}\subpageref{NWgff18-comK-1}}\RA{}
               \LA{}command-line options with exit~{\nwtagstyle{}\subpageref{NWgff18-comU-1}}\RA{}
               ) || (&warn('CMD_LINE_ERROR',$T), exit(1));
    $SIG\{__WARN__\} = 'DEFAULT';

    \LA{}open LOGFILE~{\nwtagstyle{}\subpageref{NWgff18-opeC-1}}\RA{}

    &header('',"RUNNING $PROGRAM",'',"User: $USER","Date: $DATE");

    &header("SETTING DEFAULTS");
    %DefaultVars = ();
    &set_default_vars;

    &header("CHECKING COMMAND-LINE OPTIONS");
    @data_files = ();
    &set_input_file($cmdln_stdin);
    @ARGV = (); # ensuring that command-line ARGVs array is empty

    &set_custom_files();

    &check_command_line_vars();

    &footer("COMMAND-LINE CHECKED");
\} # parse_command_line
\nwalsodefined{\\{NWgff18-ParS-2}\\{NWgff18-ParS-3}\\{NWgff18-ParS-4}\\{NWgff18-ParS-5}}\nwused{\\{NWgff18-MaiD-1}}\nwendcode{}\nwbegindocs{54}\nwdocspar

As the '{\tt{}GetOptions}' function has a problem, in the way we are using here to preserve backwards compatibility, when we want to use a single dash as a way to tell the program that '{\tt{}STDIN}' must be read in the given order (when input filenames are also given and we would load input from a pipe after/before a given file). We show the warnings printed by the program when '{\tt{}--}' is missing or missplaced in the table~\ref{tbl:STDINhandle}.

\begin{table}[!t]
\begin{center}
\begin{small}
\begin{tabular}{|c|}\hline
\begin{minipage}{15cm}
\begin{verbatim}

# CORRECT
> $BIN/gff2aplot -v -T 'Howdy World!' -- - DATA-SAMPLE.gff 
> $BIN/gff2aplot -v -T 'Howdy World!' -- DATA-SAMPLE.gff -

# WRONG???
> $BIN/gff2aplot -v -T 'Howdy World!' DATA-SAMPLE.gff -
> $BIN/gff2aplot -v -T 'Howdy World!' - DATA-SAMPLE.gff
> $BIN/gff2aplot -v -T 'Howdy World!' - -- tests/DATA-SAMPLE.gff

# COMMON MESSAGE when WRONG
### WARNING ### Error trapped while processing command-line:
                substr outside of string at lib/Getopt/Long.pm
                  (autosplit into lib/auto/Getopt/Long/FindOption.al)
                line >NNNNN<
### WARNING ### Error trapped while processing command-line:
                Use of uninitialized value at lib/Getopt/Long.pm
                  (autosplit into lib/auto/Getopt/Long/FindOption.al) 
                line >NNNNN<
### WARNING ### Error trapped while processing command-line:
                Unknown option:

\end{verbatim}
% $
\end{minipage}\\\hline
\end{tabular}
\end{small}
\caption[Fixing {\tt{}GetOptions} for using '{\tt{}-}' as '{\tt{}STDIN}' mark.]{\label{tbl:STDINhandle} Errors reported when using '{\tt{}-}' as '{\tt{}STDIN}' mark and fixed width {\tt{}\LA{}looking for STDIN~{\nwtagstyle{}\subpageref{NWgff18-looH-1}}\RA{}}.}
\end{center}
\end{table}

To avoid such errors, we capture the single dash when present in the command-line arguments list. 

'{\tt{}{\char36}cmdln{\char95}stdin}' will be used by '{\tt{}{\char38}set{\char95}input{\char95}file}' function to include the '{\tt{}STDIN}' in the correct ordering.\label{sec:stdinfix}

\nwenddocs{}\nwbegincode{55}\sublabel{NWgff18-looH-1}\nwmargintag{{\nwtagstyle{}\subpageref{NWgff18-looH-1}}}\moddef{looking for STDIN~{\nwtagstyle{}\subpageref{NWgff18-looH-1}}}\endmoddef
my $cmdln_stdin = undef;
for (my $a = 0; $a <= $#ARGV; $a++) \{ 
    next unless $ARGV[$a] =~ /^-$/o;
    $cmdln_stdin = $a - $#ARGV;
    splice(@ARGV,$a,1);
\};    
\nwused{\\{NWgff18-ParS-1}}\nwendcode{}\nwbegindocs{56}\nwdocspar

The following two code chunks define what to do if reports are sent to a file which is set in '{\tt{}{\char36}logs{\char95}filename}'.

\nwenddocs{}\nwbegincode{57}\sublabel{NWgff18-opeC-1}\nwmargintag{{\nwtagstyle{}\subpageref{NWgff18-opeC-1}}}\moddef{open LOGFILE~{\nwtagstyle{}\subpageref{NWgff18-opeC-1}}}\endmoddef
CHKLOG:
  (defined($logs_filename)) && do \{
      open(LOGFILE,"> ".$logs_filename) ||
          (&warn('FILE_NO_OPEN',$T,$logs_filename),last CHKLOG);
      $LogFile = 1;
  \};
\nwused{\\{NWgff18-ParS-1}}\nwendcode{}\nwbegindocs{58}\nwdocspar

\nwenddocs{}\nwbegincode{59}\sublabel{NWgff18-ComF-1}\nwmargintag{{\nwtagstyle{}\subpageref{NWgff18-ComF-1}}}\moddef{Common Routines~{\nwtagstyle{}\subpageref{NWgff18-ComF-1}}}\endmoddef
sub close_logfile() \{ close(LOGFILE) if $LogFile \};
\nwalsodefined{\\{NWgff18-ComF-2}\\{NWgff18-ComF-3}\\{NWgff18-ComF-4}\\{NWgff18-ComF-5}\\{NWgff18-ComF-6}\\{NWgff18-ComF-7}\\{NWgff18-ComF-8}\\{NWgff18-ComF-9}}\nwused{\\{NWgff18-APL5-1}}\nwendcode{}\nwbegindocs{60}\nwdocspar

\nwenddocs{}\nwbegincode{61}\sublabel{NWgff18-UseB-2}\nwmargintag{{\nwtagstyle{}\subpageref{NWgff18-UseB-2}}}\moddef{Use Modules~{\nwtagstyle{}\subpageref{NWgff18-UseB-1}}}\plusendmoddef
use Getopt::Long;
Getopt::Long::Configure qw/ bundling /;
\nwendcode{}\nwbegindocs{62}\nwdocspar

\nwenddocs{}\nwbegincode{63}\sublabel{NWgff18-perI-2}\nwmargintag{{\nwtagstyle{}\subpageref{NWgff18-perI-2}}}\moddef{perl requires help~{\nwtagstyle{}\subpageref{NWgff18-perI-1}}}\plusendmoddef
"Getopt::Long" - processing command-line options.
\nwendcode{}\nwbegindocs{64}\nwdocspar

See '{\tt{}man\ Getopt::Long}' for further info about this package.

\nwenddocs{}\nwbegincode{65}\sublabel{NWgff18-GloB-3}\nwmargintag{{\nwtagstyle{}\subpageref{NWgff18-GloB-3}}}\moddef{Global Vars~{\nwtagstyle{}\subpageref{NWgff18-GloB-1}}}\plusendmoddef
my ($Debug,$Verbose,$Quiet,$LogFile,$logs_filename) = (0,0,0,0,undef);
\nwendcode{}\nwbegindocs{66}\nwdocspar

\nwenddocs{}\nwbegincode{67}\sublabel{NWgff18-ward-1}\nwmargintag{{\nwtagstyle{}\subpageref{NWgff18-ward-1}}}\moddef{warnings - parsing command-line options~{\nwtagstyle{}\subpageref{NWgff18-ward-1}}}\endmoddef
UNKNOWN_CL_OPTION =>
  $Warn."Error trapped while processing command-line:\\n".(" "x16)."\\%s\\n",
CMD_LINE_ERROR =>
  $spl.$spw." Please, check your command-line options!!!\\n".$Error."\\n".
  $spw." ".("."x12)." Type \\"gff2aplot -h\\" for help.\\n".$spl,
\nwused{\\{NWgff18-GloB-G}}\nwendcode{}\nwbegindocs{68}\nwdocspar

\subsubsctn{Special command-line options (forcing exit from program)}

The first command-line checking looks for those options exiting the program: '{\tt{}help}' and '{\tt{}version}'. Both need to output to screen without any other message/warning being displayed at the same time.

\nwenddocs{}\nwbegincode{69}\sublabel{NWgff18-comU-1}\nwmargintag{{\nwtagstyle{}\subpageref{NWgff18-comU-1}}}\moddef{command-line options with exit~{\nwtagstyle{}\subpageref{NWgff18-comU-1}}}\endmoddef
"version"   => \\&prt_version, 
"h|help|?"  => \\&prt_help,
\nwused{\\{NWgff18-ParS-1}}\nwendcode{}\nwbegindocs{70}\nwdocspar
\nwenddocs{}\nwbegincode{71}\sublabel{NWgff18-comH-2}\nwmargintag{{\nwtagstyle{}\subpageref{NWgff18-comH-2}}}\moddef{command-line help~{\nwtagstyle{}\subpageref{NWgff18-comH-1}}}\plusendmoddef
-h, --help    Shows this help.
--version     Shows current version and exits.
\nwendcode{}\nwbegindocs{72}\nwdocspar
\nwenddocs{}%
%
%
\nwbegindocs{74}\nwdocspar
%<command-line latex>>=
%\clritem{-h{\x}-\/-help}
%\clritem{-\/-version}
%@

\nwenddocs{}\nwbegincode{75}\sublabel{NWgff18-ParS-2}\nwmargintag{{\nwtagstyle{}\subpageref{NWgff18-ParS-2}}}\moddef{Parsing Command-Line Options~{\nwtagstyle{}\subpageref{NWgff18-ParS-1}}}\plusendmoddef
sub prt_version() \{
    my $comment = $Messages\{'SHOW_VERSION'\};
    $comment = sprintf($comment,$PROGRAM,$VERSION);
    &prt_to_stderr($comment);
    exit(1);
\}
\nwendcode{}\nwbegindocs{76}\nwdocspar

\nwenddocs{}\nwbegincode{77}\sublabel{NWgff18-mesd-1}\nwmargintag{{\nwtagstyle{}\subpageref{NWgff18-mesd-1}}}\moddef{messages - parsing command-line options~{\nwtagstyle{}\subpageref{NWgff18-mesd-1}}}\endmoddef
SHOW_VERSION =>
  $sp."### \\%s -- \\%s\\n".$sp,
\nwused{\\{NWgff18-GloB-G}}\nwendcode{}\nwbegindocs{78}\nwdocspar

\subsubsctn{Testing command-line input filenames}

\nwenddocs{}\nwbegincode{79}\sublabel{NWgff18-ParS-3}\nwmargintag{{\nwtagstyle{}\subpageref{NWgff18-ParS-3}}}\moddef{Parsing Command-Line Options~{\nwtagstyle{}\subpageref{NWgff18-ParS-1}}}\plusendmoddef
sub set_input_file() \{
    my $stdin_flg = $F;
    \LA{}STDIN backwards compatibility~{\nwtagstyle{}\subpageref{NWgff18-STDT-1}}\RA{}
    &report("CHECKING_FILENAMES");
  FILECHK: foreach my $test_file (@ARGV) \{
        $test_file ne '-' && do \{
            -e $test_file || do \{
                &warn('FILE_NO_OPEN',$T,$test_file);
                next FILECHK;
            \};
            &report('READING_FILE',$test_file);
            push @data_files, $test_file;
            next FILECHK;
        \};
        $stdin_flg = $T;
        push @data_files, '-';
    \}; # foreach
    scalar(@data_files) == 0 && do \{
        push @data_files, '-';
        $stdin_flg = $T;
    \};
    $stdin_flg && &report('READING_STDIN');
\}
\nwendcode{}\nwbegindocs{80}\nwdocspar

\nwenddocs{}\nwbegincode{81}\sublabel{NWgff18-GloB-4}\nwmargintag{{\nwtagstyle{}\subpageref{NWgff18-GloB-4}}}\moddef{Global Vars~{\nwtagstyle{}\subpageref{NWgff18-GloB-1}}}\plusendmoddef
my (@data_files,$file);
\nwendcode{}\nwbegindocs{82}\nwdocspar

\nwenddocs{}\nwbegincode{83}\sublabel{NWgff18-outN-1}\nwmargintag{{\nwtagstyle{}\subpageref{NWgff18-outN-1}}}\moddef{warnings - input/output~{\nwtagstyle{}\subpageref{NWgff18-outN-1}}}\endmoddef
FILE_NO_OPEN =>
  $spl.$Warn."Cannot Open Current file \\"\\%s\\" . Not used !!!\\n".$spl,
\nwalsodefined{\\{NWgff18-outN-2}}\nwused{\\{NWgff18-GloB-G}}\nwendcode{}\nwbegindocs{84}\nwdocspar

\nwenddocs{}\nwbegincode{85}\sublabel{NWgff18-outN.2-1}\nwmargintag{{\nwtagstyle{}\subpageref{NWgff18-outN.2-1}}}\moddef{messages - input/output~{\nwtagstyle{}\subpageref{NWgff18-outN.2-1}}}\endmoddef
CHECKING_FILENAMES =>
  $sp."### Validating INPUT FILENAMES\\n".$sp,
READING_FILE =>
  "###---> \\"\\%s\\" exists, including as Input File.\\n",
READING_STDIN =>
  "###---> Including GFF records from standard input.\\n",  
\nwalsodefined{\\{NWgff18-outN.2-2}\\{NWgff18-outN.2-3}}\nwused{\\{NWgff18-GloB-G}}\nwendcode{}\nwbegindocs{86}\nwdocspar

Here is the fix for the explained in section~\ref{sec:stdinfix} on page~\pageref{sec:stdinfix} ({\tt{}\LA{}looking for STDIN~{\nwtagstyle{}\subpageref{NWgff18-looH-1}}\RA{}} code).

\nwenddocs{}\nwbegincode{87}\sublabel{NWgff18-STDT-1}\nwmargintag{{\nwtagstyle{}\subpageref{NWgff18-STDT-1}}}\moddef{STDIN backwards compatibility~{\nwtagstyle{}\subpageref{NWgff18-STDT-1}}}\endmoddef
my $chk_stdin = shift @_;
my $t = scalar(@ARGV);
defined($chk_stdin) && do \{
    abs($chk_stdin) > $t && ($chk_stdin = -$t);
    $chk_stdin > 0  && ($chk_stdin = 0 );
    $t += $chk_stdin;
    splice(@ARGV,$t,0,'-');
\};
\nwused{\\{NWgff18-ParS-3}}\nwendcode{}\nwbegindocs{88}\nwdocspar

\subsubsctn{Verifying command-line settings}

\label{todo:BBC}
\nwenddocs{}%
%
\nwbegindocs{90}\nwdocspar
\nwenddocs{}%
%
\nwbegindocs{92}\nwdocspar
\todo{ \item \todoBBC } % todo
\nwenddocs{}\nwbegincode{93}\sublabel{NWgff18-ParS-4}\nwmargintag{{\nwtagstyle{}\subpageref{NWgff18-ParS-4}}}\moddef{Parsing Command-Line Options~{\nwtagstyle{}\subpageref{NWgff18-ParS-1}}}\plusendmoddef
sub check_command_line_vars() \{
    &report("CHECKING_CMDLN_VARS");
\} # check_command_line
\nwendcode{}\nwbegindocs{94}\nwdocspar

\nwenddocs{}\nwbegincode{95}\sublabel{NWgff18-outN.2-2}\nwmargintag{{\nwtagstyle{}\subpageref{NWgff18-outN.2-2}}}\moddef{messages - input/output~{\nwtagstyle{}\subpageref{NWgff18-outN.2-1}}}\plusendmoddef
CHECKING_CMDLN_VARS =>
  $sp."### Checking COMMAND-LINE Settings\\n".$sp,
\nwendcode{}\nwbegindocs{96}\nwdocspar

\subsubsctn{Loading GFF elements custom variables from command-line}

\label{todo:BBD}
\nwenddocs{}%
%
%
%
%
\nwbegindocs{98}\nwdocspar
\nwenddocs{}%
%
\nwbegindocs{100}\nwdocspar
\todo{ \item \todoBBD } % todo
\nwenddocs{}\nwbegincode{101}\sublabel{NWgff18-ParS-5}\nwmargintag{{\nwtagstyle{}\subpageref{NWgff18-ParS-5}}}\moddef{Parsing Command-Line Options~{\nwtagstyle{}\subpageref{NWgff18-ParS-1}}}\plusendmoddef
#
\nwendcode{}\nwbegindocs{102}\nwdocspar

\nwenddocs{}\nwbegincode{103}\sublabel{NWgff18-comR-1}\nwmargintag{{\nwtagstyle{}\subpageref{NWgff18-comR-1}}}\moddef{command-line options custom~{\nwtagstyle{}\subpageref{NWgff18-comR-1}}}\endmoddef
"feature-var=s@"  => \\@\{ $CmdLineVars\{VARS\}\{FEATURE\} \},
"group-var=s@"    => \\@\{ $CmdLineVars\{VARS\}\{GROUP\} \},
"strand-var=s@"   => \\@\{ $CmdLineVars\{VARS\}\{STRAND\} \},
"source-var=s@"   => \\@\{ $CmdLineVars\{VARS\}\{SOURCE\} \},
"sequence-var=s@" => \\@\{ $CmdLineVars\{VARS\}\{SEQUENCE\} \},
"layout-var=s@"   => \\@\{ $CmdLineVars\{VARS\}\{LAYOUT\} \},
\nwalsodefined{\\{NWgff18-comR-2}}\nwused{\\{NWgff18-comK-3}}\nwendcode{}\nwbegindocs{104}\nwdocspar
\nwenddocs{}\nwbegincode{105}\sublabel{NWgff18-comO-1}\nwmargintag{{\nwtagstyle{}\subpageref{NWgff18-comO-1}}}\moddef{command-line help custom~{\nwtagstyle{}\subpageref{NWgff18-comO-1}}}\endmoddef
 --feature-var '<feature::variable=value>'
   --group-var '<group::variable=value>'
  --strand-var '<strand::variable=value>' 
  --source-var '<source::variable=value>'
--sequence-var '<sequence::variable=value>'
  --layout-var '<variable=value>'
         Loading a feature/group/strand/source/sequence/layout
         customization variable from command-line. You can set
         several variables by repeating any of these options, 
         i.e.:
... --feature-var cds::shape=box --feature-var cds::feature_color=blue ... 
\nwalsodefined{\\{NWgff18-comO-2}}\nwused{\\{NWgff18-comH-5}}\nwendcode{}\nwbegindocs{106}\nwdocspar
\nwenddocs{}%
%
%
%
%
%
%
%
%
%
%
%
%
\nwbegindocs{108}\nwdocspar
\nwenddocs{}%
%
%
%
%
%
%
%
%
%
%
%
%
\nwbegindocs{110}\nwdocspar

\subsctn{Setting customization parameters from files} 

Here we define the command-line option for loading customization files. Setting '{\tt{}O|custom-filename=s@}' allows us to pass more than one custom file within the same command-line. Must be noted that latest file customization variables will override the previous ones.

\nwenddocs{}\nwbegincode{111}\sublabel{NWgff18-comR-2}\nwmargintag{{\nwtagstyle{}\subpageref{NWgff18-comR-2}}}\moddef{command-line options custom~{\nwtagstyle{}\subpageref{NWgff18-comR-1}}}\plusendmoddef
"O|custom-filename=s@"  => \\@custom_files,
\nwendcode{}\nwbegindocs{112}\nwdocspar
\nwenddocs{}\nwbegincode{113}\sublabel{NWgff18-comO-2}\nwmargintag{{\nwtagstyle{}\subpageref{NWgff18-comO-2}}}\moddef{command-line help custom~{\nwtagstyle{}\subpageref{NWgff18-comO-1}}}\plusendmoddef
-O, --custom-filename <custom_filename> Load a configuration file 
         (if default \\".gff2aplotrc\\" exists is loaded before it).
\nwendcode{}\nwbegindocs{114}\nwdocspar
\nwenddocs{}%
%
%
\nwbegindocs{116}\nwdocspar
\nwenddocs{}%
%
%
\nwbegindocs{118}\nwdocspar

\subsubsctn{Testing whether custom-files exist}

Customization parameters cannot be set from '{\tt{}STDIN}'. What program does first of all, is to look whether default custom file does exist or not. But we are taking advantage of the fact that we also have to check existence of the command-line custom files (if given), so we insert as first element of the array the default custom file to the variable '{\tt{}@custom{\char95}files}' and then we check all its elements with the same function. We look for these files in current path and if they are not found in '{\tt{}{\char36}Custom{\char95}path}'.

\nwenddocs{}\nwbegincode{119}\sublabel{NWgff18-ParK-1}\nwmargintag{{\nwtagstyle{}\subpageref{NWgff18-ParK-1}}}\moddef{Parsing Custom Files~{\nwtagstyle{}\subpageref{NWgff18-ParK-1}}}\endmoddef
sub set_custom_files() \{
    unshift @custom_files, $Custom_file;
    my @files = ();
    &report("CHECKING_CUSTOM_NAMES");
  MLOOP: foreach my $test_file (@custom_files) \{
      FILECHK: \{
        -e $test_file && last FILECHK;
        ($test_file =~ m\{/\}og || $Custom_path eq '.') || do \{
            my $tmpfl = $test_file;
            $test_file = "$Custom_path/$test_file";
            &report('READING_FROM_PATH',$tmpfl,$test_file);
            -e $test_file && last FILECHK;
        \};
        scalar(@custom_files) == 1 && do \{
            &report('NO_CUSTOM_FILES',$T);
            last MLOOP;
        \};
        &warn('FILE_NO_OPEN',$T,$test_file);
        next MLOOP;
      \}; # FILECHK
        &report('READING_CUSTOM_FILE',$test_file);
        push @files, $test_file;
    \}; # MLOOP: foreach
    @custom_files = @files;
\} # set_custom_files
\nwalsodefined{\\{NWgff18-ParK-2}\\{NWgff18-ParK-3}\\{NWgff18-ParK-4}\\{NWgff18-ParK-5}\\{NWgff18-ParK-6}\\{NWgff18-ParK-7}}\nwused{\\{NWgff18-MaiD-1}}\nwendcode{}\nwbegindocs{120}\nwdocspar

\nwenddocs{}\nwbegincode{121}\sublabel{NWgff18-GloB-5}\nwmargintag{{\nwtagstyle{}\subpageref{NWgff18-GloB-5}}}\moddef{Global Vars~{\nwtagstyle{}\subpageref{NWgff18-GloB-1}}}\plusendmoddef
my @custom_files = ();
\nwendcode{}\nwbegindocs{122}\nwdocspar

\nwenddocs{}\nwbegincode{123}\sublabel{NWgff18-outN.2-3}\nwmargintag{{\nwtagstyle{}\subpageref{NWgff18-outN.2-3}}}\moddef{messages - input/output~{\nwtagstyle{}\subpageref{NWgff18-outN.2-1}}}\plusendmoddef
CHECKING_CUSTOM_NAMES =>
  $sp."### Validating CUSTOM FILENAMES\\n".$sp,
READING_FROM_PATH =>
  "###---> Custom File NOT FOUND in local path: \\"\\%s\\"\\n".
  "###     Trying to find in \\"GFF2APLOT_CUSTOMPATH\\": \\"\\%s\\"\\n",
READING_CUSTOM_FILE =>
  "###---> \\"\\%s\\" exists, including as Custom File.\\n",
NO_CUSTOM_FILES =>
  "###---> NO CUSTOM FILES found. Using program DEFAULTS.\\n",
\nwendcode{}\nwbegindocs{124}\nwdocspar

\subsubsctn{Parsing customization files}

Layout variables are defined as two fields, variable name and value, while the other variables require three fields, a feature to which is set the variable to the given value. To validate input records, '{\tt{}{\char36}var{\char95}flag}' boolean will determine whether a two fields or a three fields record we currently are going to process. '{\tt{}{\char38}varscheck}' function not only checks the customization parameters for correctness but also loads '{\tt{}{\char37}CustomVars}' hash.

\nwenddocs{}\nwbegincode{125}\sublabel{NWgff18-ParK-2}\nwmargintag{{\nwtagstyle{}\subpageref{NWgff18-ParK-2}}}\moddef{Parsing Custom Files~{\nwtagstyle{}\subpageref{NWgff18-ParK-1}}}\plusendmoddef
sub parse_custom_files() \{
    &header("READING CUSTOM FILES");
  MAIN: \{
      scalar(@custom_files) == 0 && do \{
          &report('NO_CUSTOM_FOUND',$file);
          last MAIN;
      \};
    LOAD: foreach $file (@custom_files) \{
        open(THIS,"< $file") ||
            ( &warn('FILE_NO_OPEN',$T,$file), next LOAD);
        &report('READ_CUSTOM_FILE',$file);
        ($n,$c) = (0,undef);
        while (<THIS>) \{
            my (@line,$main,$_c,$_v,$v_flag);
            /^\\#/o && do \{
                /^\\# ([LQSTGF]) \\#/o && do \{ 
                     $_c = $1; $c = '*';
                     $v_flag = ($_c ne 'L') ? $T : $F;
                     $_v = $varkeys\{$_c\};
                     next;
                \};
                $c = '.'; next;
            \};
            ($c = '.', next) if /^\\s*$/o;
            chomp;
            ($main,undef) = split /\\b\\s+\\#/o;
          TWOTHREE: \{
              $v_flag && do \{
                  $main =~ /^(.*?):\{2\}(.*?)=\{1\}(.*?)$/o &&
                      (@line = ($1,$2,$3));
                  last TWOTHREE;
              \};
              $main =~ /^(.*?)=\{1\}(.*?)$/o &&
                  (@line = ($_v,$1,$2));
          \}; # TWOTHREE
            $c = &varscheck($v_flag,$_v,\\@line) ? $_c : $noCV;
        \} continue \{
            &counter(++$n,$c) if ($Verbose && !$Quiet);
        \}; # WHILE
        &counter_end($n,$c) if ($Verbose && !$Quiet);
        close(THIS);
    \}; # LOAD
  \}; # MAIN
    print LOGFILE '>>> \\%CustomVars : '.(Dumper(\\%CustomVars))
        if ($LogFile && $Debug);
    &footer("CUSTOM FILES LOADED");
\} # parse_custom_files
\nwendcode{}\nwbegindocs{126}\nwdocspar

\begin{table}[!t]
\begin{center}
\begin{tabular}{cc}
\fbox{
\begin{tabular}[b]{c>{\bfseries}c}
\textbf{Record TYPE} & \textbf{Symbol} \\
Empty record            & . \\
Comment                 & . \\
Header (\# [LQSTGF] \#) & * \\
Layout Vars             & L \\
Sequence Vars           & Q \\
Source Vars             & S \\
Strand Vars             & T \\
Group Vars              & G \\
Feature Vars            & F \\
Unknown                 & ? \\
\end{tabular}
} %fbox
&
\parbox[b]{0.4\linewidth}{
\caption[Symbols used to report record type from custom files]{\label{tbl:CFkeysXreport} Symbols used to report record type when parsing customization files (and {\tt{}STDERR} verbose mode is enabled with '{\tt{}-v}' switch).}
} % parbox
\\
\end{tabular}
\end{center}
\end{table}

\nwenddocs{}\nwbegincode{127}\sublabel{NWgff18-GloG-3}\nwmargintag{{\nwtagstyle{}\subpageref{NWgff18-GloG-3}}}\moddef{Global Constants~{\nwtagstyle{}\subpageref{NWgff18-GloG-1}}}\plusendmoddef
my $noCV = '?';
my %varkeys = (
    L => 'LAYOUT',
    Q => 'SEQUENCE',
    S => 'SOURCE',
    T => 'STRAND',
    G => 'GROUP',
    F => 'FEATURE',
    );
\nwendcode{}\nwbegindocs{128}\nwdocspar

\nwenddocs{}\nwbegincode{129}\sublabel{NWgff18-mesV-1}\nwmargintag{{\nwtagstyle{}\subpageref{NWgff18-mesV-1}}}\moddef{messages - parsing custom files~{\nwtagstyle{}\subpageref{NWgff18-mesV-1}}}\endmoddef
READ_CUSTOM_FILE => 
  $sp."### Reading Customization Parameters from \\"\\%s\\"\\n".$sp,
NO_CUSTOM_FOUND =>
  $sp."### NO CUSTOM FILES found: Using program DEFAULTS.\\n".$sp,
\nwused{\\{NWgff18-GloB-G}}\nwendcode{}\nwbegindocs{130}\nwdocspar

\subsubsctn{Validating customization parameters} 

'{\tt{}{\char36}rec}' is a reference to an array that contains three elements: feature, variable and value. First we check if variable name is defined in , then we check if the given value matches the variable type. After that we analyse the feature element, testing if it is a regular expression or not.

\nwenddocs{}\nwbegincode{131}\sublabel{NWgff18-ParK-3}\nwmargintag{{\nwtagstyle{}\subpageref{NWgff18-ParK-3}}}\moddef{Parsing Custom Files~{\nwtagstyle{}\subpageref{NWgff18-ParK-1}}}\plusendmoddef
sub varscheck() \{
    my ($flag,$class,$rec) = @_;
    defined($DefaultVars\{$class\}) || 
        (&warn('SECTION_NOT_DEF',$F),return $F);
    my $_var = \\%\{$DefaultVars\{$class\}\};
    defined($_var->\{$$rec[1]\}) || 
        (&warn('VAR_NOT_DEFINED',$F,$class,$$rec[1]),return $F);
    &checkvarvalues($_var->\{$$rec[1]\}\{'TYPE'\},\\$$rec[2],$$rec[1]) || return $F;
    $flag && do \{
        my @tmpary = ();
        defined(@\{ $CustomVars\{$class\}\{$$rec[1]\} \}) || do \{
            @\{ $CustomVars\{$class\}\{$$rec[1]\} \} = ();
        \};
        @tmpary = &find_regexp($$rec[0]);
        (shift @tmpary) || return $F;
        push @\{ $CustomVars\{$class\}\{$$rec[1]\} \}, 
             ( @tmpary, $$rec[2] );
        return $T;
    \};
    $CustomVars\{$class\}\{$$rec[1]\} = $$rec[2];
    return $T;
\} # varscheck
\nwendcode{}\nwbegindocs{132}\nwdocspar

\label{sec:DATAtriadarray}
For the same variable name, say here {\tt{}\ {\char36}{\char36}rec[1]\ }, we will have more than one setting (for the same element or not). This requirement force us to save element-value pairs (and a third field which tells us if element contains a regular expression that must not to be matched or not). The array will preserve the input ordering of each new setting for the same variable. We want this behaviour, as the global definitions must precede specific ones in the custom files. 

\nwenddocs{}\nwbegincode{133}\sublabel{NWgff18-warV-1}\nwmargintag{{\nwtagstyle{}\subpageref{NWgff18-warV-1}}}\moddef{warnings - parsing custom files~{\nwtagstyle{}\subpageref{NWgff18-warV-1}}}\endmoddef
SECTION_NOT_DEF =>
  $Warn."You probably forgot a section header, unable to parse this record.\\n",
VAR_NOT_DEFINED =>
  $Warn."\\%s variable not defined: \\"\\%s\\" .\\n",
\nwalsodefined{\\{NWgff18-warV-2}\\{NWgff18-warV-3}\\{NWgff18-warV-4}\\{NWgff18-warV-5}\\{NWgff18-warV-6}\\{NWgff18-warV-7}\\{NWgff18-warV-8}\\{NWgff18-warV-9}\\{NWgff18-warV-A}\\{NWgff18-warV-B}}\nwused{\\{NWgff18-GloB-G}}\nwendcode{}\nwbegindocs{134}\nwdocspar

Different data requires different value tests, another function of those tests is to normalize when needed variable values to fixed internal values, so this will avoid later checkings or functions for reformating them and writing functions
will be simplest.

\nwenddocs{}\nwbegincode{135}\sublabel{NWgff18-ParK-4}\nwmargintag{{\nwtagstyle{}\subpageref{NWgff18-ParK-4}}}\moddef{Parsing Custom Files~{\nwtagstyle{}\subpageref{NWgff18-ParK-1}}}\plusendmoddef
sub checkvarvalues() \{
    my ($_test,$_val,$_var) = @_;
    my $_t = lc($$_val);
    \LA{}variable value checking~{\nwtagstyle{}\subpageref{NWgff18-varN-1}}\RA{}
    &warn('VARTYPE_NOT_DEFINED',$T,$_test,$_var);
    return $F;
\} # checkvarvalues
\nwendcode{}\nwbegindocs{136}\nwdocspar

\nwenddocs{}\nwbegincode{137}\sublabel{NWgff18-warV-2}\nwmargintag{{\nwtagstyle{}\subpageref{NWgff18-warV-2}}}\moddef{warnings - parsing custom files~{\nwtagstyle{}\subpageref{NWgff18-warV-1}}}\plusendmoddef
VARTYPE_NOT_DEFINED =>
  $Error."Variable type \\"\\%s\\" not defined,\\n".
  $spw."  could not check value for \\"\\%s\\".\\n",
\nwendcode{}\nwbegindocs{138}\nwdocspar

\subsctn{Deciphering element fields}

If an {\tt{}element} appears not to be a regular expression we transform it
to '{\tt{}{\char94}element{\char36}}' before returning it. {\tt{}{\char36}not{\char95}flg} and {\tt{}{\char36}id{\char95}flg}
will be set to a true value in case that input string contains a
'{\tt{}!}' as first character ({\tt{}!element},{\tt{}!/regexp/}, {\tt{}!/regexp/@id} and
{\tt{}!*}) and that an ID is given ({\tt{}element@id}, {\tt{}/regexp/@id} and
{\tt{}!/regexp/@id}), respectively. Both flags will be helpful to
determine what test we have to perform in order to know if we have to
set the variable value for a GFF element (see
section~\ref{sec:mapcustoms}, page~\pageref{sec:mapcustoms}). For
{\tt{}{\char36}id{\char95}flg}, the true value will consist on the feature identifier, so
that we do not need to introduce a new field in the array.

\nwenddocs{}\nwbegincode{139}\sublabel{NWgff18-ParK-5}\nwmargintag{{\nwtagstyle{}\subpageref{NWgff18-ParK-5}}}\moddef{Parsing Custom Files~{\nwtagstyle{}\subpageref{NWgff18-ParK-1}}}\plusendmoddef
sub find_regexp() \{
    my $string = $_[0];
    my ($isOK_flg,$not_flg,$id_flg,$tmpstr,$tmpid);
    $isOK_flg = $T;
    $not_flg = $F;
    $id_flg = undef;
    $string =~ s\{^!\}\{\}o && ($not_flg = $T); # not_regexp is true
    $string =~ s\{(\\\\@)$\}\{@@\}o; # scaping trailing '@' (if element ends with) 
    $string = &escape_input($string);
    ($tmpstr, $tmpid) = (undef, undef);
    ( reverse($string) =~ m\{^([^\\/@]*?)(?:@)\{1\}(.*)$\}o ) && do \{
        $tmpstr = reverse($2);
        $tmpid  = reverse($1);
    \}; # reverse($string)
    (defined($tmpid)  && $tmpid  ne "") && ($id_flg = $tmpid);
    (defined($tmpstr) && $tmpstr ne "") || do \{
        $string eq '@' && ($isOK_flg = $F);
        $tmpstr = $string;
    \}; # (defined($tmpstr) && $tmpstr ne "")
  REGEXPS: \{
      \LA{}does string contain a regular expression~{\nwtagstyle{}\subpageref{NWgff18-doee-1}}\RA{}
    \}; # REGEXPS
    $isOK_flg && do \{
        eval \{ "" =~ m\{$string\}; \} || do \{
            $isOK_flg = $F;
        \}; # check if final regexp string is OK
    \}; # $isOK_flg
    $isOK_flg || do \{
        &warn('BAD_REGEXP',$F,"$string");
        $string = "";
    \}; # NOT $isOK_flg
    return ($isOK_flg, $not_flg, $id_flg, $string);
\} # find_regexp
\nwendcode{}\nwbegindocs{140}\nwdocspar

\nwenddocs{}\nwbegincode{141}\sublabel{NWgff18-warV-3}\nwmargintag{{\nwtagstyle{}\subpageref{NWgff18-warV-3}}}\moddef{warnings - parsing custom files~{\nwtagstyle{}\subpageref{NWgff18-warV-1}}}\plusendmoddef
BAD_REGEXP =>
  $Warn."Ill-formed regular expression found in custom file:\\n".
  $spw." ---> \\%s <---\\n",
\nwendcode{}\nwbegindocs{142}\nwdocspar

Two steps are needed for parsing element fields. We must ensure first
that there is an {\tt{}@id} feature delimiter in the element field and,
after removing that delimiter, check if we have a regular expression
or not. The easiest way to extract any trailing 'id' substring from
element field is using a sexeger, a reversed regular expression
\footnote{Further information on sexeger (reversed regexes) can be
found at \url|http://www.perl.com/pub/2001/05/01/expressions.html| and
\url|http://www.pobox.com/~japhy/|.}

\nwenddocs{}\nwbegincode{143}\sublabel{NWgff18-doee-1}\nwmargintag{{\nwtagstyle{}\subpageref{NWgff18-doee-1}}}\moddef{does string contain a regular expression~{\nwtagstyle{}\subpageref{NWgff18-doee-1}}}\endmoddef
$tmpstr eq '*' && do \{
    $string = '^.*$'; #'
    last REGEXPS;
\}; # $string eq '*'
$tmpstr =~ m\{^/(.*)/$\}o && do \{    
    ($string, $isOK_flg) = &eval_regexp($1);
    last REGEXPS;
\}; # $tmpstr is a regexp
# ($tmpstr =~ m\{^/.*[^/]$\}o || $tmpstr =~ m\{^[^/].*/$\}o) && do \{
#     $isOK_flg = $F;  
#     last REGEXPS;
# \}; # $tmpstr is a bad defined regexp
$string = '^'.(quotemeta($tmpstr)).'$'; #' 
\nwused{\\{NWgff18-ParK-5}}\nwendcode{}\nwbegindocs{144}\nwdocspar

\nwenddocs{}%
%
%
%
%
%
%
%
%
%
%
%
%
%
%
%
%
%
%
%
%
%
%
%
%
%
%
%
%
%
%
\nwbegindocs{146}\nwdocspar

\nwenddocs{}\nwbegincode{147}\sublabel{NWgff18-ParK-6}\nwmargintag{{\nwtagstyle{}\subpageref{NWgff18-ParK-6}}}\moddef{Parsing Custom Files~{\nwtagstyle{}\subpageref{NWgff18-ParK-1}}}\plusendmoddef
sub eval_regexp() \{
    my $str = $_[0];
    my $flag;
    eval \{ "" =~ m\{$str\}; $flag = $T; \} || ($flag = $F);
    return ($str, $flag);
\} # eval_regexp
\nwendcode{}\nwbegindocs{148}\nwdocspar

From a list of possible tainted shell metacharacters, \\[1ex]
\centerline{{\tt{}\ ;\ <\ >\ *\ |\ {\char38}\ {\char36}\ !\ ?\ {\char35}\ (\ )\ [\ ]\ {\char123}\ {\char125}\ :\ `\ '\ "\ {\char92}\ }}\vskip 0.5ex
% # @_ =~ s/([;<>\*\|`&\$!?#\(\)\[\]\{\}:'"\\])/\\$1/g;
we quote those that are not going to be used as regular expression metacharacters, \\[1ex]
\centerline{{\tt{}\ ;\ <\ >\ {\char38}\ !\ {\char123}\ {\char125}\ `\ '\ "\ }}\vskip 0.5ex
to avoid security issues. This will make a CGI developed on {\prog} a little bit more secure because the user cannot send something like:\\[1ex]
\centerline{\shortstack{{\tt{}//;\ `/bin/rm\ -rf`;//}\ \ \ ,\\{\tt{}/\ @{\char123}[\ system('mail\ user@server\ /etc/passwd')\ ]{\char125}\ /}\ \ \ or\\{\tt{}/\ {\char36}{\char123}\ {\char92}(\ exec('chmod\ -R\ 0777\ /root')\ )\ {\char125}\ /}}} %$

\nwenddocs{}\nwbegincode{149}\sublabel{NWgff18-ParK-7}\nwmargintag{{\nwtagstyle{}\subpageref{NWgff18-ParK-7}}}\moddef{Parsing Custom Files~{\nwtagstyle{}\subpageref{NWgff18-ParK-1}}}\plusendmoddef
sub escape_input() \{
    my $var = $_[0];
    $var =~ s\{([;,<>&!\\\{\\\}`'"])\}\{\\\\$1\}g;
    return $var;
\} # escape_input
\nwendcode{}\nwbegindocs{150}\nwdocspar

\nwenddocs{}%
%
%
%
%
%
%
%
%
%
%
%
%
%
%
%
%
%
%
%
%
%
%
%
%
%
%
%
%
%
%
%
%
%
%
%
%
%
%
%
%
%
%
%
%
%
%
%
%
%
%
%
%
%
%
%
%
%
%
%
%
%
%
%
%
%
%
%
%
%
\nwbegindocs{152}\nwdocspar

\nwenddocs{}%
%
%
%
%
%
%
%
%
%
%
%
%
%
%
%
%
%
%
%
%
%
%
%
%
%
%
%
%
%
%
%
%
%
%
%
%
\nwbegindocs{154}\nwdocspar

\nwenddocs{}%
%
%
%
%
%
\nwbegindocs{156}\nwdocspar

\subsctn{Customization variable values validation}

This section is dedicated to those test we need to ensure that customization variables contain acceptable values, so we do not need to check them later in the program. Table~\ref{tbl:varformats} summarizes the variable types defined for customization parameters.

\begin{table}[!t]
\begin{center}
\fbox{
\begin{tabular}{cl}
BOOLEAN  & A true/false variable.\\
STRING   & A free text character string.\\
DIGIT    & A natural number (nor decimals neither sign).\\
INTEGER  & Positive or negative integers.\\
DECIMAL  & Numbers with decimal values.\\
REALNUM  & Any number, including scientific notation.\\
COLOR    & Available colors are defined in '{\tt{}{\char37}COLORS}' hash.\\
PAGE     & Page sizes are set in '{\tt{}{\char37}FORMATS}' variable.\\
DNABASE  & A numeric value with nucleotide units (Gb, Mb, Kb or bases).\\
RANGE    & A 'lower-limit..upper-limit' numeric pair.\\
FONT     & A {\ps} from those defined in '{\tt{}{\char37}Fonts}' hash.\\
FONT\_SZ & A numeric value with font-size units (cm, mm, in or pt ---points---).\\
\end{tabular}
} % fbox
\caption[Customization parameters available formats]{\label{tbl:varformats} Parameter available formats we are going to use for customization variables checking.}
\end{center}

\todo{
\item More format variables need to be defined, such as: {\tt{}SHAPES} from a shapes array, {\tt{}LINES}/{\tt{}LINESTYLES} (none, dotted, dashed, solid/default), {\tt{}GROUP-BRACKETS} (bracket, arrow, half-arrow, dotted-bracket --with vertical dotted lines at both ends like conecting to baseline--).
\item maybe {\tt{}UNITS} instead of {\tt{}FONT{\char95}SZ} will have broader applicability.
} % todo

\end{table}

\subsubsctn{Checking boolean variables}

We define as 'BOOLEAN' those variables used as flags or switches, which can hold only a binary value such ON/OFF. Booleans are NOT case sensitive.

\nwenddocs{}\nwbegincode{157}\sublabel{NWgff18-varN-1}\nwmargintag{{\nwtagstyle{}\subpageref{NWgff18-varN-1}}}\moddef{variable value checking~{\nwtagstyle{}\subpageref{NWgff18-varN-1}}}\endmoddef
$_test eq 'BOOLEAN' && do \{
    $_t =~ /^(1|on|t(rue)|y(es))$/o && ($$_val = $T, return $T);
    $_t =~ /^(0|off|f(alse)|n(o))$/o && ($$_val = $F, return $T);
    &warn('NOT_A_BOOLEAN',$F,$_var);
    return $F;
\};
\nwalsodefined{\\{NWgff18-varN-2}\\{NWgff18-varN-3}\\{NWgff18-varN-4}\\{NWgff18-varN-5}\\{NWgff18-varN-6}\\{NWgff18-varN-7}\\{NWgff18-varN-8}\\{NWgff18-varN-9}}\nwused{\\{NWgff18-ParK-4}}\nwendcode{}\nwbegindocs{158}\nwdocspar

\nwenddocs{}\nwbegincode{159}\sublabel{NWgff18-warV-4}\nwmargintag{{\nwtagstyle{}\subpageref{NWgff18-warV-4}}}\moddef{warnings - parsing custom files~{\nwtagstyle{}\subpageref{NWgff18-warV-1}}}\plusendmoddef
NOT_A_BOOLEAN =>
  $Warn."\\"\\%s\\" variable requires a boolean value:\\n".
  $spw."     (ON/OFF, 1/0, TRUE/FALSE, YES/NO)\\n",
\nwendcode{}\nwbegindocs{160}\nwdocspar

\subsubsctn{Protecting reserved characters in strings}

Having a 'STRING' implies to replace/protect those characters that will produce indesirable effects in {\ps} code (such unbalanced parentheses). No error is reported for strings because they are assumed to be free text.

\nwenddocs{}\nwbegincode{161}\sublabel{NWgff18-varN-2}\nwmargintag{{\nwtagstyle{}\subpageref{NWgff18-varN-2}}}\moddef{variable value checking~{\nwtagstyle{}\subpageref{NWgff18-varN-1}}}\plusendmoddef
$_test eq 'STRING' && do \{
    $$_val =~ s\{[\\\\]\}\{\\\\134\}og;
    $$_val =~ s\{[\\(]\}\{\\\\050\}og;
    $$_val =~ s\{[\\)]\}\{\\\\051\}og;
    return $T;
\};
\nwendcode{}\nwbegindocs{162}\nwdocspar

\todo{
\item Remember to check for some of the strings values such: '++none++', '++sequence++', and so on.
} % todo

\subsubsctn{Validating numbers}

The following tests look for four different number types.

\nwenddocs{}\nwbegincode{163}\sublabel{NWgff18-varN-3}\nwmargintag{{\nwtagstyle{}\subpageref{NWgff18-varN-3}}}\moddef{variable value checking~{\nwtagstyle{}\subpageref{NWgff18-varN-1}}}\plusendmoddef
$_test eq 'DIGIT' && do \{
    $$_val =~ /^\\d+$/o && (return $T);
    &warn('NOT_A_DIGIT',$F,$_var);
    return $F;
\};
$_test eq 'INTEGER' && do \{
    $$_val =~ /^[+-]?\\d+$/o && (return $T);
    &warn('NOT_AN_INTEGER',$F,$_var);
    return $F;
\};
$_test eq 'DECIMAL' && do \{
    $$_val =~ /^[+-]?(?:\\d+(?:\\.\\d*)?|\\.\\d+)$/o && (return $T);
    &warn('NOT_A_DECIMAL',$F,$_var);
    return $F;
\};
$_test eq 'REALNUM' && do \{
    $$_val =~ /^([+-]?)(?=\\d|\\.\\d)\\d*(\\.\\d*)?([Ee]([+-]?\\d+))?$/o &&
        (return $T);
    &warn('NOT_A_REAL',$F,$_var);
    return $F;
\};
\nwendcode{}\nwbegindocs{164}%$

\nwenddocs{}\nwbegincode{165}\sublabel{NWgff18-warV-5}\nwmargintag{{\nwtagstyle{}\subpageref{NWgff18-warV-5}}}\moddef{warnings - parsing custom files~{\nwtagstyle{}\subpageref{NWgff18-warV-1}}}\plusendmoddef
NOT_A_DIGIT =>
  $Warn."\\"\\%s\\" variable requires a digit (natural number).\\n",
NOT_AN_INTEGER =>
  $Warn."\\"\\%s\\" variable requires an integer.\\n",
NOT_A_DECIMAL =>
  $Warn."\\"\\%s\\" variable requires a decimal number.\\n",
NOT_A_REAL =>
  $Warn."\\"\\%s\\" variable requires a real number (with exponent).\\n",
\nwendcode{}\nwbegindocs{166}%$

\subsubsctn{Looking for color names}

Available colors are shown in section~\ref{sec:CMYKcolordef}, page~\pageref{sec:CMYKcolordef}. Here we check that customization input colors match those defined colors, including 'background' and 'foreground' colors. Color names are NOT case sensitive.

\nwenddocs{}\nwbegincode{167}\sublabel{NWgff18-varN-4}\nwmargintag{{\nwtagstyle{}\subpageref{NWgff18-varN-4}}}\moddef{variable value checking~{\nwtagstyle{}\subpageref{NWgff18-varN-1}}}\plusendmoddef
$_test eq 'COLOR' && do \{ # lc
    $_t =~ /^(b(ack)?|f(ore)?)g(round)?(color)?/o &&
        ($$_val = $_t, return $T);
    defined($COLORS\{$_t\}) && ($$_val = $_t, return $T);
    &warn('NOT_A_COLOR',$F,$_t,$_var);
    return $F;
\};
\nwendcode{}\nwbegindocs{168}\nwdocspar

\todo{
\item Set fg/bg multiple strings to a single color name such fgcolor/bgcolor for later use in the whole program (PS code too).
} % todo

\nwenddocs{}\nwbegincode{169}\sublabel{NWgff18-warV-6}\nwmargintag{{\nwtagstyle{}\subpageref{NWgff18-warV-6}}}\moddef{warnings - parsing custom files~{\nwtagstyle{}\subpageref{NWgff18-warV-1}}}\plusendmoddef
NOT_A_COLOR =>
  $Warn."\\"\\%s\\" color not defined in \\"\\%s\\" variable.\\n",
\nwendcode{}\nwbegindocs{170}\nwdocspar

\subsubsctn{Verifying named page-sizes}

Available page sizes are shown in section~\ref{sec:PAGEdef}, page~\pageref{sec:PAGEdef}.

\nwenddocs{}\nwbegincode{171}\sublabel{NWgff18-varN-5}\nwmargintag{{\nwtagstyle{}\subpageref{NWgff18-varN-5}}}\moddef{variable value checking~{\nwtagstyle{}\subpageref{NWgff18-varN-1}}}\plusendmoddef
$_test eq 'PAGE' && do \{ # lc
    defined($FORMATS\{$_t\}) && ($$_val = $_t, return $T);
    &warn('NOT_A_PAGE',$F,$_t,$_var);
    return $F;
\};
\nwendcode{}\nwbegindocs{172}\nwdocspar

\nwenddocs{}\nwbegincode{173}\sublabel{NWgff18-warV-7}\nwmargintag{{\nwtagstyle{}\subpageref{NWgff18-warV-7}}}\moddef{warnings - parsing custom files~{\nwtagstyle{}\subpageref{NWgff18-warV-1}}}\plusendmoddef
NOT_A_PAGE =>
  $Warn."\\"\\%s\\" page-size is not defined for \\"\\%s\\" variable.\\n",
\nwendcode{}\nwbegindocs{174}\nwdocspar

\subsubsctn{Nucleotide coords}

Those numbers relative to DNA positions may be given in bases, kilobases, megabases or gigabases, being last three ones a way to avoid any mistake when a big coord with lots of zeroes is provided. If no units are provided, then we asume bases.

\label{sec:DNABASE}
\nwenddocs{}\nwbegincode{175}\sublabel{NWgff18-varN-6}\nwmargintag{{\nwtagstyle{}\subpageref{NWgff18-varN-6}}}\moddef{variable value checking~{\nwtagstyle{}\subpageref{NWgff18-varN-1}}}\plusendmoddef
$_test eq 'DNABASE' && do \{ # lc
    $_t =~ /^[+-]?(?:\\d+(?:\\.\\d*)?|\\.\\d+)([gmk]?b(ases)?)?$/o && 
        ($$_val = $_t, return $T);
    &warn('NOT_A_DNABASE',$F,$_var,$_t);
    return $F;
\};
\nwendcode{}\nwbegindocs{176}\nwdocspar

\nwenddocs{}\nwbegincode{177}\sublabel{NWgff18-warV-8}\nwmargintag{{\nwtagstyle{}\subpageref{NWgff18-warV-8}}}\moddef{warnings - parsing custom files~{\nwtagstyle{}\subpageref{NWgff18-warV-1}}}\plusendmoddef
NOT_A_DNABASE =>
  $Warn."\\"\\%s\\" variable requires nucleotide units.\\n".
  $spw." \\"\\%s\\" is not valid (units must be in Gb, Mb, Kb, or bases).\\n",
\nwendcode{}\nwbegindocs{178}\nwdocspar

\subsubsctn{Accepting numeric ranges}

A range is a lower/upper limits pair for a given variable. One of the limits, or both, can be an asterisk ('*'), which means that the default value must be used.

\todo{
\item Here we must check something like '*..*'
} % todo
\nwenddocs{}\nwbegincode{179}\sublabel{NWgff18-varN-7}\nwmargintag{{\nwtagstyle{}\subpageref{NWgff18-varN-7}}}\moddef{variable value checking~{\nwtagstyle{}\subpageref{NWgff18-varN-1}}}\plusendmoddef
$_test eq 'RANGE' && do \{

    return $T;
\};
\nwendcode{}\nwbegindocs{180}\nwdocspar

\subsubsctn{Checking for available fonts and font-sizes}

Font names must be in first capital letter format for {\ps}. 

\nwenddocs{}\nwbegincode{181}\sublabel{NWgff18-varN-8}\nwmargintag{{\nwtagstyle{}\subpageref{NWgff18-varN-8}}}\moddef{variable value checking~{\nwtagstyle{}\subpageref{NWgff18-varN-1}}}\plusendmoddef
$_test eq 'FONT' && do \{ # lc
    defined($Fonts\{$_t\}) && 
        ($$_val = $Fonts\{$_t\}, return $T);
    &warn('NOT_A_FONT',$F,$_t,$_var);
    return $F;
\};
\nwendcode{}\nwbegindocs{182}\nwdocspar

\nwenddocs{}\nwbegincode{183}\sublabel{NWgff18-warV-9}\nwmargintag{{\nwtagstyle{}\subpageref{NWgff18-warV-9}}}\moddef{warnings - parsing custom files~{\nwtagstyle{}\subpageref{NWgff18-warV-1}}}\plusendmoddef
NOT_A_FONT =>
  $Warn."Sorry, \\"\\%s\\" font is not defined for \\"\\%s\\" variable.\\n",
\nwendcode{}\nwbegindocs{184}\nwdocspar

We provide three basic fonts that are standard in most {\ps} devices (a serif, a sans-serif and a monospaced font), and four series for each one (an upright ``normal'', an italic/oblique, a bold, and a bold italic/oblique series). 

\nwenddocs{}\nwbegincode{185}\sublabel{NWgff18-GloB-6}\nwmargintag{{\nwtagstyle{}\subpageref{NWgff18-GloB-6}}}\moddef{Global Vars~{\nwtagstyle{}\subpageref{NWgff18-GloB-1}}}\plusendmoddef
my %Fonts = (
    # serif
    times                   => 'Times-Roman',
    'times-roman'           => 'Times-Roman',
    'times-italic'          => 'Times-Italic',
    'times-bold'            => 'Times-Bold',
    'times-bolditalic'      => 'Times-BoldItalic',
    # sans serif
    helvetica               => 'Helvetica',
    'helvetica-oblique'     => 'Helvetica-Oblique',
    'helvetica-bold'        => 'Helvetica-Bold',
    'helvetica-boldoblique' => 'Helvetica-BoldOblique',
    # monospaced
    courier                 => 'Courier',
    'courier-oblique'       => 'Courier-Oblique',
    'courier-bold'          => 'Courier-Bold',
    'courier-boldoblique'   => 'Courier-BoldOblique',
    );
\nwendcode{}\nwbegindocs{186}\nwdocspar

\nwenddocs{}\nwbegincode{187}\sublabel{NWgff18-warV-A}\nwmargintag{{\nwtagstyle{}\subpageref{NWgff18-warV-A}}}\moddef{warnings - parsing custom files~{\nwtagstyle{}\subpageref{NWgff18-warV-1}}}\plusendmoddef
NOT_A_FONT =>
  $Warn."\\"\\%s\\" font is not available for \\"\\%s\\" variable.\\n",
\nwendcode{}\nwbegindocs{188}\nwdocspar

Font sizes can be given in points, centimeters or inches. If no units are provided, then we asume points, but always must be possitive values.

\nwenddocs{}\nwbegincode{189}\sublabel{NWgff18-varN-9}\nwmargintag{{\nwtagstyle{}\subpageref{NWgff18-varN-9}}}\moddef{variable value checking~{\nwtagstyle{}\subpageref{NWgff18-varN-1}}}\plusendmoddef
$_test eq 'FONT_SZ' && do \{
    $_t =~ /^(?:\\d+(?:\\.\\d*)?|\\.\\d+)
            (pt|cm|mm|in(ches)?|points|(centi|mili)meters)?$/ox && 
        ($$_val = $_t, return $T);
    &warn('NOT_A_FONTSIZE',$F,$_t,$_var);
    return $F;
\};
\nwendcode{}\nwbegindocs{190}\nwdocspar

\todo{
\item Assign the proper PS unit multiplier. Functions to convert to points/cm/inches, and a hash variable to reformat the values: {\tt{}{\char37}units} ({\tt{}{\char36}units{\char123}pt{\char125}\ =\ "";\ {\char36}units{\char123}cm{\char125}\ =\ "\ cm";\ {\char36}units{\char123}mm{\char125}\ =\ "\ mm";\ {\char36}units{\char123}in{\char125}\ =\ "\ in"}).
\item Can we do the same for DNABASE ? (section~\ref{sec:DNABASE})
} % todo

\nwenddocs{}\nwbegincode{191}\sublabel{NWgff18-warV-B}\nwmargintag{{\nwtagstyle{}\subpageref{NWgff18-warV-B}}}\moddef{warnings - parsing custom files~{\nwtagstyle{}\subpageref{NWgff18-warV-1}}}\plusendmoddef
NOT_A_FONTSIZE =>
  $Warn."\\"\\%s\\" is not a valid font-size for \\"\\%s\\" variable.",
\nwendcode{}\nwbegindocs{192}\nwdocspar


\newpage

\sctn{Customization Variables Description} %%%%%%%%%%%%%%%%%%%%%%%%%%%%%%%%%%%%%
    \label{sec:customvardesc}
% \subsctn{Command-line options}

\todo{
\item Write a script to produce easier the appendix/descriptions tables for the manual from the options code chunks. Transforming to a fixed record formats like:\\
\shortstack[l]{OPTION: ....\\DEFAULTS: ....\\PARAMS: ....\\ SHORT DESC: .... (for the appendix table)\\ LONG DESC: .... (for the description table)}

\item Set the final options/variables/parameters for the whole section.
} % todo


Two options are used for reporting how program is running. By default no messages are shown to standard error, only errors. One reports to a log file defined by user and the other shows the messages to standard error, both can be activated at the same time. A third one is needed to disable STDERR warnings.

\nwenddocs{}\nwbegincode{193}\sublabel{NWgff18-comK-1}\nwmargintag{{\nwtagstyle{}\subpageref{NWgff18-comK-1}}}\moddef{command-line options~{\nwtagstyle{}\subpageref{NWgff18-comK-1}}}\endmoddef
"v|verbose"   => \\$Verbose, # Print_Report
\nwalsodefined{\\{NWgff18-comK-2}\\{NWgff18-comK-3}\\{NWgff18-comK-4}\\{NWgff18-comK-5}\\{NWgff18-comK-6}\\{NWgff18-comK-7}\\{NWgff18-comK-8}\\{NWgff18-comK-9}\\{NWgff18-comK-A}\\{NWgff18-comK-B}\\{NWgff18-comK-C}\\{NWgff18-comK-D}\\{NWgff18-comK-E}\\{NWgff18-comK-F}\\{NWgff18-comK-G}\\{NWgff18-comK-H}\\{NWgff18-comK-I}\\{NWgff18-comK-J}\\{NWgff18-comK-K}\\{NWgff18-comK-L}\\{NWgff18-comK-M}\\{NWgff18-comK-N}\\{NWgff18-comK-O}\\{NWgff18-comK-P}\\{NWgff18-comK-Q}\\{NWgff18-comK-R}\\{NWgff18-comK-S}\\{NWgff18-comK-T}\\{NWgff18-comK-U}\\{NWgff18-comK-V}\\{NWgff18-comK-W}\\{NWgff18-comK-X}}\nwused{\\{NWgff18-ParS-1}}\nwendcode{}\nwbegindocs{194}\nwdocspar
\nwenddocs{}\nwbegincode{195}\sublabel{NWgff18-comH-3}\nwmargintag{{\nwtagstyle{}\subpageref{NWgff18-comH-3}}}\moddef{command-line help~{\nwtagstyle{}\subpageref{NWgff18-comH-1}}}\plusendmoddef
-v, --verbose    Verbose mode, a full report is sent to standard error 
                 (default is set to showing only WARNINGS).
\nwendcode{}\nwbegindocs{196}\nwdocspar
\nwenddocs{}%
%
%
\nwbegindocs{198}\nwdocspar
\nwenddocs{}%
%
%
%
\nwbegindocs{200}\nwdocspar


\nwenddocs{}\nwbegincode{201}\sublabel{NWgff18-comK-2}\nwmargintag{{\nwtagstyle{}\subpageref{NWgff18-comK-2}}}\moddef{command-line options~{\nwtagstyle{}\subpageref{NWgff18-comK-1}}}\plusendmoddef
"V|logs-filename=s"  => \\$logs_filename, # Print_Report -> LogFile
\nwendcode{}\nwbegindocs{202}\nwdocspar
\nwenddocs{}\nwbegincode{203}\sublabel{NWgff18-comH-4}\nwmargintag{{\nwtagstyle{}\subpageref{NWgff18-comH-4}}}\moddef{command-line help~{\nwtagstyle{}\subpageref{NWgff18-comH-1}}}\plusendmoddef
-V, --logs-filename <logs_filename>    Report is written to a log file.
\nwendcode{}\nwbegindocs{204}\nwdocspar
\nwenddocs{}%
%
%
\nwbegindocs{206}\nwdocspar
\nwenddocs{}%
%
%
%
%
\nwbegindocs{208}\nwdocspar

\nwenddocs{}\nwbegincode{209}\sublabel{NWgff18-comK-3}\nwmargintag{{\nwtagstyle{}\subpageref{NWgff18-comK-3}}}\moddef{command-line options~{\nwtagstyle{}\subpageref{NWgff18-comK-1}}}\plusendmoddef
"q|quiet"   => \\$Quiet, # Quiet_Mode
\LA{}command-line options custom~{\nwtagstyle{}\subpageref{NWgff18-comR-1}}\RA{}
\nwendcode{}\nwbegindocs{210}\nwdocspar
\nwenddocs{}\nwbegincode{211}\sublabel{NWgff18-comH-5}\nwmargintag{{\nwtagstyle{}\subpageref{NWgff18-comH-5}}}\moddef{command-line help~{\nwtagstyle{}\subpageref{NWgff18-comH-1}}}\plusendmoddef
-q, --quiet    Quiet mode, do not show any message/warning
               to standard error (only ERRORS are reported).
\LA{}command-line help custom~{\nwtagstyle{}\subpageref{NWgff18-comO-1}}\RA{}
\nwendcode{}\nwbegindocs{212}\nwdocspar
\nwenddocs{}%
%
%
%
\nwbegindocs{214}\nwdocspar
\nwenddocs{}%
%
%
%
%
%
%
\nwbegindocs{216}\nwdocspar

The following variables define the boundary/clipping coordinates for each sequence, so you can select a region to be displayed in the {\ps} output.

\nwenddocs{}\nwbegincode{217}\sublabel{NWgff18-comK-4}\nwmargintag{{\nwtagstyle{}\subpageref{NWgff18-comK-4}}}\moddef{command-line options~{\nwtagstyle{}\subpageref{NWgff18-comK-1}}}\plusendmoddef
"S|sequence1-start=i"  => \\$CmdLineVars\{LAYOUT\}\{"sequence1_start"\}, # SEQUENCE1_ORIGIN # Zoom_SEQUENCE1_ORIGIN
\nwendcode{}\nwbegindocs{218}\nwdocspar
\nwenddocs{}\nwbegincode{219}\sublabel{NWgff18-comH-6}\nwmargintag{{\nwtagstyle{}\subpageref{NWgff18-comH-6}}}\moddef{command-line help~{\nwtagstyle{}\subpageref{NWgff18-comH-1}}}\plusendmoddef
-S, --sequence1-start <pos>   Sets X-sequence first nucleotide.
\nwendcode{}\nwbegindocs{220}\nwdocspar
\nwenddocs{}%
\nwbegindocs{222}\nwdocspar
\nwenddocs{}%
%
\nwbegindocs{224}\nwdocspar

\nwenddocs{}\nwbegincode{225}\sublabel{NWgff18-comK-5}\nwmargintag{{\nwtagstyle{}\subpageref{NWgff18-comK-5}}}\moddef{command-line options~{\nwtagstyle{}\subpageref{NWgff18-comK-1}}}\plusendmoddef
"E|sequence1_end=i"  => \\$CmdLineVars\{LAYOUT\}\{"sequence1_end"\}, # SEQUENCE1_END    # Zoom_SEQUENCE1_END
\nwendcode{}\nwbegindocs{226}\nwdocspar
\nwenddocs{}\nwbegincode{227}\sublabel{NWgff18-comH-7}\nwmargintag{{\nwtagstyle{}\subpageref{NWgff18-comH-7}}}\moddef{command-line help~{\nwtagstyle{}\subpageref{NWgff18-comH-1}}}\plusendmoddef
-E, --sequence1-end <pos>     Sets X-sequence last nucleotide.
\nwendcode{}\nwbegindocs{228}\nwdocspar
\nwenddocs{}%
\nwbegindocs{230}\nwdocspar
\nwenddocs{}%
%
\nwbegindocs{232}\nwdocspar

\nwenddocs{}\nwbegincode{233}\sublabel{NWgff18-comK-6}\nwmargintag{{\nwtagstyle{}\subpageref{NWgff18-comK-6}}}\moddef{command-line options~{\nwtagstyle{}\subpageref{NWgff18-comK-1}}}\plusendmoddef
"s|sequence2_start=i"  => \\$CmdLineVars\{LAYOUT\}\{"sequence2_start"\}, # SEQUENCE2_ORIGIN # Zoom_SEQUENCE2_ORIGIN
\nwendcode{}\nwbegindocs{234}\nwdocspar
\nwenddocs{}\nwbegincode{235}\sublabel{NWgff18-comH-8}\nwmargintag{{\nwtagstyle{}\subpageref{NWgff18-comH-8}}}\moddef{command-line help~{\nwtagstyle{}\subpageref{NWgff18-comH-1}}}\plusendmoddef
-s, --sequence2-start <pos>   Sets Y-sequence first nucleotide.
\nwendcode{}\nwbegindocs{236}\nwdocspar
\nwenddocs{}%
\nwbegindocs{238}\nwdocspar
\nwenddocs{}%
%
\nwbegindocs{240}\nwdocspar

\nwenddocs{}\nwbegincode{241}\sublabel{NWgff18-comK-7}\nwmargintag{{\nwtagstyle{}\subpageref{NWgff18-comK-7}}}\moddef{command-line options~{\nwtagstyle{}\subpageref{NWgff18-comK-1}}}\plusendmoddef
"e|sequence2_end=i"  => \\$CmdLineVars\{LAYOUT\}\{"sequence2_end"\}, # SEQUENCE2_END    # Zoom_SEQUENCE2_END
\nwendcode{}\nwbegindocs{242}\nwdocspar
\nwenddocs{}\nwbegincode{243}\sublabel{NWgff18-comH-9}\nwmargintag{{\nwtagstyle{}\subpageref{NWgff18-comH-9}}}\moddef{command-line help~{\nwtagstyle{}\subpageref{NWgff18-comH-1}}}\plusendmoddef
-e, --sequence2-end <pos>     Sets Y-sequence last nucleotide.
\nwendcode{}\nwbegindocs{244}\nwdocspar
\nwenddocs{}%
\nwbegindocs{246}\nwdocspar
\nwenddocs{}%
%
\nwbegindocs{248}\nwdocspar

\nwenddocs{}\nwbegincode{249}\sublabel{NWgff18-comK-8}\nwmargintag{{\nwtagstyle{}\subpageref{NWgff18-comK-8}}}\moddef{command-line options~{\nwtagstyle{}\subpageref{NWgff18-comK-1}}}\plusendmoddef
"Z|zoom"   => \\$CmdLineVars\{LAYOUT\}\{"zoom"\}, # ZOOM_Zoom; Zoom_Marks
\nwendcode{}\nwbegindocs{250}\nwdocspar
\nwenddocs{}\nwbegincode{251}\sublabel{NWgff18-comH-A}\nwmargintag{{\nwtagstyle{}\subpageref{NWgff18-comH-A}}}\moddef{command-line help~{\nwtagstyle{}\subpageref{NWgff18-comH-1}}}\plusendmoddef
-Z, --zoom [ [-S <pos>] [-E <pos>] [-s <pos>] [-e <pos>] ]
               This option zooms an area you have selected
               with -S,-E,-s,-e (all 4 are optional).
\nwendcode{}\nwbegindocs{252}\nwdocspar
\nwenddocs{}%
\nwbegindocs{254}\nwdocspar
\nwenddocs{}%
%
\nwbegindocs{256}\nwdocspar

\nwenddocs{}\nwbegincode{257}\sublabel{NWgff18-comK-9}\nwmargintag{{\nwtagstyle{}\subpageref{NWgff18-comK-9}}}\moddef{command-line options~{\nwtagstyle{}\subpageref{NWgff18-comK-1}}}\plusendmoddef
"z|zoom-area"   => \\$CmdLineVars\{LAYOUT\}\{"zoom_area"\}, # ZOOM_Area
\nwendcode{}\nwbegindocs{258}\nwdocspar
\nwenddocs{}\nwbegincode{259}\sublabel{NWgff18-comH-B}\nwmargintag{{\nwtagstyle{}\subpageref{NWgff18-comH-B}}}\moddef{command-line help~{\nwtagstyle{}\subpageref{NWgff18-comH-1}}}\plusendmoddef
-z, --zoom-area [ [-S <pos>] [-E <pos>] [-s <pos>] [-e <pos>] ]
               This option marks a zoom area on your plot,
               but does not make a zoom.
\nwendcode{}\nwbegindocs{260}\nwdocspar
\nwenddocs{}%
\nwbegindocs{262}\nwdocspar
\nwenddocs{}%
%
\nwbegindocs{264}\nwdocspar

\nwenddocs{}\nwbegincode{265}\sublabel{NWgff18-comK-A}\nwmargintag{{\nwtagstyle{}\subpageref{NWgff18-comK-A}}}\moddef{command-line options~{\nwtagstyle{}\subpageref{NWgff18-comK-1}}}\plusendmoddef
"G|display-grid"   => \\$CmdLineVars\{LAYOUT\}\{"display_grid"\}, # Display_GRID
# "g"    => \\$CmdLineVars\{LAYOUT\}\{""\}, # Display_GRID
\nwendcode{}\nwbegindocs{266}\nwdocspar
\nwenddocs{}\nwbegincode{267}\sublabel{NWgff18-comH-C}\nwmargintag{{\nwtagstyle{}\subpageref{NWgff18-comH-C}}}\moddef{command-line help~{\nwtagstyle{}\subpageref{NWgff18-comH-1}}}\plusendmoddef
-G, --display-grid   Switches 'on' grid (default is 'off').
\nwendcode{}\nwbegindocs{268}\nwdocspar
\nwenddocs{}%
\nwbegindocs{270}\nwdocspar
\nwenddocs{}%
%
\nwbegindocs{272}\nwdocspar

\nwenddocs{}\nwbegincode{273}\sublabel{NWgff18-comK-B}\nwmargintag{{\nwtagstyle{}\subpageref{NWgff18-comK-B}}}\moddef{command-line options~{\nwtagstyle{}\subpageref{NWgff18-comK-1}}}\plusendmoddef
"P|display-percent-box"   => \\$CmdLineVars\{LAYOUT\}\{"display_percent_box"\}, # Display_PERCENT-BOX
# "p"    => \\$CmdLineVars\{LAYOUT\}\{""\}, # Display_PERCENT-BOX
\nwendcode{}\nwbegindocs{274}\nwdocspar
\nwenddocs{}\nwbegincode{275}\sublabel{NWgff18-comH-D}\nwmargintag{{\nwtagstyle{}\subpageref{NWgff18-comH-D}}}\moddef{command-line help~{\nwtagstyle{}\subpageref{NWgff18-comH-1}}}\plusendmoddef
-P, --display-percent-box   Switches 'on' Percent box (default is 'off').
\nwendcode{}\nwbegindocs{276}\nwdocspar
\nwenddocs{}%
\nwbegindocs{278}\nwdocspar
\nwenddocs{}%
%
\nwbegindocs{280}\nwdocspar

\nwenddocs{}\nwbegincode{281}\sublabel{NWgff18-comK-C}\nwmargintag{{\nwtagstyle{}\subpageref{NWgff18-comK-C}}}\moddef{command-line options~{\nwtagstyle{}\subpageref{NWgff18-comK-1}}}\plusendmoddef
"Q|display-extra-box"   => \\$CmdLineVars\{LAYOUT\}\{"display_extra_box"\}, # Display_EXTRA-BOX
# "q"    => \\$CmdLineVars\{LAYOUT\}\{""\}, # Display_EXTRA-BOX
\nwendcode{}\nwbegindocs{282}\nwdocspar
\nwenddocs{}\nwbegincode{283}\sublabel{NWgff18-comH-E}\nwmargintag{{\nwtagstyle{}\subpageref{NWgff18-comH-E}}}\moddef{command-line help~{\nwtagstyle{}\subpageref{NWgff18-comH-1}}}\plusendmoddef
-Q, --display-extra-box   Switches 'on' Extra box (default is 'off').
\nwendcode{}\nwbegindocs{284}\nwdocspar
\nwenddocs{}%
\nwbegindocs{286}\nwdocspar
\nwenddocs{}%
%
\nwbegindocs{288}\nwdocspar

\nwenddocs{}\nwbegincode{289}\sublabel{NWgff18-comK-D}\nwmargintag{{\nwtagstyle{}\subpageref{NWgff18-comK-D}}}\moddef{command-line options~{\nwtagstyle{}\subpageref{NWgff18-comK-1}}}\plusendmoddef
"T|title=s"  => \\$CmdLineVars\{LAYOUT\}\{"title"\}, # TITLE
\nwendcode{}\nwbegindocs{290}\nwdocspar
\nwenddocs{}\nwbegincode{291}\sublabel{NWgff18-comH-F}\nwmargintag{{\nwtagstyle{}\subpageref{NWgff18-comH-F}}}\moddef{command-line help~{\nwtagstyle{}\subpageref{NWgff18-comH-1}}}\plusendmoddef
-T, --title <Title>   Definning Plot Title.
\nwendcode{}\nwbegindocs{292}\nwdocspar
\nwenddocs{}%
\nwbegindocs{294}\nwdocspar
\nwenddocs{}%
%
\nwbegindocs{296}\nwdocspar

\nwenddocs{}\nwbegincode{297}\sublabel{NWgff18-comK-E}\nwmargintag{{\nwtagstyle{}\subpageref{NWgff18-comK-E}}}\moddef{command-line options~{\nwtagstyle{}\subpageref{NWgff18-comK-1}}}\plusendmoddef
"t|subtitle=s"  => \\$CmdLineVars\{LAYOUT\}\{"subtitle"\}, # SUBTITLE
\nwendcode{}\nwbegindocs{298}\nwdocspar
\nwenddocs{}\nwbegincode{299}\sublabel{NWgff18-comH-G}\nwmargintag{{\nwtagstyle{}\subpageref{NWgff18-comH-G}}}\moddef{command-line help~{\nwtagstyle{}\subpageref{NWgff18-comH-1}}}\plusendmoddef
-t, --subtitle <Subtitle>   Definning Plot SubTitle.
\nwendcode{}\nwbegindocs{300}\nwdocspar
\nwenddocs{}%
\nwbegindocs{302}\nwdocspar
\nwenddocs{}%
%
\nwbegindocs{304}\nwdocspar

\nwenddocs{}\nwbegincode{305}\sublabel{NWgff18-comK-F}\nwmargintag{{\nwtagstyle{}\subpageref{NWgff18-comK-F}}}\moddef{command-line options~{\nwtagstyle{}\subpageref{NWgff18-comK-1}}}\plusendmoddef
"X|x-label=s"  => \\$CmdLineVars\{LAYOUT\}\{"x-axis_label"\}, # X-Axis_LABEL
\nwendcode{}\nwbegindocs{306}\nwdocspar
\nwenddocs{}\nwbegincode{307}\sublabel{NWgff18-comH-H}\nwmargintag{{\nwtagstyle{}\subpageref{NWgff18-comH-H}}}\moddef{command-line help~{\nwtagstyle{}\subpageref{NWgff18-comH-1}}}\plusendmoddef
-X, --x-label <X-Label>   Definning X-Axis Label.
\nwendcode{}\nwbegindocs{308}\nwdocspar
\nwenddocs{}%
\nwbegindocs{310}\nwdocspar
\nwenddocs{}%
%
\nwbegindocs{312}\nwdocspar

\nwenddocs{}\nwbegincode{313}\sublabel{NWgff18-comK-G}\nwmargintag{{\nwtagstyle{}\subpageref{NWgff18-comK-G}}}\moddef{command-line options~{\nwtagstyle{}\subpageref{NWgff18-comK-1}}}\plusendmoddef
"Y|y-label=s"  => \\$CmdLineVars\{LAYOUT\}\{"y-axis_label"\}, # Y-Axis_LABEL
\nwendcode{}\nwbegindocs{314}\nwdocspar
\nwenddocs{}\nwbegincode{315}\sublabel{NWgff18-comH-I}\nwmargintag{{\nwtagstyle{}\subpageref{NWgff18-comH-I}}}\moddef{command-line help~{\nwtagstyle{}\subpageref{NWgff18-comH-1}}}\plusendmoddef
-Y, --y-label <Y-Label>   Defining Y-Axis Label.
\nwendcode{}\nwbegindocs{316}\nwdocspar
\nwenddocs{}%
\nwbegindocs{318}\nwdocspar
\nwenddocs{}%
%
\nwbegindocs{320}\nwdocspar


\nwenddocs{}\nwbegincode{321}\sublabel{NWgff18-comK-H}\nwmargintag{{\nwtagstyle{}\subpageref{NWgff18-comK-H}}}\moddef{command-line options~{\nwtagstyle{}\subpageref{NWgff18-comK-1}}}\plusendmoddef
"L|percent-box-label=s"  => \\$CmdLineVars\{LAYOUT\}\{"percent_box_label"\}, # Percent-Box_LABEL
\nwendcode{}\nwbegindocs{322}\nwdocspar
\nwenddocs{}\nwbegincode{323}\sublabel{NWgff18-comH-J}\nwmargintag{{\nwtagstyle{}\subpageref{NWgff18-comH-J}}}\moddef{command-line help~{\nwtagstyle{}\subpageref{NWgff18-comH-1}}}\plusendmoddef
-L, --percent-box-label <PBox-Label>   Definning Percent-Box Label.
\nwendcode{}\nwbegindocs{324}\nwdocspar
\nwenddocs{}%
\nwbegindocs{326}\nwdocspar
\nwenddocs{}%
%
\nwbegindocs{328}\nwdocspar

\nwenddocs{}\nwbegincode{329}\sublabel{NWgff18-comK-I}\nwmargintag{{\nwtagstyle{}\subpageref{NWgff18-comK-I}}}\moddef{command-line options~{\nwtagstyle{}\subpageref{NWgff18-comK-1}}}\plusendmoddef
"l|extra-box-label=s"  => \\$CmdLineVars\{LAYOUT\}\{"extra_box_label"\}, # Extra-Box_LABEL
\nwendcode{}\nwbegindocs{330}\nwdocspar
\nwenddocs{}\nwbegincode{331}\sublabel{NWgff18-comH-K}\nwmargintag{{\nwtagstyle{}\subpageref{NWgff18-comH-K}}}\moddef{command-line help~{\nwtagstyle{}\subpageref{NWgff18-comH-1}}}\plusendmoddef
-l, --extra-box-label <XBox-Label>   Definning Extra-Box Label.
\nwendcode{}\nwbegindocs{332}\nwdocspar
\nwenddocs{}%
\nwbegindocs{334}\nwdocspar
\nwenddocs{}%
%
\nwbegindocs{336}\nwdocspar

\nwenddocs{}\nwbegincode{337}\sublabel{NWgff18-comK-J}\nwmargintag{{\nwtagstyle{}\subpageref{NWgff18-comK-J}}}\moddef{command-line options~{\nwtagstyle{}\subpageref{NWgff18-comK-1}}}\plusendmoddef
"R|xy-axes-scale"   => \\$CmdLineVars\{LAYOUT\}\{"xy_axes_scale"\}, # XY_AXES_Same-SIZE
# "r"    => \\$CmdLineVars\{LAYOUT\}\{""\}, # XY_AXES_Same-SIZE
\nwendcode{}\nwbegindocs{338}\nwdocspar
\nwenddocs{}\nwbegincode{339}\sublabel{NWgff18-comH-L}\nwmargintag{{\nwtagstyle{}\subpageref{NWgff18-comH-L}}}\moddef{command-line help~{\nwtagstyle{}\subpageref{NWgff18-comH-1}}}\plusendmoddef
-R, --xy-axes-scale   X and Y axes having same scale (default is 'off').
\nwendcode{}\nwbegindocs{340}\nwdocspar
\nwenddocs{}%
\nwbegindocs{342}\nwdocspar
\nwenddocs{}%
%
\nwbegindocs{344}\nwdocspar

\nwenddocs{}\nwbegincode{345}\sublabel{NWgff18-comK-K}\nwmargintag{{\nwtagstyle{}\subpageref{NWgff18-comK-K}}}\moddef{command-line options~{\nwtagstyle{}\subpageref{NWgff18-comK-1}}}\plusendmoddef
"W|aln-scale-width"    => \\$CmdLineVars\{LAYOUT\}\{"alignment_scale_width"\}, # APlotLine_ScaleWidth; APlotLine_GroupScore
\nwendcode{}\nwbegindocs{346}\nwdocspar
\nwenddocs{}\nwbegincode{347}\sublabel{NWgff18-comH-M}\nwmargintag{{\nwtagstyle{}\subpageref{NWgff18-comH-M}}}\moddef{command-line help~{\nwtagstyle{}\subpageref{NWgff18-comH-1}}}\plusendmoddef
-W, --aln-scale-width   Scaling score on width for Aplot line.
\nwendcode{}\nwbegindocs{348}\nwdocspar
\nwenddocs{}%
\nwbegindocs{350}\nwdocspar
\nwenddocs{}%
%
\nwbegindocs{352}\nwdocspar

\nwenddocs{}\nwbegincode{353}\sublabel{NWgff18-comK-L}\nwmargintag{{\nwtagstyle{}\subpageref{NWgff18-comK-L}}}\moddef{command-line options~{\nwtagstyle{}\subpageref{NWgff18-comK-1}}}\plusendmoddef
"w|aln-scale-color"    => \\$CmdLineVars\{LAYOUT\}\{"alignment_scale_color"\}, # APlotLine_ScaleGrey; APlotLine_GroupScore
\nwendcode{}\nwbegindocs{354}\nwdocspar
\nwenddocs{}\nwbegincode{355}\sublabel{NWgff18-comH-N}\nwmargintag{{\nwtagstyle{}\subpageref{NWgff18-comH-N}}}\moddef{command-line help~{\nwtagstyle{}\subpageref{NWgff18-comH-1}}}\plusendmoddef
-w, --aln-scale-color   Scaling score on color for Aplot line.
\nwendcode{}\nwbegindocs{356}\nwdocspar
\nwenddocs{}%
\nwbegindocs{358}\nwdocspar
\nwenddocs{}%
%
\nwbegindocs{360}\nwdocspar

\nwenddocs{}\nwbegincode{361}\sublabel{NWgff18-comK-M}\nwmargintag{{\nwtagstyle{}\subpageref{NWgff18-comK-M}}}\moddef{command-line options~{\nwtagstyle{}\subpageref{NWgff18-comK-1}}}\plusendmoddef
"B|background-color=s"  => \\$CmdLineVars\{LAYOUT\}\{"background_color"\}, # BACKGROUND_COLOR
\nwendcode{}\nwbegindocs{362}\nwdocspar
\nwenddocs{}\nwbegincode{363}\sublabel{NWgff18-comH-O}\nwmargintag{{\nwtagstyle{}\subpageref{NWgff18-comH-O}}}\moddef{command-line help~{\nwtagstyle{}\subpageref{NWgff18-comH-1}}}\plusendmoddef
-B, --background-color <color>   Background color.
\nwendcode{}\nwbegindocs{364}\nwdocspar
\nwenddocs{}%
\nwbegindocs{366}\nwdocspar
\nwenddocs{}%
%
\nwbegindocs{368}\nwdocspar

\nwenddocs{}\nwbegincode{369}\sublabel{NWgff18-comK-N}\nwmargintag{{\nwtagstyle{}\subpageref{NWgff18-comK-N}}}\moddef{command-line options~{\nwtagstyle{}\subpageref{NWgff18-comK-1}}}\plusendmoddef
"F|foreground-color=s"  => \\$CmdLineVars\{LAYOUT\}\{"foreground_color"\}, # FOREGROUND_COLOR
\nwendcode{}\nwbegindocs{370}\nwdocspar
\nwenddocs{}\nwbegincode{371}\sublabel{NWgff18-comH-P}\nwmargintag{{\nwtagstyle{}\subpageref{NWgff18-comH-P}}}\moddef{command-line help~{\nwtagstyle{}\subpageref{NWgff18-comH-1}}}\plusendmoddef
-F, --foreground-color <color>   Foreground color.
\nwendcode{}\nwbegindocs{372}\nwdocspar
\nwenddocs{}%
\nwbegindocs{374}\nwdocspar
\nwenddocs{}%
%
\nwbegindocs{376}\nwdocspar

\nwenddocs{}\nwbegincode{377}\sublabel{NWgff18-comK-O}\nwmargintag{{\nwtagstyle{}\subpageref{NWgff18-comK-O}}}\moddef{command-line options~{\nwtagstyle{}\subpageref{NWgff18-comK-1}}}\plusendmoddef
"D|aplot-box-color=s"  => \\$CmdLineVars\{LAYOUT\}\{"aplot_box_color"\}, # APlotBox_BqGCOLOR
\nwendcode{}\nwbegindocs{378}\nwdocspar
\nwenddocs{}\nwbegincode{379}\sublabel{NWgff18-comH-Q}\nwmargintag{{\nwtagstyle{}\subpageref{NWgff18-comH-Q}}}\moddef{command-line help~{\nwtagstyle{}\subpageref{NWgff18-comH-1}}}\plusendmoddef
-D, --aplot-box-color <color>   Aplot main box background color.
\nwendcode{}\nwbegindocs{380}\nwdocspar
\nwenddocs{}%
\nwbegindocs{382}\nwdocspar
\nwenddocs{}%
%
\nwbegindocs{384}\nwdocspar

\nwenddocs{}\nwbegincode{385}\sublabel{NWgff18-comK-P}\nwmargintag{{\nwtagstyle{}\subpageref{NWgff18-comK-P}}}\moddef{command-line options~{\nwtagstyle{}\subpageref{NWgff18-comK-1}}}\plusendmoddef
"C|percent-box-color=s"  => \\$CmdLineVars\{LAYOUT\}\{"percent_box_color"\}, # PercentBox_BGCOLOR
\nwendcode{}\nwbegindocs{386}\nwdocspar
\nwenddocs{}\nwbegincode{387}\sublabel{NWgff18-comH-R}\nwmargintag{{\nwtagstyle{}\subpageref{NWgff18-comH-R}}}\moddef{command-line help~{\nwtagstyle{}\subpageref{NWgff18-comH-1}}}\plusendmoddef
-C, --percent-box-color <color>   Percent box background color.
\nwendcode{}\nwbegindocs{388}\nwdocspar
\nwenddocs{}%
\nwbegindocs{390}\nwdocspar
\nwenddocs{}%
%
\nwbegindocs{392}\nwdocspar

\nwenddocs{}\nwbegincode{393}\sublabel{NWgff18-comK-Q}\nwmargintag{{\nwtagstyle{}\subpageref{NWgff18-comK-Q}}}\moddef{command-line options~{\nwtagstyle{}\subpageref{NWgff18-comK-1}}}\plusendmoddef
"c|extra-box-color=s"  => \\$CmdLineVars\{LAYOUT\}\{"extra_box_color"\}, # ExtraBox_BGCOLOR
\nwendcode{}\nwbegindocs{394}\nwdocspar
\nwenddocs{}\nwbegincode{395}\sublabel{NWgff18-comH-S}\nwmargintag{{\nwtagstyle{}\subpageref{NWgff18-comH-S}}}\moddef{command-line help~{\nwtagstyle{}\subpageref{NWgff18-comH-1}}}\plusendmoddef
-c, --extra-box-color <color>   Extra box background color.
\nwendcode{}\nwbegindocs{396}\nwdocspar
\nwenddocs{}%
\nwbegindocs{398}\nwdocspar
\nwenddocs{}%
%
\nwbegindocs{400}\nwdocspar

\nwenddocs{}\nwbegincode{401}\sublabel{NWgff18-comK-R}\nwmargintag{{\nwtagstyle{}\subpageref{NWgff18-comK-R}}}\moddef{command-line options~{\nwtagstyle{}\subpageref{NWgff18-comK-1}}}\plusendmoddef
"A|alignment-name=s"  => \\$CmdLineVars\{LAYOUT\}\{"alignment_name"\}, # Align_NAME
\nwendcode{}\nwbegindocs{402}\nwdocspar
\nwenddocs{}\nwbegincode{403}\sublabel{NWgff18-comH-T}\nwmargintag{{\nwtagstyle{}\subpageref{NWgff18-comH-T}}}\moddef{command-line help~{\nwtagstyle{}\subpageref{NWgff18-comH-1}}}\plusendmoddef
-A, --alignment-name <SeqXName:SeqYName>   Defining which alignment is going to be plotted 
                  if you have more than one alignment in your gff files.
\nwendcode{}\nwbegindocs{404}\nwdocspar
\nwenddocs{}%
\nwbegindocs{406}\nwdocspar
\nwenddocs{}%
%
\nwbegindocs{408}\nwdocspar

\nwenddocs{}\nwbegincode{409}\sublabel{NWgff18-comK-S}\nwmargintag{{\nwtagstyle{}\subpageref{NWgff18-comK-S}}}\moddef{command-line options~{\nwtagstyle{}\subpageref{NWgff18-comK-1}}}\plusendmoddef
"N|x-sequence-name=s"  => \\$CmdLineVars\{LAYOUT\}\{"x-sequence_name"\}, # X-Sequence_NAME
\nwendcode{}\nwbegindocs{410}\nwdocspar
\nwenddocs{}\nwbegincode{411}\sublabel{NWgff18-comH-U}\nwmargintag{{\nwtagstyle{}\subpageref{NWgff18-comH-U}}}\moddef{command-line help~{\nwtagstyle{}\subpageref{NWgff18-comH-1}}}\plusendmoddef
-N, --x-sequence-name <SeqXName>   Defining which sequence is going to be plotted at X-axes.
\nwendcode{}\nwbegindocs{412}\nwdocspar
\nwenddocs{}%
\nwbegindocs{414}\nwdocspar
\nwenddocs{}%
%
\nwbegindocs{416}\nwdocspar

\nwenddocs{}\nwbegincode{417}\sublabel{NWgff18-comK-T}\nwmargintag{{\nwtagstyle{}\subpageref{NWgff18-comK-T}}}\moddef{command-line options~{\nwtagstyle{}\subpageref{NWgff18-comK-1}}}\plusendmoddef
"n|y-sequence-name=s"  => \\$CmdLineVars\{LAYOUT\}\{"y-sequence_name"\}, # Y-Sequence_NAME
\nwendcode{}\nwbegindocs{418}\nwdocspar
\nwenddocs{}\nwbegincode{419}\sublabel{NWgff18-comH-V}\nwmargintag{{\nwtagstyle{}\subpageref{NWgff18-comH-V}}}\moddef{command-line help~{\nwtagstyle{}\subpageref{NWgff18-comH-1}}}\plusendmoddef
-n, --y-sequence-name <SeqYName>   Defining which sequence is going to be plotted at Y-axes.
\nwendcode{}\nwbegindocs{420}\nwdocspar
\nwenddocs{}%
\nwbegindocs{422}\nwdocspar
\nwenddocs{}%
%
\nwbegindocs{424}\nwdocspar

\nwenddocs{}\nwbegincode{425}\sublabel{NWgff18-comK-U}\nwmargintag{{\nwtagstyle{}\subpageref{NWgff18-comK-U}}}\moddef{command-line options~{\nwtagstyle{}\subpageref{NWgff18-comK-1}}}\plusendmoddef
"K|show-ribbons=s"   => \\$CmdLineVars\{LAYOUT\}\{"show_ribbons"\}, # Show_Ribbons (NLRB)
\nwendcode{}\nwbegindocs{426}\nwdocspar
\nwenddocs{}\nwbegincode{427}\sublabel{NWgff18-comH-W}\nwmargintag{{\nwtagstyle{}\subpageref{NWgff18-comH-W}}}\moddef{command-line help~{\nwtagstyle{}\subpageref{NWgff18-comH-1}}}\plusendmoddef
-K, --show-ribbons <ribbon_type> Force Ribbons for all features on axes:
                 (N)one, (L)ines, (R)ibbons, (B)oth.
\nwendcode{}\nwbegindocs{428}\nwdocspar
\nwenddocs{}%
\nwbegindocs{430}\nwdocspar
\nwenddocs{}%
%
\nwbegindocs{432}\nwdocspar

\nwenddocs{}\nwbegincode{433}\sublabel{NWgff18-comK-V}\nwmargintag{{\nwtagstyle{}\subpageref{NWgff18-comK-V}}}\moddef{command-line options~{\nwtagstyle{}\subpageref{NWgff18-comK-1}}}\plusendmoddef
"I|page-size=s"  => \\$CmdLineVars\{LAYOUT\}\{"page_size"\}, # PAGE_SIZE
\nwendcode{}\nwbegindocs{434}\nwdocspar
\nwenddocs{}\nwbegincode{435}\sublabel{NWgff18-comH-X}\nwmargintag{{\nwtagstyle{}\subpageref{NWgff18-comH-X}}}\moddef{command-line help~{\nwtagstyle{}\subpageref{NWgff18-comH-1}}}\plusendmoddef
-I, --page-size <page_size> Set page size for plot (default is a4).
\nwendcode{}\nwbegindocs{436}\nwdocspar
\nwenddocs{}%
\nwbegindocs{438}\nwdocspar
\nwenddocs{}%
%
\nwbegindocs{440}\nwdocspar

\nwenddocs{}\nwbegincode{441}\sublabel{NWgff18-comK-W}\nwmargintag{{\nwtagstyle{}\subpageref{NWgff18-comK-W}}}\moddef{command-line options~{\nwtagstyle{}\subpageref{NWgff18-comK-1}}}\plusendmoddef
"a|show-credits"   => \\$CmdLineVars\{LAYOUT\}\{"show_credits"\}, # Show_Credits
\nwendcode{}\nwbegindocs{442}\nwdocspar
\nwenddocs{}\nwbegincode{443}\sublabel{NWgff18-comH-Y}\nwmargintag{{\nwtagstyle{}\subpageref{NWgff18-comH-Y}}}\moddef{command-line help~{\nwtagstyle{}\subpageref{NWgff18-comH-1}}}\plusendmoddef
-a, --show-credits  Switch off $PROGRAM CopyRight line on plot.
\nwendcode{}\nwbegindocs{444}\nwdocspar
\nwenddocs{}%
\nwbegindocs{446}\nwdocspar
\nwenddocs{}%
%
\nwbegindocs{448}\nwdocspar

\nwenddocs{}\nwbegincode{449}\sublabel{NWgff18-ChaV-1}\nwmargintag{{\nwtagstyle{}\subpageref{NWgff18-ChaV-1}}}\moddef{Changes in command-line options~{\nwtagstyle{}\subpageref{NWgff18-ChaV-1}}}\endmoddef
# "i"  => # Page_Orientation
# "F"  => \\$CmdLineVars\{LAYOUT\}\{""\}, # Function_COLOR;Display_FUNCTION;Display_EXTRA-BOX
# "f"  => \\$CmdLineVars\{LAYOUT\}\{""\}, # Display_FUNCTION; Display_EXTRA-BOX
# "M"  => \\$CmdLineVars\{LAYOUT\}\{""\}, # Display_GFF; Display_EXTRA-BOX
# "m"  => \\$CmdLineVars\{LAYOUT\}\{""\}, # Display_GFF-ReverseOrder; Display_EXTRA-BOX
\nwnotused{Changes\ in\ command-line\ options}\nwendcode{}\nwbegindocs{450}\nwdocspar

\nwenddocs{}\nwbegincode{451}\sublabel{NWgff18-defQ-1}\nwmargintag{{\nwtagstyle{}\subpageref{NWgff18-defQ-1}}}\moddef{default layout vars values~{\nwtagstyle{}\subpageref{NWgff18-defQ-1}}}\endmoddef
#                         
page_size                  => \{ TYPE => 'PAGE'   , VALUE => 'a4'   \},
background_color           => \{ TYPE => 'COLOR'  , VALUE => 'white' \},
foreground_color           => \{ TYPE => 'COLOR'  , VALUE => 'black' \},
# GLOBAL Labels           
show_title                 => \{ TYPE => 'BOOLEAN', VALUE => $T     \},
title                      => \{ TYPE => 'STRING' , VALUE => undef  \},
show_subtitle              => \{ TYPE => 'BOOLEAN', VALUE => $T     \},
subtitle                   => \{ TYPE => 'STRING' , VALUE => undef  \},
show_x_sequence_label      => \{ TYPE => 'BOOLEAN', VALUE => $T     \},
x_sequence_label           => \{ TYPE => 'STRING' , VALUE => undef  \},
show_y_sequence_label      => \{ TYPE => 'BOOLEAN', VALUE => $T     \},
x_sequence_label           => \{ TYPE => 'STRING' , VALUE => undef  \},
# TICKMARK features                                              
show_tickmark_label        => \{ TYPE => 'BOOLEAN', VALUE => $T     \},
show_aplot_x_ticks         => \{ TYPE => 'BOOLEAN', VALUE => $T     \},
show_percent_x_ticks       => \{ TYPE => 'BOOLEAN', VALUE => $T     \},
show_extrabox_x_ticks      => \{ TYPE => 'BOOLEAN', VALUE => $T     \},
show_aplot_y_ticks         => \{ TYPE => 'BOOLEAN', VALUE => $T     \},
show_percent_y_ticks       => \{ TYPE => 'BOOLEAN', VALUE => $T     \},
show_extrabox_y_ticks      => \{ TYPE => 'BOOLEAN', VALUE => $T     \},
show_onlylower_x_ticks     => \{ TYPE => 'BOOLEAN', VALUE => $F     \},
aplot_major_tickmark       => \{ TYPE => 'INTEGER', VALUE => 2      \},
aplot_minor_tickmark       => \{ TYPE => 'INTEGER', VALUE => 5      \},
percent_major_tickmark     => \{ TYPE => 'INTEGER', VALUE => 5      \},
percent_minor_tickmark     => \{ TYPE => 'INTEGER', VALUE => 5      \},
extra_major_tickmark       => \{ TYPE => 'INTEGER', VALUE => 2      \},
extra_minor_tickmark       => \{ TYPE => 'INTEGER', VALUE => 5      \},
major_tickmark_nucleotide  => \{ TYPE => 'DNABASE', VALUE => undef  \},
minor_tickmark_nucleotide  => \{ TYPE => 'DNABASE', VALUE => undef  \},
percent_box_score_range    => \{ TYPE => 'RANGE'  , VALUE => '50..100' \},
show_grid                  => \{ TYPE => 'BOOLEAN', VALUE => $F     \},
# APLOT box features      
aplot_axes_same_scale      => \{ TYPE => 'BOOLEAN', VALUE => $F     \},
aplot_box_bgcolor          => \{ TYPE => 'COLOR'  , VALUE => 'bg'   \},
#                         
sequence1_start            => \{ TYPE => 'DNABASE', VALUE => undef  \},
sequence1_end              => \{ TYPE => 'DNABASE', VALUE => undef  \},
sequence2_start            => \{ TYPE => 'DNABASE', VALUE => undef  \},
sequence2_end              => \{ TYPE => 'DNABASE', VALUE => undef  \},
#                         
sequence1_zoom_start       => \{ TYPE => 'DNABASE', VALUE => undef  \},
sequence1_zoom_end         => \{ TYPE => 'DNABASE', VALUE => undef  \},
sequence2_zoom_start       => \{ TYPE => 'DNABASE', VALUE => undef  \},
sequence2_zoom_end         => \{ TYPE => 'DNABASE', VALUE => undef  \},
zoom                       => \{ TYPE => 'BOOLEAN', VALUE => $F     \},
zoom_area                  => \{ TYPE => 'BOOLEAN', VALUE => $F     \},
zoom_marks                 => \{ TYPE => 'BOOLEAN', VALUE => $F     \},
zoom_area_mark_color       => \{ TYPE => 'COLOR'  , VALUE => 'lightred' \},
# PERCENT box features    
show_percent_box           => \{ TYPE => 'BOOLEAN', VALUE => $F     \},
percent_box_bgcolor        => \{ TYPE => 'COLOR'  , VALUE => 'bg'   \},
show_percent_box_label     => \{ TYPE => 'BOOLEAN', VALUE => $T     \},
percent_box_label          => \{ TYPE => 'STRING' , VALUE => undef  \},
percent_box_sublabel       => \{ TYPE => 'STRING' , VALUE => undef  \},
# EXTRA box features      
show_extra_box             => \{ TYPE => 'BOOLEAN', VALUE => $F     \},
extra_box_bgcolor          => \{ TYPE => 'COLOR'  , VALUE => 'bg'   \},
show_extra_box_label       => \{ TYPE => 'BOOLEAN', VALUE => $T     \},
extra_box_label            => \{ TYPE => 'STRING' , VALUE => undef  \},
extra_box_sublabel         => \{ TYPE => 'STRING' , VALUE => undef  \},
#                         
feature_x_label_angle      => \{ TYPE => 'INTEGER', VALUE => 0      \},
feature_y_label_angle      => \{ TYPE => 'INTEGER', VALUE => 0      \},
feature_label_length       => \{ TYPE => 'INTEGER', VALUE => 0      \},
feature_labels_font        => \{ TYPE => 'FONT'   , VALUE => 'helvetica' \},
feature_labels_fontsize    => \{ TYPE => 'FONT_SZ', VALUE => '5pt'  \},
group_x_label_angle        => \{ TYPE => 'INTEGER', VALUE => 0      \},
group_y_label_angle        => \{ TYPE => 'INTEGER', VALUE => 0      \},
group_label_length         => \{ TYPE => 'INTEGER', VALUE => 0      \},
group_labels_font          => \{ TYPE => 'FONT'   , VALUE => 'helvetica' \},
group_labels_fontsize      => \{ TYPE => 'FONT_SZ', VALUE => '5pt'  \},
\nwalsodefined{\\{NWgff18-defQ-2}\\{NWgff18-defQ-3}\\{NWgff18-defQ-4}}\nwused{\\{NWgff18-SetG-1}}\nwendcode{}\nwbegindocs{452}\nwdocspar

\nwenddocs{}\nwbegincode{453}\sublabel{NWgff18-defS-1}\nwmargintag{{\nwtagstyle{}\subpageref{NWgff18-defS-1}}}\moddef{default sequence vars values~{\nwtagstyle{}\subpageref{NWgff18-defS-1}}}\endmoddef
NIL => \{ TYPE => 'BOOLEAN', VALUE => $F \},
\nwused{\\{NWgff18-SetG-1}}\nwendcode{}\nwbegindocs{454}\nwdocspar

\nwenddocs{}\nwbegincode{455}\sublabel{NWgff18-defQ.2-1}\nwmargintag{{\nwtagstyle{}\subpageref{NWgff18-defQ.2-1}}}\moddef{default source vars values~{\nwtagstyle{}\subpageref{NWgff18-defQ.2-1}}}\endmoddef
NIL => \{ TYPE => 'BOOLEAN', VALUE => $F \},
\nwused{\\{NWgff18-SetG-1}}\nwendcode{}\nwbegindocs{456}\nwdocspar

\nwenddocs{}\nwbegincode{457}\sublabel{NWgff18-defQ.3-1}\nwmargintag{{\nwtagstyle{}\subpageref{NWgff18-defQ.3-1}}}\moddef{default strand vars values~{\nwtagstyle{}\subpageref{NWgff18-defQ.3-1}}}\endmoddef
NIL => \{ TYPE => 'BOOLEAN', VALUE => $F \},
\nwused{\\{NWgff18-SetG-1}}\nwendcode{}\nwbegindocs{458}\nwdocspar

\nwenddocs{}\nwbegincode{459}\sublabel{NWgff18-defP-1}\nwmargintag{{\nwtagstyle{}\subpageref{NWgff18-defP-1}}}\moddef{default group vars values~{\nwtagstyle{}\subpageref{NWgff18-defP-1}}}\endmoddef
show_group_label           => \{ TYPE => 'BOOLEAN', VALUE => $T     \}, 
show_group_rule            => \{ TYPE => 'BOOLEAN', VALUE => $T     \},
show_group_arrow           => \{ TYPE => 'BOOLEAN', VALUE => $T     \},
feature_arrows_color       => \{ TYPE => 'COLOR'  , VALUE => 'fg'   \},
Show_JOINS                 => \{ TYPE => 'BOOLEAN', VALUE => $T     \},
Join_Lines_COLOR           => \{ TYPE => 'COLOR'  , VALUE => 'fg'   \},
\nwused{\\{NWgff18-SetG-1}}\nwendcode{}\nwbegindocs{460}\nwdocspar

\nwenddocs{}\nwbegincode{461}\sublabel{NWgff18-defR-1}\nwmargintag{{\nwtagstyle{}\subpageref{NWgff18-defR-1}}}\moddef{default feature vars values~{\nwtagstyle{}\subpageref{NWgff18-defR-1}}}\endmoddef
Show_HalfHeightBOX         => \{ TYPE => 'BOOLEAN', VALUE => $T     \},
HalfSizeBox_BGCOLOR        => \{ TYPE => 'COLOR'  , VALUE => 'DEFAULT' \},
Show_FullHeightBOX         => \{ TYPE => 'BOOLEAN', VALUE => $T     \},
FullSizeBox_BGCOLOR        => \{ TYPE => 'COLOR'  , VALUE => 'DEFAULT' \},
Show_BOX_LABEL             => \{ TYPE => 'BOOLEAN', VALUE => $T     \},
Show_UserDef_BOX_LABEL     => \{ TYPE => 'BOOLEAN', VALUE => $T     \},
Show_RIBBON                => \{ TYPE => 'BOOLEAN', VALUE => $T     \},
Ribbon_BGCOLOR             => \{ TYPE => 'COLOR'  , VALUE => 'DEFAULT' \},
Show_GFF                   => \{ TYPE => 'BOOLEAN', VALUE => $F     \},
Show_GFF_ReverseOrder      => \{ TYPE => 'BOOLEAN', VALUE => $F     \},
Show_FUNCTION              => \{ TYPE => 'BOOLEAN', VALUE => $F     \},
APlotLine_GroupScore       => \{ TYPE => 'BOOLEAN', VALUE => $F     \},
APlotLine_ScaleWidth       => \{ TYPE => 'BOOLEAN', VALUE => $F     \},
APlotLine_ScaleGrey        => \{ TYPE => 'BOOLEAN', VALUE => $F     \},
Show_SELECTION_BOX         => \{ TYPE => 'BOOLEAN', VALUE => $T     \},
SelectionBox_BGCOLOR       => \{ TYPE => 'COLOR'  , VALUE => 'grey' \},
Function_COLOR             => \{ TYPE => 'COLOR'  , VALUE => 'red'  \},
\nwused{\\{NWgff18-SetG-1}}\nwendcode{}\nwbegindocs{462}\nwdocspar

\begin{comment}
## ## 
## XTR (bin): Force those <features> for which XTR is switched ON
##   to be displayed in the third panel, if the Plot Option 
##   "Display_EXTRA-BOX" is switched ON.
## ## 
## Box (FHN): Define the box size (FHN) for <feature>.
## ##
## Line (NLRB): Draw lines across all panels for the <feature>.
## ##
## Join (bin): Switch ON to connect elements of one group with splice symbols.
## ##
## Arrw (bin): Swicth ON to display an arrow that shows the <feature>'s strand.
## ##
## Lbl (bin): Swicth ON to display the <feature>'s label
##   (if it's defined in the .gff file).
## ##
## Color (clr): Fill <Box> with this color.
## ##
## CLine (clr): Fill Line <Ribbons> with this color.
## ##
## Wdth (HND): You can choose among three predefined LineWidths.
## ##
## Layr : Layer where is drawed each feature.
##   You must pass an integer for feature ordering (1..n),
##   put 0 to print the feature always on top. 
\end{comment}

\nwenddocs{}\nwbegincode{463}\sublabel{NWgff18-ChaM-1}\nwmargintag{{\nwtagstyle{}\subpageref{NWgff18-ChaM-1}}}\moddef{Changes in custom Vars~{\nwtagstyle{}\subpageref{NWgff18-ChaM-1}}}\endmoddef
# PERCENT_ORIGIN => "50",
# PERCENT_END => "100",
Align_NAME       \\
X_Sequence_NAME   >--> To be defined in SOURCE
Y_Sequence_NAME  /
### FL ## FLAGs ##
### DF ## DEFAULT VALUES ##
### OP ## OPTIONs ##
### FT ## GFF FEATUREs DEFINITION ##
  # FTcounter++,
  # FTindex[""]="",
  # FTputEXTRA[""]=OnOff(""),
  # FTBoxType[""]="",
  # FTLineType[""]="",
  # FTputJoin[""]=OnOff(""),
  # FTputArrow[""]=OnOff(""),
  # FTputLabel[""]=OnOff(""),
  # FTLineWidth[""]="",
  # FTLayer[""]="",
  # FTColor[""]=Var["BACKGROUND_COLOR"],
  # FTRbnColor[""]=Var["BACKGROUND_COLOR"],
### SP ## Special gff Features ##
  # FTGeneBounds="genebounds", NOT NEEDED NOW !!!
  FTLimits="seqbounds",
  FTAPlot="align",
  FTHighLight="boxit",
  FTRectangle="rectangle",
  FTCircle="circle",
  FTText="text",
  FTExtraFunction="function",
  FTExtraDisplay="est",
\nwnotused{Changes\ in\ custom\ Vars}\nwendcode{}\nwbegindocs{464}\nwdocspar
 
\nwenddocs{}%
\nwbegindocs{466}\nwdocspar
\nwenddocs{}%
%
%
%
\nwbegindocs{468}\nwdocspar


\newpage

\sctn{Reading GFF records} %%%%%%%%%%%%%%%%%%%%%%%%%%%%%%%%%%%%%%%%%%%%%%%%%%%%%

We have to deal with two basic formats, the \textsl{standard} General Feature Format (GFF) and the derived \textsl{aplot} format, the former having two versions with some minor differences at field level. We are going to asume the record formats shown in the table~\ref{fig:formats} for the input files.

\todo{
\item A flow chart of the GFF record parsing process.\\[6cm]
} % todo

\begin{table}[!t]
% \newcommand{\m}[1]{\multicolumn{2}{@{}c@{}}{#1}}
\begin{center}\setlength{\fboxsep}{10pt}
\fbox{
\begin{minipage}[c]{0.95\linewidth}
\begin{center}
\footnotesize
\textbf{\normalsize Record Format}\\[1ex]
\pa{seqname} \pa{source} \pa{feature} \pa{start} \pa{end} \pa{score} \pa{strand} \pa{frame} [attributes] [comments]\\[2ex]
\renewcommand{\arraystretch}{1.5}\begin{tabular}{ccc}
\textbf{Field Name} & \textbf{GFF Version 1/Version 2} & \textbf{Aplot format} \\
\pa{seqname} & {\tt{}\ [{\char94}{\char92}{\char35}\ ]+\ }     & \pa{seqname1}:\pa{seqname2} \\
\pa{source}  & {\tt{}\ [{\char94}{\char92}{\char35}\ ]+\ }     & \pa{source}[:\pa{source}]   \\
\pa{feature} & {\tt{}\ [{\char94}{\char92}{\char35}\ ]+\ }     & \pa{feature}[:\pa{feature}] \\
\pa{start}   & {\tt{}\ [+-]?[0-9]+\ } & \pa{start1}:\pa{start2}     \\
\pa{end}     & {\tt{}\ [+-]?[0-9]+\ } & \pa{end1}:\pa{end2}         \\
\pa{score}   & {\tt{}\ [+-]?[0-9]*[.]?[0-9]*(e[+-]?[0-9]+)?\ } & \pa{alignment\_score}       \\[-1.5ex]
  (no score) & GFFv1 {\tt{}=>\ 0} / GFFv2 {\tt{}=>\ .} & . \\
\pa{strand}  & {\tt{}\ [+-.]\ }  & \pa{strand1}[:\pa{strand2}] \\
\pa{frame}   & {\tt{}\ [.012]\ } & \pa{frame1}[:\pa{frame2}]   \\[0.5ex]
\textbf{Field Separator} & GFFv1 {\tt{}=>\ {\char92}s+\ } / GFFv2 {\tt{}=>\ [{\char92}t]+\ } & {\tt{}\ {\char92}s+\ } \\
\end{tabular}\\[4ex]
\textbf{\normalsize Grouping fields}\\[1ex]
\begin{tabular}{rl}
 \,[attributes]  & \hspace{0.25cm}(\textsl{GFF v1\&v2, APLOT})\\
                 & group\_name [extra\_data]\\
                 & "group\_name" [extra\_data]\\
                 & \pa{tag} "group\_name" [extra\_data]\\[1ex]
 \,[extra\_data] & \\
                 & [\,[\pa{start2} \pa{end2} [\pa{strand2}] [\pa{frame2}]\,]; \pa{tag} \pa{value} [ \ldots ; \pa{tag} \pa{value}]\,]\\[2ex]
 \,[attributes]  & \hspace{0.25cm}(\textsl{APLOT old format})\\
                 & group\_name:element\_name [extra\_data]\\[1ex]
 \,[extra\_data] & \hspace{0.25cm}(\textsl{APLOT old format})\\
                 & [\,; \pa{tag} \pa{value} [ \ldots [ ; \pa{tag} \pa{value}]\,]\,]\\
\end{tabular}\\[4ex]
\textbf{\normalsize Other definitions}\\[1ex]
\begin{tabular}{rl}
 \,[comments]    & {\tt{}{\char35}\ Whatever\ extra\ information\ about\ this\ record...} \\
\end{tabular}
\end{center}
\end{minipage}
} %fbox
\caption{\label{tbl:formats} Input records format definition for {\prog}.}
\end{center}
\end{table}


Function '{\tt{}{\char38}parse{\char95}GFF{\char95}files}' is only a wrapper for reporting the parsing process over GFF records. The main function call is '{\tt{}{\char38}fieldscheck}',which processes each GFF record and loads the values in its fields to the corresponding data structure. Although that, the wrapper checks if there are enough fields (8 required at least) and removes also any comment from the input record, assuming that a white space or a tab followed by a '{\tt{}{\char35}}' (as shown in this regular expression: '{\tt{}\ /{\char92}s+{\char92}{\char35}/\ }') is a comment mark, from there to the end of line everything is ignored.

\nwenddocs{}\nwbegincode{469}\sublabel{NWgff18-ParI-1}\nwmargintag{{\nwtagstyle{}\subpageref{NWgff18-ParI-1}}}\moddef{Parsing Input Data~{\nwtagstyle{}\subpageref{NWgff18-ParI-1}}}\endmoddef
sub parse_GFF_files() \{
    &header("PARSING INPUT GFF RECORDS");
  LOAD: foreach $file (@data_files) \{
      open(THIS,"< $file") ||
          (&warn('FILE_NO_OPEN',$T,$file), next LOAD);
      $file eq '-' && ($file = 'STANDARD INPUT');
      &report('READ_GFF_FILE',$file);
      ($n,$c) = (0,undef);
      while (<THIS>) \{
          my (@line,$main);
          ($c = '.', next) if /^\\#/o;
          ($c = '.', next) if /^\\s*$/o;
          chomp;
          # $c = $noGFF;
          ($main,undef) = split /\\s+\\#/o;
          @line = split /\\s+/o, $main, 9;        
          scalar(@line) < 8 &&
              (&warn('NOT_ENOUGH_FIELDS',$F,$file,$n,join(" ",@line)), next);
          $c = &GFF_format(&fieldscheck(\\@line));
      \} continue \{
          &counter(++$n,$c) if ($Verbose && !$Quiet);
      \}; # WHILE
      &counter_end($n,$c) if ($Verbose && !$Quiet);
      close(THIS);
  \}; # LOAD
    print LOGFILE '>>> \\%GFF_DATA : '.(Dumper(\\%GFF_DATA))
        if ($LogFile && $Debug);
    &footer("DATA LOADED");
\} # sub parse_GFF
\nwalsodefined{\\{NWgff18-ParI-2}\\{NWgff18-ParI-3}\\{NWgff18-ParI-4}\\{NWgff18-ParI-5}\\{NWgff18-ParI-6}\\{NWgff18-ParI-7}\\{NWgff18-ParI-8}\\{NWgff18-ParI-9}\\{NWgff18-ParI-A}\\{NWgff18-ParI-B}\\{NWgff18-ParI-C}\\{NWgff18-ParI-D}\\{NWgff18-ParI-E}\\{NWgff18-ParI-F}\\{NWgff18-ParI-G}\\{NWgff18-ParI-H}\\{NWgff18-ParI-I}\\{NWgff18-ParI-J}}\nwused{\\{NWgff18-MaiD-1}}\nwendcode{}\nwbegindocs{470}%$

We define the following constants, used to report which format was found in the current read record (See table~\ref{tbl:DTkeysXreport}).

\begin{table}[!t]
\begin{center}
\begin{tabular}{cc}
\fbox{
\begin{tabular}[b]{c>{\bfseries}c}
\textbf{Record TYPE} & \textbf{Symbol} \\
Empty record        & . \\
Comment             & . \\
GFF (grouped)       & X \\
GFF (ungrouped)     & x \\
GFF (alignment)     & A \\
GFF (vector)        & V \\
APLOT (grouped)     & O \\
APLOT (ungrouped)   & o \\
Unknown             & ? \\
\end{tabular}
} %fbox
&
\parbox[b]{0.4\linewidth}{
\caption[Symbols used to report record type from data files]{\label{tbl:DTkeysXreport} Symbols used to report record type when parsing input GFF files (and {\tt{}STDERR} verbose mode is enabled with '{\tt{}-v}' switch).}
} % parbox
\\
\end{tabular}
\end{center}
\end{table}

\nwenddocs{}\nwbegincode{471}\sublabel{NWgff18-GloG-4}\nwmargintag{{\nwtagstyle{}\subpageref{NWgff18-GloG-4}}}\moddef{Global Constants~{\nwtagstyle{}\subpageref{NWgff18-GloG-1}}}\plusendmoddef
my ($GFF,$GFF_NOGP,$VECTOR,$ALIGN,
    $APLOT,$APLOT_NOGP,$noGFF) =
    qw/ X x V A O o ? /;
\nwendcode{}\nwbegindocs{472}\nwdocspar

\nwenddocs{}\nwbegincode{473}\sublabel{NWgff18-GloB-7}\nwmargintag{{\nwtagstyle{}\subpageref{NWgff18-GloB-7}}}\moddef{Global Vars~{\nwtagstyle{}\subpageref{NWgff18-GloB-1}}}\plusendmoddef
my ($n,$c);
\nwendcode{}\nwbegindocs{474}\nwdocspar

\nwenddocs{}\nwbegincode{475}\sublabel{NWgff18-warS-1}\nwmargintag{{\nwtagstyle{}\subpageref{NWgff18-warS-1}}}\moddef{warnings - parsing GFF files~{\nwtagstyle{}\subpageref{NWgff18-warS-1}}}\endmoddef
NOT_ENOUGH_FIELDS =>
  $Warn."Not enough fields in file \\"\\%s\\", line \\%s :\\n\\t\\%s\\n",
\nwalsodefined{\\{NWgff18-warS-2}\\{NWgff18-warS-3}\\{NWgff18-warS-4}}\nwused{\\{NWgff18-GloB-G}}\nwendcode{}\nwbegindocs{476}\nwdocspar

\nwenddocs{}\nwbegincode{477}\sublabel{NWgff18-mesS-1}\nwmargintag{{\nwtagstyle{}\subpageref{NWgff18-mesS-1}}}\moddef{messages - parsing GFF files~{\nwtagstyle{}\subpageref{NWgff18-mesS-1}}}\endmoddef
READ_GFF_FILE => 
  $sp."### Reading GFF records from \\"\\%s\\"\\n".$sp,
\nwused{\\{NWgff18-GloB-G}}\nwendcode{}\nwbegindocs{478}\nwdocspar

We set the character being printed in the parsing status output.

\nwenddocs{}\nwbegincode{479}\sublabel{NWgff18-ParI-2}\nwmargintag{{\nwtagstyle{}\subpageref{NWgff18-ParI-2}}}\moddef{Parsing Input Data~{\nwtagstyle{}\subpageref{NWgff18-ParI-1}}}\plusendmoddef
sub GFF_format() \{
    my $gff = $_[0];
    # return "x" if $GFF == $version1;
    return $GFF        if $gff eq $GFF;        # $version2
    return $GFF_NOGP   if $gff eq $GFF_NOGP;   # $version2 (ungrouped)
    return $VECTOR     if $gff eq $VECTOR;     # VECTOR: GFFv2 particular case 
    return $ALIGN      if $gff eq $ALIGN;      # ALIGN: GFFv2 particular case
    return $APLOT      if $gff eq $APLOT;      # Old aplot format (with colons)
    return $APLOT_NOGP if $gff eq $APLOT_NOGP; # Old aplot format (ungrouped)
    return $noGFF;
\} # GFF_format
\nwendcode{}\nwbegindocs{480}\nwdocspar

Here we decide to parse records as they are in GFF or in APLOT format. '{\tt{}load{\char95}gff}' and '{\tt{}load{\char95}aplot}' do the field error checking and load the program variables '{\tt{}{\char37}GFF{\char95}DATA}' and '{\tt{}{\char37}ALN{\char95}DATA}'. Those variables are reset in '{\tt{}\LA{}Main Loop~{\nwtagstyle{}\subpageref{NWgff18-Mai9-1}}\RA{}}'.

\nwenddocs{}\nwbegincode{481}\sublabel{NWgff18-ParI-3}\nwmargintag{{\nwtagstyle{}\subpageref{NWgff18-ParI-3}}}\moddef{Parsing Input Data~{\nwtagstyle{}\subpageref{NWgff18-ParI-1}}}\plusendmoddef
sub fieldscheck() \{
    my ($list) = @_;
    my ($seqname,$start,$end) = @$list[0,3,4]; # ($list->[0],$list->[3],$list->[4]);
    (&fcolon($seqname) && &fcolon($start) && &fcolon($end)) && do \{
        return &load_aplot(\\@$list);
    \};
    return &load_gff(\\@$list);
\} # fieldscheck
\nwendcode{}\nwbegindocs{482}\nwdocspar

For historical reasons the program can work with APLOT format (see Table~\ref{tbl:formats}). Initial field checking determines whether the record being processed is under that alignment GFF-like format, by determining that first, fourth and fifth fields contain a colon char. Then we assume we are under APLOT format.

\nwenddocs{}\nwbegincode{483}\sublabel{NWgff18-ParI-4}\nwmargintag{{\nwtagstyle{}\subpageref{NWgff18-ParI-4}}}\moddef{Parsing Input Data~{\nwtagstyle{}\subpageref{NWgff18-ParI-1}}}\plusendmoddef
sub fcolon() \{ return ($_[0] =~ /.+:.+/o ? $T : $F) \}
\nwendcode{}\nwbegindocs{484}\nwdocspar

\subsctn{Validating fields}

The following functions checks whether some of the input GFF fields are well defined. They return '{\tt{}True}' ('{\tt{}{\char36}T}') if no error found and '{\tt{}False}' ('{\tt{}{\char36}F}') when input fields are wrong. The returned value forces '{\tt{}{\char38}load{\char95}aplot}' or '{\tt{}{\char38}load{\char95}gff}' functions to skip the records having format errors.

Here we test that '\pa{start}' field must be lower or equal than '\pa{end}':

\label{func:checkcoords}
\nwenddocs{}\nwbegincode{485}\sublabel{NWgff18-ParI-5}\nwmargintag{{\nwtagstyle{}\subpageref{NWgff18-ParI-5}}}\moddef{Parsing Input Data~{\nwtagstyle{}\subpageref{NWgff18-ParI-1}}}\plusendmoddef
sub check_coords() \{ # ((ori,end)_1,...,(ori,end)_n)
    my @ary = @_;
    for (my $j=0; $j<=$#ary; $j+=2) \{
        $ary[$j] > $ary[$j+1] && do \{
            &warn('ORI_GREATER_END',$F,$ary[$j],$ary[$j+1],$file,$n+1);
            return $F;
        \};
    \}; # foreach
    return $T;
\} # check_coords
\nwendcode{}\nwbegindocs{486}\nwdocspar

\nwenddocs{}\nwbegincode{487}\sublabel{NWgff18-warS-2}\nwmargintag{{\nwtagstyle{}\subpageref{NWgff18-warS-2}}}\moddef{warnings - parsing GFF files~{\nwtagstyle{}\subpageref{NWgff18-warS-1}}}\plusendmoddef
ORI_GREATER_END =>
  $Warn."Start greater than end \\"\\%s > \\%s\\" in file \\"\\%s\\" line \\"\\%s\\".\\n", 
\nwendcode{}\nwbegindocs{488}\nwdocspar

'\pa{strand}' field must contain only '+', '-' or '.' (for records having no strand).

\nwenddocs{}\nwbegincode{489}\sublabel{NWgff18-ParI-6}\nwmargintag{{\nwtagstyle{}\subpageref{NWgff18-ParI-6}}}\moddef{Parsing Input Data~{\nwtagstyle{}\subpageref{NWgff18-ParI-1}}}\plusendmoddef
sub check_strand() \{ # (str_1,...,srt_n)
    foreach my $str (@_) \{
        $str !~ /[+-.]/o && do \{
            &warn('STRAND_MISMATCH',$F,$str,$file,$n+1);
            return $F;
        \};
    \}; # foreach
    return $T;
\} # check_strand
\nwendcode{}\nwbegindocs{490}\nwdocspar

\nwenddocs{}\nwbegincode{491}\sublabel{NWgff18-warS-3}\nwmargintag{{\nwtagstyle{}\subpageref{NWgff18-warS-3}}}\moddef{warnings - parsing GFF files~{\nwtagstyle{}\subpageref{NWgff18-warS-1}}}\plusendmoddef
STRAND_MISMATCH =>
  $Warn." Strand mismatch definition \\"\\%s\\" in file \\"\\%s\\" line \\"\\%s\\".\\n",
\nwendcode{}\nwbegindocs{492}\nwdocspar

Same happens to '\pa{frame}', which only '0', '1', '2' or '.' (for those records having no frame) values are allowed.

\nwenddocs{}\nwbegincode{493}\sublabel{NWgff18-ParI-7}\nwmargintag{{\nwtagstyle{}\subpageref{NWgff18-ParI-7}}}\moddef{Parsing Input Data~{\nwtagstyle{}\subpageref{NWgff18-ParI-1}}}\plusendmoddef
sub check_frame() \{ # (frm_1,...,frm_n)
    foreach my $frm (@_) \{
        $frm !~ /[.012]/o && do \{
            &warn('FRAME_MISMATCH',$F,$frm,$file,$n+1);
            return $F;
        \};
    \}; # foreach
    return $T;
\} # check_frame
\nwendcode{}\nwbegindocs{494}\nwdocspar

\nwenddocs{}\nwbegincode{495}\sublabel{NWgff18-warS-4}\nwmargintag{{\nwtagstyle{}\subpageref{NWgff18-warS-4}}}\moddef{warnings - parsing GFF files~{\nwtagstyle{}\subpageref{NWgff18-warS-1}}}\plusendmoddef
FRAME_MISMATCH =>
  $Warn." Frame mismatch definition \\"\\%s\\" in file \\"\\%s\\" line \\"\\%s\\".\\n",
\nwendcode{}\nwbegindocs{496}\nwdocspar

\subsctn{Parsing APLOT format}\label{sec:parseAPLOT}

'{\tt{}{\char38}check{\char95}aplot{\char95}fields}' will call to the '{\tt{}{\char38}add{\char95}aplot{\char95}record}' which loads the new record if everything is OK into '{\tt{}{\char37}ALN{\char95}DATA}' (see section~\ref{sec:APLOT-DS}). The first parameter for '{\tt{}{\char38}load{\char95}grouping}' function is set to 'False' is for APLOT grouping definition ('True' for GFF grouping definition, see section~\ref{sec:parseGFF}).

\nwenddocs{}\nwbegincode{497}\sublabel{NWgff18-ParI-8}\nwmargintag{{\nwtagstyle{}\subpageref{NWgff18-ParI-8}}}\moddef{Parsing Input Data~{\nwtagstyle{}\subpageref{NWgff18-ParI-1}}}\plusendmoddef
sub load_aplot() \{ # if errors found > return $noGFF
    my ($list) = @_;
    my $w_gff;
    ($seqname_1,$seqname_2,$source_1,$source_2,$feature_1,$feature_2,
     $start_1,$start_2,$end_1,$end_2,$score_1,$score_2,
     $strand_1,$strand_2,$frame_1,$frame_2) = 
     &remove_colon(@$list[0,1,2,3,4,5,6,7]);
    $w_gff = &load_grouping($F,$list->[8]);
    &check_aplot_fields || ($w_gff=$noGFF);
    return $w_gff;
\} # load_aplot
\nwendcode{}\nwbegindocs{498}%$

\nwenddocs{}\nwbegincode{499}\sublabel{NWgff18-GloB-8}\nwmargintag{{\nwtagstyle{}\subpageref{NWgff18-GloB-8}}}\moddef{Global Vars~{\nwtagstyle{}\subpageref{NWgff18-GloB-1}}}\plusendmoddef
my ($seqname_1,$seqname_2,
    $source_1,$source_2,$feature_1,$feature_2,
    $start_1,$start_2,$end_1,$end_2,$score_1,$score_2,
    $strand_1,$strand_2,$frame_1,$frame_2); # APLOT temporary vars
my ($tag,$group,$group_id,$label,
    $group_gff_counter,$group_aplot_counter); # GROUPING temporary vars
\nwendcode{}\nwbegindocs{500}\nwdocspar

\begin{table}[!t]
\begin{center}
\fbox{
\begin{tabular}{c@{\qquad}c}
Strand
\(
\quad\Rightarrow\quad
\left\{
\begin{array}{c@{\quad\Rightarrow\quad}c}
\mbox{{\tt{}+}} & \mbox{{\tt{}+:+}}\\
\mbox{{\tt{}-}} & \mbox{{\tt{}-:-}}\\
\mbox{{\tt{}.}} & \mbox{{\tt{}.:.}}\\
\end{array}
\right.
\)
 &
Frame
\(
\quad\Rightarrow\quad
\left\{
\begin{array}{c@{\quad\Rightarrow\quad}c}
\mbox{{\tt{}2}} & \mbox{{\tt{}2:2}}\\
\mbox{{\tt{}1}} & \mbox{{\tt{}1:1}}\\
\mbox{{\tt{}0}} & \mbox{{\tt{}0:0}}\\
\mbox{{\tt{}.}} & \mbox{{\tt{}.:.}}\\
\end{array}
\right.
\)
 \\
\end{tabular}
} % fbox
\parbox{0.75\linewidth}{
\caption[Retrieving second element when missing in old APLOT format]{\label{tbl:missingfields} For {\prog} old format, when in a '\pa{element1}:\pa{element2}' field pair the second element is missing, second is set to first element.}
} % parbox
\end{center}
\end{table}

When checking for colons, we assume that any element defined without colons is equal to the same value repeated twice (see Table~\ref{tbl:missingfields}). 

\nwenddocs{}\nwbegincode{501}\sublabel{NWgff18-ParI-9}\nwmargintag{{\nwtagstyle{}\subpageref{NWgff18-ParI-9}}}\moddef{Parsing Input Data~{\nwtagstyle{}\subpageref{NWgff18-ParI-1}}}\plusendmoddef
sub remove_colon() \{
    my @ary_out = ();
    my ($a,$b) = (undef,undef);
    foreach my $fld (@_) \{
        ($a,$b) = split /:/o, $fld, 2;
        $a = '.' unless defined($a);
        $b = $a  unless defined($b);
        push @ary_out, $a, $b;
    \};
    return @ary_out;
\} # remove_colon
\nwendcode{}\nwbegindocs{502}%$

\subsubsctn{Checking fields and accepting APLOT records}

If records are OK, then we can append a new record to the variable defined for APLOT records (so called later in this section as '{\tt{}{\char37}ALN{\char95}DATA}').

\nwenddocs{}\nwbegincode{503}\sublabel{NWgff18-ParI-A}\nwmargintag{{\nwtagstyle{}\subpageref{NWgff18-ParI-A}}}\moddef{Parsing Input Data~{\nwtagstyle{}\subpageref{NWgff18-ParI-1}}}\plusendmoddef
sub check_aplot_fields() \{
    &check_coords($start_1,$end_1,$start_2,$end_2) || (return $F);
    &check_strand($strand_1,$strand_2) || (return $F);
    &check_frame($frame_1,$frame_2) || (return $F);
    &add_aplot_record;
    return $T; 
\} # check_aplot_fields
\nwendcode{}\nwbegindocs{504}\nwdocspar

\todo{
\item Append a new record to the variable defined for APLOT records
} % todo
\nwenddocs{}\nwbegincode{505}\sublabel{NWgff18-ParI-B}\nwmargintag{{\nwtagstyle{}\subpageref{NWgff18-ParI-B}}}\moddef{Parsing Input Data~{\nwtagstyle{}\subpageref{NWgff18-ParI-1}}}\plusendmoddef
sub add_aplot_record() \{
    return;
\} # add_aplot_record
\nwendcode{}\nwbegindocs{506}%$

\label{sec:APLOT-DS}

\nwenddocs{}\nwbegincode{507}\sublabel{NWgff18-GloB-9}\nwmargintag{{\nwtagstyle{}\subpageref{NWgff18-GloB-9}}}\moddef{Global Vars~{\nwtagstyle{}\subpageref{NWgff18-GloB-1}}}\plusendmoddef
my (%ALN_DATA);
\nwendcode{}\nwbegindocs{508}\nwdocspar

\subsctn{Parsing standard GFF format}\label{sec:parseGFF}

'{\tt{}{\char38}check{\char95}gff{\char95}fields}' will call to the '{\tt{}{\char38}add{\char95}gff{\char95}record}' which loads the new record if everything is OK into '{\tt{}{\char37}GFF{\char95}DATA}' (see section~\ref{sec:GFF-DS}). The first parameter for '{\tt{}{\char38}load{\char95}grouping}' function is set to 'True' for GFF grouping definition ('False' is for APLOT grouping definition, see section~\ref{sec:parseAPLOT}). 

\nwenddocs{}\nwbegincode{509}\sublabel{NWgff18-ParI-C}\nwmargintag{{\nwtagstyle{}\subpageref{NWgff18-ParI-C}}}\moddef{Parsing Input Data~{\nwtagstyle{}\subpageref{NWgff18-ParI-1}}}\plusendmoddef
sub load_gff() \{ # if errors found > return $noGFF
    my ($list) = @_ ;
    my $w_gff;
    ($seqname,$source,$feature,$start,$end,
     $score,$strand,$frame) = @$list[0,1,2,3,4,5,6,7];
    $w_gff = &load_grouping($T,$list->[8]);
    &check_gff_fields($w_gff) || ($w_gff=$noGFF);
    return $w_gff;
\} # load_gff
\nwendcode{}\nwbegindocs{510}\nwdocspar

\nwenddocs{}\nwbegincode{511}\sublabel{NWgff18-GloB-A}\nwmargintag{{\nwtagstyle{}\subpageref{NWgff18-GloB-A}}}\moddef{Global Vars~{\nwtagstyle{}\subpageref{NWgff18-GloB-1}}}\plusendmoddef
my ($seqname,$source,$feature,$start,
    $end,$score,$strand,$frame); # GFF temporary vars
\nwendcode{}\nwbegindocs{512}\nwdocspar

\subsubsctn{Checking fields and accepting GFF records}

\nwenddocs{}\nwbegincode{513}\sublabel{NWgff18-ParI-D}\nwmargintag{{\nwtagstyle{}\subpageref{NWgff18-ParI-D}}}\moddef{Parsing Input Data~{\nwtagstyle{}\subpageref{NWgff18-ParI-1}}}\plusendmoddef
sub check_gff_fields() \{
    &check_coords($start,$end) || (return $F);
    &check_strand($strand) || (return $F);
    &check_frame($frame) || (return $F);
    &add_gff_record($_[0]);
    return $T; 
\} # check_gff_fields
\nwendcode{}\nwbegindocs{514}\nwdocspar

Once fields checking is done, we proceed to load the data structure defined for the GFF records as '{\tt{}{\char37}GFF{\char95}DATA}' hash.
The outline of the inner structure of that variable is shown in table~\ref{tbl:gffdata}.

\begin{table}[!ht]
\begin{center}
\input tables/DataStructure_GFF.tex
\caption[GFF internal data structure for {\prog}]{\label{tbl:gffdata} GFF internal data structure for {\prog}. The topmost hash corresponds to '{\tt{}{\char37}GFF{\char95}DATA}', the others are anonymous lists/hashes expanding from it.}
\end{center}
\end{table}

\todo{
\item Define DATA STRUCTURE for {\tt{}{\char37}ALN{\char95}DATA}.
} % todo

%%%%%%%%%%%%%%%%%%%%%%%%%%%%%%%%%%%%%%%%%%%%%%%%%%%%%%%%%%%%%%%%%
\nwenddocs{}%
%
%
%
%
%
%
%
%
%
%
%
%
%
%
%
%
%
%
%
%
%
%
%
%
%
%
%
%
%
%
%
%
%
%
%
%
%
%
%
%
%
%
%
%
%
%
%
%
%
%
%
%
%
%
%
%
%
%
%
%
%
%
%
%
%
%
%
%
%
\nwbegindocs{516}\nwdocspar
%%%%%%%%%%%%%%%%%%%%%%%%%%%%%%%%%%%%%%%%%%%%%%%%%%%%%%%%%%%%%%%%%%%

\nwenddocs{}\nwbegincode{517}\sublabel{NWgff18-ParI-E}\nwmargintag{{\nwtagstyle{}\subpageref{NWgff18-ParI-E}}}\moddef{Parsing Input Data~{\nwtagstyle{}\subpageref{NWgff18-ParI-1}}}\plusendmoddef
sub add_gff_record() \{
    my $_gff = $_[0];
    \LA{}add align records to aplot DS~{\nwtagstyle{}\subpageref{NWgff18-addT-1}}\RA{}
    my ($VarName,$Counter,$Type);
    ($VarName,$Counter,$Type) = (
        \\%GFF_DATA,
        \\$seq_COUNT,
        'SEQUENCE' );
    &load_gff_var($seqname,$VarName,$Counter,$Type);
    ($VarName,$Counter,$Type) = (
        \\%\{$VarName->\{$seqname\}[$_element]\},
        \\$$VarName\{$seqname\}[$_counter][$_elemNum],
        'SOURCE' );
    &load_gff_var($source,$VarName,$Counter,$Type);
    ($VarName,$Counter,$Type) = ( 
        \\%\{$VarName->\{$source\}[$_element]\},
        \\$$VarName\{$source\}[$_counter][$_elemNum],
        'STRAND' );
    &load_gff_var($strand,$VarName,$Counter,$Type);
    ($VarName,$Counter,$Type) = (
        \\%\{$VarName->\{$strand\}[$_element]\},
        \\$$VarName\{$strand\}[$_counter][$_elemNum],
        'GROUP' );
    &load_gff_var($group,$VarName,$Counter,$Type);
    \LA{}adding new feature elements~{\nwtagstyle{}\subpageref{NWgff18-addR-1}}\RA{}
    return;
\} # add_gff_record
\nwendcode{}\nwbegindocs{518}\nwdocspar

First we load the plain GFF attributes for a given record, when such record is defining a vector we add a new element to the feature anonymous array. That element is a list of the scores provided by the '{\tt{}{\char38}load{\char95}GFF{\char95}vector}' function defined in section~\ref{sec:loadvector}, page~\pageref{sec:loadvector}.

\nwenddocs{}\nwbegincode{519}\sublabel{NWgff18-addR-1}\nwmargintag{{\nwtagstyle{}\subpageref{NWgff18-addR-1}}}\moddef{adding new feature elements~{\nwtagstyle{}\subpageref{NWgff18-addR-1}}}\endmoddef
push @\{$VarName->\{$group\}[$_element]\},
    [
      'G',      # Type == plain GFF
      \{\},       # Properties hash, now empty
      $feature, # GFF feature (3rd field)
      $label,   # Record ID if exist, order# otherwise
      $start,
      $end,
      $score,
      $frame,
    ];
my $t = ++$VarName->\{$group\}[$_counter][$_elemNum];
&set_var_defaults('FEATURE',
                  \\%\{$VarName->\{$group\}[$_element][($t-1)][$_prop]\});
$_gff eq $VECTOR && do \{
    @\{$VarName->\{$group\}[$_element][8]\} = [ @vect_ary ];
\};
\nwused{\\{NWgff18-ParI-E}}\nwendcode{}\nwbegindocs{520}\nwdocspar

If the standard GFF alignment format is found, we process them as if it was an aplot records, so that we have to set all the variables required to fill such data structure ('{\tt{}{\char37}ALN{\char95}DATA}' in section~\ref{sec:APLOT-DS} , page~\pageref{sec:APLOT-DS}).

\nwenddocs{}\nwbegincode{521}\sublabel{NWgff18-addT-1}\nwmargintag{{\nwtagstyle{}\subpageref{NWgff18-addT-1}}}\moddef{add align records to aplot DS~{\nwtagstyle{}\subpageref{NWgff18-addT-1}}}\endmoddef
$_gff eq $ALIGN && do \{
    &add_aplot_record;
    return;
\};
\nwused{\\{NWgff18-ParI-E}}\nwendcode{}\nwbegindocs{522}\nwdocspar

We declare here the main variable containing all the GFF data, and auxiliary variables containing the array indexes used in the anonymous arrays defined within that variable.\label{sec:GFF-DS}

\nwenddocs{}\nwbegincode{523}\sublabel{NWgff18-GloB-B}\nwmargintag{{\nwtagstyle{}\subpageref{NWgff18-GloB-B}}}\moddef{Global Vars~{\nwtagstyle{}\subpageref{NWgff18-GloB-1}}}\plusendmoddef
my (%GFF_DATA,$seq_COUNT);
my ($_counter,$_prop,$_element) = (0..2);
my ($_order,$_elemNum,$_ori,$_end) = (0..3);
my ($_fttype,$_ftprop,$_ftname,$_ftid,$_ftori,$_ftend,$_ftsco,$_ftfrm) = (0..7);
\nwendcode{}\nwbegindocs{524}\nwdocspar

\label{func:loadgffvar}
\nwenddocs{}\nwbegincode{525}\sublabel{NWgff18-ParI-F}\nwmargintag{{\nwtagstyle{}\subpageref{NWgff18-ParI-F}}}\moddef{Parsing Input Data~{\nwtagstyle{}\subpageref{NWgff18-ParI-1}}}\plusendmoddef
sub load_gff_var() \{
    my ($_value,$_var,$_cnt,$_type) = @_;
    defined($$_var\{$_value\}) || do \{
        $$_var\{$_value\}[$_counter] = [ ++$$_cnt, 0, 0, 0 ];
        &set_var_defaults($_type,\\%\{$$_var\{$_value\}[$_prop]\});
    \};
    return;
\} # load_gff_var
\nwendcode{}\nwbegindocs{526}%$

We initialize properties for each new element as references to '{\tt{}{\char37}DefaultVars}' given level corresponding values. 

\nwenddocs{}\nwbegincode{527}\sublabel{NWgff18-ParI-G}\nwmargintag{{\nwtagstyle{}\subpageref{NWgff18-ParI-G}}}\moddef{Parsing Input Data~{\nwtagstyle{}\subpageref{NWgff18-ParI-1}}}\plusendmoddef
sub set_var_defaults() \{
    my ($sect,$varhash) = @_;
    \LA{}P.I.D: set reference to all properties~{\nwtagstyle{}\subpageref{NWgff18-P.Ic-1}}\RA{}
    return;
\} # load_gff_var
\nwendcode{}\nwbegindocs{528}\nwdocspar

We can set default properties of GFF elements in two ways: making a reference to the default properties hash for that element, as shown in the {\tt{}\LA{}P.I.D: set reference to all properties~{\nwtagstyle{}\subpageref{NWgff18-P.Ic-1}}\RA{}}, or reference each variable from the properties hash to a new key, as described in {\tt{}\LA{}P.I.D: set reference for each property~{\nwtagstyle{}\subpageref{NWgff18-P.Ic.2-1}}\RA{}}.

\label{sec:DATAdefaultvars}
\nwenddocs{}\nwbegincode{529}\sublabel{NWgff18-P.Ic-1}\nwmargintag{{\nwtagstyle{}\subpageref{NWgff18-P.Ic-1}}}\moddef{P.I.D: set reference to all properties~{\nwtagstyle{}\subpageref{NWgff18-P.Ic-1}}}\endmoddef
$$varhash = \\%\{$Defaults\{$sect\}\};
\nwused{\\{NWgff18-ParI-G}}\nwendcode{}\nwbegindocs{530}\nwdocspar

\nwenddocs{}\nwbegincode{531}\sublabel{NWgff18-P.Ic.2-1}\nwmargintag{{\nwtagstyle{}\subpageref{NWgff18-P.Ic.2-1}}}\moddef{P.I.D: set reference for each property~{\nwtagstyle{}\subpageref{NWgff18-P.Ic.2-1}}}\endmoddef
foreach my $nm (keys %\{$DefaultVars\{$sect\}\}) \{
    $$varhash\{$nm\} = \\$DefaultVars\{$sect\}\{$nm\}\{'VALUE'\};
\}; # foreach $nm
\nwnotused{P.I.D:\ set\ reference\ for\ each\ property}\nwendcode{}\nwbegindocs{532}\nwdocspar

We choose the first approach, and we will replace the hash reference by the keys with references to defaults if needed when setting variables from custom parameters for a given GFF element (see section~\ref{sec:mapcustoms}, page~\pageref{sec:mapcustoms}). That choice will save memory used by the elements properties if user does not customize any of the element properties, but requires a temporary default variable (due to the special structure of the {\tt{}{\char37}DefaultVars} hash which contains not only default variable values but also variable type), named {\tt{}{\char37}Defaults} and defined in section~\ref{sec:tmpdefaults}, page~\pageref{sec:tmpdefaults}.

\subsctn{Parsing grouping attributes}

'{\tt{}GFF{\char95}CHOICE}' block sets some variables depending on the record format, standard GFF or APLOT GFF-like. Once we have the attribute string from the GFF/APLOT record, we first check if it is empty, then we split by semicolons which will have or not white spaces or tabs before and/or after, as shown in this regular expression:\\
\centerline{'{\tt{}/{\char92}b{\char92}s*;{\char92}s*{\char92}b/}'}\\
This will define tag-value pairs (and maybe some extra fields).  

\nwenddocs{}\nwbegincode{533}\sublabel{NWgff18-ParI-H}\nwmargintag{{\nwtagstyle{}\subpageref{NWgff18-ParI-H}}}\moddef{Parsing Input Data~{\nwtagstyle{}\subpageref{NWgff18-ParI-1}}}\plusendmoddef
sub load_grouping() \{
    my ($_type,$attributes) = @_;
    my ($grp_string,$grp_counter,$grp_GP,$grp_NOGP,$grp_tag);
  GFF_CHOICE: \{
    $_type && do \{ 
        $grp_string = "$seqname\\_$source\\_$strand";
        $grp_counter = ++$group_gff_counter;
        $grp_GP = $GFF;
        $grp_NOGP = $GFF_NOGP;
        $grp_tag = '';
        last GFF_CHOICE;
    \};
    $grp_string = "$seqname_1\\_$seqname_2\\_$source_1\\_$strand_1$strand_2";
    $grp_counter = ++$group_aplot_counter;
    $grp_GP = $APLOT;
    $grp_NOGP = $APLOT_NOGP;
    $grp_tag = $SOURCE\{'align_tag'\}; # %SOURCE is a temporary hash name
  \};
    $label = $group_id = $grp_counter;
    defined($attributes) || do \{
        $group = "$grp_string\\_$group_id";
        return $grp_NOGP;
    \};
    my @grouping_list = ();
    @grouping_list = split /\\b\\s*;\\s*\\b/o, $attributes;
    \LA{}parse grouping attributes~{\nwtagstyle{}\subpageref{NWgff18-parP-1}}\RA{}
    \LA{}parse other attributes~{\nwtagstyle{}\subpageref{NWgff18-parM-1}}\RA{}
    return $grp_GP;
\}
\nwendcode{}\nwbegindocs{534}\nwdocspar

We may find four basic grouping field structures (detailed in Table~\ref{tbl:formats}) within the first element of '{\tt{}@grouping{\char95}list}'. We check first for double-quotes in the first or second field within that element. If second field is double-quoted, the first field is set as 'Tag' and the quoted as 'Value', else 'Tag' is set to default value (empty string for GFF records and 'target' for APLOT GFF-like ones) and 'Value' is set with the first field. Then we are looking in GFF grouping attributes for extra fields defining start, end, strand and frame (all related to the second sequence); in APLOT GFF-like format, those values are deprecated (because they must be defined in the first eight fields following the first sequence values and a colon as shown in table~\ref{tbl:formats}). 

\nwenddocs{}\nwbegincode{535}\sublabel{NWgff18-parP-1}\nwmargintag{{\nwtagstyle{}\subpageref{NWgff18-parP-1}}}\moddef{parse grouping attributes~{\nwtagstyle{}\subpageref{NWgff18-parP-1}}}\endmoddef
my $grp_flag = 0;
my @new_group = ();
my $groupregexp = '^(.*?)(?:"(.+?)"(?:\\s+\\b(.*))?)?$'; #'
my $group_string = shift @grouping_list;
($group_string =~ /$groupregexp/o) && 
    (@new_group = ($1,$2,$3));
$new_group[0] =~ s/\\b\\s*$//o;
$new_group[0] || do \{  # type 2 attributes
    $grp_flag = 1;
    $new_group[0] = $grp_tag;
\};
$new_group[1] || do \{  # type 1 attributes
    $grp_flag = 1;
    $new_group[1] = $new_group[0];
    $new_group[0] = $grp_tag;
\};
($tag,$group) = (lc($new_group[0]),$new_group[1]);
# Here looking for colon field separator in aplot GFF-like grouping
($grp_flag && $group =~ /^(.*?):(.*?)$/o) && (($group,$label) = ($1,$2));
\nwused{\\{NWgff18-ParI-H}}\nwendcode{}\nwbegindocs{536}%$

Here we look for other attribute fields, those defining coords for the target sequence match in standard GFF format and those defining a scoring vector, both formats are explained more in detail in the following two subsections.

\nwenddocs{}\nwbegincode{537}\sublabel{NWgff18-parM-1}\nwmargintag{{\nwtagstyle{}\subpageref{NWgff18-parM-1}}}\moddef{parse other attributes~{\nwtagstyle{}\subpageref{NWgff18-parM-1}}}\endmoddef
$_type && do \{ # GFF grouping
    $tag =~ /$SOURCE\{'align_tag'\}/ && do \{
        &load_GFF_align;
        return $ALIGN;
    \}; 
    $tag =~ /$SOURCE\{'vector_tag'\}/ && do \{
        &load_GFF_vector;
        return $VECTOR;
    \};
\};
\nwalsodefined{\\{NWgff18-parM-2}}\nwused{\\{NWgff18-ParI-H}}\nwendcode{}\nwbegindocs{538}\nwdocspar

We are interested in finding 'id' tag, that defines the label for a single record, this is implemented in the '{\tt{}{\char38}load{\char95}GFF{\char95}align}' and '{\tt{}{\char38}load{\char95}GFF{\char95}vector}'functions but has to be parsed in plain GFF records.

\nwenddocs{}\nwbegincode{539}\sublabel{NWgff18-parM-2}\nwmargintag{{\nwtagstyle{}\subpageref{NWgff18-parM-2}}}\moddef{parse other attributes~{\nwtagstyle{}\subpageref{NWgff18-parM-1}}}\plusendmoddef
scalar(@grouping_list) > 0 && do\{ 
    foreach my $element (@grouping_list) \{
        ($element =~ /$groupregexp/o) && (@new_group = ($1,$2,$3));
        lc($new_group[0]) =~ $SOURCE\{'label_tag'\} && do \{
            $label = $new_group[1];
            $label eq "" && do \{
                (undef,$label,undef) = split /\\s+/og, $new_group[0];
            \};
        \};
    \};
\};
\nwendcode{}\nwbegindocs{540}\nwdocspar

\subsubsctn{Layout variables involved in grouping tags}

\nwenddocs{}\nwbegincode{541}\sublabel{NWgff18-defQ-2}\nwmargintag{{\nwtagstyle{}\subpageref{NWgff18-defQ-2}}}\moddef{default layout vars values~{\nwtagstyle{}\subpageref{NWgff18-defQ-1}}}\plusendmoddef
align_tag                  => \{ TYPE => 'STRING' , VALUE => 'target' \},
\nwendcode{}\nwbegindocs{542}\nwdocspar
\nwenddocs{}%
\nwbegindocs{544}\nwdocspar
\nwenddocs{}%
%
%
%
\nwbegindocs{546}\nwdocspar

\nwenddocs{}\nwbegincode{547}\sublabel{NWgff18-defQ-3}\nwmargintag{{\nwtagstyle{}\subpageref{NWgff18-defQ-3}}}\moddef{default layout vars values~{\nwtagstyle{}\subpageref{NWgff18-defQ-1}}}\plusendmoddef
vector_tag                 => \{ TYPE => 'STRING' , VALUE => 'vector' \},
\nwendcode{}\nwbegindocs{548}\nwdocspar
\nwenddocs{}%
\nwbegindocs{550}\nwdocspar
\nwenddocs{}%
%
%
%
\nwbegindocs{552}\nwdocspar

\nwenddocs{}\nwbegincode{553}\sublabel{NWgff18-defQ-4}\nwmargintag{{\nwtagstyle{}\subpageref{NWgff18-defQ-4}}}\moddef{default layout vars values~{\nwtagstyle{}\subpageref{NWgff18-defQ-1}}}\plusendmoddef
label_tag                  => \{ TYPE => 'STRING' , VALUE => 'id'   \},
\nwendcode{}\nwbegindocs{554}\nwdocspar
\nwenddocs{}%
\nwbegindocs{556}\nwdocspar
\nwenddocs{}%
%
%
%
\nwbegindocs{558}\nwdocspar

% \subsctn{Parsing extended GFF format}\label{sec:parseXTRA}

The following record formats share the field structure of a GFF standard record, but there are special tags in the grouping fields containing data that will be used by {\prog}. Both formats require a fixed tag string (by default 'Vector' for the scoring vectors and 'Target' for the alignment data), which can be redefined by users from custom files.

\begin{comment}
\todo{
\item {\tt{}{\char37}SOURCE} is a \textbf{provisional varname} that must be replaced by the corresponding {\tt{}{\char36}DefaultVars{\char123}LAYOUT{\char125}}, {\tt{}{\char36}CustomVars{\char123}LAYOUT{\char125}} or {\tt{}{\char36}CmdLineVars{\char123}LAYOUT{\char125}}. Those 'LAYOUT' variables must be set before reading GFF input files because some settings are used by parser functions.
} % todo
\end{comment}

\subsubsctn{Parsing alignment records}

\todo{
\item Parsing alignment records:\\
\centerline{{\tt{}aln{\char95}tag\ "value"\ target{\char95}ori\ target{\char95}end\ target{\char95}frame\ target{\char95}strand\ (...)}}
} % todo
\nwenddocs{}\nwbegincode{559}\sublabel{NWgff18-ParI-I}\nwmargintag{{\nwtagstyle{}\subpageref{NWgff18-ParI-I}}}\moddef{Parsing Input Data~{\nwtagstyle{}\subpageref{NWgff18-ParI-1}}}\plusendmoddef
sub load_GFF_align() \{

\} # load_GFF_align
\nwendcode{}\nwbegindocs{560}\nwdocspar

\subsubsctn{Parsing scoring vectors} \label{sec:loadvector}

\todo{
\item Parsing scoring vectors:\\
\centerline{{\tt{}vector{\char95}tag\ "value"\ vector{\char95}type;\ step\ value;\ window\ value;\ scores\ sco\ ...\ sco}}
} % todo
\nwenddocs{}\nwbegincode{561}\sublabel{NWgff18-ParI-J}\nwmargintag{{\nwtagstyle{}\subpageref{NWgff18-ParI-J}}}\moddef{Parsing Input Data~{\nwtagstyle{}\subpageref{NWgff18-ParI-1}}}\plusendmoddef
sub load_GFF_vector() \{
    @vect_ary = ();
\} # load_GFF_vector
\nwendcode{}\nwbegindocs{562}\nwdocspar

\nwenddocs{}\nwbegincode{563}\sublabel{NWgff18-GloB-C}\nwmargintag{{\nwtagstyle{}\subpageref{NWgff18-GloB-C}}}\moddef{Global Vars~{\nwtagstyle{}\subpageref{NWgff18-GloB-1}}}\plusendmoddef
my @vect_ary;
\nwendcode{}\nwbegindocs{564}\nwdocspar


\newpage

\sctn{Mapping custom features to figure elements} %%%%%%%%%%%%%%%%%%%%%%%%%%%%%%

\todo{
\item Set variable mapping for {\tt{}{\char37}ALN{\char95}DATA} elements.
} % todo

\subsctn{Mapping customization variables to working variable}

We define here {\tt{}{\char38}merge{\char95}custom{\char95}vars} to combine customization
parameters from {\tt{}{\char37}CmdLineVars} and {\tt{}{\char37}CustomVars} into {\tt{}{\char37}Vars}
before start parsing GFF input files. The main reason to do this
variable mapping at this point is that there are some LAYOUT variables
that are used when parsing the GFF records (basically
{\tt{}align{\char95}tag},{\tt{}vector{\char95}tag} and {\tt{}label{\char95}tag}). Another reason is
that from now on, we only have to take care of a single variable,
which summarizes all the settings from different inputs, say here the
program built-in defaults, the custom files and the command-line
options. The values set on command-line override those from custom
files and those set from custom files override defaults. We are going
to concatenate each element-value pairs for each variable (not for the
layout variables but for all the rest), so that we will have a
variable name reference linked to a list of element-value pairs, which
will be used when mapping those values for a given element in the data
variables. When loading variables, defaults are set to {\tt{}{\char37}Defaults}
(see section~\ref{sec:DATAdefaultvars},
page~\pageref{sec:DATAdefaultvars}), so we only take care here of
those values that are redefined by user (we do not use {\tt{}{\char37}DefaultVars} because we already checked variable values and {\tt{}{\char37}Defaults} variable has a simpler structure than it).

\nwenddocs{}\nwbegincode{565}\sublabel{NWgff18-FeaG-1}\nwmargintag{{\nwtagstyle{}\subpageref{NWgff18-FeaG-1}}}\moddef{Features Setting~{\nwtagstyle{}\subpageref{NWgff18-FeaG-1}}}\endmoddef
sub merge_custom_vars() \{
    &header("MAPPING CUSTOMIZATION INPUTS TO MAIN VARS");
    \LA{}setting layout variables~{\nwtagstyle{}\subpageref{NWgff18-setO-1}}\RA{}
    \LA{}setting non-layout variables~{\nwtagstyle{}\subpageref{NWgff18-setS-1}}\RA{}
    &footer("VALUES SET for MAIN VARS");
\} # merge_custom_vars
\nwalsodefined{\\{NWgff18-FeaG-2}\\{NWgff18-FeaG-3}\\{NWgff18-FeaG-4}\\{NWgff18-FeaG-5}}\nwused{\\{NWgff18-MaiD-1}}\nwendcode{}\nwbegindocs{566}\nwdocspar

Layout variables are simpler than the rest, they are defined as a key-value pair, so that mapping is straightforward.

\nwenddocs{}\nwbegincode{567}\sublabel{NWgff18-setO-1}\nwmargintag{{\nwtagstyle{}\subpageref{NWgff18-setO-1}}}\moddef{setting layout variables~{\nwtagstyle{}\subpageref{NWgff18-setO-1}}}\endmoddef
foreach my $var_name (%\{ $Defaults\{LAYOUT\} \}) \{
    defined($CmdLineVars\{LAYOUT\}\{$var_name\}) && do \{
        $Vars\{LAYOUT\}\{$var_name\} = $CmdLineVars\{LAYOUT\}\{$var_name\};
        next;
    \};
    defined($CustomVars\{LAYOUT\}\{$var_name\}) && do \{
        $Vars\{LAYOUT\}\{$var_name\} = $CustomVars\{LAYOUT\}\{$var_name\};
        next;
    \};
    $Vars\{LAYOUT\}\{$var_name\} = $Defaults\{LAYOUT\}\{$var_name\};
\}; # foreach $var_name
\nwused{\\{NWgff18-FeaG-1}}\nwendcode{}\nwbegindocs{568}\nwdocspar

For the rest of variable classes (SEQUENCE, SOURCE, STRAND, GROUP and
FEATURE), which are defined as an element-key-value triad, we can have
more than one element and with a different value settings, for each
key (say here variable name). We store that information in a list. We
know that every three values of the list we have a variable definition
for a given element and if it is a negation or not (i.e., see
section~\ref{sec:DATAtriadarray}, page~\pageref{sec:DATAtriadarray},
for its definition in {\tt{}{\char37}CustomVars}). The structure of that subset
of the list is squetched here:\\

\centerline{\ldots , 'negate regexp flag', 'element regular expression', 'variable value' , \ldots}

\nwenddocs{}\nwbegincode{569}\sublabel{NWgff18-setS-1}\nwmargintag{{\nwtagstyle{}\subpageref{NWgff18-setS-1}}}\moddef{setting non-layout variables~{\nwtagstyle{}\subpageref{NWgff18-setS-1}}}\endmoddef
foreach my $_sec (keys %Defaults) \{
    $_sec eq 'LAYOUT' && next; # skip layout variables
    foreach my $_var (keys %\{$Defaults\{$_sec\}\}) \{
        defined($CustomVars\{$_sec\}\{$_var\}) && do \{
            push @\{ $Vars\{$_sec\}\{$_var\} \}, @\{ $CustomVars\{$_sec\}\{$_var\} \};
        \}; # defined($CustomVars\{$_sec\}\{$_var\})
        defined($CmdLineVars\{$_sec\}\{$_var\}) && do \{
            push @\{ $Vars\{$_sec\}\{$_var\} \}, @\{ $CmdLineVars\{$_sec\}\{$_var\} \};
        \}; # defined($CustomVars\{$_sec\}\{$_var\})        
    \}; # foreach $vnm
\}; # foreach $sct
\nwused{\\{NWgff18-FeaG-1}}\nwendcode{}\nwbegindocs{570}\nwdocspar

Here we load the custom files definitions before the command-line ones, because for those variables in {\tt{}{\char37}Vars} ordering is important, and later definitions in the array will override previous ones. 

\subsctn{Mapping customization to GFF elements}

Once we have set the variables from different sources as we did in the previous section, and once we loaded the GFF data from input files, we have to assign the corresponding settings to each GFF element. Defaults were already set as a reference to {\tt{}{\char37}Defaults} hash (see section~\ref{sec:DATAdefaultvars}, page~\pageref{sec:DATAdefaultvars}). Perhaps this will be the most time consuming function of the program as we have to traverse all the variables for each class (if they are set in {\tt{}{\char37}Vars}, which will save much of that time if they are not defined) throught all the GFF elements.
\label{sec:mapcustoms}

\nwenddocs{}\nwbegincode{571}\sublabel{NWgff18-FeaG-2}\nwmargintag{{\nwtagstyle{}\subpageref{NWgff18-FeaG-2}}}\moddef{Features Setting~{\nwtagstyle{}\subpageref{NWgff18-FeaG-1}}}\plusendmoddef
sub map_vars_data() \{
    &header("SETTING CUSTOM VALUES TO GFF ELEMENTS");
    foreach my $v_sec (keys %Vars) \{
        $v_sec eq 'LAYOUT' && next; # skip layout variables
        foreach my $v_var (keys %\{$Vars\{$v_sec\}\}) \{
            my @v_values = @\{ $Vars\{$v_sec\}\{$v_var\} \}
            for (my $foo = 0; $foo < $#v_values; $foo+=4) \{
                my ($neg_flg, $id_flg, $reg_exp, $the_value, $name_test);
                ($neg_flg, $id_flg, $reg_exp, $the_value) = @v_values[$foo..($foo + 3)];
                $name_test = $neg_flg ? \\&match_regexp_neg : \\&match_regexp;
                \LA{}set reference to GFF data~{\nwtagstyle{}\subpageref{NWgff18-setP-1}}\RA{}
            \}; # for $foo
        \}; # foreach $vnm
    \}; # foreach $sct
    &footer("VALUES SET for GFF ELEMENTS");
\} # map_vars_data
\nwendcode{}\nwbegindocs{572}\nwdocspar

We have to set the properties hash corresponding to the GFF elements present in the current section (say here SEQUENCE, SOURCE, STRAND, GROUP or FEATURE). To do that we have to define a reference to that anonymous hash and also retrieve the element string ({\tt{}{\char36}element}) to check if it match to the regular expression ({\tt{}{\char36}reg{\char95}exp}) which determines if the variable of that element is re-assigned to the current value ({\tt{}{\char36}the{\char95}value}).

\nwenddocs{}\nwbegincode{573}\sublabel{NWgff18-setP-1}\nwmargintag{{\nwtagstyle{}\subpageref{NWgff18-setP-1}}}\moddef{set reference to GFF data~{\nwtagstyle{}\subpageref{NWgff18-setP-1}}}\endmoddef
foreach my $seq (keys %GFF_DATA) \{
    $v_sec eq 'SEQUENCE' && do \{
        my $seq_ref = \\@\{ $GFF_DATA\{$seq\} \};
        &does_feat_match($seq,$reg_exp,$seq_ref,
                         $_prop,$v_sec,$v_var,$the_value);
        next;
    \};
    \LA{}set reference to GFF data: SOURCES~{\nwtagstyle{}\subpageref{NWgff18-setY-1}}\RA{}
\}; # foreach my $seq
\nwused{\\{NWgff18-FeaG-2}}\nwendcode{}\nwbegindocs{574}\nwdocspar

\nwenddocs{}\nwbegincode{575}\sublabel{NWgff18-setY-1}\nwmargintag{{\nwtagstyle{}\subpageref{NWgff18-setY-1}}}\moddef{set reference to GFF data: SOURCES~{\nwtagstyle{}\subpageref{NWgff18-setY-1}}}\endmoddef
foreach my $src (keys $seq_ref->[$_element]) \{
    $v_sec eq 'SOURCE' && do \{
        my $src_ref = \\@\{ $seq_ref->[$_element]\{$src\} \};
        &does_feat_match($src,$reg_exp,$src_ref,
                         $_prop,$v_sec,$v_var,$the_value);
        next;
    \};
    \LA{}set reference to GFF data: STRANDS~{\nwtagstyle{}\subpageref{NWgff18-setY.2-1}}\RA{}
\}; # foreach my $src
\nwused{\\{NWgff18-setP-1}}\nwendcode{}\nwbegindocs{576}\nwdocspar

\nwenddocs{}\nwbegincode{577}\sublabel{NWgff18-setY.2-1}\nwmargintag{{\nwtagstyle{}\subpageref{NWgff18-setY.2-1}}}\moddef{set reference to GFF data: STRANDS~{\nwtagstyle{}\subpageref{NWgff18-setY.2-1}}}\endmoddef
foreach my $str (keys $src_ref->[$_element]) \{
    $v_sec eq 'STRAND' && do \{
        my $str_ref = \\@\{ $src_ref->[$_element]\{$str\} \};
        &does_feat_match($str,$reg_exp,$str_ref,
                         $_prop,$v_sec,$v_var,$the_value);
        next;
    \};
    \LA{}set reference to GFF data: GROUPS~{\nwtagstyle{}\subpageref{NWgff18-setX-1}}\RA{}
\}; # foreach my $src
\nwused{\\{NWgff18-setY-1}}\nwendcode{}\nwbegindocs{578}\nwdocspar

\nwenddocs{}\nwbegincode{579}\sublabel{NWgff18-setX-1}\nwmargintag{{\nwtagstyle{}\subpageref{NWgff18-setX-1}}}\moddef{set reference to GFF data: GROUPS~{\nwtagstyle{}\subpageref{NWgff18-setX-1}}}\endmoddef
foreach my $grp (keys $str_ref->[$_element]) \{
    $v_sec eq 'GROUP' && do \{
        my $grp_ref = \\@\{ $str_ref->[$_element]\{$grp\} \};
        &does_feat_match($grp,$reg_exp,$grp_ref,
                         $_prop,$v_sec,$v_var,$the_value);
        next;
    \};
    \LA{}set reference to GFF data: FEATURES~{\nwtagstyle{}\subpageref{NWgff18-setZ-1}}\RA{}
\}; # foreach my $src
\nwused{\\{NWgff18-setY.2-1}}\nwendcode{}\nwbegindocs{580}\nwdocspar

\nwenddocs{}\nwbegincode{581}\sublabel{NWgff18-setZ-1}\nwmargintag{{\nwtagstyle{}\subpageref{NWgff18-setZ-1}}}\moddef{set reference to GFF data: FEATURES~{\nwtagstyle{}\subpageref{NWgff18-setZ-1}}}\endmoddef
foreach my $feat (0..$#\{ $grp_ref->[$_element] \}) \{
    my ($feat_ref, $ft_name, $ft_id, $ft_vars);
    $feat_ref = \\@\{ $grp_ref->[$_element][$feat] \};
    ($ft_name, $ft_id) = ($feat_ref->[$_ftname],$feat_ref->[$_ftid]);
    # check ID
    $id_flg ne $NULL && \{
        $ft_id ~ /^$id_flg$/o && do \{
            &does_feat_match($ft_name,$reg_exp,$feat_ref,
                             $_ftprop,$v_sec,$v_var,$the_value);
        \}; # $ft_id ~ /^$id_flg$/o
        next;
    \}; # $id_flg ne $NULL
    # check regexp
    &does_feat_match($ft_name,$reg_exp,$feat_ref,
                     $_ftprop,$v_sec,$v_var,$the_value);
\}; # foreach my $src
\nwused{\\{NWgff18-setX-1}}\nwendcode{}\nwbegindocs{582}\nwdocspar

\nwenddocs{}\nwbegincode{583}\sublabel{NWgff18-FeaG-3}\nwmargintag{{\nwtagstyle{}\subpageref{NWgff18-FeaG-3}}}\moddef{Features Setting~{\nwtagstyle{}\subpageref{NWgff18-FeaG-1}}}\plusendmoddef
sub does_feat_match() \{
    my ($name,$rexp,$gffref,$prop,$sct,$var,$value) = @_;
    &$name_test($name,$rexp) && do \{
        (ref($gffref->[$prop]) eq 'REF') &&
            &set_all_defaults($gffref, $sct);
        $gffref->[$prop]\{$var\} = $value;
    \};
\} # does_feat_match
\nwendcode{}\nwbegindocs{584}\nwdocspar

The following two functions are checking if a GFF element match or not a given regular expression (minimal expression consists in a GFF element string). They return a true/flase result that decides if a variable is set or not for a given GFF element.

\nwenddocs{}\nwbegincode{585}\sublabel{NWgff18-FeaG-4}\nwmargintag{{\nwtagstyle{}\subpageref{NWgff18-FeaG-4}}}\moddef{Features Setting~{\nwtagstyle{}\subpageref{NWgff18-FeaG-1}}}\plusendmoddef
sub match_regexp() \{
    $_[0]  ~ /$_[1]/o && return $T;
    return $F;
\} # match_regexp
sub match_regexp_neg() \{
    $_[0] !~ /$_[1]/o && return $T;
    return $F;
\} # match_regexp_neg
\nwendcode{}\nwbegindocs{586}\nwdocspar

\nwenddocs{}\nwbegincode{587}\sublabel{NWgff18-FeaG-5}\nwmargintag{{\nwtagstyle{}\subpageref{NWgff18-FeaG-5}}}\moddef{Features Setting~{\nwtagstyle{}\subpageref{NWgff18-FeaG-1}}}\plusendmoddef
sub set_all_defaults() \{
    my ($the_hash,$the_sec) = @_;
    $the_hash->[$_prop] = ();
    foreach my $vnm (keys %\{$Defaults\{$the_sec\}\}) \{
        $the_hash->[$_prop]\{$vnm\} = \\$Defaults\{$the_sec\}\{$vnm\};
    \}; # foreach $nm
\} # set_all_defaults
\nwendcode{}\nwbegindocs{588}\nwdocspar


\newpage

\sctn{Sorting GFF features} %%%%%%%%%%%%%%%%%%%%%%%%%%%%%%%%%%%%%%%%%%%%

We need to sort the GFF elements by acceptor, it is more or less easy to sort them because they were defined in {\tt{}{\char37}GFF{\char95}DATA} or {\tt{}{\char37}ALN{\char95}DATA}
as arrays (it is implemented with the {\tt{}map}/{\tt{}sort}/{\tt{}map} technique as it can be appreciated in {\tt{}\LA{}sorting group elements~{\nwtagstyle{}\subpageref{NWgff18-sorM-1}}\RA{}} code chunk). But we also need to sort groups by their coordinates and to sort sequences and sources by their input order (as they were read from input file I mean). This will be difficult to do on the original hashes because coords and order were placed in a subarray for a given group/source/sequence key name (that key point to an anonymous array containing that counters as you can see in table~\ref{tbl:gffdata}, page~\pageref{tbl:gffdata}). As we are looking for the GFF elements that belong to a group, which is related to a given strand of a source within a sequence, we can build an auxiliary variable that holds the proper ordering at each level as we go deep in the {\tt{}{\char37}GFF{\char95}DATA}/{\tt{}{\char37}ALN{\char95}DATA} variables hierarchy.
At the same time we can fill the {\tt{}{\char36}{\char95}ori} and {\tt{}{\char36}{\char95}end} fields we left set to zero for the counters array in {\tt{}{\char38}load{\char95}gff{\char95}var} (see function definition in page~\pageref{func:loadgffvar}).

\todo{
\item Set ordering for {\tt{}{\char37}ALN{\char95}DATA}.
} % todo
\nwenddocs{}\nwbegincode{589}\sublabel{NWgff18-SorG-1}\nwmargintag{{\nwtagstyle{}\subpageref{NWgff18-SorG-1}}}\moddef{Sorting Features~{\nwtagstyle{}\subpageref{NWgff18-SorG-1}}}\endmoddef
sub sort_elements() \{
    &header("SORTING ELEMENTS BY ACCEPTOR (START)");
    %Order = ();
    # sorting %GFF_DATA contents 
    scalar(%GFF_DATA) && do \{
        &report('SORT_GFF','*- ','ANNOTATION DATA');
        my ($v_max,$v_min);
        my $sq_ord = \\@\{ $Order\{GFF\} \} ;
        @\{ $sq_ord \} = ();
        my $s_ref = \\%GFF_DATA;
        foreach my $s_seq (keys %\{ $s_ref \}) \{
            \LA{}sorting sequences~{\nwtagstyle{}\subpageref{NWgff18-sorH-1}}\RA{}
        \}; # foreach $s_seq
        &sort_by_inputorder($sq_ord);
        print LOGFILE '>>> \\%Order\{GFF\} : '.(Dumper(\\%\{ $Order\{GFF\} \}))
            if ($LogFile && $Debug);
    \}; # scalar(%GFF_DATA) > 0
    # sorting %ALN_DATA contents 
    scalar(%ALN_DATA) && do \{
        my $s_ref = \\%ALN_DATA;
        &report('SORT_GFF','*- ','ALIGNMENT DATA');
    \}; # scalar(%ALN_DATA) > 0
  # print LOGFILE '>>> \\%GFF_DATA : '.(Dumper(\\%GFF_DATA))
  #     if ($LogFile && $Debug);
    &footer("ELEMENTS SORTED");
\} # sort_elements
\nwalsodefined{\\{NWgff18-SorG-2}\\{NWgff18-SorG-3}}\nwused{\\{NWgff18-MaiD-1}}\nwendcode{}\nwbegindocs{590}\nwdocspar

Once we get into {\tt{}{\char38}sort{\char95}elements} function we initialize the arrays within the {\tt{}{\char37}Order} hash as we are requiring them.

\nwenddocs{}\nwbegincode{591}\sublabel{NWgff18-GloB-D}\nwmargintag{{\nwtagstyle{}\subpageref{NWgff18-GloB-D}}}\moddef{Global Vars~{\nwtagstyle{}\subpageref{NWgff18-GloB-1}}}\plusendmoddef
my %Order;
\nwendcode{}\nwbegindocs{592}\nwdocspar

\nwenddocs{}\nwbegincode{593}\sublabel{NWgff18-mesR-1}\nwmargintag{{\nwtagstyle{}\subpageref{NWgff18-mesR-1}}}\moddef{messages - sorting GFF data~{\nwtagstyle{}\subpageref{NWgff18-mesR-1}}}\endmoddef
SORT_GFF => "\\%sSorting \\%s\\n",
\nwalsodefined{\\{NWgff18-mesR-2}\\{NWgff18-mesR-3}\\{NWgff18-mesR-4}\\{NWgff18-mesR-5}}\nwused{\\{NWgff18-GloB-G}}\nwendcode{}\nwbegindocs{594}\nwdocspar

\subsctn{Descending from sequences to groups}

\nwenddocs{}\nwbegincode{595}\sublabel{NWgff18-sorH-1}\nwmargintag{{\nwtagstyle{}\subpageref{NWgff18-sorH-1}}}\moddef{sorting sequences~{\nwtagstyle{}\subpageref{NWgff18-sorH-1}}}\endmoddef
&report('SORT_SEQ','|  *- ',$s_seq);
my @sc_coords = ();
push @\{ $sq_ord \},
        [ $s_seq,
          $s_ref->\{$s_seq\}[$_counter][$_order],
          () ];
my $sc_ord = \\@\{ $sq_ord->[ $#\{$sq_ord\} ][2] \};
my $ss_ref = \\%\{ $s_ref->\{$s_seq\}[$_element] \};
foreach my $s_src (keys %\{ $ss_ref \}) \{
    \LA{}sorting sources~{\nwtagstyle{}\subpageref{NWgff18-sorF-1}}\RA{}
\}; # foreach $s_src
&sort_by_inputorder($sc_ord);
\LA{}set sequence elements boundaries~{\nwtagstyle{}\subpageref{NWgff18-setW-1}}\RA{}
\nwused{\\{NWgff18-SorG-1}}\nwendcode{}\nwbegindocs{596}\nwdocspar

\nwenddocs{}\nwbegincode{597}\sublabel{NWgff18-sorF-1}\nwmargintag{{\nwtagstyle{}\subpageref{NWgff18-sorF-1}}}\moddef{sorting sources~{\nwtagstyle{}\subpageref{NWgff18-sorF-1}}}\endmoddef
&report('SORT_SRC',(('|  ' x 2).'*- '),$s_src);
my @sr_coords = ();
push @\{ $sc_ord \},
        [ $s_src,
          $ss_ref->\{$s_src\}[$_counter][$_order],
          () ];
my $sr_ord = \\@\{ $sc_ord->[ $#\{$sc_ord\} ][2] \};
my $sc_ref = \\%\{ $ss_ref->\{$s_src\}[$_element] \};
foreach my $s_str (keys %\{ $sc_ref \}) \{
    \LA{}sorting GFF groups~{\nwtagstyle{}\subpageref{NWgff18-sorI-1}}\RA{}
\}; # foreach $s_str
\LA{}set source elements boundaries~{\nwtagstyle{}\subpageref{NWgff18-setU-1}}\RA{}
\nwused{\\{NWgff18-sorH-1}}\nwendcode{}\nwbegindocs{598}\nwdocspar

\nwenddocs{}\nwbegincode{599}\sublabel{NWgff18-mesR-2}\nwmargintag{{\nwtagstyle{}\subpageref{NWgff18-mesR-2}}}\moddef{messages - sorting GFF data~{\nwtagstyle{}\subpageref{NWgff18-mesR-1}}}\plusendmoddef
SORT_SEQ => "\\%sSequence: \\%s\\n",
SORT_SRC => "\\%sSource: \\%s\\n",
\nwendcode{}\nwbegindocs{600}\nwdocspar

\subsctn{Sorting features for groups}

\nwenddocs{}\nwbegincode{601}\sublabel{NWgff18-sorI-1}\nwmargintag{{\nwtagstyle{}\subpageref{NWgff18-sorI-1}}}\moddef{sorting GFF groups~{\nwtagstyle{}\subpageref{NWgff18-sorI-1}}}\endmoddef
&report('SORT_STR',(('|  ' x 3).'*- '),$s_str);
my @ft_coords = ();
push @\{ $sr_ord \},
        [ $s_str,
          # $sc_ref->\{$s_str\}[$_counter][$_order],
          () ]; # if uncomment '$sc_ref' set next to [2] instead of [1].
my $st_ord = \\@\{ $sr_ord->[ $#\{$sr_ord\} ][1] \};
my $sr_ref = \\%\{ $sc_ref->\{$s_str\}[$_element] \};
\LA{}set forward/reverse sorting functon~{\nwtagstyle{}\subpageref{NWgff18-revZ-1}}\RA{}
my $s_elem;
foreach my $s_grp (keys %\{ $sr_ref \}) \{
    \LA{}sorting group elements~{\nwtagstyle{}\subpageref{NWgff18-sorM-1}}\RA{}
\}; # foreach $s_grp
\LA{}sorting group coords by acceptor~{\nwtagstyle{}\subpageref{NWgff18-sorW-1}}\RA{}
\nwused{\\{NWgff18-sorF-1}}\nwendcode{}\nwbegindocs{602}\nwdocspar

\nwenddocs{}\nwbegincode{603}\sublabel{NWgff18-mesR-3}\nwmargintag{{\nwtagstyle{}\subpageref{NWgff18-mesR-3}}}\moddef{messages - sorting GFF data~{\nwtagstyle{}\subpageref{NWgff18-mesR-1}}}\plusendmoddef
SORT_STR => "\\%sStrand: \\%s\\n",
\nwendcode{}\nwbegindocs{604}\nwdocspar

The main difference between this piece of code and the previous three ones is that we also find the boundaries of the groups here ({\tt{}{\char36}ft{\char95}min}/{\tt{}{\char36}ft{\char95}max} variables). This make possible to sort them and to know the first and the last coords that we can find in a strand/source/sequence (as we do in the last lines of the following piece of code, and in the next one {\tt{}\LA{}sorting group coords by acceptor~{\nwtagstyle{}\subpageref{NWgff18-sorW-1}}\RA{}}).

\nwenddocs{}\nwbegincode{605}\sublabel{NWgff18-sorM-1}\nwmargintag{{\nwtagstyle{}\subpageref{NWgff18-sorM-1}}}\moddef{sorting group elements~{\nwtagstyle{}\subpageref{NWgff18-sorM-1}}}\endmoddef
&report('SORT_GRP',(('|  ' x 4).'*- '),$s_grp);
my $sg_ref = \\@\{ $sr_ref->\{$s_grp\}[$_element] \};
$s_elem = $sr_ref->\{$s_grp\}[$_counter][$_elemNum];
$s_elem > 1 && do \{
    @\{ $sg_ref \} = map \{ $_->[2] \}
                   sort \{ &$sortfunct \}
                   map \{ [ $_->[4],
                           $_->[5],
                           $_ ] \} @\{ $sg_ref \}; # maps 'start 'end 'arrayelement
\}; # $s_elem > 1
@ft_coords = ( map \{ $_->[4], $_->[5] \} @\{ $sg_ref \} );
$v_min = &min(@ft_coords);
$v_max = &max(@ft_coords);
$sr_ref->\{$s_grp\}[$_counter][$_ori] = $v_min;
$sr_ref->\{$s_grp\}[$_counter][$_end] = $v_max;
push @\{ $st_ord \}, [ $s_grp, $v_min, $v_max ];
&report('SORT_FTR',('|  ' x 5),$s_elem);
\nwused{\\{NWgff18-sorI-1}}\nwendcode{}\nwbegindocs{606}\nwdocspar

Something that really boost the ordering, if we compare to the old {\tt{}gawk} version of {\prog}, is that we are performing the sort only on those groups that have more that one element, and we are also ordering short sets of GFF elements instead of sorting the whole input records first and then sort the groups we found in those records.

\nwenddocs{}\nwbegincode{607}\sublabel{NWgff18-mesR-4}\nwmargintag{{\nwtagstyle{}\subpageref{NWgff18-mesR-4}}}\moddef{messages - sorting GFF data~{\nwtagstyle{}\subpageref{NWgff18-mesR-1}}}\plusendmoddef
SORT_GRP => "\\%sGroup: \\%s\\n",
SORT_FTR => "\\%s       Sorted \\%s elements.\\n",
\nwendcode{}\nwbegindocs{608}\nwdocspar

\subsctn{Sorting groups by coords and upper features by input order}

Here we take advantage of having the group coords in a list of lists ({\tt{}@{\char123}\ {\char36}st{\char95}ord\ {\char125}}) instead of having it as a list within a key/value hash of lists.

\nwenddocs{}\nwbegincode{609}\sublabel{NWgff18-sorW-1}\nwmargintag{{\nwtagstyle{}\subpageref{NWgff18-sorW-1}}}\moddef{sorting group coords by acceptor~{\nwtagstyle{}\subpageref{NWgff18-sorW-1}}}\endmoddef
@ft_coords = ( map \{ $_->[1], $_->[2] \} @\{ $st_ord \} );
$v_min = &min(@ft_coords);
$v_max = &max(@ft_coords);
$sc_ref->\{$s_str\}[$_counter][$_ori] = $v_min;
$sc_ref->\{$s_str\}[$_counter][$_end] = $v_max;
push @sr_coords, $v_min, $v_max;
#
@\{ $st_ord \} = map \{ $_->[2] \}
               sort \{ &$sortfunct \}
               map \{ [ $_->[1], $_->[2], $_->[0] ] \} @\{ $st_ord \};
$s_elem = $sc_ref->\{$s_str\}[$_counter][$_elemNum];
&report('SORT_GPN',('|  ' x 4),$s_elem);
\nwused{\\{NWgff18-sorI-1}}\nwendcode{}\nwbegindocs{610}\nwdocspar

We reduced the group order subarray just to contain the group names sorted (their coords were saved in the corresponding variable within )

\nwenddocs{}\nwbegincode{611}\sublabel{NWgff18-mesR-5}\nwmargintag{{\nwtagstyle{}\subpageref{NWgff18-mesR-5}}}\moddef{messages - sorting GFF data~{\nwtagstyle{}\subpageref{NWgff18-mesR-1}}}\plusendmoddef
SORT_GPN => "\\%s\\`Sorted \\%s groups.\\n",
\nwendcode{}\nwbegindocs{612}\nwdocspar

We are going to skip the strand sorting, because the program will process always strands in a fixed order, that is: '+', '.' and '-' (so forward features first, followed by those features that does not have a defined strand, and finally reverse features).

\nwenddocs{}\nwbegincode{613}\sublabel{NWgff18-setU-1}\nwmargintag{{\nwtagstyle{}\subpageref{NWgff18-setU-1}}}\moddef{set source elements boundaries~{\nwtagstyle{}\subpageref{NWgff18-setU-1}}}\endmoddef
$v_min = &min(@sr_coords);
$v_max = &max(@sr_coords);
$ss_ref->\{$s_src\}[$_counter][$_ori] = $v_min;
$ss_ref->\{$s_src\}[$_counter][$_end] = $v_max;
push @sc_coords, $v_min, $v_max;
\nwused{\\{NWgff18-sorF-1}}\nwendcode{}\nwbegindocs{614}\nwdocspar

\nwenddocs{}\nwbegincode{615}\sublabel{NWgff18-setW-1}\nwmargintag{{\nwtagstyle{}\subpageref{NWgff18-setW-1}}}\moddef{set sequence elements boundaries~{\nwtagstyle{}\subpageref{NWgff18-setW-1}}}\endmoddef
$s_ref ->\{$s_seq\}[$_counter][$_ori] = &min(@sc_coords);
$s_ref ->\{$s_seq\}[$_counter][$_end] = &max(@sc_coords);
\nwused{\\{NWgff18-sorH-1}}\nwendcode{}\nwbegindocs{616}\nwdocspar

\nwenddocs{}\nwbegincode{617}\sublabel{NWgff18-SorG-2}\nwmargintag{{\nwtagstyle{}\subpageref{NWgff18-SorG-2}}}\moddef{Sorting Features~{\nwtagstyle{}\subpageref{NWgff18-SorG-1}}}\plusendmoddef
sub sort_by_inputorder() \{
    my $ref = $_[0];
    @\{ $ref \} = map \{ [ $_->[0], $_->[2] ] \}
                sort \{ $a->[1] <=> $b->[1] \}
                map \{ [ $_->[0], $_->[1], $_->[2] ] \} @\{ $ref \};
\} # sort_by_inputorder
\nwendcode{}\nwbegindocs{618}\nwdocspar

\subsctn{Sorting forward/reverse strands}

As we are saving feature boundaries always in forward coords (meaning that feature start is always smaller than feature end, this was already checked by {\tt{}{\char38}check{\char95}coords} which is defined in page~\pageref{func:checkcoords}), we need two sort functions, one for sorting forward features that is slightly different than the one used for sorting reverse ones. The main reason is that we want to avoid feature overlapping upstream (so that an element that ends downstream a previous element but starts within the boundaries of this one do not hide it ), this works from left to right for forward features, but must be done counterclockwise (say here from right to left) on reverse features.

\nwenddocs{}\nwbegincode{619}\sublabel{NWgff18-SorG-3}\nwmargintag{{\nwtagstyle{}\subpageref{NWgff18-SorG-3}}}\moddef{Sorting Features~{\nwtagstyle{}\subpageref{NWgff18-SorG-1}}}\plusendmoddef
sub sort_forward \{
    $a->[0] <=> $b->[0]  # sorting by start
             or
    $b->[1] <=> $a->[1]; # reverse sorting by end if same start
\} # sort_forward
#
sub sort_reverse \{
    $b->[1] <=> $a->[1] # reverse sorting by end
             or
    $a->[0] <=> $b->[0];  # sorting by start if same end
\} # sort_forward
\nwendcode{}\nwbegindocs{620}\nwdocspar

We choose between those two functions (both defining which fields to order and the direction --ascending/descending--), assigning by reference to {\tt{}{\char36}sortfunct} variable which is de-referenced within the {\tt{}sort} functions.

\nwenddocs{}\nwbegincode{621}\sublabel{NWgff18-revZ-1}\nwmargintag{{\nwtagstyle{}\subpageref{NWgff18-revZ-1}}}\moddef{set forward/reverse sorting functon~{\nwtagstyle{}\subpageref{NWgff18-revZ-1}}}\endmoddef
my $sortfunct = ($s_str eq '-') ? \\&sort_reverse : \\&sort_forward;
\nwused{\\{NWgff18-sorI-1}}\nwendcode{}\nwbegindocs{622}\nwdocspar


\newpage

\sctn{Obtaining {\ps} output} %%%%%%%%%%%%%%%%%%%%%%%%%%%%%%%%%%%%%%%%%%%%%%%%%%

\todo{
\item Description of Document Structuring Convention (DSC) and plot PS file structure.
} % todo

\nwenddocs{}\nwbegincode{623}\sublabel{NWgff18-MakH-1}\nwmargintag{{\nwtagstyle{}\subpageref{NWgff18-MakH-1}}}\moddef{Making PS Figures~{\nwtagstyle{}\subpageref{NWgff18-MakH-1}}}\endmoddef
sub make_plot() \{
    &header("WRITING POSTSCRIPT TO STDOUT");

    &ps_header;
    &ps_colors;
    &ps_page_formats;
    &ps_variables;
    &ps_main;

    &ps_plot; 

    &ps_trailer;

    &footer("WRITING POSTSCRIPT FINISHED");
\} # make_plot
\nwused{\\{NWgff18-MaiD-1}}\nwendcode{}\nwbegindocs{624}\nwdocspar

\nwenddocs{}\nwbegincode{625}\sublabel{NWgff18-PosM-1}\nwmargintag{{\nwtagstyle{}\subpageref{NWgff18-PosM-1}}}\moddef{PostScript CODE Chunks~{\nwtagstyle{}\subpageref{NWgff18-PosM-1}}}\endmoddef
\LA{}PostScript PLOT~{\nwtagstyle{}\subpageref{NWgff18-PosF.3-1}}\RA{}
\LA{}PostScript HEADER~{\nwtagstyle{}\subpageref{NWgff18-PosH-1}}\RA{}
\LA{}PostScript COLORS~{\nwtagstyle{}\subpageref{NWgff18-PosH.2-1}}\RA{}
\LA{}PostScript FORMATS~{\nwtagstyle{}\subpageref{NWgff18-PosI-1}}\RA{}
\LA{}PostScript VARS~{\nwtagstyle{}\subpageref{NWgff18-PosF-1}}\RA{}
\LA{}PostScript MAIN~{\nwtagstyle{}\subpageref{NWgff18-PosF.2-1}}\RA{}
\LA{}Postscript OPEN PAGE~{\nwtagstyle{}\subpageref{NWgff18-PosK-1}}\RA{}
\LA{}Postscript CLOSE PAGE~{\nwtagstyle{}\subpageref{NWgff18-PosL-1}}\RA{}
\LA{}Postscript TRAILER~{\nwtagstyle{}\subpageref{NWgff18-PosI.2-1}}\RA{}
\nwused{\\{NWgff18-APL5-1}}\nwendcode{}\nwbegindocs{626}\nwdocspar

\subsctn{Header comments and constants definition}
 
\nwenddocs{}\nwbegincode{627}\sublabel{NWgff18-PosH-1}\nwmargintag{{\nwtagstyle{}\subpageref{NWgff18-PosH-1}}}\moddef{PostScript HEADER~{\nwtagstyle{}\subpageref{NWgff18-PosH-1}}}\endmoddef
sub ps_header() \{
    print STDOUT << "+++HEADER+++";
%!PS-Adobe-3.0
%%Title: title
%%Creator: $PROGRAM
%%Version: $VERSION
%%CreationDate: $DATE
%%For: $USER
%%Pages: 1
%%Orientation: Portrait
%%BoundingBox: 0 0 595 842
%%EndComments
%
\LA{}GNU License PostScript~{\nwtagstyle{}\subpageref{NWgff18-GNUM-1}}\RA{}
%
% $LAST_UPDATE
%
% Report BUGS to: jabril@\\imim.es 
%
%%BeginProlog
%
\LA{}POSTSCRIPT shortnames~{\nwtagstyle{}\subpageref{NWgff18-POSL-1}}\RA{}
\LA{}POSTSCRIPT constants~{\nwtagstyle{}\subpageref{NWgff18-POSK-1}}\RA{}
+++HEADER+++
\} # ps_header
\nwused{\\{NWgff18-PosM-1}}\nwendcode{}\nwbegindocs{628}\nwdocspar

\subsctn{CMYK color definition}

\nwenddocs{}\nwbegincode{629}\sublabel{NWgff18-PosH.2-1}\nwmargintag{{\nwtagstyle{}\subpageref{NWgff18-PosH.2-1}}}\moddef{PostScript COLORS~{\nwtagstyle{}\subpageref{NWgff18-PosH.2-1}}}\endmoddef
sub ps_colors() \{
  my %tmp = ();
  print STDOUT "%% Fixed Color Variables (CMYK)\\n";
  print STDOUT "/colordict ".($colors + 28)." dict def colordict begin %% ".
               $colors." colors + 28 definitions\\n";
  foreach my $key (keys %COLORS) \{ $tmp\{$COLORS\{$key\}->[0]\} = $key \};
  for (my $j = 1; $j <= $colors; $j++) \{ 
      my $name = $tmp\{$j\};
      my $ref = \\$COLORS\{$name\};
      my $cmyk = "$$ref->[1] $$ref->[2] $$ref->[3] $$ref->[4]";
      print STDOUT "/".(&fill_right($name,20," "))."\{ $cmyk \} def\\n";
      \};
  print STDOUT "end %% colordict\\n";
\} # ps_colors
\nwused{\\{NWgff18-PosM-1}}\nwendcode{}\nwbegindocs{630}\nwdocspar

\nwenddocs{}\nwbegincode{631}\sublabel{NWgff18-GloB-E}\nwmargintag{{\nwtagstyle{}\subpageref{NWgff18-GloB-E}}}\moddef{Global Vars~{\nwtagstyle{}\subpageref{NWgff18-GloB-1}}}\plusendmoddef
my $colors = 0;
my %COLORS = (    # [ ColorNUMBER, qw/ CYAN MAGENTA YELLOW BLACK / ]
  \LA{}cmyk colors perl definition~{\nwtagstyle{}\subpageref{NWgff18-cmyR-1}}\RA{}
  ); # %COLORS
\nwendcode{}\nwbegindocs{632}\nwdocspar

\subsctn{Page formats definition}

\nwenddocs{}\nwbegincode{633}\sublabel{NWgff18-PosI-1}\nwmargintag{{\nwtagstyle{}\subpageref{NWgff18-PosI-1}}}\moddef{PostScript FORMATS~{\nwtagstyle{}\subpageref{NWgff18-PosI-1}}}\endmoddef
sub ps_page_formats() \{
  my %tmp = ();
  print STDOUT "%% Paper Sizes (in points)\\n";
  print STDOUT "/pagedict ".($formats + 2)." dict def pagedict begin %% ".
               $formats." formats + 2 definitions\\n";
  foreach my $key (keys %FORMATS) \{ $tmp\{$FORMATS\{$key\}->[0]\} = $key \};
  for (my $j = 1; $j <= $formats; $j++) \{ 
      my $name = $tmp\{$j\};
      my $ref = \\$FORMATS\{$name\};
      my $pgsz = &fill_left($$ref->[1],4," ").&fill_left($$ref->[2],5," ");
      print STDOUT "/pg".(&fill_right($name,10," "))."\{ $pgsz \} def\\n";
      \};
  print STDOUT "end %% pagedict\\n";\}
\nwused{\\{NWgff18-PosM-1}}\nwendcode{}\nwbegindocs{634}\nwdocspar

\nwenddocs{}\nwbegincode{635}\sublabel{NWgff18-GloB-F}\nwmargintag{{\nwtagstyle{}\subpageref{NWgff18-GloB-F}}}\moddef{Global Vars~{\nwtagstyle{}\subpageref{NWgff18-GloB-1}}}\plusendmoddef
my $formats = 0;
my %FORMATS = (   # [ FormatNUMBER, X(short edge), Y(long edge) ]
  \LA{}page sizes perl definition~{\nwtagstyle{}\subpageref{NWgff18-pagQ-1}}\RA{}
  );
\nwendcode{}\nwbegindocs{636}\nwdocspar

\subsctn{Setting up {\ps} variables}

\todo{
\item Planning: 2001/08/28
} % todo
\nwenddocs{}\nwbegincode{637}\sublabel{NWgff18-PosF-1}\nwmargintag{{\nwtagstyle{}\subpageref{NWgff18-PosF-1}}}\moddef{PostScript VARS~{\nwtagstyle{}\subpageref{NWgff18-PosF-1}}}\endmoddef
sub ps_variables() \{
    print STDOUT << '+++PSVARS+++';
%%BeginProcSet: Setting_Vars 1.0 0
%
%%EndProcSet:   Setting_Vars 1.0 0
%
+++PSVARS+++
\} # ps_variables
\nwused{\\{NWgff18-PosM-1}}\nwendcode{}\nwbegindocs{638}\nwdocspar

\subsctn{Main {\ps} functions}

\todo{
\item Page layout must be redefined (taking into account page size too).\\[6cm]
} % todo

\nwenddocs{}\nwbegincode{639}\sublabel{NWgff18-PosF.2-1}\nwmargintag{{\nwtagstyle{}\subpageref{NWgff18-PosF.2-1}}}\moddef{PostScript MAIN~{\nwtagstyle{}\subpageref{NWgff18-PosF.2-1}}}\endmoddef
sub ps_main() \{
    print STDOUT << '+++MAINProcs+++';
\LA{}POSTSCRIPT layout~{\nwtagstyle{}\subpageref{NWgff18-POSH-1}}\RA{}
\LA{}POSTSCRIPT text functions~{\nwtagstyle{}\subpageref{NWgff18-POSP-1}}\RA{}
\LA{}POSTSCRIPT aplotdict~{\nwtagstyle{}\subpageref{NWgff18-POSK.2-1}}\RA{}
\LA{}POSTSCRIPT main function calls~{\nwtagstyle{}\subpageref{NWgff18-POSU-1}}\RA{}
%
%%EndProlog
%
%%BeginSetup
%
% initgraphics
% true setpacking
true setstrokeadjust
0.125 setlinewidth
0 setlinejoin
0 setlinecap
%
%%EndSetup
%
+++MAINProcs+++
\} # ps_main
\nwused{\\{NWgff18-PosM-1}}\nwendcode{}\nwbegindocs{640}\nwdocspar

\subsctn{Plotting data into the figure}

\todo{
\item Planning: 2001/08/29
} % todo
\nwenddocs{}\nwbegincode{641}\sublabel{NWgff18-PosF.3-1}\nwmargintag{{\nwtagstyle{}\subpageref{NWgff18-PosF.3-1}}}\moddef{PostScript PLOT~{\nwtagstyle{}\subpageref{NWgff18-PosF.3-1}}}\endmoddef
sub ps_plot()\{
\} # ps_plot
\nwused{\\{NWgff18-PosM-1}}\nwendcode{}\nwbegindocs{642}\nwdocspar

\subsctn{{\ps} code chunks}

\subsubsctn{Page boundaries}

\nwenddocs{}\nwbegincode{643}\sublabel{NWgff18-PosK-1}\nwmargintag{{\nwtagstyle{}\subpageref{NWgff18-PosK-1}}}\moddef{Postscript OPEN PAGE~{\nwtagstyle{}\subpageref{NWgff18-PosK-1}}}\endmoddef
sub ps_open_page() \{
    print STDOUT << '+++OPEN+++';
%%Page: 1 1
%%BeginPageSetup
%
% Saving current page settings
/pgsave save def
% Setting BGcolor for sheet
Dpage 0 0 bbox S BGcolor scmyk fill R clip newpath
% Setting page-size scale
1 CSF dup F
%%EndPageSetup
%
+++OPEN+++
\} # ps_open_page
\nwused{\\{NWgff18-PosM-1}}\nwendcode{}\nwbegindocs{644}\nwdocspar

\nwenddocs{}\nwbegincode{645}\sublabel{NWgff18-PosL-1}\nwmargintag{{\nwtagstyle{}\subpageref{NWgff18-PosL-1}}}\moddef{Postscript CLOSE PAGE~{\nwtagstyle{}\subpageref{NWgff18-PosL-1}}}\endmoddef
sub ps_close_page() \{
    print STDOUT << '+++CLOSE+++';
%
flgcrd \{ s_credits \} if
grestoreall
pgsave restore
showpage
%
%%PageEND: 1 1
%
+++CLOSE+++
\} # ps_close_page
\nwused{\\{NWgff18-PosM-1}}\nwendcode{}\nwbegindocs{646}\nwdocspar

\subsubsctn{{\ps} end of file}

\nwenddocs{}\nwbegincode{647}\sublabel{NWgff18-PosI.2-1}\nwmargintag{{\nwtagstyle{}\subpageref{NWgff18-PosI.2-1}}}\moddef{Postscript TRAILER~{\nwtagstyle{}\subpageref{NWgff18-PosI.2-1}}}\endmoddef
sub ps_trailer() \{
    print STDOUT << "+++EOF+++";
%%Trailer
%
%%Pages: 1
%%Orientation: Portrait
%%BoundingBox: 0 0 595 842
%%EOF
+++EOF+++
\} # ps_trailer
\nwused{\\{NWgff18-PosM-1}}\nwendcode{}\nwbegindocs{648}\nwdocspar


\newpage

\sctn{CMYK color definition} %%%%%%%%%%%%%%%%%%%%%%%%%%%%%%%%%%%%%%%%%%%%%%%%%%%
\label{sec:CMYKcolordef}

\todo{
\item Move this section and the following (colors and page formats definition) to perl package {\tt{}gfftools::PostScript}.
} % todo

To avoid mistakes when working with three separate files, we define here CMYK color names used by this perl program, their {\LaTeX} definition ('{\tt{}AplotColorDefs.tex}') and the table sumarizing them for the manual ('{\tt{}AplotColorTbl.tex}'). % outlined in page~\pageref{sec:cmykfiles}.

\subsctn{Black and White}

\newcommand{\clrtbl}[5]{\vskip -2.5ex \noindent\hrulefill\cln{#1}\cln{#2}\cln{#3}\cln{#4}\cln{#5}}
\newcommand{\clrtblg}[7]{\vskip -2.5ex \noindent\hrulefill\cln{#1}\cln{#2}\cln{#3}\cln{#4}\cln{#5}\cln{#6}\cln{#7}}

\clrtblg{black}{verydarkgrey}{darkgrey}{grey}{lightgrey}{verylightgrey}{white}

\nwenddocs{}\nwbegincode{649}\sublabel{NWgff18-cmyR-1}\nwmargintag{{\nwtagstyle{}\subpageref{NWgff18-cmyR-1}}}\moddef{cmyk colors perl definition~{\nwtagstyle{}\subpageref{NWgff18-cmyR-1}}}\endmoddef
# black+grey+white
black              => [ ++$colors, qw/ 0.00 0.00 0.00 1.00 / ],
verydarkgrey       => [ ++$colors, qw/ 0.00 0.00 0.00 0.80 / ],
darkgrey           => [ ++$colors, qw/ 0.00 0.00 0.00 0.60 / ],
grey               => [ ++$colors, qw/ 0.00 0.00 0.00 0.40 / ],
lightgrey          => [ ++$colors, qw/ 0.00 0.00 0.00 0.20 / ],
verylightgrey      => [ ++$colors, qw/ 0.00 0.00 0.00 0.10 / ],
white              => [ ++$colors, qw/ 0.00 0.00 0.00 0.00 / ],
\nwalsodefined{\\{NWgff18-cmyR-2}\\{NWgff18-cmyR-3}\\{NWgff18-cmyR-4}\\{NWgff18-cmyR-5}\\{NWgff18-cmyR-6}\\{NWgff18-cmyR-7}\\{NWgff18-cmyR-8}\\{NWgff18-cmyR-9}\\{NWgff18-cmyR-A}\\{NWgff18-cmyR-B}\\{NWgff18-cmyR-C}\\{NWgff18-cmyR-D}}\nwused{\\{NWgff18-GloB-E}}\nwendcode{}\nwbegindocs{650}\nwdocspar
\nwenddocs{}\nwbegincode{651}\sublabel{NWgff18-cmyS-1}\nwmargintag{{\nwtagstyle{}\subpageref{NWgff18-cmyS-1}}}\moddef{cmyk colors latex definition~{\nwtagstyle{}\subpageref{NWgff18-cmyS-1}}}\endmoddef
% black+grey+white
\\definecolor\{black\}              \{cmyk\}\{0.00,0.00,0.00,1.00\}
\\definecolor\{verydarkgrey\}       \{cmyk\}\{0.00,0.00,0.00,0.80\}
\\definecolor\{darkgrey\}           \{cmyk\}\{0.00,0.00,0.00,0.60\}
\\definecolor\{grey\}               \{cmyk\}\{0.00,0.00,0.00,0.40\}
\\definecolor\{lightgrey\}          \{cmyk\}\{0.00,0.00,0.00,0.20\}
\\definecolor\{verylightgrey\}      \{cmyk\}\{0.00,0.00,0.00,0.10\}
\\definecolor\{white\}              \{cmyk\}\{0.00,0.00,0.00,0.00\}
\nwalsodefined{\\{NWgff18-cmyS-2}\\{NWgff18-cmyS-3}\\{NWgff18-cmyS-4}\\{NWgff18-cmyS-5}\\{NWgff18-cmyS-6}\\{NWgff18-cmyS-7}\\{NWgff18-cmyS-8}\\{NWgff18-cmyS-9}\\{NWgff18-cmyS-A}\\{NWgff18-cmyS-B}\\{NWgff18-cmyS-C}\\{NWgff18-cmyS-D}}\nwused{\\{NWgff18-LATH.2-1}}\nwendcode{}\nwbegindocs{652}\nwdocspar
\nwenddocs{}\nwbegincode{653}\sublabel{NWgff18-cmyN-1}\nwmargintag{{\nwtagstyle{}\subpageref{NWgff18-cmyN-1}}}\moddef{cmyk color table - left~{\nwtagstyle{}\subpageref{NWgff18-cmyN-1}}}\endmoddef
% black+grey+white
\\clspc
\\clrow\{black\}                    & 0.00 & 0.00 & 0.00 & 1.00 \\\\
\\clrow\{verydarkgrey\}             & 0.00 & 0.00 & 0.00 & 0.80 \\\\
\\clrow\{darkgrey\}                 & 0.00 & 0.00 & 0.00 & 0.60 \\\\
\\clrow\{grey\}                     & 0.00 & 0.00 & 0.00 & 0.40 \\\\
\\clrow\{lightgrey\}                & 0.00 & 0.00 & 0.00 & 0.20 \\\\
\\clrow\{verylightgrey\}            & 0.00 & 0.00 & 0.00 & 0.10 \\\\
\\clrow\{white\}                    & 0.00 & 0.00 & 0.00 & 0.00 \\\\
\nwalsodefined{\\{NWgff18-cmyN-2}\\{NWgff18-cmyN-3}\\{NWgff18-cmyN-4}\\{NWgff18-cmyN-5}\\{NWgff18-cmyN-6}}\nwused{\\{NWgff18-LATI.2-1}}\nwendcode{}\nwbegindocs{654}\nwdocspar

\subsctn{Magenta}

\clrtbl{verydarkmagenta}{darkmagenta}{magenta}{lightmagenta}{verylightmagenta}

\nwenddocs{}\nwbegincode{655}\sublabel{NWgff18-cmyR-2}\nwmargintag{{\nwtagstyle{}\subpageref{NWgff18-cmyR-2}}}\moddef{cmyk colors perl definition~{\nwtagstyle{}\subpageref{NWgff18-cmyR-1}}}\plusendmoddef
# magenta                 
verydarkmagenta    => [ ++$colors, qw/ 0.00 1.00 0.00 0.30 / ],
darkmagenta        => [ ++$colors, qw/ 0.00 0.80 0.00 0.05 / ],
magenta            => [ ++$colors, qw/ 0.00 0.60 0.00 0.00 / ],
lightmagenta       => [ ++$colors, qw/ 0.00 0.40 0.00 0.00 / ],
verylightmagenta   => [ ++$colors, qw/ 0.00 0.20 0.00 0.00 / ],
\nwendcode{}\nwbegindocs{656}\nwdocspar
\nwenddocs{}\nwbegincode{657}\sublabel{NWgff18-cmyS-2}\nwmargintag{{\nwtagstyle{}\subpageref{NWgff18-cmyS-2}}}\moddef{cmyk colors latex definition~{\nwtagstyle{}\subpageref{NWgff18-cmyS-1}}}\plusendmoddef
% magenta
\\definecolor\{verydarkmagenta\}    \{cmyk\}\{0.00,1.00,0.00,0.30\}
\\definecolor\{darkmagenta\}        \{cmyk\}\{0.00,0.80,0.00,0.05\}
\\definecolor\{magenta\}            \{cmyk\}\{0.00,0.60,0.00,0.00\}
\\definecolor\{lightmagenta\}       \{cmyk\}\{0.00,0.40,0.00,0.00\}
\\definecolor\{verylightmagenta\}   \{cmyk\}\{0.00,0.20,0.00,0.00\}
\nwendcode{}\nwbegindocs{658}\nwdocspar
\nwenddocs{}\nwbegincode{659}\sublabel{NWgff18-cmyN-2}\nwmargintag{{\nwtagstyle{}\subpageref{NWgff18-cmyN-2}}}\moddef{cmyk color table - left~{\nwtagstyle{}\subpageref{NWgff18-cmyN-1}}}\plusendmoddef
% magenta
\\clspc
\\clrow\{verydarkmagenta\}          & 0.00 & 1.00 & 0.00 & 0.30 \\\\
\\clrow\{darkmagenta\}              & 0.00 & 0.80 & 0.00 & 0.05 \\\\
\\clrow\{magenta\}                  & 0.00 & 0.60 & 0.00 & 0.00 \\\\
\\clrow\{lightmagenta\}             & 0.00 & 0.40 & 0.00 & 0.00 \\\\
\\clrow\{verylightmagenta\}         & 0.00 & 0.20 & 0.00 & 0.00 \\\\
\nwendcode{}\nwbegindocs{660}\nwdocspar

\subsctn{Violet}

\clrtbl{verydarkviolet}{darkviolet}{violet}{lightviolet}{verylightviolet}

\nwenddocs{}\nwbegincode{661}\sublabel{NWgff18-cmyR-3}\nwmargintag{{\nwtagstyle{}\subpageref{NWgff18-cmyR-3}}}\moddef{cmyk colors perl definition~{\nwtagstyle{}\subpageref{NWgff18-cmyR-1}}}\plusendmoddef
# violet                  
verydarkviolet     => [ ++$colors, qw/ 0.45 0.85 0.00 0.00 / ],
darkviolet         => [ ++$colors, qw/ 0.30 0.65 0.00 0.00 / ],
violet             => [ ++$colors, qw/ 0.22 0.55 0.00 0.00 / ],
lightviolet        => [ ++$colors, qw/ 0.15 0.40 0.00 0.00 / ],
verylightviolet    => [ ++$colors, qw/ 0.10 0.20 0.00 0.00 / ],
\nwendcode{}\nwbegindocs{662}\nwdocspar
\nwenddocs{}\nwbegincode{663}\sublabel{NWgff18-cmyS-3}\nwmargintag{{\nwtagstyle{}\subpageref{NWgff18-cmyS-3}}}\moddef{cmyk colors latex definition~{\nwtagstyle{}\subpageref{NWgff18-cmyS-1}}}\plusendmoddef
% violet
\\definecolor\{verydarkviolet\}     \{cmyk\}\{0.45,0.85,0.00,0.00\}
\\definecolor\{darkviolet\}         \{cmyk\}\{0.30,0.65,0.00,0.00\}
\\definecolor\{violet\}             \{cmyk\}\{0.22,0.55,0.00,0.00\}
\\definecolor\{lightviolet\}        \{cmyk\}\{0.15,0.40,0.00,0.00\}
\\definecolor\{verylightviolet\}    \{cmyk\}\{0.10,0.20,0.00,0.00\}
\nwendcode{}\nwbegindocs{664}\nwdocspar
\nwenddocs{}\nwbegincode{665}\sublabel{NWgff18-cmyN-3}\nwmargintag{{\nwtagstyle{}\subpageref{NWgff18-cmyN-3}}}\moddef{cmyk color table - left~{\nwtagstyle{}\subpageref{NWgff18-cmyN-1}}}\plusendmoddef
% violet
\\clspc
\\clrow\{verydarkviolet\}           & 0.45 & 0.85 & 0.00 & 0.00 \\\\
\\clrow\{darkviolet\}               & 0.30 & 0.65 & 0.00 & 0.00 \\\\
\\clrow\{violet\}                   & 0.22 & 0.55 & 0.00 & 0.00 \\\\
\\clrow\{lightviolet\}              & 0.15 & 0.40 & 0.00 & 0.00 \\\\
\\clrow\{verylightviolet\}          & 0.10 & 0.20 & 0.00 & 0.00 \\\\
\nwendcode{}\nwbegindocs{666}\nwdocspar

\subsctn{Blue}

\clrtbl{verydarkblue}{darkblue}{blue}{lightblue}{verylightblue}

\nwenddocs{}\nwbegincode{667}\sublabel{NWgff18-cmyR-4}\nwmargintag{{\nwtagstyle{}\subpageref{NWgff18-cmyR-4}}}\moddef{cmyk colors perl definition~{\nwtagstyle{}\subpageref{NWgff18-cmyR-1}}}\plusendmoddef
# blue                
verydarkblue       => [ ++$colors, qw/ 1.00 1.00 0.00 0.20 / ],
darkblue           => [ ++$colors, qw/ 0.90 0.90 0.00 0.00 / ],
blue               => [ ++$colors, qw/ 0.75 0.75 0.00 0.00 / ],
lightblue          => [ ++$colors, qw/ 0.50 0.50 0.00 0.00 / ],
verylightblue      => [ ++$colors, qw/ 0.30 0.30 0.00 0.00 / ],
\nwendcode{}\nwbegindocs{668}\nwdocspar
\nwenddocs{}\nwbegincode{669}\sublabel{NWgff18-cmyS-4}\nwmargintag{{\nwtagstyle{}\subpageref{NWgff18-cmyS-4}}}\moddef{cmyk colors latex definition~{\nwtagstyle{}\subpageref{NWgff18-cmyS-1}}}\plusendmoddef
% blue
\\definecolor\{verydarkblue\}       \{cmyk\}\{1.00,1.00,0.00,0.20\}
\\definecolor\{darkblue\}           \{cmyk\}\{0.90,0.90,0.00,0.00\}
\\definecolor\{blue\}               \{cmyk\}\{0.75,0.75,0.00,0.00\}
\\definecolor\{lightblue\}          \{cmyk\}\{0.50,0.50,0.00,0.00\}
\\definecolor\{verylightblue\}      \{cmyk\}\{0.30,0.30,0.00,0.00\}
\nwendcode{}\nwbegindocs{670}\nwdocspar
\nwenddocs{}\nwbegincode{671}\sublabel{NWgff18-cmyN-4}\nwmargintag{{\nwtagstyle{}\subpageref{NWgff18-cmyN-4}}}\moddef{cmyk color table - left~{\nwtagstyle{}\subpageref{NWgff18-cmyN-1}}}\plusendmoddef
% blue
\\clspc
\\clrow\{verydarkblue\}             & 1.00 & 1.00 & 0.00 & 0.20 \\\\
\\clrow\{darkblue\}                 & 0.90 & 0.90 & 0.00 & 0.00 \\\\
\\clrow\{blue\}                     & 0.75 & 0.75 & 0.00 & 0.00 \\\\
\\clrow\{lightblue\}                & 0.50 & 0.50 & 0.00 & 0.00 \\\\
\\clrow\{verylightblue\}            & 0.30 & 0.30 & 0.00 & 0.00 \\\\
\nwendcode{}\nwbegindocs{672}\nwdocspar

\subsctn{Skyblue}

\clrtbl{verydarkskyblue}{darkskyblue}{skyblue}{lightskyblue}{verylightskyblue}

\nwenddocs{}\nwbegincode{673}\sublabel{NWgff18-cmyR-5}\nwmargintag{{\nwtagstyle{}\subpageref{NWgff18-cmyR-5}}}\moddef{cmyk colors perl definition~{\nwtagstyle{}\subpageref{NWgff18-cmyR-1}}}\plusendmoddef
# skyblue                 
verydarkskyblue    => [ ++$colors, qw/ 0.90 0.50 0.00 0.15 / ],
darkskyblue        => [ ++$colors, qw/ 0.75 0.45 0.00 0.00 / ],
skyblue            => [ ++$colors, qw/ 0.60 0.38 0.00 0.00 / ],
lightskyblue       => [ ++$colors, qw/ 0.45 0.25 0.00 0.00 / ],
verylightskyblue   => [ ++$colors, qw/ 0.30 0.15 0.00 0.00 / ],
\nwendcode{}\nwbegindocs{674}\nwdocspar
\nwenddocs{}\nwbegincode{675}\sublabel{NWgff18-cmyS-5}\nwmargintag{{\nwtagstyle{}\subpageref{NWgff18-cmyS-5}}}\moddef{cmyk colors latex definition~{\nwtagstyle{}\subpageref{NWgff18-cmyS-1}}}\plusendmoddef
% skyblue
\\definecolor\{verydarkskyblue\}    \{cmyk\}\{0.90,0.50,0.00,0.15\}
\\definecolor\{darkskyblue\}        \{cmyk\}\{0.75,0.45,0.00,0.00\}
\\definecolor\{skyblue\}            \{cmyk\}\{0.60,0.38,0.00,0.00\}
\\definecolor\{lightskyblue\}       \{cmyk\}\{0.45,0.25,0.00,0.00\}
\\definecolor\{verylightskyblue\}   \{cmyk\}\{0.30,0.15,0.00,0.00\}
\nwendcode{}\nwbegindocs{676}\nwdocspar
\nwenddocs{}\nwbegincode{677}\sublabel{NWgff18-cmyN-5}\nwmargintag{{\nwtagstyle{}\subpageref{NWgff18-cmyN-5}}}\moddef{cmyk color table - left~{\nwtagstyle{}\subpageref{NWgff18-cmyN-1}}}\plusendmoddef
% skyblue
\\clspc
\\clrow\{verydarkskyblue\}          & 0.90 & 0.50 & 0.00 & 0.15 \\\\
\\clrow\{darkskyblue\}              & 0.75 & 0.45 & 0.00 & 0.00 \\\\
\\clrow\{skyblue\}                  & 0.60 & 0.38 & 0.00 & 0.00 \\\\
\\clrow\{lightskyblue\}             & 0.45 & 0.25 & 0.00 & 0.00 \\\\
\\clrow\{verylightskyblue\}         & 0.30 & 0.15 & 0.00 & 0.00 \\\\
\nwendcode{}\nwbegindocs{678}\nwdocspar

\subsctn{Cyan}

\clrtbl{verydarkcyan}{darkcyan}{cyan}{lightcyan}{verylightcyan}

\nwenddocs{}\nwbegincode{679}\sublabel{NWgff18-cmyR-6}\nwmargintag{{\nwtagstyle{}\subpageref{NWgff18-cmyR-6}}}\moddef{cmyk colors perl definition~{\nwtagstyle{}\subpageref{NWgff18-cmyR-1}}}\plusendmoddef
# cyan                
verydarkcyan       => [ ++$colors, qw/ 1.00 0.00 0.00 0.10 / ],
darkcyan           => [ ++$colors, qw/ 0.80 0.00 0.00 0.00 / ],
cyan               => [ ++$colors, qw/ 0.60 0.00 0.00 0.00 / ],
lightcyan          => [ ++$colors, qw/ 0.40 0.00 0.00 0.00 / ],
verylightcyan      => [ ++$colors, qw/ 0.20 0.00 0.00 0.00 / ],
\nwendcode{}\nwbegindocs{680}\nwdocspar
\nwenddocs{}\nwbegincode{681}\sublabel{NWgff18-cmyS-6}\nwmargintag{{\nwtagstyle{}\subpageref{NWgff18-cmyS-6}}}\moddef{cmyk colors latex definition~{\nwtagstyle{}\subpageref{NWgff18-cmyS-1}}}\plusendmoddef
% cyan
\\definecolor\{verydarkcyan\}       \{cmyk\}\{1.00,0.00,0.00,0.10\}
\\definecolor\{darkcyan\}           \{cmyk\}\{0.80,0.00,0.00,0.00\}
\\definecolor\{cyan\}               \{cmyk\}\{0.60,0.00,0.00,0.00\}
\\definecolor\{lightcyan\}          \{cmyk\}\{0.40,0.00,0.00,0.00\}
\\definecolor\{verylightcyan\}      \{cmyk\}\{0.20,0.00,0.00,0.00\}
\nwendcode{}\nwbegindocs{682}\nwdocspar
\nwenddocs{}\nwbegincode{683}\sublabel{NWgff18-cmyN-6}\nwmargintag{{\nwtagstyle{}\subpageref{NWgff18-cmyN-6}}}\moddef{cmyk color table - left~{\nwtagstyle{}\subpageref{NWgff18-cmyN-1}}}\plusendmoddef
% cyan
\\clspc
\\clrow\{verydarkcyan\}             & 1.00 & 0.00 & 0.00 & 0.10 \\\\
\\clrow\{darkcyan\}                 & 0.80 & 0.00 & 0.00 & 0.00 \\\\
\\clrow\{cyan\}                     & 0.60 & 0.00 & 0.00 & 0.00 \\\\
\\clrow\{lightcyan\}                & 0.40 & 0.00 & 0.00 & 0.00 \\\\
\\clrow\{verylightcyan\}            & 0.20 & 0.00 & 0.00 & 0.00 \\\\
\nwendcode{}\nwbegindocs{684}\nwdocspar

\subsctn{Seagreen}

\clrtbl{verydarkseagreen}{darkseagreen}{seagreen}{lightseagreen}{verylightseagreen}

\nwenddocs{}\nwbegincode{685}\sublabel{NWgff18-cmyR-7}\nwmargintag{{\nwtagstyle{}\subpageref{NWgff18-cmyR-7}}}\moddef{cmyk colors perl definition~{\nwtagstyle{}\subpageref{NWgff18-cmyR-1}}}\plusendmoddef
# seagreen            
verydarkseagreen   => [ ++$colors, qw/ 0.75 0.00 0.45 0.00 / ],
darkseagreen       => [ ++$colors, qw/ 0.62 0.00 0.38 0.00 / ],
seagreen           => [ ++$colors, qw/ 0.50 0.00 0.30 0.00 / ],
lightseagreen      => [ ++$colors, qw/ 0.38 0.00 0.22 0.00 / ],
verylightseagreen  => [ ++$colors, qw/ 0.25 0.00 0.15 0.00 / ],
\nwendcode{}\nwbegindocs{686}\nwdocspar
\nwenddocs{}\nwbegincode{687}\sublabel{NWgff18-cmyS-7}\nwmargintag{{\nwtagstyle{}\subpageref{NWgff18-cmyS-7}}}\moddef{cmyk colors latex definition~{\nwtagstyle{}\subpageref{NWgff18-cmyS-1}}}\plusendmoddef
% seagreen
\\definecolor\{verydarkseagreen\}   \{cmyk\}\{0.75,0.00,0.45,0.00\}
\\definecolor\{darkseagreen\}       \{cmyk\}\{0.62,0.00,0.38,0.00\}
\\definecolor\{seagreen\}           \{cmyk\}\{0.50,0.00,0.30,0.00\}
\\definecolor\{lightseagreen\}      \{cmyk\}\{0.38,0.00,0.22,0.00\}
\\definecolor\{verylightseagreen\}  \{cmyk\}\{0.25,0.00,0.15,0.00\}
\nwendcode{}\nwbegindocs{688}\nwdocspar
\nwenddocs{}\nwbegincode{689}\sublabel{NWgff18-cmyO-1}\nwmargintag{{\nwtagstyle{}\subpageref{NWgff18-cmyO-1}}}\moddef{cmyk color table - right~{\nwtagstyle{}\subpageref{NWgff18-cmyO-1}}}\endmoddef
% seagreen
\\clspc
\\clrow\{verydarkseagreen\}         & 0.75 & 0.00 & 0.45 & 0.00 \\\\
\\clrow\{darkseagreen\}             & 0.62 & 0.00 & 0.38 & 0.00 \\\\
\\clrow\{seagreen\}                 & 0.50 & 0.00 & 0.30 & 0.00 \\\\
\\clrow\{lightseagreen\}            & 0.38 & 0.00 & 0.22 & 0.00 \\\\
\\clrow\{verylightseagreen\}        & 0.25 & 0.00 & 0.15 & 0.00 \\\\
\nwalsodefined{\\{NWgff18-cmyO-2}\\{NWgff18-cmyO-3}\\{NWgff18-cmyO-4}\\{NWgff18-cmyO-5}\\{NWgff18-cmyO-6}\\{NWgff18-cmyO-7}}\nwused{\\{NWgff18-LATI.2-1}}\nwendcode{}\nwbegindocs{690}\nwdocspar

\subsctn{Green}

\clrtbl{verydarkgreen}{darkgreen}{green}{lightgreen}{verylightgreen}

\nwenddocs{}\nwbegincode{691}\sublabel{NWgff18-cmyR-8}\nwmargintag{{\nwtagstyle{}\subpageref{NWgff18-cmyR-8}}}\moddef{cmyk colors perl definition~{\nwtagstyle{}\subpageref{NWgff18-cmyR-1}}}\plusendmoddef
# green               
verydarkgreen      => [ ++$colors, qw/ 1.00 0.00 1.00 0.25 / ],
darkgreen          => [ ++$colors, qw/ 0.80 0.00 0.80 0.00 / ],
green              => [ ++$colors, qw/ 0.60 0.00 0.60 0.00 / ],
lightgreen         => [ ++$colors, qw/ 0.40 0.00 0.40 0.00 / ],
verylightgreen     => [ ++$colors, qw/ 0.20 0.00 0.20 0.00 / ],
\nwendcode{}\nwbegindocs{692}\nwdocspar
\nwenddocs{}\nwbegincode{693}\sublabel{NWgff18-cmyS-8}\nwmargintag{{\nwtagstyle{}\subpageref{NWgff18-cmyS-8}}}\moddef{cmyk colors latex definition~{\nwtagstyle{}\subpageref{NWgff18-cmyS-1}}}\plusendmoddef
% green
\\definecolor\{verydarkgreen\}      \{cmyk\}\{1.00,0.00,1.00,0.25\}
\\definecolor\{darkgreen\}          \{cmyk\}\{0.80,0.00,0.80,0.00\}
\\definecolor\{green\}              \{cmyk\}\{0.60,0.00,0.60,0.00\}
\\definecolor\{lightgreen\}         \{cmyk\}\{0.40,0.00,0.40,0.00\}
\\definecolor\{verylightgreen\}     \{cmyk\}\{0.20,0.00,0.20,0.00\}
\nwendcode{}\nwbegindocs{694}\nwdocspar
\nwenddocs{}\nwbegincode{695}\sublabel{NWgff18-cmyO-2}\nwmargintag{{\nwtagstyle{}\subpageref{NWgff18-cmyO-2}}}\moddef{cmyk color table - right~{\nwtagstyle{}\subpageref{NWgff18-cmyO-1}}}\plusendmoddef
% green
\\clspc
\\clrow\{verydarkgreen\}            & 1.00 & 0.00 & 1.00 & 0.25 \\\\
\\clrow\{darkgreen\}                & 0.80 & 0.00 & 0.80 & 0.00 \\\\
\\clrow\{green\}                    & 0.60 & 0.00 & 0.60 & 0.00 \\\\
\\clrow\{lightgreen\}               & 0.40 & 0.00 & 0.40 & 0.00 \\\\
\\clrow\{verylightgreen\}           & 0.20 & 0.00 & 0.20 & 0.00 \\\\
\nwendcode{}\nwbegindocs{696}\nwdocspar

\subsctn{Limegreen}

\clrtbl{verydarklimegreen}{darklimegreen}{limegreen}{lightlimegreen}{verylightlimegreen}

\nwenddocs{}\nwbegincode{697}\sublabel{NWgff18-cmyR-9}\nwmargintag{{\nwtagstyle{}\subpageref{NWgff18-cmyR-9}}}\moddef{cmyk colors perl definition~{\nwtagstyle{}\subpageref{NWgff18-cmyR-1}}}\plusendmoddef
# limegreen           
verydarklimegreen  => [ ++$colors, qw/ 0.50 0.00 1.00 0.10 / ],
darklimegreen      => [ ++$colors, qw/ 0.40 0.00 0.95 0.00 / ],
limegreen          => [ ++$colors, qw/ 0.30 0.00 0.80 0.00 / ],
lightlimegreen     => [ ++$colors, qw/ 0.20 0.00 0.65 0.00 / ],
verylightlimegreen => [ ++$colors, qw/ 0.10 0.00 0.50 0.00 / ],
\nwendcode{}\nwbegindocs{698}\nwdocspar
\nwenddocs{}\nwbegincode{699}\sublabel{NWgff18-cmyS-9}\nwmargintag{{\nwtagstyle{}\subpageref{NWgff18-cmyS-9}}}\moddef{cmyk colors latex definition~{\nwtagstyle{}\subpageref{NWgff18-cmyS-1}}}\plusendmoddef
% limegreen
\\definecolor\{verydarklimegreen\}  \{cmyk\}\{0.50,0.00,1.00,0.10\}
\\definecolor\{darklimegreen\}      \{cmyk\}\{0.40,0.00,0.95,0.00\}
\\definecolor\{limegreen\}          \{cmyk\}\{0.30,0.00,0.80,0.00\}
\\definecolor\{lightlimegreen\}     \{cmyk\}\{0.20,0.00,0.65,0.00\}
\\definecolor\{verylightlimegreen\} \{cmyk\}\{0.10,0.00,0.50,0.00\}
\nwendcode{}\nwbegindocs{700}\nwdocspar
\nwenddocs{}\nwbegincode{701}\sublabel{NWgff18-cmyO-3}\nwmargintag{{\nwtagstyle{}\subpageref{NWgff18-cmyO-3}}}\moddef{cmyk color table - right~{\nwtagstyle{}\subpageref{NWgff18-cmyO-1}}}\plusendmoddef
% limegreen
\\clspc
\\clrow\{verydarklimegreen\}        & 0.50 & 0.00 & 1.00 & 0.10 \\\\
\\clrow\{darklimegreen\}            & 0.40 & 0.00 & 0.95 & 0.00 \\\\
\\clrow\{limegreen\}                & 0.30 & 0.00 & 0.80 & 0.00 \\\\
\\clrow\{lightlimegreen\}           & 0.20 & 0.00 & 0.65 & 0.00 \\\\
\\clrow\{verylightlimegreen\}       & 0.10 & 0.00 & 0.50 & 0.00 \\\\
\nwendcode{}\nwbegindocs{702}\nwdocspar

\subsctn{Yellow}

\clrtbl{verydarkyellow}{darkyellow}{yellow}{lightyellow}{verylightyellow}

\nwenddocs{}\nwbegincode{703}\sublabel{NWgff18-cmyR-A}\nwmargintag{{\nwtagstyle{}\subpageref{NWgff18-cmyR-A}}}\moddef{cmyk colors perl definition~{\nwtagstyle{}\subpageref{NWgff18-cmyR-1}}}\plusendmoddef
# yellow                  
verydarkyellow     => [ ++$colors, qw/ 0.00 0.00 1.00 0.25 / ],
darkyellow         => [ ++$colors, qw/ 0.00 0.00 1.00 0.10 / ],
yellow             => [ ++$colors, qw/ 0.00 0.00 1.00 0.00 / ],
lightyellow        => [ ++$colors, qw/ 0.00 0.00 0.50 0.00 / ],
verylightyellow    => [ ++$colors, qw/ 0.00 0.00 0.25 0.00 / ],
\nwendcode{}\nwbegindocs{704}\nwdocspar
\nwenddocs{}\nwbegincode{705}\sublabel{NWgff18-cmyS-A}\nwmargintag{{\nwtagstyle{}\subpageref{NWgff18-cmyS-A}}}\moddef{cmyk colors latex definition~{\nwtagstyle{}\subpageref{NWgff18-cmyS-1}}}\plusendmoddef
% yellow
\\definecolor\{verydarkyellow\}     \{cmyk\}\{0.00,0.00,1.00,0.25\}
\\definecolor\{darkyellow\}         \{cmyk\}\{0.00,0.00,1.00,0.10\}
\\definecolor\{yellow\}             \{cmyk\}\{0.00,0.00,1.00,0.00\}
\\definecolor\{lightyellow\}        \{cmyk\}\{0.00,0.00,0.50,0.00\}
\\definecolor\{verylightyellow\}    \{cmyk\}\{0.00,0.00,0.25,0.00\}
\nwendcode{}\nwbegindocs{706}\nwdocspar
\nwenddocs{}\nwbegincode{707}\sublabel{NWgff18-cmyO-4}\nwmargintag{{\nwtagstyle{}\subpageref{NWgff18-cmyO-4}}}\moddef{cmyk color table - right~{\nwtagstyle{}\subpageref{NWgff18-cmyO-1}}}\plusendmoddef
% yellow
\\clspc
\\clrow\{verydarkyellow\}           & 0.00 & 0.00 & 1.00 & 0.25 \\\\
\\clrow\{darkyellow\}               & 0.00 & 0.00 & 1.00 & 0.10 \\\\
\\clrow\{yellow\}                   & 0.00 & 0.00 & 1.00 & 0.00 \\\\
\\clrow\{lightyellow\}              & 0.00 & 0.00 & 0.50 & 0.00 \\\\
\\clrow\{verylightyellow\}          & 0.00 & 0.00 & 0.25 & 0.00 \\\\
\nwendcode{}\nwbegindocs{708}\nwdocspar

\subsctn{Orange}

\clrtbl{verydarkorange}{darkorange}{orange}{lightorange}{verylightorange}

\nwenddocs{}\nwbegincode{709}\sublabel{NWgff18-cmyR-B}\nwmargintag{{\nwtagstyle{}\subpageref{NWgff18-cmyR-B}}}\moddef{cmyk colors perl definition~{\nwtagstyle{}\subpageref{NWgff18-cmyR-1}}}\plusendmoddef
# orange                  
verydarkorange     => [ ++$colors, qw/ 0.00 0.50 0.80 0.10 / ],
darkorange         => [ ++$colors, qw/ 0.00 0.40 0.80 0.00 / ],
orange             => [ ++$colors, qw/ 0.00 0.30 0.80 0.00 / ],
lightorange        => [ ++$colors, qw/ 0.00 0.20 0.75 0.00 / ],
verylightorange    => [ ++$colors, qw/ 0.00 0.15 0.70 0.00 / ],
\nwendcode{}\nwbegindocs{710}\nwdocspar
\nwenddocs{}\nwbegincode{711}\sublabel{NWgff18-cmyS-B}\nwmargintag{{\nwtagstyle{}\subpageref{NWgff18-cmyS-B}}}\moddef{cmyk colors latex definition~{\nwtagstyle{}\subpageref{NWgff18-cmyS-1}}}\plusendmoddef
% orange
\\definecolor\{verydarkorange\}     \{cmyk\}\{0.00,0.50,0.80,0.10\}
\\definecolor\{darkorange\}         \{cmyk\}\{0.00,0.40,0.80,0.00\}
\\definecolor\{orange\}             \{cmyk\}\{0.00,0.30,0.80,0.00\}
\\definecolor\{lightorange\}        \{cmyk\}\{0.00,0.20,0.75,0.00\}
\\definecolor\{verylightorange\}    \{cmyk\}\{0.00,0.15,0.70,0.00\}
\nwendcode{}\nwbegindocs{712}\nwdocspar
\nwenddocs{}\nwbegincode{713}\sublabel{NWgff18-cmyO-5}\nwmargintag{{\nwtagstyle{}\subpageref{NWgff18-cmyO-5}}}\moddef{cmyk color table - right~{\nwtagstyle{}\subpageref{NWgff18-cmyO-1}}}\plusendmoddef
% orange
\\clspc
\\clrow\{verydarkorange\}           & 0.00 & 0.50 & 0.80 & 0.10 \\\\
\\clrow\{darkorange\}               & 0.00 & 0.40 & 0.80 & 0.00 \\\\
\\clrow\{orange\}                   & 0.00 & 0.30 & 0.80 & 0.00 \\\\
\\clrow\{lightorange\}              & 0.00 & 0.20 & 0.75 & 0.00 \\\\
\\clrow\{verylightorange\}          & 0.00 & 0.15 & 0.70 & 0.00 \\\\
\nwendcode{}\nwbegindocs{714}\nwdocspar

\subsctn{Red}

\clrtbl{verydarkred}{darkred}{red}{lightred}{verylightred}

\nwenddocs{}\nwbegincode{715}\sublabel{NWgff18-cmyR-C}\nwmargintag{{\nwtagstyle{}\subpageref{NWgff18-cmyR-C}}}\moddef{cmyk colors perl definition~{\nwtagstyle{}\subpageref{NWgff18-cmyR-1}}}\plusendmoddef
# red                     
verydarkred        => [ ++$colors, qw/ 0.00 1.00 1.00 0.15 / ],
darkred            => [ ++$colors, qw/ 0.00 0.80 0.80 0.00 / ],
red                => [ ++$colors, qw/ 0.00 0.60 0.60 0.00 / ],
lightred           => [ ++$colors, qw/ 0.00 0.40 0.40 0.00 / ],
verylightred       => [ ++$colors, qw/ 0.00 0.20 0.20 0.00 / ],
\nwendcode{}\nwbegindocs{716}\nwdocspar
\nwenddocs{}\nwbegincode{717}\sublabel{NWgff18-cmyS-C}\nwmargintag{{\nwtagstyle{}\subpageref{NWgff18-cmyS-C}}}\moddef{cmyk colors latex definition~{\nwtagstyle{}\subpageref{NWgff18-cmyS-1}}}\plusendmoddef
% red
\\definecolor\{verydarkred\}        \{cmyk\}\{0.00,1.00,1.00,0.15\}
\\definecolor\{darkred\}            \{cmyk\}\{0.00,0.80,0.80,0.00\}
\\definecolor\{red\}                \{cmyk\}\{0.00,0.60,0.60,0.00\}
\\definecolor\{lightred\}           \{cmyk\}\{0.00,0.40,0.40,0.00\}
\\definecolor\{verylightred\}       \{cmyk\}\{0.00,0.20,0.20,0.00\}
\nwendcode{}\nwbegindocs{718}\nwdocspar
\nwenddocs{}\nwbegincode{719}\sublabel{NWgff18-cmyO-6}\nwmargintag{{\nwtagstyle{}\subpageref{NWgff18-cmyO-6}}}\moddef{cmyk color table - right~{\nwtagstyle{}\subpageref{NWgff18-cmyO-1}}}\plusendmoddef
% red
\\clspc
\\clrow\{verydarkred\}              & 0.00 & 1.00 & 1.00 & 0.15 \\\\
\\clrow\{darkred\}                  & 0.00 & 0.80 & 0.80 & 0.00 \\\\
\\clrow\{red\}                      & 0.00 & 0.60 & 0.60 & 0.00 \\\\
\\clrow\{lightred\}                 & 0.00 & 0.40 & 0.40 & 0.00 \\\\
\\clrow\{verylightred\}             & 0.00 & 0.20 & 0.20 & 0.00 \\\\
\nwendcode{}\nwbegindocs{720}\nwdocspar

\subsctn{Brown}

\clrtbl{verydarkbrown}{darkbrown}{brown}{lightbrown}{verylightbrown}

\nwenddocs{}\nwbegincode{721}\sublabel{NWgff18-cmyR-D}\nwmargintag{{\nwtagstyle{}\subpageref{NWgff18-cmyR-D}}}\moddef{cmyk colors perl definition~{\nwtagstyle{}\subpageref{NWgff18-cmyR-1}}}\plusendmoddef
# brown               
verydarkbrown      => [ ++$colors, qw/ 0.35 0.85 1.00 0.40 / ],
darkbrown          => [ ++$colors, qw/ 0.30 0.70 1.00 0.35 / ],
brown              => [ ++$colors, qw/ 0.25 0.75 1.00 0.25 / ],
lightbrown         => [ ++$colors, qw/ 0.20 0.60 0.70 0.15 / ],
verylightbrown     => [ ++$colors, qw/ 0.15 0.45 0.55 0.00 / ],
\nwendcode{}\nwbegindocs{722}\nwdocspar
\nwenddocs{}\nwbegincode{723}\sublabel{NWgff18-cmyS-D}\nwmargintag{{\nwtagstyle{}\subpageref{NWgff18-cmyS-D}}}\moddef{cmyk colors latex definition~{\nwtagstyle{}\subpageref{NWgff18-cmyS-1}}}\plusendmoddef
% brown
\\definecolor\{verydarkbrown\}     \{cmyk\}\{0.35,0.85,1.00,0.40\}
\\definecolor\{darkbrown\}         \{cmyk\}\{0.30,0.70,1.00,0.35\}
\\definecolor\{brown\}             \{cmyk\}\{0.25,0.75,1.00,0.25\}
\\definecolor\{lightbrown\}        \{cmyk\}\{0.20,0.60,0.70,0.15\}
\\definecolor\{verylightbrown\}    \{cmyk\}\{0.15,0.45,0.55,0.00\}
\nwendcode{}\nwbegindocs{724}\nwdocspar
\nwenddocs{}\nwbegincode{725}\sublabel{NWgff18-cmyO-7}\nwmargintag{{\nwtagstyle{}\subpageref{NWgff18-cmyO-7}}}\moddef{cmyk color table - right~{\nwtagstyle{}\subpageref{NWgff18-cmyO-1}}}\plusendmoddef
% brown
\\clspc
\\clrow\{verydarkbrown\}            & 0.35 & 0.85 & 1.00 & 0.40 \\\\
\\clrow\{darkbrown\}                & 0.30 & 0.70 & 1.00 & 0.35 \\\\
\\clrow\{brown\}                    & 0.25 & 0.75 & 1.00 & 0.25 \\\\
\\clrow\{lightbrown\}               & 0.20 & 0.60 & 0.70 & 0.15 \\\\
\\clrow\{verylightbrown\}           & 0.15 & 0.45 & 0.55 & 0.00 \\\\
\nwendcode{}\nwbegindocs{726}\nwdocspar

\subsctn{Colors Help}

\nwenddocs{}\nwbegincode{727}\sublabel{NWgff18-colB-1}\nwmargintag{{\nwtagstyle{}\subpageref{NWgff18-colB-1}}}\moddef{colors help~{\nwtagstyle{}\subpageref{NWgff18-colB-1}}}\endmoddef
Those are the colors defined in $PROGRAM:
+ Basic Colors: black white.
+ Variable Colors: 
      grey magenta violet blue skyblue cyan seagreen
         green limegreen yellow orange red brown
  You can get five color shades from Variable Colors with
  \\"verydark\\", \\"dark\\", \\"light\\" and \\"verylight\\" prefixes,
  as example: 
    verydarkblue, darkblue, blue, lightblue and verylightblue.
\nwused{\\{NWgff18-MaiD-2}}\nwendcode{}\nwbegindocs{728}\nwdocspar


\newpage

\sctn{Page format definition} %%%%%%%%%%%%%%%%%%%%%%%%%%%%%%%%%%%%%%%%%%%%%%%%%%
\label{sec:PAGEdef}

\subsctn{ISO standard page sizes: A series}

\nwenddocs{}\nwbegincode{729}\sublabel{NWgff18-pagQ-1}\nwmargintag{{\nwtagstyle{}\subpageref{NWgff18-pagQ-1}}}\moddef{page sizes perl definition~{\nwtagstyle{}\subpageref{NWgff18-pagQ-1}}}\endmoddef
  a0        => [ ++$formats, 2384, 3370 ],
  a1        => [ ++$formats, 1684, 2384 ],
  a2        => [ ++$formats, 1190, 1684 ],
  a3        => [ ++$formats,  842, 1190 ],
  a4        => [ ++$formats,  595,  842 ],
  a5        => [ ++$formats,  420,  595 ],
  a6        => [ ++$formats,  297,  420 ],
  a7        => [ ++$formats,  210,  297 ],
  a8        => [ ++$formats,  148,  210 ],
  a9        => [ ++$formats,  105,  148 ],
  a10       => [ ++$formats,   73,  105 ],
\nwalsodefined{\\{NWgff18-pagQ-2}\\{NWgff18-pagQ-3}}\nwused{\\{NWgff18-GloB-F}}\nwendcode{}\nwbegindocs{730}\nwdocspar
\nwenddocs{}\nwbegincode{731}\sublabel{NWgff18-pagR-1}\nwmargintag{{\nwtagstyle{}\subpageref{NWgff18-pagR-1}}}\moddef{page sizes latex definition~{\nwtagstyle{}\subpageref{NWgff18-pagR-1}}}\endmoddef
%               points    -  centimeters  -     inches
a0          & 2384 & 3370 &  84.1 & 118.9 &  33.1 &  46.8 \\\\
a1          & 1684 & 2384 &  59.4 &  84.1 &  23.4 &  33.1 \\\\
a2          & 1190 & 1684 &  42.0 &  59.4 &  16.5 &  23.4 \\\\
a3          &  842 & 1190 &  29.7 &  42.0 &  11.7 &  16.5 \\\\
a4          &  595 &  842 &  21.0 &  29.7 &   8.3 &  11.7 \\\\
a5          &  420 &  595 &  14.8 &  21.0 &   5.8 &   8.3 \\\\
a6          &  297 &  420 &  10.5 &  14.8 &   4.1 &   5.8 \\\\
a7          &  210 &  297 &   7.4 &  10.5 &   2.9 &   4.1 \\\\
a8          &  148 &  210 &   5.2 &   7.4 &   2.1 &   2.9 \\\\
a9          &  105 &  148 &   3.7 &   5.2 &   1.5 &   2.1 \\\\
a10         &   73 &  105 &   2.6 &   3.7 &   1.0 &   1.5 \\\\ \\hline\\hline
\nwalsodefined{\\{NWgff18-pagR-2}\\{NWgff18-pagR-3}}\nwused{\\{NWgff18-LATN.2-1}}\nwendcode{}\nwbegindocs{732}\nwdocspar

\subsctn{ISO standard page sizes: B series}

\nwenddocs{}\nwbegincode{733}\sublabel{NWgff18-pagQ-2}\nwmargintag{{\nwtagstyle{}\subpageref{NWgff18-pagQ-2}}}\moddef{page sizes perl definition~{\nwtagstyle{}\subpageref{NWgff18-pagQ-1}}}\plusendmoddef
  b0        => [ ++$formats, 2920, 4127 ],
  b1        => [ ++$formats, 2064, 2920 ],
  b2        => [ ++$formats, 1460, 2064 ],
  b3        => [ ++$formats, 1032, 1460 ],
  b4        => [ ++$formats,  729, 1032 ],
  b5        => [ ++$formats,  516,  729 ],
  b6        => [ ++$formats,  363,  516 ],
  b7        => [ ++$formats,  258,  363 ],
  b8        => [ ++$formats,  181,  258 ],
  b9        => [ ++$formats,  127,  181 ],
  b10       => [ ++$formats,   91,  127 ],
\nwendcode{}\nwbegindocs{734}\nwdocspar
\nwenddocs{}\nwbegincode{735}\sublabel{NWgff18-pagR-2}\nwmargintag{{\nwtagstyle{}\subpageref{NWgff18-pagR-2}}}\moddef{page sizes latex definition~{\nwtagstyle{}\subpageref{NWgff18-pagR-1}}}\plusendmoddef
%               points    -  centimeters  -     inches
b0          & 2920 & 4127 & 103.0 & 145.6 &  40.6 &  57.3 \\\\
b1          & 2064 & 2920 &  72.8 & 103.0 &  28.7 &  40.6 \\\\
b2          & 1460 & 2064 &  51.5 &  72.8 &  20.3 &  28.7 \\\\
b3          & 1032 & 1460 &  36.4 &  51.5 &  14.3 &  20.3 \\\\
b4          &  729 & 1032 &  25.7 &  36.4 &  10.1 &  14.3 \\\\
b5          &  516 &  729 &  18.2 &  25.7 &   7.2 &  10.1 \\\\
b6          &  363 &  516 &  12.8 &  18.2 &   5.0 &   7.2 \\\\
b7          &  258 &  363 &   9.1 &  12.8 &   3.6 &   5.0 \\\\
b8          &  181 &  258 &   6.4 &   9.1 &   2.5 &   3.6 \\\\
b9          &  127 &  181 &   4.5 &   6.4 &   1.8 &   2.5 \\\\
b10         &   91 &  127 &   3.2 &   4.5 &   1.3 &   1.8 \\\\ \\hline\\hline
\nwendcode{}\nwbegindocs{736}\nwdocspar

\subsctn{US standard page sizes}

\nwenddocs{}\nwbegincode{737}\sublabel{NWgff18-pagQ-3}\nwmargintag{{\nwtagstyle{}\subpageref{NWgff18-pagQ-3}}}\moddef{page sizes perl definition~{\nwtagstyle{}\subpageref{NWgff18-pagQ-1}}}\plusendmoddef
  executive => [ ++$formats,  540,  720 ],
  folio     => [ ++$formats,  612,  936 ],
  legal     => [ ++$formats,  612, 1008 ],
  letter    => [ ++$formats,  612,  792 ],
  quarto    => [ ++$formats,  610,  780 ],
  statement => [ ++$formats,  396,  612 ],
 '10x14'    => [ ++$formats,  720, 1008 ],
  ledger    => [ ++$formats, 1224,  792 ],
  tabloid   => [ ++$formats,  792, 1224 ],
\nwendcode{}\nwbegindocs{738}\nwdocspar
\nwenddocs{}\nwbegincode{739}\sublabel{NWgff18-pagR-3}\nwmargintag{{\nwtagstyle{}\subpageref{NWgff18-pagR-3}}}\moddef{page sizes latex definition~{\nwtagstyle{}\subpageref{NWgff18-pagR-1}}}\plusendmoddef
%               points    -  centimeters  -     inches
executive   &  540 &  720 &  19.0 &  25.4 &   7.5 &  10.0 \\\\
folio       &  612 &  936 &  21.6 &  33.0 &   8.5 &  13.0 \\\\
legal       &  612 & 1008 &  21.6 &  35.6 &   8.5 &  14.0 \\\\
letter      &  612 &  792 &  21.6 &  27.9 &   8.5 &  11.0 \\\\
quarto      &  610 &  780 &  21.5 &  27.5 &   8.5 &  10.8 \\\\
statement   &  396 &  612 &  14.0 &  21.6 &   5.5 &   8.5 \\\\ \\hline\\hline
10x14       &  720 & 1008 &  25.4 &  35.6 &  10.0 &  14.0 \\\\
ledger      & 1224 &  792 &  43.2 &  27.9 &  17.0 &  11.0 \\\\
tabloid     &  792 & 1224 &  27.9 &  43.2 &  11.0 &  17.0 \\\\ \\hline
\nwendcode{}\nwbegindocs{740}\nwdocspar

\subsctn{Page sizes help}

\nwenddocs{}\nwbegincode{741}\sublabel{NWgff18-pagA-1}\nwmargintag{{\nwtagstyle{}\subpageref{NWgff18-pagA-1}}}\moddef{pages help~{\nwtagstyle{}\subpageref{NWgff18-pagA-1}}}\endmoddef
The following page sizes are available: from A0 to A10, 
from B0 to B10, 10x14, executive, folio, ledger, legal, 
letter, quarto, statement and tabloid.
\nwused{\\{NWgff18-MaiD-2}}\nwendcode{}\nwbegindocs{742}\nwdocspar


\newpage

\sctn{{\ps} main program and routines} %%%%%%%%%%%%%%%%%%%%%%%%%%%%%%%%%%%%%%%%%

\subsctn{Abbreviated function names}

\nwenddocs{}\nwbegincode{743}\sublabel{NWgff18-POSL-1}\nwmargintag{{\nwtagstyle{}\subpageref{NWgff18-POSL-1}}}\moddef{POSTSCRIPT shortnames~{\nwtagstyle{}\subpageref{NWgff18-POSL-1}}}\endmoddef
%%BeginProcSet: Short_names 1.0 0
%
/tflg false def % test flag
/bdf \{ bind def \} bind def
/xdf \{ exch def \} bdf
/cm \{ 28.35 mul \} bdf
/ivcm \{ 28.35 div \} bdf
/in \{ 72    mul \} bdf
/F \{ scale \} bdf
/T \{ translate \} bdf
/S \{ gsave \} bdf
/R \{ grestore \} bdf
/m \{ moveto \} bdf
/rm \{ rmoveto \} bdf
/l \{ lineto \} bdf
/rl \{ rlineto \} bdf
/K \{ stroke \} bdf
/scmyk \{ setcmykcolor \} bdf
/slw \{ setlinewidth \} bdf
/bbox \{ 4 copy 3 1 roll exch 6 2 roll 8 -2 roll m l l l closepath \} bdf
/dotted \{ [ 1 ] 0 setdash \} def
%
%%EndProcSet:   Short_names 1.0 0
%
\nwused{\\{NWgff18-PosH-1}}\nwendcode{}\nwbegindocs{744}\nwdocspar

\subsctn{Defining PS constants}

\nwenddocs{}\nwbegincode{745}\sublabel{NWgff18-POSK-1}\nwmargintag{{\nwtagstyle{}\subpageref{NWgff18-POSK-1}}}\moddef{POSTSCRIPT constants~{\nwtagstyle{}\subpageref{NWgff18-POSK-1}}}\endmoddef
%%BeginProcSet: Constants 1.0 0
%
% Printing Offset
/VUpOS 0.25 cm def  % offset defines non printable
/VDnOS 0.25 cm def  % paper area for pages (printer outlimits).
/HLtOS 0.25 cm def
/HRtOS 0.25 cm def
/htag   0 def
/Xmarg  5.0 cm def       % Starting Point (upper left corner)
/Ymarg  842 2 cm sub def
/Y Ymarg def
/PlotWidth  14 cm def    % Blocks Size
/Spacer   1.00 cm def    % BBox relative to dotplot
/BBoxX  PlotWidth def    % SBox relative to percent box
/BBoxY   14.00 cm def    % XBox relative to extra box
/SBoxX  PlotWidth def
/SBoxY    2.25 cm def
/XBoxX  PlotWidth def
/XBoxY    3.00 cm def
/WBox     0.50 cm def    % TagBox Size
/HWBox WBox 2 div    def
/Warw  WBox 0.75 mul def
/HWarw Warw 2 div    def
%
%%EndProcSet:   Constants 1.0 0
%
\nwused{\\{NWgff18-PosH-1}}\nwendcode{}\nwbegindocs{746}\nwdocspar

\subsctn{Page layout}

\nwenddocs{}\nwbegincode{747}\sublabel{NWgff18-POSH-1}\nwmargintag{{\nwtagstyle{}\subpageref{NWgff18-POSH-1}}}\moddef{POSTSCRIPT layout~{\nwtagstyle{}\subpageref{NWgff18-POSH-1}}}\endmoddef
%%BeginProcSet: Page_Layout 1.0 0
%
/TitleFont    \{ 24 /Times-Bold  \} def
/SubTtFont    \{ 16 /Times-Roman \} def
/ElmFont      \{ 12 FTLbsc mul /Times-Roman \} def
/TagFont      \{ 14 GPLbsc mul /Times-Roman \} def
/TagLabelFont \{ 16 /Times-Bold  \} def
/TickFont     \{ 10 /Helvetica \} def
%
/xBDspl  1.8 putExon add putExLbl add putGnLbl add def % 1.25
/xGLDspl 1.0 putExon add putExLbl add def              % 0.85
/xGDspl  0.75 putExon add putExLbl add def
/FBDspl 0.60 def % For example, for mRNA.
%
/vertical   \{ /htag 0 def \} bdf
/horizontal \{ /htag 1 def \} bdf
%
/FSF 4 def % Point size for Credits for A4
/CSF \{ pagedict begin pga4 pop end Dpage pop exch div mul \} bdf
%
% checking if margins are within the defined offset
flglscape \{
 UpM HLtOS lt \{ /UpM HLtOS def \} if % Checking margins for flglscape mode
 DnM HRtOS lt \{ /DnM HRtOS def \} if
 LtM VDnOS lt \{ /LtM VDnOS def \} if
 RtM VUpOS lt \{ /RtM VUpOS def \} if
 \} \{
  UpM VUpOS lt \{ /UpM VUpOS def \} if % Checking margins for portrait mode
  DnM VDnOS lt \{ /DnM VDnOS def \} if
  LtM HLtOS lt \{ /LtM HLtOS def \} if
  RtM HRtOS lt \{ /RtM HRtOS def \} if
  \} ifelse
% defining pagelimits and X - Y scales (Xlim Ylim)
/pglim \{ Dpage pop LtM RtM add sub Dpage exch pop UpM DnM add sub \} def
% Defining starting point on page.
/XORI LtM def
/YORI UpM def
%
%%EndProcSet:   Page_Layout 1.0 0
%
\nwused{\\{NWgff18-PosF.2-1}}\nwendcode{}\nwbegindocs{748}\nwdocspar

\subsctn{Text related functions}

\nwenddocs{}\nwbegincode{749}\sublabel{NWgff18-POSP-1}\nwmargintag{{\nwtagstyle{}\subpageref{NWgff18-POSP-1}}}\moddef{POSTSCRIPT text functions~{\nwtagstyle{}\subpageref{NWgff18-POSP-1}}}\endmoddef
%%BeginProcSet: text_functions 1.0 0
%
/sfont \{ findfont exch scalefont setfont \} bdf
/tshow \{ S scmyk sfont m rotate show R \} bdf
/ctshow \{
  10 -1 roll dup 11 1 roll 7 -2 roll 2 copy 9 2 roll 
  S sfont stringwidth pop R 2 div 
  htag 1 eq \{
    9 -1 roll exch sub 8 1 roll
    \} \{
      8 -1 roll exch sub 7 1 roll
    \} ifelse
  tshow
  \} bdf
/ltshow \{
  10 -1 roll dup 11 1 roll 7 -2 roll 2 copy 9 2 roll 
  S sfont stringwidth pop R 
  htag 1 eq \{
    9 -1 roll exch sub 8 1 roll
    \} \{
      8 -1 roll exch sub 7 1 roll
    \} ifelse
  tshow
  \} bdf
%
% X Y angle string valign halign fnt color ttxt
 % valign : tv (top)  cv (middle) bv (bottom)
 % halign : lh (left) ch (center) rh (right)
/chrh \{ 
  S newpath 0 0 m false 
  charpath flattenpath pathbbox exch pop
  3 -1 roll pop R
  \} bdf
/strh \{
  2 dict begin /lly 0.0 def /ury 0.0 def 
  \{ ( ) dup 0 4 -1 roll put chrh 
    dup ury gt \{ /ury xdf \} \{ pop \} ifelse
    dup lly lt \{ /lly xdf \} \{ pop \} ifelse
    \} forall
  ury end
  \} bdf
/ttxt \{
  S scmyk sfont 8 dict begin 
  /h xdf /v xdf /lbl xdf /angle xdf /y xdf /x xdf 
  /hs lbl stringwidth pop neg def /vs lbl strh neg def
  x y T angle rotate
  h (rh) eq \{ hs
    \} \{
      h (ch) eq \{ hs 2 div \} \{ 0 \} ifelse
    \} ifelse
  v (tv) eq \{ vs
    \} \{
      v (cv) eq \{ vs 2 div \} \{ 0 \} ifelse
    \} ifelse
  m lbl show end R
  \} bdf
%
/Title \{ S 0 Xmarg Y TitleFont FGcolor tshow R /Y Y 0.75 cm sub def \} bdf 
/SubTitle \{ S 0 Xmarg Y SubTtFont FGcolor tshow R /Y Y 2.5 cm sub def \} bdf 
%
%%EndProcSet:   text_functions 1.0 0
%
\nwused{\\{NWgff18-PosF.2-1}}\nwendcode{}\nwbegindocs{750}\nwdocspar

\subsctn{The APLOT dictionary}

\nwenddocs{}\nwbegincode{751}\sublabel{NWgff18-POSK.2-1}\nwmargintag{{\nwtagstyle{}\subpageref{NWgff18-POSK.2-1}}}\moddef{POSTSCRIPT aplotdict~{\nwtagstyle{}\subpageref{NWgff18-POSK.2-1}}}\endmoddef
%%BeginProcSet: aplotdict 1.0 0
%
/aplot 120 dict def aplot begin
/min \{ 2 copy gt \{ exch \} if pop \} bdf 
/max \{ 2 copy lt \{ exch \} if pop \} bdf
/Xscm \{ Xscale mul \} bdf /Xscme \{ Xscm exch \} bdf
/Yscm \{ Yscale mul \} bdf /Yscme \{ Yscm exch \} bdf
/fmt \{ Yscme Xscme m \} bdf /flt \{ Yscme Xscme l \} bdf
/line \{ scmyk slw m l K \} bdf
/uline \{
  scmyk slw m S htag 1 eq \{ 0 -0.1 cm \} \{ 0.1 cm 0 \} ifelse
  rl K R l S htag 1 eq \{ 0 -0.1 cm \} \{ 0.1 cm 0 \} ifelse
  rl K R K
  \} bdf
/ZoomTicks \{
  /zmdict 3 dict def zmdict begin
  /r \{ TxWB BDspl mul Xscm \} def /s \{ TyWB BDspl mul Yscm \} def
  /corner \{ S 10 -2 roll m 
    2 \{ rl S BGcolor scmyk .4 slw K R rl
      S FGcolor scmyk 2 slw [1 3] 0 setdash K R
      \} repeat R 
    \} def
  4 copy 3 1 roll exch
  0 s 0 s neg r neg 0 r 0 corner
  0 s neg 0 s r 0 r neg 0 corner
  0 s neg 0 s r neg 0 r 0 corner
  0 s 0 s neg r 0 r neg 0 corner end
  \} def
/Line \{
  S 9 5 roll 
    2 \{ Yscm 4 1 roll Xscm 4 1 roll \} repeat
    9 4 roll line
  R \} bdf
/Xline \{ 
  S 3 1 roll Xscme Xscme 3 -1 roll Yscm xwdt sub dup
    3 1 roll xwdt 9 -4 roll line
  R \} bdf
/Bline \{
  S htag 1 eq \{ Xscme Xscme y0 Yscme y1 Yscm bbox \}
     \{ Yscme Yscme X0 Xscm 3 1 roll X1 Xscme bbox \} ifelse
  S scmyk fill 
  R scmyk slw K
  R \} bdf
/Bsquare \{
  S Yscme Yscme 4 2 roll Xscme Xscme 4 1 roll exch bbox 
  S scmyk fill 
  R scmyk slw K
  R \} bdf
/Msquare \{
  S Yscme Yscme 4 2 roll Xscme Xscme 4 1 roll exch bbox
    scmyk slw K
  R \} bdf
/Mcircle \{
  S Yscme Xscme 3 -1 roll Xscm 0 360 arc closepath scmyk slw K
  R \} bdf
/Join \{
  S 2 copy 2 copy pop sub 2 div add exch
    htag 1 eq \{
      y1 TyWB add dup dup TyWB 0.75 mul add 4 1 roll 5 1 roll
      \} \{
        X0 TxWB sub dup dup TxWB 0.75 mul sub 5 1 roll 6 1 roll exch
      \} ifelse
    3 \{ Yscme Xscme 6 2 roll \} repeat
    m l l 0.25 slw scmyk K
  R \} bdf
/Arrow \{
  /acol [ 9 -4 roll ] def /acolor \{ acol aload pop \} def
  S 1 eq \{ exch /sn \{ 1 mul \} def \} \{ /sn \{ 1 neg mul \} def \} ifelse
    htag 1 eq \{ Xscm dup y1 Yscm HWBox add m
      HWarw sn HWarw rl 0 Warw neg rl HWarw neg sn HWarw rl closepath
      S acolor scmyk fill R
      y1 Yscm HWBox add m Xscm y1 Yscm HWBox add l acolor scmyk 1 slw K
      \} \{ Yscm dup X0 Xscm HWBox sub exch m
        HWarw neg HWarw sn rl Warw 0 rl HWarw neg HWarw neg sn rl closepath
        S acolor scmyk fill R
        X0 Xscm HWBox sub exch m Yscm X0 Xscm HWBox sub exch l
        acolor scmyk 1 slw K
      \} ifelse
  R \} bdf
/FBox \{
  htag 1 eq \{
    Xscme Xscme y1 Yscme y1 TyWB FBDspl mul add Yscm bbox
    \} \{
      Yscme Yscme X0 Xscm 3 1 roll X0 TxWB FBDspl mul sub Xscme bbox
    \} ifelse
  S scmyk fill R FGcolor scmyk 1 slw K
  \} bdf
/FTalgn \{
  htag 1 eq \{
    FTXangle 0 eq \{ (bv) (ch) \} \{ (cv) (lh) \} ifelse
    \} \{
      FTYangle 0 eq \{ (cv) (rh) \} \{ (bv) (rh) \} ifelse
    \} ifelse
  \} bdf
/GPalgn \{
  htag 1 eq \{
    GPXangle 0 eq \{ (bv) (ch) \} \{ (cv) (lh) \} ifelse
    \} \{
      GPYangle 0 eq \{ (cv) (rh) \} \{ (bv) (rh) \} ifelse
    \} ifelse
  \} bdf
/Box \{
  2 copy 2 copy pop sub 2 div add 7 1 roll
  S htag 1 eq \{
    Xscme Xscme y1 Yscme y1 TyWB add Yscm bbox
    \} \{
      Yscme Yscme X0 Xscm 3 1 roll X0 TxWB sub Xscme bbox
    \} ifelse
  S scmyk fill R
  FGcolor scmyk 1 slw K
  htag 1 eq \{
    Xscm y1 TyWB 1.75 mul add Yscm FTXangle
    \} \{
      Yscm X0 TxWB 1.75 mul sub Xscme FTYangle
    \} ifelse
  6 -1 roll FTalgn 8 -2 roll FGcolor ttxt
  R \} bdf
%
/GDmore \{
  MxFtLBL S ElmFont sfont (M) stringwidth pop R mul
  htag 1 eq \{ FTXangle sin \} \{ FTYangle cos \} ifelse
  abs mul ivcm add
  \} bdf
/BDmore \{
  MxGpLBL S TagFont sfont (M) stringwidth pop R mul
  htag 1 eq \{ GPXangle sin \} \{ GPYangle cos \} ifelse
  abs mul ivcm add
  \} bdf
%
/GnBanner\{
  S 1 eq \{
      2 copy
      htag 1 eq \{
        Xscme Xscme y1 TyWB GDspl mul add dup Yscme Yscm 3 1 roll
        \} \{
          Yscme Yscme X0 TxWB GDspl mul sub dup Xscme Xscme 4 1 roll exch
        \} ifelse
      0.5 FGcolor uline
      \} if
    2 copy pop sub 2 div add
    htag 1 eq \{
      Xscm y1 TyWB GLDspl mul add Yscm GPXangle
      \} \{
        Yscm X0 TxWB GLDspl mul sub Xscme GPYangle
      \} ifelse
    6 -1 roll GPalgn 8 -2 roll FGcolor ttxt
  R \} bdf
/SbBanner \{
  S htag 1 eq \{
      0 X0 X1 X0 sub 2 div add Xscm y1 TyWB GDspl mul add Yscm
      \} \{
        90 X0 TxWB GDspl mul sub Xscm y0 y1 y0 sub 2 div add Yscm
      \} ifelse
    5 -2 roll FGcolor ctshow
  R \} bdf
/Banner \{
  S htag 1 eq \{
      0 X0 X1 X0 sub 2 div add Xscm y1 TyWB BDspl mul add Yscm
      \} \{
        90 X0 TxWB BDspl mul sub Xscm y0 y1 y0 sub 2 div add Yscm
      \} ifelse
    5 -2 roll FGcolor ctshow
  R \} bdf
%
% mxt mnt xp yp ori end htick
/tckdict 15 dict def
 tckdict begin
 /mkmxt \{
  dup 0 lt \{ neg \} if
  dup 10 lt \{
    10 mul log round 10 exch exp cvi
    \} \{
      log round 10 exch exp cvi
    \} ifelse
  \} def
 /nwmod \{
  dup 1 le \{
    100 mul cvi exch 100 mul cvi exch mod 100 div
    \} \{
      exch dup
      1 le \{
        100 mul cvi exch 100 mul cvi mod 100 div
        \} \{
          cvi exch cvi mod
        \} ifelse
    \} ifelse
  \} def
 /isltone \{ dup 1 lt \{ 100 mul cvi 100 div \} if \} def
end
/htick \{ 
  S tckdict begin horizontal 
    /yp exch Yscm def /xp exch Xscm def /xend xdf /xori xdf
    /nmnt exch cvi def /nmxt exch cvi def
    /mxt exch dup 0 lt \{ pop xend xori sub mkmxt \} if def
    /mnt exch dup 0 lt \{ pop mxt nmnt div \} if def
    /lori xori dup mnt nwmod sub mnt add def
    /lend xend dup mnt nwmod sub mnt sub def
    xp yp T
    lori mnt lend \{
      isltone dup dup xori gt exch xend lt and \{
        dup Xscm 0 m dup dup Xscme
        mxt nwmod 0 eq \{ 7.5 \} \{ 4 \} ifelse neg l
        1 slw FGcolor scmyk K
        GridON 1 eq \{
          S 0 y0 Yscm neg T dup Xscm dup y0 Yscm m y1 Yscm l
          0.1 slw verylightgrey scmyk K R
          \} if
        dup mxt nwmod 0 eq \{
          dup xend mnt sub exch ge \{
            dup dup 1 ge \{ cvi \} if
            10 string cvs 0 3 -1 roll Xscm -18 TickFont FGcolor ctshow
            \} if
          \} \{
            pop
          \} ifelse
        \} if
      \} for
    xori Xscm dup 0 m 7.5 neg l 2 slw FGcolor scmyk K
    xori 10 string cvs 0 xori Xscm ZoomON 1 eq \{ 4 sub \} if
    -18 TickFont FGcolor ZoomON 0 eq \{ ctshow \} \{ ltshow \} ifelse
    xend Xscm dup 0 m 7.5 neg l 2 slw FGcolor scmyk K
    xend 10 string cvs 0 xend Xscm ZoomON 1 eq \{ 4 add \} if
    -18 TickFont FGcolor ZoomON 0 eq \{ ctshow \} \{ tshow \} ifelse
  end R \} def
/vtick \{
  S tckdict begin vertical
    /yp exch Yscm def /xp exch Xscm def /xend xdf /xori xdf
    /nmnt exch cvi def /nmxt exch cvi def
    /mxt exch dup 0 lt \{ pop xend xori sub mkmxt \} if def
    /mnt exch dup 0 lt \{ pop mxt nmnt div \} if def
    /lori xori dup mnt nwmod sub mnt add def
    /lend xend dup mnt nwmod sub mnt sub def
    xp yp T
    lori mnt lend \{
      isltone dup dup xori gt exch xend lt and \{
        dup Yscm 0 exch m dup dup Yscme
        mxt nwmod 0 eq \{ 7.5 \} \{ 4 \} ifelse exch l
        1 slw FGcolor scmyk K
        GridON 1 eq \{
          S X0 Xscm 0 T dup Yscm dup X0 neg Xscme m
          X1 neg Xscme l 0.1 slw verylightgrey scmyk K R
          \} if
        dup 0 eq \{
          S dup dup X0 Xscme Yscm m X1 neg Xscme Yscm l
          0.1 slw FGcolor scmyk K R
          \} if
        dup mxt nwmod 0 eq \{
          dup xend mnt sub exch ge \{
            dup dup 1 ge \{ cvi \} if
            10 string cvs 0 10 4 -1 roll Yscm 2.5 sub TickFont FGcolor tshow
            \} if
          \} \{
            pop
          \} ifelse
        \} if
      \} for
    7.5 xori Yscm dup 0 exch m l 2 slw FGcolor scmyk K
    xori 10 string cvs 0 10 xori Yscm
    ZoomON 0 eq \{ 2.5 sub \} \{ 2.5 add \} ifelse
    TickFont FGcolor tshow 7.5 xend Yscm dup 0 exch m l
    2 slw FGcolor scmyk K xend 10 string cvs 0 10 xend Yscm
    ZoomON 0 eq \{ 2.5 sub \} \{ 2.5 add \} ifelse
    TickFont FGcolor tshow end
  R \} def
/nucltick \{ tickmn tickmx maxtck mintck X0 X1 0 y0 htick \} def
/pctmarks \{
  S y0 10 y1 \{
      Yscm dup X0 Xscm X1 Xscm 3 1 roll 4 1 roll 0.5 FGcolor line
      \} for
  R \} bdf
%
/beginfunct \{
  R S 3 dict begin
    /maxx exch ceiling def /minx exch floor def
    /Yscale XBoxY maxx minx sub div def
    minx 0 lt \{ 0 minx neg Yscm T \} if
    S XBYtickflg 1 eq \{
        -1 -1 xtrmxt xtrmnt minx maxx X1 0 vtick
        S X0 Xscm 0 m X1 Xscm 0 l dotted FGcolor scmyk K R
        \} if
    R
  \} def /endfunct \{ end R \} def
end % aplot dict
%
%%EndProcSet:   aplotdict 1.0 0
%
\nwused{\\{NWgff18-PosF.2-1}}\nwendcode{}\nwbegindocs{752}\nwdocspar

\subsctn{Main functions definition}

\nwenddocs{}\nwbegincode{753}\sublabel{NWgff18-POSU-1}\nwmargintag{{\nwtagstyle{}\subpageref{NWgff18-POSU-1}}}\moddef{POSTSCRIPT main function calls~{\nwtagstyle{}\subpageref{NWgff18-POSU-1}}}\endmoddef
%%BeginProcSet: main_function_calls 1.0 0
%
/s_credits \{
  S 1 dict begin
    /fs_cd FSF def
    pagedict begin
    pga4 pop end 1 cm sub DnM T 0 0 0
(This plot has been obtained using GFF2APLOT. The most recent version of GFF2APLOT is freely available at \\042http:\\/\\/www1.imim.es/software/gfftools/APLOT.html\\042. Copyright      1999 by Josep F. ABRIL, Thomas WIEHE & Roderic GUIGO)
    (cv) (rh) fs_cd /Times-Roman FGcolor ttxt
    S fs_cd /Times-Roman sfont
    (   1999 by Josep F. ABRIL, Thomas WIEHE & Roderic GUIGO)
    stringwidth pop neg R 0 0 (\\343) (cv) (ch) fs_cd /Symbol FGcolor ttxt
    end
  R \} bdf
%
/estlbl \{
  S 3 dict begin
    /lbl xdf /ypos xdf /xpos xdf
    horizontal
      0 Xlwdt 4 div neg T
      lbl 0 xpos Xscm ypos Yscm 5 /Helvetica black ctshow
    vertical
    end
  R \} bdf
%
/GsclonX \{
  /Y exch Y Spacer sub exch sub def
  Xmarg Y T
  axesp 0 eq \{
    /Xscale \{ BBoxX X1 X0 sub Y1 Y0 sub max div \} def
    \} \{
      /Xscale \{ BBoxX X1 X0 sub div \} def
    \} ifelse
  \} bdf
%
% DOTPLOT BOX
/begindata \{
 aplot begin
  S /y1 Y1 def /y0 Y0 def
    BBoxY GsclonX
    axesp 0 eq \{
      /Yscale Xscale def
      \} \{
        /Yscale \{ BBoxY y1 y0 sub div \} def
      \} ifelse
    /TxWB \{ WBox Xscale div \} def /TyWB \{ WBox Yscale div \} def
    /GDspl \{ xGDspl GDmore \} def /GLDspl \{ xGLDspl GDmore \} def
    /BDspl \{ xBDspl GDmore BDmore \} def
    X0 Xscm neg y0 Yscm neg T
    S X0 Xscm y0 Yscm X1 Xscm y1 Yscm bbox BBoxcol scmyk fill R
    S X0 Xscm y0 Yscm X1 Xscm y1 Yscm
      ZoomON 1 eq \{ 4 copy ZoomTicks \} if bbox
      2 slw FGcolor scmyk K
    R
    S BBXtickflg 1 eq \{
        nucltick /Spacer 1 cm def
        \} \{
          /Spacer 0.5 cm def
        \} ifelse
      BBYtickflg 1 eq \{
        tickmn tickmx maxtck mintck y0 y1 X1 0 vtick
        \} if
    R newpath
 \} def
/enddata \{ R /GDspl xGDspl def /GLDspl xGLDspl def /BDspl xBDspl def end \} def
%
% PERCENT BOX
/beginmatches \{
 aplot begin 
   /ZoomON 0 def S /y1 xdf /y0 xdf SBoxY GsclonX
   /Yscale SBoxY y1 y0 sub div def
   X0 Xscm neg y0 Yscm neg T S X0 Xscm y0 Yscm X1 Xscm y1 Yscm bbox
   S
     S SBoxcol scmyk fill R
     S SBoxLab TagLabelFont Banner SBoxSLab TagFont SbBanner
       SBXtickflg 1 eq \{
         nucltick /Spacer 1 cm def
         \} \{
           /Spacer 0.5 cm def
         \} ifelse
     R
     S SBYtickflg 1 eq \{ -1 -1 pctmxt pctmnt y0 y1 X1 0 vtick \} if
     R
     S X0 Xscm y0 Yscm X1 Xscm y1 Yscm bbox 2 slw FGcolor scmyk K
     R
   R clip newpath
 \} def /endmatches \{ R R end \} def
%
% EXTRA BOX
/beginextra \{
 aplot begin
   S /nlines xdf /y0 0 def XBoxY GsclonX
     /Yscale XBoxY nlines 1 add div def
     /Xlwdt Yscale 0.75 mul def /xwdt Yscale 2 div def
     /y1 XBoxY Yscale div def
     X0 Xscm neg 0 T
     S X0 Xscm y0 Yscm X1 Xscm y1 Yscm bbox
       S XBoxcol scmyk fill R
       S XBoxLab TagLabelFont Banner XBoxSLab TagFont SbBanner
         XBXtickflg 1 eq \{ nucltick \} if
       R newpath
  \} def
/endextra \{
  /Yscale XBoxY nlines 1 add div def
  X0 Xscm y0 Yscm X1 Xscm y1 Yscm bbox
  2 slw FGcolor scmyk K R
  R end
  \} def
%
%%EndProcSet:   main_function_calls 1.0 0
%
\nwused{\\{NWgff18-PosF.2-1}}\nwendcode{}\nwbegindocs{754}\nwdocspar


\newpage

\sctn{Reporting Program Execution} %%%%%%%%%%%%%%%%%%%%%%%%%%%%%%%%%%%%%%%%%%%%%

\subsctn{Signal trapping}

In this version of {\prog} we are trapping signals, to exit from this program but reporting that was a user action, not a program error. We will close '{\tt{}LOGFILE}' if opened. 

\nwenddocs{}\nwbegincode{755}\sublabel{NWgff18-TraG-1}\nwmargintag{{\nwtagstyle{}\subpageref{NWgff18-TraG-1}}}\moddef{Trapping signals~{\nwtagstyle{}\subpageref{NWgff18-TraG-1}}}\endmoddef
$SIG\{HUP\}  = \\&trap_signals_prog;
$SIG\{ABRT\} = \\&trap_signals;
$SIG\{INT\}  = \\&trap_signals;
$SIG\{QUIT\} = \\&trap_signals;
$SIG\{TERM\} = \\&trap_signals;
$SIG\{KILL\} = \\&trap_signals;
$SIG\{CHLD\} = 'IGNORE';
\nwused{\\{NWgff18-UseB-1}}\nwendcode{}\nwbegindocs{756}\nwdocspar

\nwenddocs{}\nwbegincode{757}\sublabel{NWgff18-ComF-2}\nwmargintag{{\nwtagstyle{}\subpageref{NWgff18-ComF-2}}}\moddef{Common Routines~{\nwtagstyle{}\subpageref{NWgff18-ComF-1}}}\plusendmoddef
sub trap_signals() \{
    &prt_to_logfile($Messages\{'USER_HALT'\});
    &close_logfile();
    die($Messages\{'USER_HALT'\});
\}
\nwendcode{}\nwbegindocs{758}\nwdocspar

\nwenddocs{}\nwbegincode{759}\sublabel{NWgff18-ComF-3}\nwmargintag{{\nwtagstyle{}\subpageref{NWgff18-ComF-3}}}\moddef{Common Routines~{\nwtagstyle{}\subpageref{NWgff18-ComF-1}}}\plusendmoddef
sub trap_signals_prog() \{
    &prt_to_logfile($Messages\{'PROCESS_HALT'\});
    &close_logfile();
    die($Messages\{'PROCESS_HALT'\});
\}
\nwendcode{}\nwbegindocs{760}\nwdocspar

\nwenddocs{}\nwbegincode{761}\sublabel{NWgff18-outN-2}\nwmargintag{{\nwtagstyle{}\subpageref{NWgff18-outN-2}}}\moddef{warnings - input/output~{\nwtagstyle{}\subpageref{NWgff18-outN-1}}}\plusendmoddef
USER_HALT =>
  $spl.$Warn."$PROGRAM has been stopped by user !!!\\n".
  $spl.$Warn."---------- Exiting NOW !!! ----------\\n".$spl,
PROCESS_HALT =>
  $spl.$Warn."------- $PROGRAM is down !!! -------\\n".
  $spl.$Warn."---------- Exiting NOW !!! ----------\\n".$spl,
\nwendcode{}\nwbegindocs{762}\nwdocspar

\subsctn{Dumping complex data structures}

With '{\tt{}Data::Dumper}' we are able to pretty print complex data structures for debugging them. We include also a command-line switch to enable this feature, but it requires a log file definition with '{\tt{}-V}' option.

\nwenddocs{}\nwbegincode{763}\sublabel{NWgff18-UseB-3}\nwmargintag{{\nwtagstyle{}\subpageref{NWgff18-UseB-3}}}\moddef{Use Modules~{\nwtagstyle{}\subpageref{NWgff18-UseB-1}}}\plusendmoddef
use Data::Dumper;
local $Data::Dumper::Purity = 0;
local $Data::Dumper::Deepcopy = 1;
\nwendcode{}\nwbegindocs{764}\nwdocspar

\nwenddocs{}\nwbegincode{765}\sublabel{NWgff18-perI-3}\nwmargintag{{\nwtagstyle{}\subpageref{NWgff18-perI-3}}}\moddef{perl requires help~{\nwtagstyle{}\subpageref{NWgff18-perI-1}}}\plusendmoddef
"Data::Dumper" - pretty printing data structures for debugging (*).
\nwendcode{}\nwbegindocs{766}\nwdocspar

\nwenddocs{}\nwbegincode{767}\sublabel{NWgff18-comK-X}\nwmargintag{{\nwtagstyle{}\subpageref{NWgff18-comK-X}}}\moddef{command-line options~{\nwtagstyle{}\subpageref{NWgff18-comK-1}}}\plusendmoddef
"debug"  => \\$Debug, # Dumps Vars -> LogFile
\nwendcode{}\nwbegindocs{768}\nwdocspar
\nwenddocs{}\nwbegincode{769}\sublabel{NWgff18-comH-Z}\nwmargintag{{\nwtagstyle{}\subpageref{NWgff18-comH-Z}}}\moddef{command-line help~{\nwtagstyle{}\subpageref{NWgff18-comH-1}}}\plusendmoddef
--debug    Reporting variable contents when testing the program.
\nwendcode{}\nwbegindocs{770}\nwdocspar

\subsctn{Reporting program status}

\nwenddocs{}\nwbegincode{771}\sublabel{NWgff18-GloB-G}\nwmargintag{{\nwtagstyle{}\subpageref{NWgff18-GloB-G}}}\moddef{Global Vars~{\nwtagstyle{}\subpageref{NWgff18-GloB-1}}}\plusendmoddef
# Program status strings.
my %Messages = (
    # ERROR MESSAGES
    \LA{}warnings - input/output~{\nwtagstyle{}\subpageref{NWgff18-outN-1}}\RA{}
    \LA{}warnings - parsing command-line options~{\nwtagstyle{}\subpageref{NWgff18-ward-1}}\RA{}
    \LA{}warnings - parsing custom files~{\nwtagstyle{}\subpageref{NWgff18-warV-1}}\RA{}
    \LA{}warnings - parsing GFF files~{\nwtagstyle{}\subpageref{NWgff18-warS-1}}\RA{}
    # WORKING MESSAGES
    \LA{}messages - input/output~{\nwtagstyle{}\subpageref{NWgff18-outN.2-1}}\RA{}
    \LA{}messages - parsing command-line options~{\nwtagstyle{}\subpageref{NWgff18-mesd-1}}\RA{}
    \LA{}messages - parsing custom files~{\nwtagstyle{}\subpageref{NWgff18-mesV-1}}\RA{}
    \LA{}messages - parsing GFF files~{\nwtagstyle{}\subpageref{NWgff18-mesS-1}}\RA{}
    \LA{}messages - sorting GFF data~{\nwtagstyle{}\subpageref{NWgff18-mesR-1}}\RA{}
   ); # %Messages
\nwendcode{}\nwbegindocs{772}\nwdocspar

\subsubsctn{Reporting errors and warnings}

\nwenddocs{}\nwbegincode{773}\sublabel{NWgff18-GloG-5}\nwmargintag{{\nwtagstyle{}\subpageref{NWgff18-GloG-5}}}\moddef{Global Constants~{\nwtagstyle{}\subpageref{NWgff18-GloG-1}}}\plusendmoddef
my $Error = "\\<\\<\\<  ERROR  \\>\\>\\> ";
my $Warn  = "\\<\\<\\< WARNING \\>\\>\\> ";
my $spl   = "\\<\\<\\<\\-\\-\\-\\-\\-\\-\\-\\-\\-\\>\\>\\>\\n";
my $spw   = "\\<\\<\\<         \\>\\>\\> ";
\nwendcode{}\nwbegindocs{774}\nwdocspar

\nwenddocs{}\nwbegincode{775}\sublabel{NWgff18-ComF-4}\nwmargintag{{\nwtagstyle{}\subpageref{NWgff18-ComF-4}}}\moddef{Common Routines~{\nwtagstyle{}\subpageref{NWgff18-ComF-1}}}\plusendmoddef
sub warn() \{
    my $type = shift @_;
    my $screen_flg = shift @_;
    my $comment = sprintf($Messages\{$type\}, @_);
    # ALWAYS to STDERR if $screen_flg==$T unless $Quiet==$T
    $screen_flg && ($Quiet || print STDERR $comment); 
    &prt_to_logfile($comment);
\} # warn
\nwendcode{}\nwbegindocs{776}\nwdocspar

\nwenddocs{}\nwbegincode{777}\sublabel{NWgff18-ComF-5}\nwmargintag{{\nwtagstyle{}\subpageref{NWgff18-ComF-5}}}\moddef{Common Routines~{\nwtagstyle{}\subpageref{NWgff18-ComF-1}}}\plusendmoddef
sub prt_to_logfile() \{ $LogFile && (print LOGFILE $_[0]) \}
sub prt_to_stderr()  \{ $Verbose && ($Quiet || print STDERR $_[0]) \}
\nwendcode{}\nwbegindocs{778}\nwdocspar

\subsctn{Reporting messages}

\nwenddocs{}\nwbegincode{779}\sublabel{NWgff18-GloG-6}\nwmargintag{{\nwtagstyle{}\subpageref{NWgff18-GloG-6}}}\moddef{Global Constants~{\nwtagstyle{}\subpageref{NWgff18-GloG-1}}}\plusendmoddef
my $line = ("#" x 80)."\\n";
my $sp = "###\\n";
\nwendcode{}\nwbegindocs{780}\nwdocspar

\nwenddocs{}\nwbegincode{781}\sublabel{NWgff18-ComF-6}\nwmargintag{{\nwtagstyle{}\subpageref{NWgff18-ComF-6}}}\moddef{Common Routines~{\nwtagstyle{}\subpageref{NWgff18-ComF-1}}}\plusendmoddef
sub report() \{
    my $type = shift @_;
    my $comment = sprintf($Messages\{$type\},@_);
    &prt_to_stderr($comment);
    &prt_to_logfile($comment);
\} # report
\nwendcode{}\nwbegindocs{782}\nwdocspar

\nwenddocs{}\nwbegincode{783}\sublabel{NWgff18-ComF-7}\nwmargintag{{\nwtagstyle{}\subpageref{NWgff18-ComF-7}}}\moddef{Common Routines~{\nwtagstyle{}\subpageref{NWgff18-ComF-1}}}\plusendmoddef
sub header() \{
    my $comment = $line;
    foreach my $ln (@_) \{ 
        $comment .= "### ".&fill_mid("$ln",72," ")." ###\\n";
        \};
    $comment .= $line;
    &prt_to_stderr($comment);
    &prt_to_logfile($comment);
\} # header
sub footer() \{
    $total_time = &timing($F);
    &header(@_,$total_time);
    &prt_to_stderr("###\\n");
    &prt_to_logfile("###\\n");
\}
\nwendcode{}\nwbegindocs{784}\nwdocspar

\subsctn{Timing procedures}

The '{\tt{}Benchmark}' module encapsulates a number of routines to help to figure out how long it takes to execute a piece of code and the whole script.

\nwenddocs{}\nwbegincode{785}\sublabel{NWgff18-UseB-4}\nwmargintag{{\nwtagstyle{}\subpageref{NWgff18-UseB-4}}}\moddef{Use Modules~{\nwtagstyle{}\subpageref{NWgff18-UseB-1}}}\plusendmoddef
use Benchmark;
  \LA{}Timer ON~{\nwtagstyle{}\subpageref{NWgff18-Tim8-1}}\RA{}
\nwendcode{}\nwbegindocs{786}\nwdocspar

\nwenddocs{}\nwbegincode{787}\sublabel{NWgff18-perI-4}\nwmargintag{{\nwtagstyle{}\subpageref{NWgff18-perI-4}}}\moddef{perl requires help~{\nwtagstyle{}\subpageref{NWgff18-perI-1}}}\plusendmoddef
"Benchmark" - checking and comparing running times of code.
\nwendcode{}\nwbegindocs{788}\nwdocspar

See '{\tt{}man\ Benchmark}' for further info about this package. 
We set an array to keep record of timing for each section.

\nwenddocs{}\nwbegincode{789}\sublabel{NWgff18-Tim8-1}\nwmargintag{{\nwtagstyle{}\subpageref{NWgff18-Tim8-1}}}\moddef{Timer ON~{\nwtagstyle{}\subpageref{NWgff18-Tim8-1}}}\endmoddef
my @Timer = (new Benchmark);
\nwused{\\{NWgff18-UseB-4}}\nwendcode{}\nwbegindocs{790}\nwdocspar

\nwenddocs{}\nwbegincode{791}\sublabel{NWgff18-ComF-8}\nwmargintag{{\nwtagstyle{}\subpageref{NWgff18-ComF-8}}}\moddef{Common Routines~{\nwtagstyle{}\subpageref{NWgff18-ComF-1}}}\plusendmoddef
sub timing() \{
    push @Timer, (new Benchmark);
    # partial time 
    $_[0] || 
        (return timestr(timediff($Timer[$#Timer],$Timer[($#Timer - 1)])));
    # total time
    return timestr(timediff($Timer[$#Timer],$Timer[0]));
\}
\nwendcode{}\nwbegindocs{792}\nwdocspar

We also set here the date when the script is running and who is the user running it.

\nwenddocs{}\nwbegincode{793}\sublabel{NWgff18-GloB-H}\nwmargintag{{\nwtagstyle{}\subpageref{NWgff18-GloB-H}}}\moddef{Global Vars~{\nwtagstyle{}\subpageref{NWgff18-GloB-1}}}\plusendmoddef
my $total_time = 0;
my $DATE = localtime;
my $USER = defined($ENV\{USER\}) ? $ENV\{USER\} : 'Child Process';
\nwendcode{}\nwbegindocs{794}\nwdocspar

%%%%%%%%%%%%%%%%%%%% BACKMATTER

% \newpage
% 
% \bibliographystyle{apalike}
% \bibliography{/home1/rguigo/docs/biblio/References}

\appendix

\nwenddocs{}%
%
%
%
%
\nwbegindocs{796}\nwdocspar

\newpage

\sctn{\LaTeX{} wrappers} %%%%%%%%%%%%%%%%%%%%%%%%%%%%%%%%%%%%%%%%

\subsctn{Customization parameters}

Summarizing command-line options for their description in the {\prog} manual:

\nwenddocs{}\nwbegincode{797}\sublabel{NWgff18-LATI-1}\nwmargintag{{\nwtagstyle{}\subpageref{NWgff18-LATI-1}}}\moddef{LATEX command-line~{\nwtagstyle{}\subpageref{NWgff18-LATI-1}}}\endmoddef
%
% AplotCLoptions.tex
%
% Command-Line Options for "gff2aplot".
%
% \LA{}Version Control Id Tag~{\nwtagstyle{}\subpageref{NWgff18-VerM-1}}\RA{}
%
\LA{}latex wrappers code~{\nwtagstyle{}\subpageref{NWgff18-latJ-1}}\RA{}
%
\\newcommand\{\\cltitem\}[4]\{
  \\item[\\op\{#1 \\pr\{#2\}\}]\\hfill\\op\{#3\}=\\op\{#4\}\\\\[1ex]
  \}
\\newcommand\{\\clpitem\}[3]\{
  \\item[\\op\{#1 \\pr\{#3\}\}]\\hfill\\op\{#2\}=\\pr\{#3\}\\\\[1ex]
  \}
\\newcommand\{\\clsitem\}[3]\{
  \\item[\\op\{#1\}]\\hfill\\op\{#2\}=\\op\{#3\}\\\\[1ex]
  \}
\\newcommand\{\\clritem\}[1]\{
  \\item[\\op\{#1\}]\\ \\\\[1ex]
  \}
\\newcommand\{\\clmitem\}[1]\{
  \\item[\\op\{\\shortstack[l]\{\\ #1\}\}]\\ \\\\[1ex]
  \}
%
%%%%%%%%%%%%%%%%%%%%%%%%%%%%%%%%%%%%%%%%%%%%%%%%%%%%%%%%%%%%
\\begin\{description\}
\LA{}HIDE: command-line latex~{\nwtagstyle{}\subpageref{NWgff18-HIDO.2-1}}\RA{}
\\end\{description\}
%%%%%%%%%%%%%%%%%%%%%%%%%%%%%%%%%%%%%%%%%%%%%%%%%%%%%%%%%%%%
\nwnotused{LATEX\ command-line}\nwendcode{}\nwbegindocs{798}\nwdocspar

We build the command-line table:

\nwenddocs{}\nwbegincode{799}\sublabel{NWgff18-LATO-1}\nwmargintag{{\nwtagstyle{}\subpageref{NWgff18-LATO-1}}}\moddef{LATEX command-line table~{\nwtagstyle{}\subpageref{NWgff18-LATO-1}}}\endmoddef
%
% AplotCLoptionsTbl.tex
%
% Command-Line Options for "gff2aplot".
%
% \LA{}Version Control Id Tag~{\nwtagstyle{}\subpageref{NWgff18-VerM-1}}\RA{}
%
\LA{}latex wrappers code~{\nwtagstyle{}\subpageref{NWgff18-latJ-1}}\RA{}
%
\\label\{sec:CLoptions\}
\\newcommand\{\\rw\}[2]\{ \\op\{ #1 \} & #2 \\\\\}
%
%%%%%%%%%%%%%%%%%%%%%%%%%%%%%%%%%%%%%%%%%%%%%%%%%%%%%%%%%%%%
\\vfill 
%\\setlength\{\\intextsep\}\{0ex\}
%\\setlength\{\\textfloatsep\}\{0ex\}
%\\setlength\{\\floatsep\}\{0ex\}
\\begin\{table\}[!ht]
\\begin\{center\}
\\label\{tbl:CLopt\}
% \\refstepcounter\{section\}\\refstepcounter\{table\}
% \\addcontentsline\{lot\}\{section\}\{
%   \\thesection\\hspace\{1em\}Shell command-line options for \\prog.\}
% \\addcontentsline\{toc\}\{section\}\{
%   \\thesection\\hspace\{1em\}Shell command-line options for \\prog.\}
\\setlength\{\\fboxsep\}\{10pt\}
\\footnotesize
\\fbox\{
  \\begin\{tabular\}\{rl\}
\LA{}HIDE: command-line latex short~{\nwtagstyle{}\subpageref{NWgff18-HIDU-1}}\RA{}
  \\end\{tabular\}
\} % fbox
\\vspace\{0.5cm\}

%\\vfill
\\fbox\{
\\begin\{tabular\}\{rl\}
\\pa\{int\}       &
  An integer value. \\\\
\\pa\{float\}     &
  A float value. \\\\
\\pa\{string\}    &
  A free text string, single or double-quoted \\\\[-0.5ex]
 &  if special chars or white-spaces/tabs are present respectively. \\\\
\\pa\{format\}    &
  Page format, see available values on Appendix~\\ref\{tbl:PageSztbl\} table. \\\\
\\pa\{color\}     &
  A color name chosen from table on Appendix~\\ref\{tbl:CMYKcolortbl\}. \\\\
\\pa\{file\}      &
  A valid file name (including path if necessary). \\\\
NOTE:          &
  When a parameter is required, it applies for both short and long options. \\\\
\\end\{tabular\}
\} % fbox
%
\\end\{center\}
\\end\{table\}
\\vfill
%%%%%%%%%%%%%%%%%%%%%%%%%%%%%%%%%%%%%%%%%%%%%%%%%%%%%%%%%%%%
\nwnotused{LATEX\ command-line\ table}\nwendcode{}\nwbegindocs{800}\nwdocspar

We collect all customization variables for the manual too:

\nwenddocs{}\nwbegincode{801}\sublabel{NWgff18-LATH-1}\nwmargintag{{\nwtagstyle{}\subpageref{NWgff18-LATH-1}}}\moddef{LATEX custom-vars~{\nwtagstyle{}\subpageref{NWgff18-LATH-1}}}\endmoddef
%
% AplotCustomVars.tex
%
% Customization Variables for "gff2aplot".
%
% \LA{}Version Control Id Tag~{\nwtagstyle{}\subpageref{NWgff18-VerM-1}}\RA{}
%
\LA{}latex wrappers code~{\nwtagstyle{}\subpageref{NWgff18-latJ-1}}\RA{}
%
\\newcommand\{\\vnlitem\}[3]\{
  \\item[\\op\{#1=\\pr\{#2\}\}]\\hfill\\textbf\{Default\\quad=\\quad\}\\op\{#3\}\\ \\\\[1ex]
  \}
\\newcommand\{\\vnfitem\}[4]\{
  \\item[\\pa\{#1\}\\op\{::#2=\\pr\{#3\}\}]\\hfill\\textbf\{Default\\quad=\\quad\}\\op\{#4\}\\ \\\\[1ex]
  \}
%
%%%%%%%%%%%%%%%%%%%%%%%%%%%%%%%%%%%%%%%%%%%%%%%%%%%%%%%%%%%%
\\begin\{description\}
\LA{}HIDE: custom-vars latex~{\nwtagstyle{}\subpageref{NWgff18-HIDN.2-1}}\RA{}
\\end\{description\}
%%%%%%%%%%%%%%%%%%%%%%%%%%%%%%%%%%%%%%%%%%%%%%%%%%%%%%%%%%%%
\nwnotused{LATEX\ custom-vars}\nwendcode{}\nwbegindocs{802}\nwdocspar

Generating the custom variables table:

\nwenddocs{}\nwbegincode{803}\sublabel{NWgff18-LATN-1}\nwmargintag{{\nwtagstyle{}\subpageref{NWgff18-LATN-1}}}\moddef{LATEX custom-vars table~{\nwtagstyle{}\subpageref{NWgff18-LATN-1}}}\endmoddef
%
% AplotCustomVarsTbl.tex
%
% Customization Variables for "gff2aplot".
%
% \LA{}Version Control Id Tag~{\nwtagstyle{}\subpageref{NWgff18-VerM-1}}\RA{}
%
\LA{}latex wrappers code~{\nwtagstyle{}\subpageref{NWgff18-latJ-1}}\RA{}
%
%%%%%%%%%%%%%%%%%%%%%%%%%%%%%%%%%%%%%%%%%%%%%%%%%%%%%%%%%%%%
\LA{}HIDE: custom-vars latex short~{\nwtagstyle{}\subpageref{NWgff18-HIDT-1}}\RA{}
%%%%%%%%%%%%%%%%%%%%%%%%%%%%%%%%%%%%%%%%%%%%%%%%%%%%%%%%%%%%
\nwnotused{LATEX\ custom-vars\ table}\nwendcode{}\nwbegindocs{804}\nwdocspar

And here the common code for those files:

\nwenddocs{}\nwbegincode{805}\sublabel{NWgff18-latJ-1}\nwmargintag{{\nwtagstyle{}\subpageref{NWgff18-latJ-1}}}\moddef{latex wrappers code~{\nwtagstyle{}\subpageref{NWgff18-latJ-1}}}\endmoddef
% \\newcommand\{\\op\}[1]\{\\bfseries\\sffamily #1\}
% \\newcommand\{\\tp\}[1]\{'\{\\bfseries\\sffamily #1\}'\}
% \\newcommand\{\\pr\}[1]\{\\mdseries\\sffamily\\slshape $<$#1$>$\}
% \\newcommand\{\\x\}\{\\textmd\{\\,,\\,\}\}
\nwused{\\{NWgff18-LATI-1}\\{NWgff18-LATO-1}\\{NWgff18-LATH-1}\\{NWgff18-LATN-1}}\nwendcode{}\nwbegindocs{806}\nwdocspar

\subsctn{CMYK color palette}

Building the color definition file used by this file to produce the CMYK color definitions used in this document and the ``{\prog} User's Manual''.
% Table~\ref{tbl:CMYKcolortbl}.
\label{sec:cmykfiles}

\nwenddocs{}\nwbegincode{807}\sublabel{NWgff18-LATH.2-1}\nwmargintag{{\nwtagstyle{}\subpageref{NWgff18-LATH.2-1}}}\moddef{LATEX cmyk colors~{\nwtagstyle{}\subpageref{NWgff18-LATH.2-1}}}\endmoddef
%
% AplotColorDefs.tex
%
% Color CMYK definition used in "gff2aplot".
%
% \LA{}Version Control Id Tag~{\nwtagstyle{}\subpageref{NWgff18-VerM-1}}\RA{}
%
%%%%%%%%%%%%%%%%%%%%%%%%%%%%%%%%%%%%%%%%%%%%%%%%%%%%%%%%%%%%
\LA{}cmyk colors latex definition~{\nwtagstyle{}\subpageref{NWgff18-cmyS-1}}\RA{}
%%%%%%%%%%%%%%%%%%%%%%%%%%%%%%%%%%%%%%%%%%%%%%%%%%%%%%%%%%%%
\nwnotused{LATEX\ cmyk\ colors}\nwendcode{}\nwbegindocs{808}\nwdocspar

Building the CMYK color table for the ``{\prog} User's Manual''.
% Table~\ref{tbl:CMYKcolortbl} \LaTeX{} file.

\nwenddocs{}\nwbegincode{809}\sublabel{NWgff18-LATI.2-1}\nwmargintag{{\nwtagstyle{}\subpageref{NWgff18-LATI.2-1}}}\moddef{LATEX colors table~{\nwtagstyle{}\subpageref{NWgff18-LATI.2-1}}}\endmoddef
%
% AplotColorTbl.tex
%
% Colors used in "gff2aplot": CMYK values table.
%
\\label\{sec:colortable\}
\\newcommand\{\\clrow\}[1]\{
  \\fcolorbox\{black\}\{#1\}\{
    \\textcolor\{#1\}\{\\rule[-.3ex]\{1cm\}\{1.8ex\}\}
    \} % fcolorbox
  & #1
  \} % newcommand
\\newcommand\{\\clspc\}\{&&&&&\\\\\}
%
%%%%%%%%%%%%%%%%%%%%%%%%%%%%%%%%%%%%%%%%%%%%%%%%%%%%%%%%%%%%
\\vfill
\\begin\{table\}[!ht]
%\\setlength\{\\parindent\}\{-0.5cm\}
\\begin\{center\}
\\begin\{scriptsize\}
  \\begin\{tabular\}\{c@\{\\quad\}c\}
    \\begin\{tabular\}\{|c|c|cccc|\} \\hline
      \LA{}cmyk color table - left~{\nwtagstyle{}\subpageref{NWgff18-cmyN-1}}\RA{}
      \\clspc
      \\hline
    \\end\{tabular\} 
   &
    \\begin\{tabular\}\{|c|c|cccc|\} \\hline
      \LA{}cmyk color table - right~{\nwtagstyle{}\subpageref{NWgff18-cmyO-1}}\RA{}
      \\clspc
      \\hline
    \\end\{tabular\}
   \\\\
  \\end\{tabular\}
\\end\{scriptsize\}
%\\begin\{center\}
  \\caption\{\\label\{tbl:CMYKcolortbl\}
    \{\\prog\} CMYK color definition table and Color Names.
    \} % caption
  %\\refstepcounter\{table\}
  %\\addcontentsline\{lot\}\{section\}\{
  %   \\thetable\\hspace\{1em\}\{\\prog\}\\ CMYK color definition table.
  %   \}
\\end\{center\}
\\end\{table\}
\\vfill
%%%%%%%%%%%%%%%%%%%%%%%%%%%%%%%%%%%%%%%%%%%%%%%%%%%%%%%%%%%%
\nwnotused{LATEX\ colors\ table}\nwendcode{}\nwbegindocs{810}\nwdocspar

\subsctn{Available page sizes}

\nwenddocs{}\nwbegincode{811}\sublabel{NWgff18-LATN.2-1}\nwmargintag{{\nwtagstyle{}\subpageref{NWgff18-LATN.2-1}}}\moddef{LATEX page format table~{\nwtagstyle{}\subpageref{NWgff18-LATN.2-1}}}\endmoddef
%
% AplotPageSizeTbl.tex
%
% Page Sizes used in "gff2aplot".
%
% \LA{}Version Control Id Tag~{\nwtagstyle{}\subpageref{NWgff18-VerM-1}}\RA{}
%
\\label\{sec:pagesizes\}
%
%%%%%%%%%%%%%%%%%%%%%%%%%%%%%%%%%%%%%%%%%%%%%%%%%%%%%%%%%%%%
\\vfill
\\begin\{table\}[!ht]
\\begin\{center\}
\\setlength\{\\fboxsep\}\{2pt\}
%\\setlength\{\\arrayrulewidth\}\{1pt\}
\\fbox\{
 \\begin\{tabular\}\{|c||r|r||r|r||r|r|\} \\hline
  \\raisebox\{-0.5ex\}[0pt]\{PAGE\} &
  \\multicolumn\{6\}\{c|\}\{PAGE SIZE\}\\\\ \\cline\{2-7\} \\raisebox\{0.25ex\}[0pt]\{FORMAT\} &
  \\multicolumn\{2\}\{c||\}\{(in points)\} &
  \\multicolumn\{2\}\{c||\}\{(in cms)\} &
  \\multicolumn\{2\}\{c|\}\{(in inches)\}\\\\ \\hline\\hline
  \LA{}page sizes latex definition~{\nwtagstyle{}\subpageref{NWgff18-pagR-1}}\RA{}
 \\end\{tabular\}
\} % fbox
\\caption\{\\label\{tbl:PageSztbl\}Page Sizes defined in \{\\prog\}.\}\\hspace\{1cm\}
  %\\refstepcounter\{table\}
  %\\addcontentsline\{lot\}\{section\}\{
  %   \\thetable\\hspace\{1em\}Page Sizes available at \{\\prog\}.\}
\\end\{center\}
\\end\{table\}
\\vfill
%%%%%%%%%%%%%%%%%%%%%%%%%%%%%%%%%%%%%%%%%%%%%%%%%%%%%%%%%%%%
\nwnotused{LATEX\ page\ format\ table}\nwendcode{}\nwbegindocs{812}\nwdocspar

\newpage

\sctn{Common code blocks} %%%%%%%%%%%%%%%%%%%%%%%%%%%%%%%%%%%%%%%%%%%%%%%%%%%%%%

\subsctn{PERL scripts}

\nwenddocs{}\nwbegincode{813}\sublabel{NWgff18-PERC-1}\nwmargintag{{\nwtagstyle{}\subpageref{NWgff18-PERC-1}}}\moddef{PERL shebang~{\nwtagstyle{}\subpageref{NWgff18-PERC-1}}}\endmoddef
#!/usr/bin/perl -w
# This is perl, version 5.005_03 built for i386-linux
\nwused{\\{NWgff18-APL5-1}}\nwendcode{}\nwbegindocs{814}\nwdocspar

The 'shebang' line can be emulated in several ways. Those systems that do not support '{\tt{}{\char35}!}' will emulate it with:

\nwenddocs{}\nwbegincode{815}\sublabel{NWgff18-PERM-1}\nwmargintag{{\nwtagstyle{}\subpageref{NWgff18-PERM-1}}}\moddef{PERL shebang emulation~{\nwtagstyle{}\subpageref{NWgff18-PERM-1}}}\endmoddef
#!/usr/bin/perl -w
eval 'exec /usr/bin/perl -w -S $0 $\{1+"$@"\}' if $running_under_some_shell;
\nwnotused{PERL\ shebang\ emulation}\nwendcode{}\nwbegindocs{816}\nwdocspar

Another way to emulate the 'shebang' line is the following devious construct that will work under any of csh, sh, or perl system.

\nwenddocs{}\nwbegincode{817}\sublabel{NWgff18-PERU-1}\nwmargintag{{\nwtagstyle{}\subpageref{NWgff18-PERU-1}}}\moddef{PERL shebang general emulation~{\nwtagstyle{}\subpageref{NWgff18-PERU-1}}}\endmoddef
eval '\{exit $?0\}' && 
  eval 'exec /usr/bin/perl -w -S $0 $\{1+"$@"\}' &&
    eval 'exec /usr/bin/perl -w -S $0 $argv:q' 
      if 0;
\nwnotused{PERL\ shebang\ general\ emulation}\nwendcode{}\nwbegindocs{818}\nwdocspar

We also include a copy of the GNU-GPL as a comment at first lines on the program, and the version control tag.

\nwenddocs{}\nwbegincode{819}\sublabel{NWgff18-ProC-1}\nwmargintag{{\nwtagstyle{}\subpageref{NWgff18-ProC-1}}}\moddef{Program Info~{\nwtagstyle{}\subpageref{NWgff18-ProC-1}}}\endmoddef
#
\LA{}GNU License PERL~{\nwtagstyle{}\subpageref{NWgff18-GNUG-1}}\RA{}
#
\LA{}Version Control Id Tag~{\nwtagstyle{}\subpageref{NWgff18-VerM-1}}\RA{}
#
\nwused{\\{NWgff18-APL5-1}}\nwendcode{}\nwbegindocs{820}\nwdocspar

We are using here {\tt{}perl\ -w} to show all warnings when running the script. We also {\tt{}use\ strict} to force all variable names to be declared.

\nwenddocs{}\nwbegincode{821}\sublabel{NWgff18-SkiV-1}\nwmargintag{{\nwtagstyle{}\subpageref{NWgff18-SkiV-1}}}\moddef{Skip comments and empty records~{\nwtagstyle{}\subpageref{NWgff18-SkiV-1}}}\endmoddef
next if /^\\#/o;
next if /^\\s*$/o;
chomp;
\nwnotused{Skip\ comments\ and\ empty\ records}\nwendcode{}\nwbegindocs{822}\nwdocspar

\nwenddocs{}\nwbegincode{823}\sublabel{NWgff18-ComF-9}\nwmargintag{{\nwtagstyle{}\subpageref{NWgff18-ComF-9}}}\moddef{Common Routines~{\nwtagstyle{}\subpageref{NWgff18-ComF-1}}}\plusendmoddef
#
sub max() \{
    my $z = shift @_;
    foreach my $l (@_) \{ $z = $l if $l > $z \};
    return $z;
\} # max
sub min() \{
    my $z = shift @_;
    foreach my $l (@_) \{ $z = $l if $l < $z \};
    return $z;
\} # min
#
sub fill_right() \{ $_[0].($_[2] x ($_[1] - length($_[0]))) \}
sub fill_left()  \{ ($_[2] x ($_[1] - length($_[0]))).$_[0] \}
sub fill_mid()   \{ 
    my $l = length($_[0]);
    my $k = int(($_[1] - $l)/2);
    ($_[2] x $k).$_[0].($_[2] x ($_[1] - ($l+$k)));
\} # fill_mid
#
sub counter \{ # $_[0]~current_pos++ $_[1]~char
    print STDERR "$_[1]";
    (($_[0] % 50) == 0) && (print STDERR "[".&fill_left($_[0],6,"0")."]\\n");
\} # counter
#
sub counter_end \{ # $_[0]~current_pos   $_[1]~char
    (($_[0] % 50) != 0) && (print STDERR "[".&fill_left($_[0],6,"0")."]\\n");
\} # counter_end
\nwendcode{}\nwbegindocs{824}\nwdocspar

\subsctn{BASH scripts}

\nwenddocs{}\nwbegincode{825}\sublabel{NWgff18-BASC-1}\nwmargintag{{\nwtagstyle{}\subpageref{NWgff18-BASC-1}}}\moddef{BASH shebang~{\nwtagstyle{}\subpageref{NWgff18-BASC-1}}}\endmoddef
#!/usr/bin/bash
# GNU bash, version 2.03.6(1)-release (i386-redhat-linux-gnu)
\LA{}Version Control Id Tag~{\nwtagstyle{}\subpageref{NWgff18-VerM-1}}\RA{}
#
SECONDS=0 # Reset Timing
# Which script are we running...
L="####################"
\{ echo "$L$L$L$L";
  echo "### RUNNING [$0]";
  echo "### Current date:`date`";
  echo "###"; \} 1>&2;
\nwused{\\{NWgff18-wea7-1}\\{NWgff18-LaT8-1}\\{NWgff18-TheA-1}}\nwendcode{}\nwbegindocs{826}\nwdocspar

\nwenddocs{}\nwbegincode{827}\sublabel{NWgff18-BASJ-1}\nwmargintag{{\nwtagstyle{}\subpageref{NWgff18-BASJ-1}}}\moddef{BASH script closure~{\nwtagstyle{}\subpageref{NWgff18-BASJ-1}}}\endmoddef
\{ echo "###"; echo "### Execution time for [$0] : $SECONDS secs";
  echo "$L$L$L$L";
  echo ""; \} 1>&2;
#
exit 0
\nwused{\\{NWgff18-wea7-1}\\{NWgff18-LaT8-1}\\{NWgff18-TheA-1}}\nwendcode{}\nwbegindocs{828}\nwdocspar

\subsctn{Version control tags}

This document is under Revision Control System (RCS). The version you are currently reading is the following:

\nwenddocs{}\nwbegincode{829}\sublabel{NWgff18-VerM-1}\nwmargintag{{\nwtagstyle{}\subpageref{NWgff18-VerM-1}}}\moddef{Version Control Id Tag~{\nwtagstyle{}\subpageref{NWgff18-VerM-1}}}\endmoddef
# $Id: gff2aplot.tex,v 1.2 2001-09-06 18:33:36 jabril Exp $
\nwused{\\{NWgff18-LATI-1}\\{NWgff18-LATO-1}\\{NWgff18-LATH-1}\\{NWgff18-LATN-1}\\{NWgff18-LATH.2-1}\\{NWgff18-LATN.2-1}\\{NWgff18-ProC-1}\\{NWgff18-BASC-1}}\nwendcode{}\nwbegindocs{830}\nwdocspar

\subsctn{GNU General Public License}

\nwenddocs{}\nwbegincode{831}\sublabel{NWgff18-GNUG-1}\nwmargintag{{\nwtagstyle{}\subpageref{NWgff18-GNUG-1}}}\moddef{GNU License PERL~{\nwtagstyle{}\subpageref{NWgff18-GNUG-1}}}\endmoddef
# %%%%%%%%%%%%%%%%%%%%%%%%%%%%%%%%%%%%%%%%%%%%%%%%%%%%%%%%%%%%%%%%%%%%
# %                          GFF2APLOT                               %
# %%%%%%%%%%%%%%%%%%%%%%%%%%%%%%%%%%%%%%%%%%%%%%%%%%%%%%%%%%%%%%%%%%%%
# 
#    Converting alignments in GFF format to PostScript dotplots.
# 
#     Copyright (C) 1999 - Josep Francesc ABRIL FERRANDO  
#                                  Thomas WIEHE                   
#                                 Roderic GUIGO SERRA       
#
# This program is free software; you can redistribute it and/or modify
# it under the terms of the GNU General Public License as published by
# the Free Software Foundation; either version 2 of the License, or
# (at your option) any later version.
# 
# This program is distributed in the hope that it will be useful,
# but WITHOUT ANY WARRANTY; without even the implied warranty of
# MERCHANTABILITY or FITNESS FOR A PARTICULAR PURPOSE.  See the
# GNU General Public License for more details.
# 
# You should have received a copy of the GNU General Public License
# along with this program; if not, write to the Free Software
# Foundation, Inc., 675 Mass Ave, Cambridge, MA 02139, USA.
# 
# %%%%%%%%%%%%%%%%%%%%%%%%%%%%%%%%%%%%%%%%%%%%%%%%%%%%%%%%%%%%%%%%%%%%
\nwused{\\{NWgff18-ProC-1}}\nwendcode{}\nwbegindocs{832}\nwdocspar

\nwenddocs{}\nwbegincode{833}\sublabel{NWgff18-GNUM-1}\nwmargintag{{\nwtagstyle{}\subpageref{NWgff18-GNUM-1}}}\moddef{GNU License PostScript~{\nwtagstyle{}\subpageref{NWgff18-GNUM-1}}}\endmoddef
% %%%%%%%%%%%%%%%%%%%%%%%%%%%%%%%%%%%%%%%%%%%%%%%%%%%%%%%%%%%%%%%%%%%%
% %                          GFF2APLOT                               %
% %%%%%%%%%%%%%%%%%%%%%%%%%%%%%%%%%%%%%%%%%%%%%%%%%%%%%%%%%%%%%%%%%%%%
% 
%    Converting alignments in GFF format to PostScript dotplots.
% 
%     Copyright (C) 1999 - Josep Francesc ABRIL FERRANDO  
%                                  Thomas WIEHE                   
%                                 Roderic GUIGO SERRA       
%
% This program is free software; you can redistribute it and/or modify
% it under the terms of the GNU General Public License as published by
% the Free Software Foundation; either version 2 of the License, or
% (at your option) any later version.
% 
% This program is distributed in the hope that it will be useful,
% but WITHOUT ANY WARRANTY; without even the implied warranty of
% MERCHANTABILITY or FITNESS FOR A PARTICULAR PURPOSE.  See the
% GNU General Public License for more details.
% 
% You should have received a copy of the GNU General Public License
% along with this program; if not, write to the Free Software
% Foundation, Inc., 675 Mass Ave, Cambridge, MA 02139, USA.
% 
% %%%%%%%%%%%%%%%%%%%%%%%%%%%%%%%%%%%%%%%%%%%%%%%%%%%%%%%%%%%%%%%%%%%%
\nwused{\\{NWgff18-PosH-1}}\nwendcode{}\nwbegindocs{834}\nwdocspar

\newpage

\sctn{Extracting code blocks from this document} %%%%%%%%%%%%%%%%%%%%%%%%%%%%%%%

From this file we can obtain both the code and the
documentation. The following instructions are needed:

\subsctn{Extracting Script code chunks from the {\tt{}noweb} file} % \\[-0.5ex]

\nwenddocs{}\nwbegincode{835}\sublabel{NWgff18-tan8-1}\nwmargintag{{\nwtagstyle{}\subpageref{NWgff18-tan8-1}}}\moddef{tangling~{\nwtagstyle{}\subpageref{NWgff18-tan8-1}}}\endmoddef
# showing line numbering comments in program
notangle -L -R"APLOT" $WORK/$nwfile.nw | \\
    perl -ne '$.>1 && print' - | cpif $BIN/gff2aplot.pl ;
chmod a+x $BIN/gff2aplot.pl ;
#
\nwalsodefined{\\{NWgff18-tan8-2}\\{NWgff18-tan8-3}\\{NWgff18-tan8-4}\\{NWgff18-tan8-5}}\nwnotused{tangling}\nwendcode{}\nwbegindocs{836}\nwdocspar
 
\nwenddocs{}%
%
%
%
%
\nwbegindocs{838}\nwdocspar
We use {\perltidy} to reformat final versions of perl scripts (without line numbering comments, correct indentations, etc...) and to pretty-print in html format.

\nwenddocs{}\nwbegincode{839}\sublabel{NWgff18-tan8-2}\nwmargintag{{\nwtagstyle{}\subpageref{NWgff18-tan8-2}}}\moddef{tangling~{\nwtagstyle{}\subpageref{NWgff18-tan8-1}}}\plusendmoddef
# reformating program with perltidy
notangle -R"APLOT" $WORK/$nwfile.nw | \\
    perltidy - | cpif $BIN/$nwfile.pl ;
# html pretty-printing program with perltidy
notangle -R"APLOT" $WORK/$nwfile.nw | \\
    perltidy -html - | cpif $DOCS/html/$nwfile.html ;
#
\nwendcode{}\nwbegindocs{840}\nwdocspar

\subsctn{Extracting different Config Files} % \\[-0.5ex]

\nwenddocs{}\nwbegincode{841}\sublabel{NWgff18-tan8-3}\nwmargintag{{\nwtagstyle{}\subpageref{NWgff18-tan8-3}}}\moddef{tangling~{\nwtagstyle{}\subpageref{NWgff18-tan8-1}}}\plusendmoddef
\nwendcode{}\nwbegindocs{842}\nwdocspar

\subsctn{Extracting documentation and \LaTeX{}'ing it} % \\[-0.5ex] %'

\nwenddocs{}\nwbegincode{843}\sublabel{NWgff18-tan8-4}\nwmargintag{{\nwtagstyle{}\subpageref{NWgff18-tan8-4}}}\moddef{tangling~{\nwtagstyle{}\subpageref{NWgff18-tan8-1}}}\plusendmoddef
# extracting main document processors
notangle -R"weaving"    $WORK/$nwfile.nw | cpif $WORK/nw2tex ;
notangle -R"LaTeXing"   $WORK/$nwfile.nw | cpif $WORK/ltx ;
notangle -R"The Manual" $WORK/$nwfile.nw | cpif $WORK/mkdoc ;
#
\nwendcode{}\nwbegindocs{844}\nwdocspar

{\tt{}cpif} is an auxiliary tool for use with {\tt{}notangle} and {\tt{}make}. {\tt{}cpif} compares its standard input to each of the named files, overwriting the file if the comparison succeeds or if the named file does not exist.

\nwenddocs{}\nwbegincode{845}\sublabel{NWgff18-tanY-1}\nwmargintag{{\nwtagstyle{}\subpageref{NWgff18-tanY-1}}}\moddef{tangling complementary LaTeX files~{\nwtagstyle{}\subpageref{NWgff18-tanY-1}}}\endmoddef
# extracting complementary material 
echo "Extracting \\"$DOCS/defs.tex\\"..." 1>&2 ;
notangle -R"HIDE: LaTeX deploy new definitions" $WORK/$nwfile.nw | \\
         cpif $DOCS/defs.tmp  ;
noweave -v -t4 -delay $DOCS/defs.tmp | cpif $DOCS/defs.tex ;
echo "Extracting \\"$DOCS/todo.tex\\"..." 1>&2 ;
notangle -R"HIDE: TODO" $WORK/$nwfile.nw | cpif $DOCS/todo.tmp ; 
noweave -v -t4 -delay $DOCS/todo.tmp | cpif $DOCS/todo.tex ;
echo "Extracting \\"$DOCS/localrefs.bib\\"..." 1>&2 ;
notangle -R"HIDE: BiBTeX references" $WORK/$nwfile.nw | \\
         cpif $DOCS/localrefs.bib ;
echo "Extracting \\"$DOCS/AplotCLoptions.tex\\"..." 1>&2 ;
notangle -R"LATEX command-line" $WORK/$nwfile.nw | \\
         cpif $DOCS/tables/AplotCLoptions.tex ;
echo "Extracting \\"$DOCS/AplotCLoptionsTbl.tex\\"..." ;
notangle -R"LATEX command-line table" $WORK/$nwfile.nw | \\
         cpif $DOCS/tables/AplotCLoptionsTbl.tex ;
echo "Extracting \\"$DOCS/AplotCustomVars.tex\\"..." 1>&2 ;
notangle -R"LATEX custom-vars" $WORK/$nwfile.nw  | \\
         cpif $DOCS/tables/AplotCustomVars.tex ;
echo "Extracting \\"$DOCS/AplotCustomVarsTbl.tex\\"..." 1>&2 ;
notangle -R"LATEX custom-vars table" $WORK/$nwfile.nw  | \\
         cpif $DOCS/tables/AplotCustomVarsTbl.tex ;
echo "Extracting \\"$DOCS/AplotColorDefs.tex\\"..." 1>&2 ;
notangle -R"LATEX cmyk colors" $WORK/$nwfile.nw  | \\
         cpif $DOCS/tables/AplotColorDefs.tex ;
echo "Extracting \\"$DOCS/AplotColorTbl.tex\\"..." 1>&2 ;
notangle -R"LATEX colors table" $WORK/$nwfile.nw  | \\
         cpif $DOCS/tables/AplotColorTbl.tex ;
echo "Extracting \\"$DOCS/AplotPageSizeTbl.tex\\"..." 1>&2 ;
notangle -R"LATEX page format table" $WORK/$nwfile.nw | \\
         cpif $DOCS/tables/AplotPageSizeTbl.tex ;
echo "Extracting \\"$DOCS/DataStructure_GFF.tex\\"..." 1>&2 ;
notangle -R"HIDE: GFF DATA STRUCTURE" $WORK/$nwfile.nw | \\
         cpif $DOCS/tables/DataStructure_GFF.tex ;
#
\nwused{\\{NWgff18-wea7-1}}\nwendcode{}\nwbegindocs{846}\nwdocspar

{\tt{}noweave} options:

\begin{tabular}{>{\bfseries}rp{14cm}}\hline
{\tt{}-v} & be verbose. \\
{\tt{}-t4} & tabs are made of 4 spaces. \\
{\tt{}-delay} & putting file-name and other information into the output before the first chunk of the  program, enabling a user to put a specialized \LaTeX {\tt{}{\char92}documentclass} command and other preamble material in the first documentation chunk. It also forces trailing cross-referencing information to be emitted just before the final chunk, instead of at the end of the document; the final chunk is expected to contain {\tt{}{\char92}end{\char123}document{\char125}}. Also implies {\tt{}-n} option, meanning not to wrap the output with a \LaTeX header/trailer. \\
\multicolumn{2}{l}{{\tt{}-filter\ 'elide\ "HIDE:\ *"'}} \\
 & this filter removes those code chunks we define in the pattern, in this case those starting with `{\tt{}HIDE:\ *}'. \\
{\tt{}-index} & \\\hline
\end{tabular}

\nwenddocs{}\nwbegincode{847}\sublabel{NWgff18-wea7-1}\nwmargintag{{\nwtagstyle{}\subpageref{NWgff18-wea7-1}}}\moddef{weaving~{\nwtagstyle{}\subpageref{NWgff18-wea7-1}}}\endmoddef
\LA{}BASH shebang~{\nwtagstyle{}\subpageref{NWgff18-BASC-1}}\RA{}
# weaving and LaTeXing
\LA{}BASH Environment Variables~{\nwtagstyle{}\subpageref{NWgff18-BASQ-1}}\RA{}
\LA{}tangling complementary LaTeX files~{\nwtagstyle{}\subpageref{NWgff18-tanY-1}}\RA{}
# extracting main documentation
echo "Weaving \\"$DOCS/$nwfile.tex\\"..." ;
noweave -v -t4 -delay -x -filter 'elide "HIDE: *"' \\
        $WORK/$nwfile.nw | cpif $DOCS/$nwfile.tex ;
# LaTeXing 
echo "LaTeXing \\"$DOCS/$nwfile.tex\\"..." ;
pushd $DOCS/ ;
latex $nwfile.tex ;
dvips $nwfile.dvi -o $nwfile.ps -t a4 ;
popd;
\LA{}BASH script closure~{\nwtagstyle{}\subpageref{NWgff18-BASJ-1}}\RA{}
\nwnotused{weaving}\nwendcode{}\nwbegindocs{848}\nwdocspar

\nwenddocs{}\nwbegincode{849}\sublabel{NWgff18-LaT8-1}\nwmargintag{{\nwtagstyle{}\subpageref{NWgff18-LaT8-1}}}\moddef{LaTeXing~{\nwtagstyle{}\subpageref{NWgff18-LaT8-1}}}\endmoddef
\LA{}BASH shebang~{\nwtagstyle{}\subpageref{NWgff18-BASC-1}}\RA{}
# only LaTeXing
\LA{}BASH Environment Variables~{\nwtagstyle{}\subpageref{NWgff18-BASQ-1}}\RA{}
pushd $DOCS/ ;
#
echo "### RUNNING LaTeX on $nwfile.tex" 1>&2 ;
latex $nwfile.tex ; 
latex $nwfile.tex ; 
latex $nwfile.tex ;
dvips $nwfile.dvi -o $nwfile.ps -t a4 ;
#
echo "### CONVERTING PS to PDF: $nwfile" 1>&2 ;
ps2pdf $nwfile.ps $nwfile.pdf ;
#
popd ;
\LA{}BASH script closure~{\nwtagstyle{}\subpageref{NWgff18-BASJ-1}}\RA{}
\nwnotused{LaTeXing}\nwendcode{}\nwbegindocs{850}%$

\nwenddocs{}\nwbegincode{851}\sublabel{NWgff18-TheA-1}\nwmargintag{{\nwtagstyle{}\subpageref{NWgff18-TheA-1}}}\moddef{The Manual~{\nwtagstyle{}\subpageref{NWgff18-TheA-1}}}\endmoddef
\LA{}BASH shebang~{\nwtagstyle{}\subpageref{NWgff18-BASC-1}}\RA{}
# only LaTeXing
\LA{}BASH Environment Variables~{\nwtagstyle{}\subpageref{NWgff18-BASQ-1}}\RA{}
MANUAL="GFF2APLOT_MANUAL";
pushd $DOCS/ ;
#
echo "### RUNNING LaTeX on $MANUAL.tex" 1>&2 ;
latex $MANUAL.tex ; 
latex $MANUAL.tex ; 
latex $MANUAL.tex ;
dvips $MANUAL.dvi -o $MANUAL.ps -t a4 ;
#
echo "### CONVERTING PS to PDF: $MANUAL" 1>&2 ;
ps2pdf $MANUAL.ps $MANUAL.pdf ;
#
popd ;
\LA{}BASH script closure~{\nwtagstyle{}\subpageref{NWgff18-BASJ-1}}\RA{}
\nwnotused{The\ Manual}\nwendcode{}\nwbegindocs{852}%$

\subsctn{Defining working shell variables for the current project} % \\[-0.5ex]

\nwenddocs{}\nwbegincode{853}\sublabel{NWgff18-BASQ-1}\nwmargintag{{\nwtagstyle{}\subpageref{NWgff18-BASQ-1}}}\moddef{BASH Environment Variables~{\nwtagstyle{}\subpageref{NWgff18-BASQ-1}}}\endmoddef
# Global Variables
WORK="$HOME/development/softjabril/gfftools/gff2aplot" ;
BIN="$WORK/bin" ;
PARAM="$BIN/param" ;
DOCS="$WORK/docs" ;
DATA="$WORK/data" ;
TEST="$WORK/tests" ;
nwfile="gff2aplot" ;
export WORK BIN PARAM DOCS DATA TEST nwfile ;
\nwused{\\{NWgff18-wea7-1}\\{NWgff18-LaT8-1}\\{NWgff18-TheA-1}}\nwendcode{}\nwbegindocs{854}\nwdocspar

\nwenddocs{}\nwbegincode{855}\sublabel{NWgff18-tan8-5}\nwmargintag{{\nwtagstyle{}\subpageref{NWgff18-tan8-5}}}\moddef{tangling~{\nwtagstyle{}\subpageref{NWgff18-tan8-1}}}\plusendmoddef
#
# BASH Environment Variables
notangle -R'BASH Environment Variables' $WORK/$nwfile.nw | \\
         cpif $WORK/.bash_VARS ; 
source $WORK/.bash_VARS ;
#
\nwendcode{}

\nwixlogsorted{c}{{APLOT}{NWgff18-APL5-1}{\nwixd{NWgff18-APL5-1}}}%
\nwixlogsorted{c}{{BASH Environment Variables}{NWgff18-BASQ-1}{\nwixu{NWgff18-wea7-1}\nwixu{NWgff18-LaT8-1}\nwixu{NWgff18-TheA-1}\nwixd{NWgff18-BASQ-1}}}%
\nwixlogsorted{c}{{BASH script closure}{NWgff18-BASJ-1}{\nwixd{NWgff18-BASJ-1}\nwixu{NWgff18-wea7-1}\nwixu{NWgff18-LaT8-1}\nwixu{NWgff18-TheA-1}}}%
\nwixlogsorted{c}{{BASH shebang}{NWgff18-BASC-1}{\nwixd{NWgff18-BASC-1}\nwixu{NWgff18-wea7-1}\nwixu{NWgff18-LaT8-1}\nwixu{NWgff18-TheA-1}}}%
\nwixlogsorted{c}{{Basic common block for gfftools}{NWgff18-BasV-1}{\nwixu{NWgff18-Mai9-1}\nwixd{NWgff18-BasV-1}}}%
\nwixlogsorted{c}{{Changes in command-line options}{NWgff18-ChaV-1}{\nwixd{NWgff18-ChaV-1}}}%
\nwixlogsorted{c}{{Changes in custom Vars}{NWgff18-ChaM-1}{\nwixd{NWgff18-ChaM-1}}}%
\nwixlogsorted{c}{{Common Routines}{NWgff18-ComF-1}{\nwixu{NWgff18-APL5-1}\nwixd{NWgff18-ComF-1}\nwixd{NWgff18-ComF-2}\nwixd{NWgff18-ComF-3}\nwixd{NWgff18-ComF-4}\nwixd{NWgff18-ComF-5}\nwixd{NWgff18-ComF-6}\nwixd{NWgff18-ComF-7}\nwixd{NWgff18-ComF-8}\nwixd{NWgff18-ComF-9}}}%
\nwixlogsorted{c}{{Features Setting}{NWgff18-FeaG-1}{\nwixu{NWgff18-MaiD-1}\nwixd{NWgff18-FeaG-1}\nwixd{NWgff18-FeaG-2}\nwixd{NWgff18-FeaG-3}\nwixd{NWgff18-FeaG-4}\nwixd{NWgff18-FeaG-5}}}%
\nwixlogsorted{c}{{GNU License PERL}{NWgff18-GNUG-1}{\nwixu{NWgff18-ProC-1}\nwixd{NWgff18-GNUG-1}}}%
\nwixlogsorted{c}{{GNU License PostScript}{NWgff18-GNUM-1}{\nwixu{NWgff18-PosH-1}\nwixd{NWgff18-GNUM-1}}}%
\nwixlogsorted{c}{{Global Constants}{NWgff18-GloG-1}{\nwixu{NWgff18-APL5-1}\nwixd{NWgff18-GloG-1}\nwixd{NWgff18-GloG-2}\nwixd{NWgff18-GloG-3}\nwixd{NWgff18-GloG-4}\nwixd{NWgff18-GloG-5}\nwixd{NWgff18-GloG-6}}}%
\nwixlogsorted{c}{{Global Vars}{NWgff18-GloB-1}{\nwixu{NWgff18-APL5-1}\nwixd{NWgff18-GloB-1}\nwixd{NWgff18-GloB-2}\nwixd{NWgff18-GloB-3}\nwixd{NWgff18-GloB-4}\nwixd{NWgff18-GloB-5}\nwixd{NWgff18-GloB-6}\nwixd{NWgff18-GloB-7}\nwixd{NWgff18-GloB-8}\nwixd{NWgff18-GloB-9}\nwixd{NWgff18-GloB-A}\nwixd{NWgff18-GloB-B}\nwixd{NWgff18-GloB-C}\nwixd{NWgff18-GloB-D}\nwixd{NWgff18-GloB-E}\nwixd{NWgff18-GloB-F}\nwixd{NWgff18-GloB-G}\nwixd{NWgff18-GloB-H}}}%
\nwixlogsorted{c}{{HIDE: BiBTeX references}{NWgff18-HIDN-1}{\nwixd{NWgff18-HIDN-1}}}%
\nwixlogsorted{c}{{HIDE: GFF DATA STRUCTURE}{NWgff18-HIDO.3-1}{\nwixd{NWgff18-HIDO.3-1}}}%
\nwixlogsorted{c}{{HIDE: LaTeX deploy new definitions}{NWgff18-HIDY-1}{\nwixd{NWgff18-HIDY-1}}}%
\nwixlogsorted{c}{{HIDE: TODO}{NWgff18-HIDA-1}{\nwixd{NWgff18-HIDA-1}\nwixd{NWgff18-HIDA-2}\nwixd{NWgff18-HIDA-3}\nwixd{NWgff18-HIDA-4}\nwixd{NWgff18-HIDA-5}}}%
\nwixlogsorted{c}{{HIDE: command-line latex}{NWgff18-HIDO.2-1}{\nwixd{NWgff18-HIDO.2-1}\nwixd{NWgff18-HIDO.2-2}\nwixd{NWgff18-HIDO.2-3}\nwixd{NWgff18-HIDO.2-4}\nwixd{NWgff18-HIDO.2-5}\nwixd{NWgff18-HIDO.2-6}\nwixd{NWgff18-HIDO.2-7}\nwixd{NWgff18-HIDO.2-8}\nwixd{NWgff18-HIDO.2-9}\nwixd{NWgff18-HIDO.2-A}\nwixd{NWgff18-HIDO.2-B}\nwixd{NWgff18-HIDO.2-C}\nwixd{NWgff18-HIDO.2-D}\nwixd{NWgff18-HIDO.2-E}\nwixd{NWgff18-HIDO.2-F}\nwixd{NWgff18-HIDO.2-G}\nwixd{NWgff18-HIDO.2-H}\nwixd{NWgff18-HIDO.2-I}\nwixd{NWgff18-HIDO.2-J}\nwixd{NWgff18-HIDO.2-K}\nwixd{NWgff18-HIDO.2-L}\nwixd{NWgff18-HIDO.2-M}\nwixd{NWgff18-HIDO.2-N}\nwixd{NWgff18-HIDO.2-O}\nwixd{NWgff18-HIDO.2-P}\nwixd{NWgff18-HIDO.2-Q}\nwixd{NWgff18-HIDO.2-R}\nwixd{NWgff18-HIDO.2-S}\nwixd{NWgff18-HIDO.2-T}\nwixd{NWgff18-HIDO.2-U}\nwixd{NWgff18-HIDO.2-V}\nwixd{NWgff18-HIDO.2-W}\nwixu{NWgff18-LATI-1}}}%
\nwixlogsorted{c}{{HIDE: command-line latex custom}{NWgff18-HIDV-1}{\nwixd{NWgff18-HIDV-1}\nwixd{NWgff18-HIDV-2}\nwixu{NWgff18-HIDO.2-3}}}%
\nwixlogsorted{c}{{HIDE: command-line latex short}{NWgff18-HIDU-1}{\nwixd{NWgff18-HIDU-1}\nwixd{NWgff18-HIDU-2}\nwixd{NWgff18-HIDU-3}\nwixd{NWgff18-HIDU-4}\nwixd{NWgff18-HIDU-5}\nwixd{NWgff18-HIDU-6}\nwixd{NWgff18-HIDU-7}\nwixd{NWgff18-HIDU-8}\nwixd{NWgff18-HIDU-9}\nwixd{NWgff18-HIDU-A}\nwixd{NWgff18-HIDU-B}\nwixd{NWgff18-HIDU-C}\nwixd{NWgff18-HIDU-D}\nwixd{NWgff18-HIDU-E}\nwixd{NWgff18-HIDU-F}\nwixd{NWgff18-HIDU-G}\nwixd{NWgff18-HIDU-H}\nwixd{NWgff18-HIDU-I}\nwixd{NWgff18-HIDU-J}\nwixd{NWgff18-HIDU-K}\nwixd{NWgff18-HIDU-L}\nwixd{NWgff18-HIDU-M}\nwixd{NWgff18-HIDU-N}\nwixd{NWgff18-HIDU-O}\nwixd{NWgff18-HIDU-P}\nwixd{NWgff18-HIDU-Q}\nwixd{NWgff18-HIDU-R}\nwixd{NWgff18-HIDU-S}\nwixd{NWgff18-HIDU-T}\nwixd{NWgff18-HIDU-U}\nwixd{NWgff18-HIDU-V}\nwixd{NWgff18-HIDU-W}\nwixd{NWgff18-HIDU-X}\nwixu{NWgff18-LATO-1}}}%
\nwixlogsorted{c}{{HIDE: command-line latex short custom}{NWgff18-HIDb-1}{\nwixd{NWgff18-HIDb-1}\nwixd{NWgff18-HIDb-2}\nwixu{NWgff18-HIDU-4}}}%
\nwixlogsorted{c}{{HIDE: custom-vars latex}{NWgff18-HIDN.2-1}{\nwixd{NWgff18-HIDN.2-1}\nwixd{NWgff18-HIDN.2-2}\nwixd{NWgff18-HIDN.2-3}\nwixd{NWgff18-HIDN.2-4}\nwixu{NWgff18-LATH-1}}}%
\nwixlogsorted{c}{{HIDE: custom-vars latex short}{NWgff18-HIDT-1}{\nwixd{NWgff18-HIDT-1}\nwixd{NWgff18-HIDT-2}\nwixd{NWgff18-HIDT-3}\nwixd{NWgff18-HIDT-4}\nwixu{NWgff18-LATN-1}}}%
\nwixlogsorted{c}{{HIDE: new LaTeX commands}{NWgff18-HIDO.4-1}{\nwixu{NWgff18-HIDY-1}\nwixd{NWgff18-HIDO.4-1}}}%
\nwixlogsorted{c}{{HIDE: new LaTeX definitions}{NWgff18-HIDR-1}{\nwixu{NWgff18-HIDY-1}\nwixd{NWgff18-HIDR-1}}}%
\nwixlogsorted{c}{{HIDE: new LaTeX pstricks}{NWgff18-HIDO-1}{\nwixu{NWgff18-HIDY-1}\nwixd{NWgff18-HIDO-1}}}%
\nwixlogsorted{c}{{HIDE: new LaTeX urls}{NWgff18-HIDK-1}{\nwixu{NWgff18-HIDY-1}\nwixd{NWgff18-HIDK-1}}}%
\nwixlogsorted{c}{{HIDE: new defs TODO}{NWgff18-HIDJ-1}{\nwixu{NWgff18-HIDY-1}\nwixd{NWgff18-HIDJ-1}\nwixd{NWgff18-HIDJ-2}\nwixd{NWgff18-HIDJ-3}\nwixd{NWgff18-HIDJ-4}\nwixd{NWgff18-HIDJ-5}}}%
\nwixlogsorted{c}{{HIDE: test file for the previous regexp tester}{NWgff18-HIDk-1}{\nwixd{NWgff18-HIDk-1}}}%
\nwixlogsorted{c}{{HIDE: testing regexes: regexes.test}{NWgff18-HIDZ-1}{\nwixd{NWgff18-HIDZ-1}}}%
\nwixlogsorted{c}{{HIDE: testing regexes: regexptester.pl}{NWgff18-HIDc-1}{\nwixd{NWgff18-HIDc-1}}}%
\nwixlogsorted{c}{{HIDE: testing regular expressions for element@id}{NWgff18-HIDm-1}{\nwixd{NWgff18-HIDm-1}}}%
\nwixlogsorted{c}{{LATEX cmyk colors}{NWgff18-LATH.2-1}{\nwixd{NWgff18-LATH.2-1}}}%
\nwixlogsorted{c}{{LATEX colors table}{NWgff18-LATI.2-1}{\nwixd{NWgff18-LATI.2-1}}}%
\nwixlogsorted{c}{{LATEX command-line}{NWgff18-LATI-1}{\nwixd{NWgff18-LATI-1}}}%
\nwixlogsorted{c}{{LATEX command-line table}{NWgff18-LATO-1}{\nwixd{NWgff18-LATO-1}}}%
\nwixlogsorted{c}{{LATEX custom-vars}{NWgff18-LATH-1}{\nwixd{NWgff18-LATH-1}}}%
\nwixlogsorted{c}{{LATEX custom-vars table}{NWgff18-LATN-1}{\nwixd{NWgff18-LATN-1}}}%
\nwixlogsorted{c}{{LATEX page format table}{NWgff18-LATN.2-1}{\nwixd{NWgff18-LATN.2-1}}}%
\nwixlogsorted{c}{{LaTeXing}{NWgff18-LaT8-1}{\nwixd{NWgff18-LaT8-1}}}%
\nwixlogsorted{c}{{Main Loop}{NWgff18-Mai9-1}{\nwixu{NWgff18-APL5-1}\nwixd{NWgff18-Mai9-1}}}%
\nwixlogsorted{c}{{Main Routines}{NWgff18-MaiD-1}{\nwixu{NWgff18-APL5-1}\nwixd{NWgff18-MaiD-1}\nwixd{NWgff18-MaiD-2}}}%
\nwixlogsorted{c}{{Making PS Figures}{NWgff18-MakH-1}{\nwixu{NWgff18-MaiD-1}\nwixd{NWgff18-MakH-1}}}%
\nwixlogsorted{c}{{PERL shebang}{NWgff18-PERC-1}{\nwixu{NWgff18-APL5-1}\nwixd{NWgff18-PERC-1}}}%
\nwixlogsorted{c}{{PERL shebang emulation}{NWgff18-PERM-1}{\nwixd{NWgff18-PERM-1}}}%
\nwixlogsorted{c}{{PERL shebang general emulation}{NWgff18-PERU-1}{\nwixd{NWgff18-PERU-1}}}%
\nwixlogsorted{c}{{P.I.D: set reference for each property}{NWgff18-P.Ic.2-1}{\nwixd{NWgff18-P.Ic.2-1}}}%
\nwixlogsorted{c}{{P.I.D: set reference to all properties}{NWgff18-P.Ic-1}{\nwixu{NWgff18-ParI-G}\nwixd{NWgff18-P.Ic-1}}}%
\nwixlogsorted{c}{{POSTSCRIPT aplotdict}{NWgff18-POSK.2-1}{\nwixu{NWgff18-PosF.2-1}\nwixd{NWgff18-POSK.2-1}}}%
\nwixlogsorted{c}{{POSTSCRIPT constants}{NWgff18-POSK-1}{\nwixu{NWgff18-PosH-1}\nwixd{NWgff18-POSK-1}}}%
\nwixlogsorted{c}{{POSTSCRIPT layout}{NWgff18-POSH-1}{\nwixu{NWgff18-PosF.2-1}\nwixd{NWgff18-POSH-1}}}%
\nwixlogsorted{c}{{POSTSCRIPT main function calls}{NWgff18-POSU-1}{\nwixu{NWgff18-PosF.2-1}\nwixd{NWgff18-POSU-1}}}%
\nwixlogsorted{c}{{POSTSCRIPT shortnames}{NWgff18-POSL-1}{\nwixu{NWgff18-PosH-1}\nwixd{NWgff18-POSL-1}}}%
\nwixlogsorted{c}{{POSTSCRIPT text functions}{NWgff18-POSP-1}{\nwixu{NWgff18-PosF.2-1}\nwixd{NWgff18-POSP-1}}}%
\nwixlogsorted{c}{{Parsing Command-Line Options}{NWgff18-ParS-1}{\nwixu{NWgff18-MaiD-1}\nwixd{NWgff18-ParS-1}\nwixd{NWgff18-ParS-2}\nwixd{NWgff18-ParS-3}\nwixd{NWgff18-ParS-4}\nwixd{NWgff18-ParS-5}}}%
\nwixlogsorted{c}{{Parsing Custom Files}{NWgff18-ParK-1}{\nwixu{NWgff18-MaiD-1}\nwixd{NWgff18-ParK-1}\nwixd{NWgff18-ParK-2}\nwixd{NWgff18-ParK-3}\nwixd{NWgff18-ParK-4}\nwixd{NWgff18-ParK-5}\nwixd{NWgff18-ParK-6}\nwixd{NWgff18-ParK-7}}}%
\nwixlogsorted{c}{{Parsing Input Data}{NWgff18-ParI-1}{\nwixu{NWgff18-MaiD-1}\nwixd{NWgff18-ParI-1}\nwixd{NWgff18-ParI-2}\nwixd{NWgff18-ParI-3}\nwixd{NWgff18-ParI-4}\nwixd{NWgff18-ParI-5}\nwixd{NWgff18-ParI-6}\nwixd{NWgff18-ParI-7}\nwixd{NWgff18-ParI-8}\nwixd{NWgff18-ParI-9}\nwixd{NWgff18-ParI-A}\nwixd{NWgff18-ParI-B}\nwixd{NWgff18-ParI-C}\nwixd{NWgff18-ParI-D}\nwixd{NWgff18-ParI-E}\nwixd{NWgff18-ParI-F}\nwixd{NWgff18-ParI-G}\nwixd{NWgff18-ParI-H}\nwixd{NWgff18-ParI-I}\nwixd{NWgff18-ParI-J}}}%
\nwixlogsorted{c}{{PostScript CODE Chunks}{NWgff18-PosM-1}{\nwixu{NWgff18-APL5-1}\nwixd{NWgff18-PosM-1}}}%
\nwixlogsorted{c}{{PostScript COLORS}{NWgff18-PosH.2-1}{\nwixu{NWgff18-PosM-1}\nwixd{NWgff18-PosH.2-1}}}%
\nwixlogsorted{c}{{PostScript FORMATS}{NWgff18-PosI-1}{\nwixu{NWgff18-PosM-1}\nwixd{NWgff18-PosI-1}}}%
\nwixlogsorted{c}{{PostScript HEADER}{NWgff18-PosH-1}{\nwixu{NWgff18-PosM-1}\nwixd{NWgff18-PosH-1}}}%
\nwixlogsorted{c}{{PostScript MAIN}{NWgff18-PosF.2-1}{\nwixu{NWgff18-PosM-1}\nwixd{NWgff18-PosF.2-1}}}%
\nwixlogsorted{c}{{PostScript PLOT}{NWgff18-PosF.3-1}{\nwixu{NWgff18-PosM-1}\nwixd{NWgff18-PosF.3-1}}}%
\nwixlogsorted{c}{{PostScript VARS}{NWgff18-PosF-1}{\nwixu{NWgff18-PosM-1}\nwixd{NWgff18-PosF-1}}}%
\nwixlogsorted{c}{{Postscript CLOSE PAGE}{NWgff18-PosL-1}{\nwixu{NWgff18-PosM-1}\nwixd{NWgff18-PosL-1}}}%
\nwixlogsorted{c}{{Postscript OPEN PAGE}{NWgff18-PosK-1}{\nwixu{NWgff18-PosM-1}\nwixd{NWgff18-PosK-1}}}%
\nwixlogsorted{c}{{Postscript TRAILER}{NWgff18-PosI.2-1}{\nwixu{NWgff18-PosM-1}\nwixd{NWgff18-PosI.2-1}}}%
\nwixlogsorted{c}{{Program Info}{NWgff18-ProC-1}{\nwixu{NWgff18-APL5-1}\nwixd{NWgff18-ProC-1}}}%
\nwixlogsorted{c}{{STDIN backwards compatibility}{NWgff18-STDT-1}{\nwixu{NWgff18-ParS-3}\nwixd{NWgff18-STDT-1}}}%
\nwixlogsorted{c}{{Setting Defaults}{NWgff18-SetG-1}{\nwixu{NWgff18-MaiD-1}\nwixd{NWgff18-SetG-1}}}%
\nwixlogsorted{c}{{Skip comments and empty records}{NWgff18-SkiV-1}{\nwixd{NWgff18-SkiV-1}}}%
\nwixlogsorted{c}{{Sorting Features}{NWgff18-SorG-1}{\nwixu{NWgff18-MaiD-1}\nwixd{NWgff18-SorG-1}\nwixd{NWgff18-SorG-2}\nwixd{NWgff18-SorG-3}}}%
\nwixlogsorted{c}{{The Manual}{NWgff18-TheA-1}{\nwixd{NWgff18-TheA-1}}}%
\nwixlogsorted{c}{{Timer ON}{NWgff18-Tim8-1}{\nwixu{NWgff18-UseB-4}\nwixd{NWgff18-Tim8-1}}}%
\nwixlogsorted{c}{{Trapping signals}{NWgff18-TraG-1}{\nwixu{NWgff18-UseB-1}\nwixd{NWgff18-TraG-1}}}%
\nwixlogsorted{c}{{UNDEFINED VARS}{NWgff18-UNDE-1}{\nwixu{NWgff18-APL5-1}\nwixd{NWgff18-UNDE-1}}}%
\nwixlogsorted{c}{{Use Modules}{NWgff18-UseB-1}{\nwixu{NWgff18-APL5-1}\nwixd{NWgff18-UseB-1}\nwixd{NWgff18-UseB-2}\nwixd{NWgff18-UseB-3}\nwixd{NWgff18-UseB-4}}}%
\nwixlogsorted{c}{{Version Control Id Tag}{NWgff18-VerM-1}{\nwixu{NWgff18-LATI-1}\nwixu{NWgff18-LATO-1}\nwixu{NWgff18-LATH-1}\nwixu{NWgff18-LATN-1}\nwixu{NWgff18-LATH.2-1}\nwixu{NWgff18-LATN.2-1}\nwixu{NWgff18-ProC-1}\nwixu{NWgff18-BASC-1}\nwixd{NWgff18-VerM-1}}}%
\nwixlogsorted{c}{{add align records to aplot DS}{NWgff18-addT-1}{\nwixu{NWgff18-ParI-E}\nwixd{NWgff18-addT-1}}}%
\nwixlogsorted{c}{{adding new feature elements}{NWgff18-addR-1}{\nwixu{NWgff18-ParI-E}\nwixd{NWgff18-addR-1}}}%
\nwixlogsorted{c}{{cmyk color table - left}{NWgff18-cmyN-1}{\nwixd{NWgff18-cmyN-1}\nwixd{NWgff18-cmyN-2}\nwixd{NWgff18-cmyN-3}\nwixd{NWgff18-cmyN-4}\nwixd{NWgff18-cmyN-5}\nwixd{NWgff18-cmyN-6}\nwixu{NWgff18-LATI.2-1}}}%
\nwixlogsorted{c}{{cmyk color table - right}{NWgff18-cmyO-1}{\nwixd{NWgff18-cmyO-1}\nwixd{NWgff18-cmyO-2}\nwixd{NWgff18-cmyO-3}\nwixd{NWgff18-cmyO-4}\nwixd{NWgff18-cmyO-5}\nwixd{NWgff18-cmyO-6}\nwixd{NWgff18-cmyO-7}\nwixu{NWgff18-LATI.2-1}}}%
\nwixlogsorted{c}{{cmyk colors latex definition}{NWgff18-cmyS-1}{\nwixd{NWgff18-cmyS-1}\nwixd{NWgff18-cmyS-2}\nwixd{NWgff18-cmyS-3}\nwixd{NWgff18-cmyS-4}\nwixd{NWgff18-cmyS-5}\nwixd{NWgff18-cmyS-6}\nwixd{NWgff18-cmyS-7}\nwixd{NWgff18-cmyS-8}\nwixd{NWgff18-cmyS-9}\nwixd{NWgff18-cmyS-A}\nwixd{NWgff18-cmyS-B}\nwixd{NWgff18-cmyS-C}\nwixd{NWgff18-cmyS-D}\nwixu{NWgff18-LATH.2-1}}}%
\nwixlogsorted{c}{{cmyk colors perl definition}{NWgff18-cmyR-1}{\nwixu{NWgff18-GloB-E}\nwixd{NWgff18-cmyR-1}\nwixd{NWgff18-cmyR-2}\nwixd{NWgff18-cmyR-3}\nwixd{NWgff18-cmyR-4}\nwixd{NWgff18-cmyR-5}\nwixd{NWgff18-cmyR-6}\nwixd{NWgff18-cmyR-7}\nwixd{NWgff18-cmyR-8}\nwixd{NWgff18-cmyR-9}\nwixd{NWgff18-cmyR-A}\nwixd{NWgff18-cmyR-B}\nwixd{NWgff18-cmyR-C}\nwixd{NWgff18-cmyR-D}}}%
\nwixlogsorted{c}{{colors help}{NWgff18-colB-1}{\nwixu{NWgff18-MaiD-2}\nwixd{NWgff18-colB-1}}}%
\nwixlogsorted{c}{{command-line help}{NWgff18-comH-1}{\nwixu{NWgff18-MaiD-2}\nwixd{NWgff18-comH-1}\nwixd{NWgff18-comH-2}\nwixd{NWgff18-comH-3}\nwixd{NWgff18-comH-4}\nwixd{NWgff18-comH-5}\nwixd{NWgff18-comH-6}\nwixd{NWgff18-comH-7}\nwixd{NWgff18-comH-8}\nwixd{NWgff18-comH-9}\nwixd{NWgff18-comH-A}\nwixd{NWgff18-comH-B}\nwixd{NWgff18-comH-C}\nwixd{NWgff18-comH-D}\nwixd{NWgff18-comH-E}\nwixd{NWgff18-comH-F}\nwixd{NWgff18-comH-G}\nwixd{NWgff18-comH-H}\nwixd{NWgff18-comH-I}\nwixd{NWgff18-comH-J}\nwixd{NWgff18-comH-K}\nwixd{NWgff18-comH-L}\nwixd{NWgff18-comH-M}\nwixd{NWgff18-comH-N}\nwixd{NWgff18-comH-O}\nwixd{NWgff18-comH-P}\nwixd{NWgff18-comH-Q}\nwixd{NWgff18-comH-R}\nwixd{NWgff18-comH-S}\nwixd{NWgff18-comH-T}\nwixd{NWgff18-comH-U}\nwixd{NWgff18-comH-V}\nwixd{NWgff18-comH-W}\nwixd{NWgff18-comH-X}\nwixd{NWgff18-comH-Y}\nwixd{NWgff18-comH-Z}}}%
\nwixlogsorted{c}{{command-line help custom}{NWgff18-comO-1}{\nwixd{NWgff18-comO-1}\nwixd{NWgff18-comO-2}\nwixu{NWgff18-comH-5}}}%
\nwixlogsorted{c}{{command-line options}{NWgff18-comK-1}{\nwixu{NWgff18-ParS-1}\nwixd{NWgff18-comK-1}\nwixd{NWgff18-comK-2}\nwixd{NWgff18-comK-3}\nwixd{NWgff18-comK-4}\nwixd{NWgff18-comK-5}\nwixd{NWgff18-comK-6}\nwixd{NWgff18-comK-7}\nwixd{NWgff18-comK-8}\nwixd{NWgff18-comK-9}\nwixd{NWgff18-comK-A}\nwixd{NWgff18-comK-B}\nwixd{NWgff18-comK-C}\nwixd{NWgff18-comK-D}\nwixd{NWgff18-comK-E}\nwixd{NWgff18-comK-F}\nwixd{NWgff18-comK-G}\nwixd{NWgff18-comK-H}\nwixd{NWgff18-comK-I}\nwixd{NWgff18-comK-J}\nwixd{NWgff18-comK-K}\nwixd{NWgff18-comK-L}\nwixd{NWgff18-comK-M}\nwixd{NWgff18-comK-N}\nwixd{NWgff18-comK-O}\nwixd{NWgff18-comK-P}\nwixd{NWgff18-comK-Q}\nwixd{NWgff18-comK-R}\nwixd{NWgff18-comK-S}\nwixd{NWgff18-comK-T}\nwixd{NWgff18-comK-U}\nwixd{NWgff18-comK-V}\nwixd{NWgff18-comK-W}\nwixd{NWgff18-comK-X}}}%
\nwixlogsorted{c}{{command-line options custom}{NWgff18-comR-1}{\nwixd{NWgff18-comR-1}\nwixd{NWgff18-comR-2}\nwixu{NWgff18-comK-3}}}%
\nwixlogsorted{c}{{command-line options with exit}{NWgff18-comU-1}{\nwixu{NWgff18-ParS-1}\nwixd{NWgff18-comU-1}}}%
\nwixlogsorted{c}{{default feature vars values}{NWgff18-defR-1}{\nwixu{NWgff18-SetG-1}\nwixd{NWgff18-defR-1}}}%
\nwixlogsorted{c}{{default group vars values}{NWgff18-defP-1}{\nwixu{NWgff18-SetG-1}\nwixd{NWgff18-defP-1}}}%
\nwixlogsorted{c}{{default layout vars values}{NWgff18-defQ-1}{\nwixu{NWgff18-SetG-1}\nwixd{NWgff18-defQ-1}\nwixd{NWgff18-defQ-2}\nwixd{NWgff18-defQ-3}\nwixd{NWgff18-defQ-4}}}%
\nwixlogsorted{c}{{default sequence vars values}{NWgff18-defS-1}{\nwixu{NWgff18-SetG-1}\nwixd{NWgff18-defS-1}}}%
\nwixlogsorted{c}{{default source vars values}{NWgff18-defQ.2-1}{\nwixu{NWgff18-SetG-1}\nwixd{NWgff18-defQ.2-1}}}%
\nwixlogsorted{c}{{default strand vars values}{NWgff18-defQ.3-1}{\nwixu{NWgff18-SetG-1}\nwixd{NWgff18-defQ.3-1}}}%
\nwixlogsorted{c}{{does string contain a regular expression}{NWgff18-doee-1}{\nwixu{NWgff18-ParK-5}\nwixd{NWgff18-doee-1}}}%
\nwixlogsorted{c}{{environment vars help}{NWgff18-envL-1}{\nwixu{NWgff18-MaiD-2}\nwixd{NWgff18-envL-1}}}%
\nwixlogsorted{c}{{latex wrappers code}{NWgff18-latJ-1}{\nwixu{NWgff18-LATI-1}\nwixu{NWgff18-LATO-1}\nwixu{NWgff18-LATH-1}\nwixu{NWgff18-LATN-1}\nwixd{NWgff18-latJ-1}}}%
\nwixlogsorted{c}{{looking for STDIN}{NWgff18-looH-1}{\nwixu{NWgff18-ParS-1}\nwixd{NWgff18-looH-1}}}%
\nwixlogsorted{c}{{map default vars to defaults working copy}{NWgff18-mapf-1}{\nwixu{NWgff18-SetG-1}\nwixd{NWgff18-mapf-1}}}%
\nwixlogsorted{c}{{messages - input/output}{NWgff18-outN.2-1}{\nwixd{NWgff18-outN.2-1}\nwixd{NWgff18-outN.2-2}\nwixd{NWgff18-outN.2-3}\nwixu{NWgff18-GloB-G}}}%
\nwixlogsorted{c}{{messages - parsing GFF files}{NWgff18-mesS-1}{\nwixd{NWgff18-mesS-1}\nwixu{NWgff18-GloB-G}}}%
\nwixlogsorted{c}{{messages - parsing command-line options}{NWgff18-mesd-1}{\nwixd{NWgff18-mesd-1}\nwixu{NWgff18-GloB-G}}}%
\nwixlogsorted{c}{{messages - parsing custom files}{NWgff18-mesV-1}{\nwixd{NWgff18-mesV-1}\nwixu{NWgff18-GloB-G}}}%
\nwixlogsorted{c}{{messages - sorting GFF data}{NWgff18-mesR-1}{\nwixd{NWgff18-mesR-1}\nwixd{NWgff18-mesR-2}\nwixd{NWgff18-mesR-3}\nwixd{NWgff18-mesR-4}\nwixd{NWgff18-mesR-5}\nwixu{NWgff18-GloB-G}}}%
\nwixlogsorted{c}{{open LOGFILE}{NWgff18-opeC-1}{\nwixu{NWgff18-ParS-1}\nwixd{NWgff18-opeC-1}}}%
\nwixlogsorted{c}{{page sizes latex definition}{NWgff18-pagR-1}{\nwixd{NWgff18-pagR-1}\nwixd{NWgff18-pagR-2}\nwixd{NWgff18-pagR-3}\nwixu{NWgff18-LATN.2-1}}}%
\nwixlogsorted{c}{{page sizes perl definition}{NWgff18-pagQ-1}{\nwixu{NWgff18-GloB-F}\nwixd{NWgff18-pagQ-1}\nwixd{NWgff18-pagQ-2}\nwixd{NWgff18-pagQ-3}}}%
\nwixlogsorted{c}{{pages help}{NWgff18-pagA-1}{\nwixu{NWgff18-MaiD-2}\nwixd{NWgff18-pagA-1}}}%
\nwixlogsorted{c}{{parse grouping attributes}{NWgff18-parP-1}{\nwixu{NWgff18-ParI-H}\nwixd{NWgff18-parP-1}}}%
\nwixlogsorted{c}{{parse other attributes}{NWgff18-parM-1}{\nwixu{NWgff18-ParI-H}\nwixd{NWgff18-parM-1}\nwixd{NWgff18-parM-2}}}%
\nwixlogsorted{c}{{perl requires help}{NWgff18-perI-1}{\nwixu{NWgff18-MaiD-2}\nwixd{NWgff18-perI-1}\nwixd{NWgff18-perI-2}\nwixd{NWgff18-perI-3}\nwixd{NWgff18-perI-4}}}%
\nwixlogsorted{c}{{set forward/reverse sorting functon}{NWgff18-revZ-1}{\nwixu{NWgff18-sorI-1}\nwixd{NWgff18-revZ-1}}}%
\nwixlogsorted{c}{{set reference to GFF data}{NWgff18-setP-1}{\nwixu{NWgff18-FeaG-2}\nwixd{NWgff18-setP-1}}}%
\nwixlogsorted{c}{{set reference to GFF data: FEATURES}{NWgff18-setZ-1}{\nwixu{NWgff18-setX-1}\nwixd{NWgff18-setZ-1}}}%
\nwixlogsorted{c}{{set reference to GFF data: GROUPS}{NWgff18-setX-1}{\nwixu{NWgff18-setY.2-1}\nwixd{NWgff18-setX-1}}}%
\nwixlogsorted{c}{{set reference to GFF data: SOURCES}{NWgff18-setY-1}{\nwixu{NWgff18-setP-1}\nwixd{NWgff18-setY-1}}}%
\nwixlogsorted{c}{{set reference to GFF data: STRANDS}{NWgff18-setY.2-1}{\nwixu{NWgff18-setY-1}\nwixd{NWgff18-setY.2-1}}}%
\nwixlogsorted{c}{{set sequence elements boundaries}{NWgff18-setW-1}{\nwixu{NWgff18-sorH-1}\nwixd{NWgff18-setW-1}}}%
\nwixlogsorted{c}{{set source elements boundaries}{NWgff18-setU-1}{\nwixu{NWgff18-sorF-1}\nwixd{NWgff18-setU-1}}}%
\nwixlogsorted{c}{{setting layout variables}{NWgff18-setO-1}{\nwixu{NWgff18-FeaG-1}\nwixd{NWgff18-setO-1}}}%
\nwixlogsorted{c}{{setting non-layout variables}{NWgff18-setS-1}{\nwixu{NWgff18-FeaG-1}\nwixd{NWgff18-setS-1}}}%
\nwixlogsorted{c}{{sorting GFF groups}{NWgff18-sorI-1}{\nwixu{NWgff18-sorF-1}\nwixd{NWgff18-sorI-1}}}%
\nwixlogsorted{c}{{sorting group coords by acceptor}{NWgff18-sorW-1}{\nwixu{NWgff18-sorI-1}\nwixd{NWgff18-sorW-1}}}%
\nwixlogsorted{c}{{sorting group elements}{NWgff18-sorM-1}{\nwixu{NWgff18-sorI-1}\nwixd{NWgff18-sorM-1}}}%
\nwixlogsorted{c}{{sorting sequences}{NWgff18-sorH-1}{\nwixu{NWgff18-SorG-1}\nwixd{NWgff18-sorH-1}}}%
\nwixlogsorted{c}{{sorting sources}{NWgff18-sorF-1}{\nwixu{NWgff18-sorH-1}\nwixd{NWgff18-sorF-1}}}%
\nwixlogsorted{c}{{tangling}{NWgff18-tan8-1}{\nwixd{NWgff18-tan8-1}\nwixd{NWgff18-tan8-2}\nwixd{NWgff18-tan8-3}\nwixd{NWgff18-tan8-4}\nwixd{NWgff18-tan8-5}}}%
\nwixlogsorted{c}{{tangling complementary LaTeX files}{NWgff18-tanY-1}{\nwixd{NWgff18-tanY-1}\nwixu{NWgff18-wea7-1}}}%
\nwixlogsorted{c}{{variable value checking}{NWgff18-varN-1}{\nwixu{NWgff18-ParK-4}\nwixd{NWgff18-varN-1}\nwixd{NWgff18-varN-2}\nwixd{NWgff18-varN-3}\nwixd{NWgff18-varN-4}\nwixd{NWgff18-varN-5}\nwixd{NWgff18-varN-6}\nwixd{NWgff18-varN-7}\nwixd{NWgff18-varN-8}\nwixd{NWgff18-varN-9}}}%
\nwixlogsorted{c}{{warnings - input/output}{NWgff18-outN-1}{\nwixd{NWgff18-outN-1}\nwixd{NWgff18-outN-2}\nwixu{NWgff18-GloB-G}}}%
\nwixlogsorted{c}{{warnings - parsing GFF files}{NWgff18-warS-1}{\nwixd{NWgff18-warS-1}\nwixd{NWgff18-warS-2}\nwixd{NWgff18-warS-3}\nwixd{NWgff18-warS-4}\nwixu{NWgff18-GloB-G}}}%
\nwixlogsorted{c}{{warnings - parsing command-line options}{NWgff18-ward-1}{\nwixd{NWgff18-ward-1}\nwixu{NWgff18-GloB-G}}}%
\nwixlogsorted{c}{{warnings - parsing custom files}{NWgff18-warV-1}{\nwixd{NWgff18-warV-1}\nwixd{NWgff18-warV-2}\nwixd{NWgff18-warV-3}\nwixd{NWgff18-warV-4}\nwixd{NWgff18-warV-5}\nwixd{NWgff18-warV-6}\nwixd{NWgff18-warV-7}\nwixd{NWgff18-warV-8}\nwixd{NWgff18-warV-9}\nwixd{NWgff18-warV-A}\nwixd{NWgff18-warV-B}\nwixu{NWgff18-GloB-G}}}%
\nwixlogsorted{c}{{weaving}{NWgff18-wea7-1}{\nwixd{NWgff18-wea7-1}}}%
\nwbegindocs{856}\nwdocspar

\end{document}
\nwenddocs{}
