%
% AplotCustomVars.tex
%
% Customization Variables for "gff2aplot".
%
% # $Id: AplotCustomVars.tex,v 1.1 2001-08-21 14:54:21 jabril Exp $
%
% \newcommand{\op}[1]{\bfseries\sffamily #1}
% \newcommand{\tp}[1]{'{\bfseries\sffamily #1}'}
% \newcommand{\pr}[1]{\mdseries\sffamily\slshape $<$#1$>$}
% \newcommand{\x}{\textmd{\,,\,}}
%
\newcommand{\vnlitem}[3]{
  \item[\op{#1=\pr{#2}}]\hfill\textbf{Default\quad=\quad}\op{#3}\ \\[1ex]
  }
\newcommand{\vnfitem}[4]{
  \item[\pa{#1}\op{::#2=\pr{#3}}]\hfill\textbf{Default\quad=\quad}\op{#4}\ \\[1ex]
  }
%
%%%%%%%%%%%%%%%%%%%%%%%%%%%%%%%%%%%%%%%%%%%%%%%%%%%%%%%%%%%%
\begin{description}
\vnlitem{title}{string}{align\_name} \tp{title} sets the main title
for the figure, by default showing \pr{align\_name} in the form of
'sequence1\_name x sequence2\_name'.
\vnlitem{align\_tag}{string}{Target} \tp{align\_tag} holds the specific tag (for a grouping Tag-Value pair) to detect those GFF records coding for alignment data and having the following group structure:\\[1ex]
\centerline{\shortstack[r]{align\_tag "group\_name" \pr{seq2\_start} \pr{seq2\_end} [ ; Strand \pr{seq2\_strand} \eoline\\\,[ ; Frame \pr{seq2\_frame} [ ; E\_value \pr{score} ]\,]\,]\hspace{3ex}}}\\[0.75ex]
so the program can recover the coords for the target sequence from the grouping fields. Remember that tags are not case-sensitive. 
\vnlitem{vector\_tag}{string}{Vector} \tp{vector\_tag} sets the specific tag (for a grouping Tag-Value pair) to detect scoring vector GFF records, which have the following group structure:\\[1ex]
\centerline{\shortstack[r]{vector\_tag "group\_name" \pr{vector\_type} [ ; Window \pr{window\_length} \eoline\\\, [ ; Step \pr{step\_length} ]\,] ; Scores \pr{score$_0$} ... \pr{score$_n$}\hspace{3ex}}}\\[0.75ex]
so the program can parse the list of single scores from the grouping fields. Remember that tags are not case-sensitive.
\vnlitem{label\_tag}{string}{Id} \tp{label\_tag} sets the specific tag (for a grouping Tag-Value pair) that allows to identify a single record. The program looks for the grouping Tag-Value pair for which the tag matches \tp{label\_tag} and takes \tp{value} as the specific record label. You must have an attribute like this within the grouping attribute list:\\[0.75ex]
\centerline{\ldots ; label\_tag "element\_label" [ ; \ldots }\\[0.5ex]
Using this attribute makes easy to set specific properties for one or more elements. Remember that tags are not case-sensitive.
\end{description}
%%%%%%%%%%%%%%%%%%%%%%%%%%%%%%%%%%%%%%%%%%%%%%%%%%%%%%%%%%%%
