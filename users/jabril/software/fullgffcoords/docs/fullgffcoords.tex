% -*- mode: Noweb; noweb-code-mode: perl-mode; tab-width: 4 -*-
\documentclass[11pt]{article}
%
%2345678901234567890123456789012345678901234567890123456789012345678901234567890
%        1         2         3         4         5         6         7         8
%
% # $Id: fullgffcoords.tex,v 1.1 2001-10-22 14:44:13 jabril Exp $ 
%
\usepackage{noweb}
\usepackage[a4paper,offset={0pt,0pt},hmargin={2cm,2cm},vmargin={1cm,1cm}]{geometry}
\usepackage{graphics}
\usepackage[dvips]{graphicx}
%% pstricks
\usepackage[dvips]{pstcol}
\usepackage{pstricks}
%\usepackage{pst-node}
%\usepackage{pst-char}
%\usepackage{pst-grad}
%% bibliography
\usepackage{natbib}
%% latex2html
\usepackage{url}
\usepackage{html}     
\usepackage{htmllist} 
%% tables    
\usepackage{dcolumn}
%\usepackage{colortbl}
%\usepackage{multirow}
%\usepackage{hhline}
%\usepackage{tabularx}
%% seminar
%\usepackage{semcolor,semlayer,semrot,semhelv,sem-page,slidesec}
%% draft watermark
%\usepackage[all,dvips]{draftcopy}
%\draftcopySetGrey{0.9}
%\draftcopyName{CONFIDENTIAL}{100}
%% layout
\usepackage{fancyhdr} % Do not use \usepackage{fancybox} -> TOCs disappear
%\usepackage{lscape}
%\usepackage{rotating}
%\usepackage{multicol}
%% fonts
\usepackage{times}\fontfamily{ptm}\selectfont
\usepackage{t1enc}

% noweb options
\noweboptions{smallcode}
\def\nwendcode{\endtrivlist \endgroup} % relax page breaking scheme
\let\nwdocspar=\par                    %

\input defs.tex % from <LaTeX new definitions> chunk

%
%%%%%%%%%%%%%%%%%%%%%%%%%%%%%%%%%%%%%%%%%%%%%%%%%%%%%%%%%%%%%%%%%%%%%%%%%%%%%%%%
%
\begin{document}
%
\nwfilename{/home/ug/jabril/development/softjabril/fullgffcoords/fullgffcoords.nw}%
%
%
%
%
%
%
%
%
%
%
%
%
%
%
%
%
%
%
%
%
%
%
%
%
%
%
%
%
%
%
%
%
%
%
%
%
%
%
%
%
%
%
%
%
%
%
%
%
%
%
%
%
%
%
%
%
%
%
%
%
%
%
%
%
%
%
%
%
%
%
%
%
%
%
%
%
%
%
%
%
%
%
%
%
%
%
%
%
%
%
%
%
%
%
%
\nwbegindocs{2}\nwdocspar

\nwenddocs{}%
\nwbegindocs{4}\nwdocspar
\nwenddocs{}%
\nwbegindocs{6}\nwdocspar
\nwenddocs{}%
\nwbegindocs{8}\nwdocspar
\nwenddocs{}%
\nwbegindocs{10}\nwdocspar
\nwenddocs{}%
\nwbegindocs{12}\nwdocspar

\thispagestyle{empty}

\begin{titlepage}

\ \vfill
\begin{center}
\begin{bfseries}
\begin{large}
\newlength{\lttbl}\setlength{\lttbl}{0.25\linewidth}
\newlength{\rttbl}\setlength{\rttbl}{0.70\linewidth}
%\fbox{
%\vskip 2ex
\begin{tabular}{>{\scshape}r@{\quad}l}
\rule{\lttbl}{0pt} & \rule{\rttbl}{0pt} \\[2ex]
\multicolumn{2}{c}{\shortstack{\rule[0ex]{0.95\linewidth}{2pt}\\[0ex]
                               \rule[1ex]{0.95\linewidth}{2pt}}}\\[2ex]
Program Name: & {\Huge\progname}                       \\[3ex]
\multicolumn{2}{c}{\rule[0.5ex]{0.95\linewidth}{2pt}}\\[2ex]
      Author: & {\Large
                 \begin{minipage}[t]{0.95\rttbl}
                 \authorslist
                 \end{minipage}}                       \\[2ex]
     License: & {\license}                             \\[2ex]
 Last Update: & {\today}                               \\[2ex]
 Description: & {\large\mdseries
                 \begin{minipage}[t]{0.95\rttbl}
                 \progdesc
                 \end{minipage}}                       \\[2ex]
\\
\multicolumn{2}{c}{\shortstack{\rule[0ex]{0.95\linewidth}{2pt}\\[0ex]
                               \rule[1ex]{0.95\linewidth}{2pt}}}\\[2ex]
\end{tabular}
%} % fbox
\end{large}
\end{bfseries}
\end{center}

\vfill

\begin{raggedleft}
\showaffiliation
\end{raggedleft}

\end{titlepage} %'

%
%%%%%%%%%%%%%%%%%%%% FRONTMATTER

\newpage %%%%%%%%%%%%%%%%%%%%%%%%%%%%%%%%%%%%%%%%%%%%%%%%%
\pagenumbering{roman}
\setcounter{page}{1}
\pagestyle{fancy}
% Marks redefinition must go here because pagestyle 
% resets the values to the default ones.
\renewcommand{\sectionmark}[1]{\markboth{}{\thesection.\ #1}}
\renewcommand{\subsectionmark}[1]{\markboth{}{\thesubsection.\ \textsl{#1}}}

\tableofcontents
\listoftables
\listoffigures

\vfill
\begin{center}
{\small$<$ \verb$Id: fullgffcoords.tex,v 1.1 2001-10-22 14:44:13 jabril Exp $$>$ }
\end{center}

%%%%%%%%%%%%%%%%%%%% MAINMATTER

\newpage %%%%%%%%%%%%%%%%%%%%%%%%%%%%%%%%%%%%%%%%%%%%%%%%%
\pagenumbering{arabic}
\setcounter{page}{1}

\sctn{Introduction}

\subsctn{Program description}

\label{todo:AAA}\label{todo:AAB}
\nwenddocs{}%
%
%
\nwbegindocs{14}\nwdocspar
\nwenddocs{}%
%
%
\nwbegindocs{16}\nwdocspar
\todo{ \item \todoAAA 
       \item \todoAAB } % todo

\subsctn{Input}

It reads GFF files from genomic annotations (not alignments) and uses grouping field to determine each gene boundaries.

\subsctn{Output}

Basic output is a GFF file similar to input one, to which we have appended, to the end of the grouping field and for every record, the CDS and protein coords for that GFF record.

{\tt{}\ Field{\char95}1\ ...\ Field{\char95}8\ \ Grouping;\ \ CDS\ ori\ end;\ \ AA\ ori\ end;}

% \subsctn{Comments}

\subsctn{To Do}

\begin{itemize}
 \input todo.tex
\end{itemize}

\newpage %%%%%%%%%%%%%%%%%%%%%%%%%%%%%%%%%%%%%%%%%%%%%%%%%

\sctn{Implementation}

\nwenddocs{}\nwbegincode{17}\sublabel{NWful17-ProC-1}\nwmargintag{{\nwtagstyle{}\subpageref{NWful17-ProC-1}}}\moddef{Program Info~{\nwtagstyle{}\subpageref{NWful17-ProC-1}}}\endmoddef
my $PROGRAM = 'fullgffcoords.pl';
my $VERSION = '0.1';
\nwused{\\{NWful17-PERC-1}}\nwendcode{}\nwbegindocs{18}\nwdocspar
\nwenddocs{}\nwbegincode{19}\sublabel{NWful17-ProA-1}\nwmargintag{{\nwtagstyle{}\subpageref{NWful17-ProA-1}}}\moddef{Prog USAGE~{\nwtagstyle{}\subpageref{NWful17-ProA-1}}}\endmoddef
$PROGRAM [options] < input_files > output_files
\nwused{\\{NWful17-ParS-4}}\nwendcode{}\nwbegindocs{20}\nwdocspar
\nwenddocs{}\nwbegincode{21}\sublabel{NWful17-Pro9-1}\nwmargintag{{\nwtagstyle{}\subpageref{NWful17-Pro9-1}}}\moddef{Prog DESC~{\nwtagstyle{}\subpageref{NWful17-Pro9-1}}}\endmoddef
Retrieving CDS and protein coords from GFF mapped on genomic coords.
\nwused{\\{NWful17-ParS-4}}\nwendcode{}\nwbegindocs{22}\nwdocspar

\nwenddocs{}\nwbegincode{23}\sublabel{NWful17-ProJ-1}\nwmargintag{{\nwtagstyle{}\subpageref{NWful17-ProJ-1}}}\moddef{Program Description~{\nwtagstyle{}\subpageref{NWful17-ProJ-1}}}\endmoddef
# #----------------------------------------------------------------#
# #                          fullgffcoords                         #
# #----------------------------------------------------------------#
# 
#  fullgffcoords [options] < input_files > output_files
#
#  Retrieving CDS and protein coords from GFF mapped on genomic coords.
#
#
#     Copyright (C) 2001 - Josep Francesc ABRIL FERRANDO  
\nwused{\\{NWful17-PERC-1}}\nwendcode{}\nwbegindocs{24}\nwdocspar

\subsctn{Program outline}

\nwenddocs{}\nwbegincode{25}\sublabel{NWful17-fulD-1}\nwmargintag{{\nwtagstyle{}\subpageref{NWful17-fulD-1}}}\moddef{fullgffcoords~{\nwtagstyle{}\subpageref{NWful17-fulD-1}}}\endmoddef
\LA{}PERL shebang~{\nwtagstyle{}\subpageref{NWful17-PERC-1}}\RA{}
#
# MODULES
#
\LA{}Use Modules~{\nwtagstyle{}\subpageref{NWful17-UseB-1}}\RA{}
#
# VARIABLES
#
\LA{}Global Vars~{\nwtagstyle{}\subpageref{NWful17-GloB-1}}\RA{}
#
# MAIN LOOP
#
\LA{}Main Loop~{\nwtagstyle{}\subpageref{NWful17-Mai9-1}}\RA{}
#
# FUNCTIONS
#
\LA{}Functions~{\nwtagstyle{}\subpageref{NWful17-Fun9-1}}\RA{}
\nwnotused{fullgffcoords}\nwendcode{}\nwbegindocs{26}\nwdocspar

\nwenddocs{}\nwbegincode{27}\sublabel{NWful17-UseB-1}\nwmargintag{{\nwtagstyle{}\subpageref{NWful17-UseB-1}}}\moddef{Use Modules~{\nwtagstyle{}\subpageref{NWful17-UseB-1}}}\endmoddef
\LA{}Use Modules - Benchmark~{\nwtagstyle{}\subpageref{NWful17-UseN-1}}\RA{}
\LA{}Use Modules - Getopt~{\nwtagstyle{}\subpageref{NWful17-UseK-1}}\RA{}
\nwused{\\{NWful17-fulD-1}}\nwendcode{}\nwbegindocs{28}\nwdocspar

\nwenddocs{}\nwbegincode{29}\sublabel{NWful17-GloB-1}\nwmargintag{{\nwtagstyle{}\subpageref{NWful17-GloB-1}}}\moddef{Global Vars~{\nwtagstyle{}\subpageref{NWful17-GloB-1}}}\endmoddef
\LA{}Boolean~{\nwtagstyle{}\subpageref{NWful17-Boo7-1}}\RA{}
\LA{}Counter vars~{\nwtagstyle{}\subpageref{NWful17-CouC-1}}\RA{}
\LA{}Stderr subs vars~{\nwtagstyle{}\subpageref{NWful17-StdG-1}}\RA{}
my ($id,$seq) = ('','');
my %CmdLineVar = (
                 \LA{}cmdline defaults - format~{\nwtagstyle{}\subpageref{NWful17-cmdP-1}}\RA{}
                 \LA{}command-line defaults - GFF~{\nwtagstyle{}\subpageref{NWful17-comR-1}}\RA{}
                 );
\nwalsodefined{\\{NWful17-GloB-2}\\{NWful17-GloB-3}\\{NWful17-GloB-4}\\{NWful17-GloB-5}}\nwused{\\{NWful17-fulD-1}}\nwendcode{}\nwbegindocs{30}\nwdocspar

\nwenddocs{}\nwbegincode{31}\sublabel{NWful17-Mai9-1}\nwmargintag{{\nwtagstyle{}\subpageref{NWful17-Mai9-1}}}\moddef{Main Loop~{\nwtagstyle{}\subpageref{NWful17-Mai9-1}}}\endmoddef
&main();

exit(0);
\nwused{\\{NWful17-fulD-1}}\nwendcode{}\nwbegindocs{32}\nwdocspar

\nwenddocs{}\nwbegincode{33}\sublabel{NWful17-mesQ-1}\nwmargintag{{\nwtagstyle{}\subpageref{NWful17-mesQ-1}}}\moddef{messages - program running~{\nwtagstyle{}\subpageref{NWful17-mesQ-1}}}\endmoddef
'PROG-START'  => "$line$s\\n$s Running $PROGRAM\\n$s\\n".
                 "$s HOST: $host\\n".
                 "$s USER: $USER\\n".
                 "$s DATE: $DATE\\n$s\\n$line$s\\n" ,
'PROG-FINISH' => "$s\\n$line$s\\n$s $PROGRAM FINISHED\\n$s\\n".
                 "$s TOTAL TIME: \\%s\\n$line" ,
\nwalsodefined{\\{NWful17-mesQ-2}\\{NWful17-mesQ-3}\\{NWful17-mesQ-4}\\{NWful17-mesQ-5}}\nwused{\\{NWful17-StdG-2}}\nwendcode{}\nwbegindocs{34}\nwdocspar

\nwenddocs{}\nwbegincode{35}\sublabel{NWful17-Fun9-1}\nwmargintag{{\nwtagstyle{}\subpageref{NWful17-Fun9-1}}}\moddef{Functions~{\nwtagstyle{}\subpageref{NWful17-Fun9-1}}}\endmoddef
\LA{}Parsing command line options~{\nwtagstyle{}\subpageref{NWful17-ParS-1}}\RA{}
\LA{}Main program functions~{\nwtagstyle{}\subpageref{NWful17-MaiM-1}}\RA{}
\LA{}Common PERL subs - Text fill~{\nwtagstyle{}\subpageref{NWful17-ComS.2-1}}\RA{}
\LA{}Common PERL subs - Counter~{\nwtagstyle{}\subpageref{NWful17-ComQ.2-1}}\RA{}
\LA{}Common PERL subs - Min Max~{\nwtagstyle{}\subpageref{NWful17-ComQ-1}}\RA{}
\LA{}Common PERL subs - Benchmark~{\nwtagstyle{}\subpageref{NWful17-ComS-1}}\RA{}
\LA{}Common PERL subs - STDERR~{\nwtagstyle{}\subpageref{NWful17-ComP-1}}\RA{}
\nwused{\\{NWful17-fulD-1}}\nwendcode{}\nwbegindocs{36}\nwdocspar


\sctn{Program functions}

\subsctn{Processing command-line options}

\nwenddocs{}\nwbegincode{37}\sublabel{NWful17-UseK-1}\nwmargintag{{\nwtagstyle{}\subpageref{NWful17-UseK-1}}}\moddef{Use Modules - Getopt~{\nwtagstyle{}\subpageref{NWful17-UseK-1}}}\endmoddef
use Getopt::Long;
Getopt::Long::Configure qw/ bundling /;
\nwused{\\{NWful17-UseB-1}}\nwendcode{}\nwbegindocs{38}\nwdocspar

\nwenddocs{}\nwbegincode{39}\sublabel{NWful17-perM-1}\nwmargintag{{\nwtagstyle{}\subpageref{NWful17-perM-1}}}\moddef{perl requires - Getopt~{\nwtagstyle{}\subpageref{NWful17-perM-1}}}\endmoddef
"Getopt::Long" - processing command-line options.
\nwused{\\{NWful17-perI-1}}\nwendcode{}\nwbegindocs{40}\nwdocspar

See '{\tt{}man\ Getopt::Long}' for further info about this package.

\nwenddocs{}\nwbegincode{41}\sublabel{NWful17-ParS-1}\nwmargintag{{\nwtagstyle{}\subpageref{NWful17-ParS-1}}}\moddef{Parsing command line options~{\nwtagstyle{}\subpageref{NWful17-ParS-1}}}\endmoddef
sub parse_cmdline() \{
    \LA{}looking for STDIN~{\nwtagstyle{}\subpageref{NWful17-looH-1}}\RA{}

    $SIG\{__WARN__\} = sub \{ &warn('UNKNOWN_CL_OPTION',$T,$_[0]) \};
    GetOptions(
               \LA{}command-line options - format~{\nwtagstyle{}\subpageref{NWful17-comT-1}}\RA{}
               \LA{}command-line options - GFF~{\nwtagstyle{}\subpageref{NWful17-comQ.2-1}}\RA{}
               \LA{}command-line options with exit~{\nwtagstyle{}\subpageref{NWful17-comU-1}}\RA{}
               ) || (&warn('CMD_LINE_ERROR',$T), exit(1));
    $SIG\{__WARN__\} = 'DEFAULT';

    &report("PROG-START");
    @data_files = ();
    &set_input_file($cmdln_stdin);
    @ARGV = (); # ensuring that command-line ARGVs array is empty
    
\} # parse_cmdline
\nwalsodefined{\\{NWful17-ParS-2}\\{NWful17-ParS-3}\\{NWful17-ParS-4}}\nwused{\\{NWful17-Fun9-1}}\nwendcode{}\nwbegindocs{42}\nwdocspar

\nwenddocs{}\nwbegincode{43}\sublabel{NWful17-warN-1}\nwmargintag{{\nwtagstyle{}\subpageref{NWful17-warN-1}}}\moddef{warnings - command-line~{\nwtagstyle{}\subpageref{NWful17-warN-1}}}\endmoddef
'UNKNOWN_CL_OPTION' =>
  $Warn."Error trapped while processing command-line:\\n".(" "x16)."\\%s\\n",
'CMD_LINE_ERROR' =>
  $spl.$spw." Please, check your command-line options!!!\\n".$Error."\\n".
  $spw." ".("."x12)." Type \\"$PROGRAM -h\\" for help.\\n".$spl,
\nwused{\\{NWful17-StdG-2}}\nwendcode{}\nwbegindocs{44}\nwdocspar

To avoid errors reported when using '{\tt{}-}' as '{\tt{}STDIN}' mark and {\tt{}GetOptions} to parse command-line parameters, we capture the single dash when present in the command-line arguments list. '{\tt{}{\char36}cmdln{\char95}stdin}' will be used by '{\tt{}{\char38}set{\char95}input{\char95}file}' function to include the '{\tt{}STDIN}' in the correct ordering.\label{sec:stdinfix}

\nwenddocs{}\nwbegincode{45}\sublabel{NWful17-looH-1}\nwmargintag{{\nwtagstyle{}\subpageref{NWful17-looH-1}}}\moddef{looking for STDIN~{\nwtagstyle{}\subpageref{NWful17-looH-1}}}\endmoddef
my $cmdln_stdin = undef;
for (my $a = 0; $a <= $#ARGV; $a++) \{ 
    next unless $ARGV[$a] =~ /^-$/o;
    $cmdln_stdin = $a - $#ARGV;
    splice(@ARGV,$a,1);
\};    
\nwused{\\{NWful17-ParS-1}}\nwendcode{}\nwbegindocs{46}\nwdocspar

\subsubsctn{Testing command-line input filenames}

\label{todo:BAA}
\nwenddocs{}%
%
\nwbegindocs{48}\nwdocspar
\nwenddocs{}%
%
\nwbegindocs{50}\nwdocspar
\todo{ \item \todoBAA } % todo

\nwenddocs{}\nwbegincode{51}\sublabel{NWful17-ParS-2}\nwmargintag{{\nwtagstyle{}\subpageref{NWful17-ParS-2}}}\moddef{Parsing command line options~{\nwtagstyle{}\subpageref{NWful17-ParS-1}}}\plusendmoddef
sub set_input_file() \{
    my $stdin_flg = $F;
    \LA{}STDIN backwards compatibility~{\nwtagstyle{}\subpageref{NWful17-STDT-1}}\RA{}
    &report("CHECKING_FILENAMES");
  FILECHK: foreach my $test_file (@ARGV) \{
        $test_file ne '-' && do \{
            -e $test_file || do \{
                &warn('FILE_NO_OPEN',$T,$test_file);
                next FILECHK;
            \};
            &report('READING_FILE',$test_file);
            push @data_files, $test_file;
            next FILECHK;
        \};
        $stdin_flg = $T;
        push @data_files, '-';
    \}; # foreach
    scalar(@data_files) == 0 && do \{
        push @data_files, '-';
        $stdin_flg = $T;
    \};
    $stdin_flg && &report('READING_STDIN');
\} # set_input_file
\nwendcode{}\nwbegindocs{52}\nwdocspar

\nwenddocs{}\nwbegincode{53}\sublabel{NWful17-GloB-2}\nwmargintag{{\nwtagstyle{}\subpageref{NWful17-GloB-2}}}\moddef{Global Vars~{\nwtagstyle{}\subpageref{NWful17-GloB-1}}}\plusendmoddef
my @data_files = ();
\nwendcode{}\nwbegindocs{54}\nwdocspar

\nwenddocs{}\nwbegincode{55}\sublabel{NWful17-outN-1}\nwmargintag{{\nwtagstyle{}\subpageref{NWful17-outN-1}}}\moddef{warnings - input/output~{\nwtagstyle{}\subpageref{NWful17-outN-1}}}\endmoddef
FILE_NO_OPEN =>
  $spl.$Warn."Cannot Open Current file \\"\\%s\\" . Not used !!!\\n".$spl,
\nwalsodefined{\\{NWful17-outN-2}}\nwused{\\{NWful17-StdG-2}}\nwendcode{}\nwbegindocs{56}\nwdocspar

\nwenddocs{}\nwbegincode{57}\sublabel{NWful17-outN.2-1}\nwmargintag{{\nwtagstyle{}\subpageref{NWful17-outN.2-1}}}\moddef{messages - input/output~{\nwtagstyle{}\subpageref{NWful17-outN.2-1}}}\endmoddef
CHECKING_FILENAMES =>
  $sp."### Validating INPUT FILENAMES\\n".$sp,
READING_FILE =>
  "###---> \\"\\%s\\" exists, including as Input File.\\n",
READING_STDIN =>
  "###---> Including GFF records from standard input.\\n",  
\nwused{\\{NWful17-StdG-2}}\nwendcode{}\nwbegindocs{58}\nwdocspar

Here is the fix for the explained in section~\ref{sec:stdinfix} on page~\pageref{sec:stdinfix} ({\tt{}\LA{}looking for STDIN~{\nwtagstyle{}\subpageref{NWful17-looH-1}}\RA{}} code).

\nwenddocs{}\nwbegincode{59}\sublabel{NWful17-STDT-1}\nwmargintag{{\nwtagstyle{}\subpageref{NWful17-STDT-1}}}\moddef{STDIN backwards compatibility~{\nwtagstyle{}\subpageref{NWful17-STDT-1}}}\endmoddef
my $chk_stdin = shift @_;
my $t = scalar(@ARGV);
defined($chk_stdin) && do \{
    abs($chk_stdin) > $t && ($chk_stdin = -$t);
    $chk_stdin > 0  && ($chk_stdin = 0 );
    $t += $chk_stdin;
    splice(@ARGV,$t,0,'-');
\};
\nwused{\\{NWful17-ParS-2}}\nwendcode{}\nwbegindocs{60}\nwdocspar

\subsubsctn{Defining help and auxiliary code chunks}

The following command-line checkings look for those options exiting the program: '{\tt{}help}' and '{\tt{}version}'. Both need to output to screen without any other message/warning being displayed at the same time.

\nwenddocs{}\nwbegincode{61}\sublabel{NWful17-comU-1}\nwmargintag{{\nwtagstyle{}\subpageref{NWful17-comU-1}}}\moddef{command-line options with exit~{\nwtagstyle{}\subpageref{NWful17-comU-1}}}\endmoddef
"version"   => \\&prt_version, 
"h|help|?"  => \\&prt_help,
\nwused{\\{NWful17-ParS-1}}\nwendcode{}\nwbegindocs{62}\nwdocspar
\nwenddocs{}\nwbegincode{63}\sublabel{NWful17-comO-1}\nwmargintag{{\nwtagstyle{}\subpageref{NWful17-comO-1}}}\moddef{command-line help - help~{\nwtagstyle{}\subpageref{NWful17-comO-1}}}\endmoddef
-h, --help            Shows this help.
--version             Shows current version and exits.
\nwused{\\{NWful17-comH-1}}\nwendcode{}\nwbegindocs{64}\nwdocspar

\nwenddocs{}\nwbegincode{65}\sublabel{NWful17-ParS-3}\nwmargintag{{\nwtagstyle{}\subpageref{NWful17-ParS-3}}}\moddef{Parsing command line options~{\nwtagstyle{}\subpageref{NWful17-ParS-1}}}\plusendmoddef
sub prt_version() \{
    &report('SHOW_VERSION',$PROGRAM,$VERSION);
    exit(1);
\} # prt_version
\nwendcode{}\nwbegindocs{66}\nwdocspar

\nwenddocs{}\nwbegincode{67}\sublabel{NWful17-mesV-1}\nwmargintag{{\nwtagstyle{}\subpageref{NWful17-mesV-1}}}\moddef{messages - parsing command-line~{\nwtagstyle{}\subpageref{NWful17-mesV-1}}}\endmoddef
'SHOW_VERSION' => $sp."### \\%s -- Version: \\%s\\n".$sp,
\nwused{\\{NWful17-StdG-2}}\nwendcode{}\nwbegindocs{68}\nwdocspar

Printing command-line help to {\tt{}STDERR}:

\nwenddocs{}\nwbegincode{69}\sublabel{NWful17-ParS-4}\nwmargintag{{\nwtagstyle{}\subpageref{NWful17-ParS-4}}}\moddef{Parsing command line options~{\nwtagstyle{}\subpageref{NWful17-ParS-1}}}\plusendmoddef
sub prt_help() \{
    print STDERR <<"+++EndOfHelp+++";
PROGRAM:
                        $PROGRAM - $VERSION

    \LA{}Prog DESC~{\nwtagstyle{}\subpageref{NWful17-Pro9-1}}\RA{}

USAGE:    \LA{}Prog USAGE~{\nwtagstyle{}\subpageref{NWful17-ProA-1}}\RA{}


DESCRIPTION:

    \LA{}Prog DESC~{\nwtagstyle{}\subpageref{NWful17-Pro9-1}}\RA{}


REQUIRES:

    \LA{}perl requires help~{\nwtagstyle{}\subpageref{NWful17-perI-1}}\RA{}


COMMAND-LINE OPTIONS:

    \LA{}command-line help~{\nwtagstyle{}\subpageref{NWful17-comH-1}}\RA{}


BUGS:    Report any problem to 'jabril\\@imim.es'.

AUTHOR:  $PROGRAM is under GNU-GPL (C) 2000 - Josep F. Abril

+++EndOfHelp+++
    exit(1);
\} # prt_help
\nwendcode{}\nwbegindocs{70}\nwdocspar

\nwenddocs{}\nwbegincode{71}\sublabel{NWful17-perI-1}\nwmargintag{{\nwtagstyle{}\subpageref{NWful17-perI-1}}}\moddef{perl requires help~{\nwtagstyle{}\subpageref{NWful17-perI-1}}}\endmoddef
$PROGRAM needs the following Perl modules 
installed in your system, we used those available 
from the standard Perl distribution. Those that 
are not in the standard distribution are marked 
with an '(*)', in such cases make sure that you 
already have downloaded them from CPAN 
(http://www.perl.com/CPAN) and installed.

  \LA{}perl requires - Getopt~{\nwtagstyle{}\subpageref{NWful17-perM-1}}\RA{}
  \LA{}perl requires - Benchmark~{\nwtagstyle{}\subpageref{NWful17-perP-1}}\RA{}
\nwused{\\{NWful17-ParS-4}}\nwendcode{}\nwbegindocs{72}%$

\nwenddocs{}\nwbegincode{73}\sublabel{NWful17-comH-1}\nwmargintag{{\nwtagstyle{}\subpageref{NWful17-comH-1}}}\moddef{command-line help~{\nwtagstyle{}\subpageref{NWful17-comH-1}}}\endmoddef
A double dash on itself "--" signals end of the options
and start of file names (if present). After double dash,
you can use a single dash "-" as STDIN placeholder. 
Available options and a short description are listed here:

+ General options:

  \LA{}command-line help - help~{\nwtagstyle{}\subpageref{NWful17-comO-1}}\RA{}

  \LA{}command-line help - format~{\nwtagstyle{}\subpageref{NWful17-comQ-1}}\RA{}

+ GFF output:

  \LA{}command-line help - GFF~{\nwtagstyle{}\subpageref{NWful17-comN-1}}\RA{}
\nwused{\\{NWful17-ParS-4}}\nwendcode{}\nwbegindocs{74}\nwdocspar

\subsubsctn{Defining command-line options}

\nwenddocs{}\nwbegincode{75}\sublabel{NWful17-comT-1}\nwmargintag{{\nwtagstyle{}\subpageref{NWful17-comT-1}}}\moddef{command-line options - format~{\nwtagstyle{}\subpageref{NWful17-comT-1}}}\endmoddef
"g|genomic-coords" => sub \{ $CmdLineVar\{GFFCOORDS\} = 1 \},
"c|cds-coords"     => sub \{ $CmdLineVar\{GFFCOORDS\} = 2 \},
"p|protein-coords" => sub \{ $CmdLineVar\{GFFCOORDS\} = 3 \},
"i|include-groups" => \\$CmdLineVar\{SHOWGROUPS\},
\nwused{\\{NWful17-ParS-1}}\nwendcode{}\nwbegindocs{76}\nwdocspar
\nwenddocs{}\nwbegincode{77}\sublabel{NWful17-cmdP-1}\nwmargintag{{\nwtagstyle{}\subpageref{NWful17-cmdP-1}}}\moddef{cmdline defaults - format~{\nwtagstyle{}\subpageref{NWful17-cmdP-1}}}\endmoddef
GFFCOORDS  => 1,
SHOWGROUPS => 0,
\nwused{\\{NWful17-GloB-1}}\nwendcode{}\nwbegindocs{78}\nwdocspar
\nwenddocs{}\nwbegincode{79}\sublabel{NWful17-comQ-1}\nwmargintag{{\nwtagstyle{}\subpageref{NWful17-comQ-1}}}\moddef{command-line help - format~{\nwtagstyle{}\subpageref{NWful17-comQ-1}}}\endmoddef
-g, --genomic-coords   Output GFF coords fields (columns 4th and 5th)
                       are set as genomic coords (by default).
-c, --cds-coords       Output GFF coords fields (columns 4th and 5th)
                       are set to CDS coords.
-p, --protein-coords   Output GFF coords fields (columns 4th and 5th)
                       are set to protein coords (using partial codon
                       notation as explained in ... ).
-i, --include-groups   Output a GFF record with the start/end coords
                       for each group in the input GFF file.
\nwused{\\{NWful17-comH-1}}\nwendcode{}\nwbegindocs{80}\nwdocspar

\label{todo:CAA}\label{todo:CAB}
\nwenddocs{}%
%
%
\nwbegindocs{82}\nwdocspar
\nwenddocs{}%
%
%
\nwbegindocs{84}\nwdocspar
\todo{ \item \todoCAA 
       \item \todoCAB } % todo

\nwenddocs{}\nwbegincode{85}\sublabel{NWful17-comQ.2-1}\nwmargintag{{\nwtagstyle{}\subpageref{NWful17-comQ.2-1}}}\moddef{command-line options - GFF~{\nwtagstyle{}\subpageref{NWful17-comQ.2-1}}}\endmoddef
"1|gff-version1"    => sub \{ $CmdLineVar\{GFFVERSION\} = 1 \},
"2|gff-version2"    => sub \{ $CmdLineVar\{GFFVERSION\} = 2 \},
\nwused{\\{NWful17-ParS-1}}\nwendcode{}\nwbegindocs{86}\nwdocspar
\nwenddocs{}\nwbegincode{87}\sublabel{NWful17-comR-1}\nwmargintag{{\nwtagstyle{}\subpageref{NWful17-comR-1}}}\moddef{command-line defaults - GFF~{\nwtagstyle{}\subpageref{NWful17-comR-1}}}\endmoddef
GFFVERSION => 1,
\nwused{\\{NWful17-GloB-1}}\nwendcode{}\nwbegindocs{88}\nwdocspar
\nwenddocs{}\nwbegincode{89}\sublabel{NWful17-comN-1}\nwmargintag{{\nwtagstyle{}\subpageref{NWful17-comN-1}}}\moddef{command-line help - GFF~{\nwtagstyle{}\subpageref{NWful17-comN-1}}}\endmoddef
-2, --gff-version2
-1, --gff-version1     Output GFF version, default is 1, where version 1
                       means simple grouping field and version 2 forces
                       grouping to be tag-value pair grouping field, 
                       putting value between double quotes.
\nwused{\\{NWful17-comH-1}}\nwendcode{}\nwbegindocs{90}\nwdocspar

\subsctn{Main program functions}

\nwenddocs{}\nwbegincode{91}\sublabel{NWful17-MaiM-1}\nwmargintag{{\nwtagstyle{}\subpageref{NWful17-MaiM-1}}}\moddef{Main program functions~{\nwtagstyle{}\subpageref{NWful17-MaiM-1}}}\endmoddef
sub main() \{
    &parse_cmdline(); # PROG-START
    &set_output_subs();
    foreach my $lfile (@data_files) \{
        &parse_input_files($lfile);
    \}; # foreach $lfile
    &sort_by_acceptor();
    &output_extended_GFF();
    &report('PROG-FINISH',&timing($T));
\} # main
\nwalsodefined{\\{NWful17-MaiM-2}\\{NWful17-MaiM-3}\\{NWful17-MaiM-4}\\{NWful17-MaiM-5}\\{NWful17-MaiM-6}\\{NWful17-MaiM-7}\\{NWful17-MaiM-8}\\{NWful17-MaiM-9}\\{NWful17-MaiM-A}\\{NWful17-MaiM-B}}\nwused{\\{NWful17-Fun9-1}}\nwendcode{}\nwbegindocs{92}\nwdocspar


\subsubsctn{Defining referenced functions}

Here we describe how we set the variables that call by reference a funcion. They point to different subroutines depending on a given command-line options set.

Groups, when shown in the output, share the same format as the other GFF fields, so that, we do not have to define a new variable containing the output format for both data types.

\nwenddocs{}\nwbegincode{93}\sublabel{NWful17-MaiM-2}\nwmargintag{{\nwtagstyle{}\subpageref{NWful17-MaiM-2}}}\moddef{Main program functions~{\nwtagstyle{}\subpageref{NWful17-MaiM-1}}}\plusendmoddef
sub set_output_subs() \{
    &report("SETTING_VARS");
    my $base = "\\%s\\t" x 8 ;
    if ($CmdLineVar\{GFFVERSION\} == 1) \{
        $GFF = $base."\\%s  \\%s  \\%s\\n" ; # base + group coordsA coordsB
    \} else \{                        # base + tag "group"; coordsA; coordsB
        $GFF = $base.'gene_id "%s"; %s; %s'."\\n" ;
    \};
    CHECKCOORDS: \{
        $CmdLineVar\{GFFCOORDS\} == 3 && do \{
            $printGFF = \\&gff_protein;
            last CHECKCOORDS;
        \};
        $CmdLineVar\{GFFCOORDS\} == 2 && do \{
            $printGFF = \\&gff_cds;
            last CHECKCOORDS;
        \};
        $printGFF = \\&gff_genomic;
    \};
\} # set_output_subs
\nwendcode{}\nwbegindocs{94}\nwdocspar

\nwenddocs{}\nwbegincode{95}\sublabel{NWful17-GloB-3}\nwmargintag{{\nwtagstyle{}\subpageref{NWful17-GloB-3}}}\moddef{Global Vars~{\nwtagstyle{}\subpageref{NWful17-GloB-1}}}\plusendmoddef
my ($GFF,$printGFF);
\nwendcode{}\nwbegindocs{96}\nwdocspar
\nwenddocs{}\nwbegincode{97}\sublabel{NWful17-mesQ-2}\nwmargintag{{\nwtagstyle{}\subpageref{NWful17-mesQ-2}}}\moddef{messages - program running~{\nwtagstyle{}\subpageref{NWful17-mesQ-1}}}\plusendmoddef
SETTING_VARS => $sp."### Variable Definition Finished...\\n".$sp,
\nwendcode{}\nwbegindocs{98}\nwdocspar

The array passed to the {\tt{}{\char38}{\char36}printGFF} function has the same fields structure as it has the {\tt{}{\char38}gff{\char95}genomic} function. The other two functions just take a different ordering of such array.

\nwenddocs{}\nwbegincode{99}\sublabel{NWful17-MaiM-3}\nwmargintag{{\nwtagstyle{}\subpageref{NWful17-MaiM-3}}}\moddef{Main program functions~{\nwtagstyle{}\subpageref{NWful17-MaiM-1}}}\plusendmoddef
sub gff_genomic() \{
    my @data = @_;
    printf STDOUT $GFF, @data[0..8],
                        "CDS @data[9..10]", "AA @data[11..12]";
\} # gff_genomic
sub gff_cds() \{
    my @data = @_;
    printf STDOUT $GFF, @data[0..2,9..10,5..8],
                        "BASE @data[3..4]", "AA @data[11..12]";
\} # gff_cds
sub gff_protein() \{
    my @data = @_;
    printf STDOUT $GFF, @data[0..2,11..12,5..8],
                        "BASE @data[3..4]", "CDS @data[9..10]";
\} # gff_protein
\nwendcode{}\nwbegindocs{100}\nwdocspar


\subsubsctn{Parsing GFF records}

\nwenddocs{}\nwbegincode{101}\sublabel{NWful17-MaiM-4}\nwmargintag{{\nwtagstyle{}\subpageref{NWful17-MaiM-4}}}\moddef{Main program functions~{\nwtagstyle{}\subpageref{NWful17-MaiM-1}}}\plusendmoddef
sub parse_input_files() \{
    my $ifile = $_[0];
    &report("PARSING_GFF",$ifile);
    open(IFILE,"< $ifile") || do \{
        &warn('FILE_NO_OPEN',$T,$ifile);
        return;
    \};
    ($n,$c) = (0,undef);
    while (<IFILE>) \{
        my (@f,$start,$end,$group,$strand);
        $c = '*';
        \LA{}Skip comments and empty records~{\nwtagstyle{}\subpageref{NWful17-SkiV-1}}\RA{}
        @f = split /\\s+/o, $_, 9;
        ($group,$c) = &checkgroup($f[8]);  
        ($start,$end,$strand) = (@f[3,4],"$f[6]");
        ($start > $end) && do \{
            &warn('SWAPPINGCOORDS',$T,$start,$end);
            ($start,$end) = ($end,$start);
        \};
        defined($GeneList\{$group\}\{STRAND\}) || 
            ($GeneList\{$group\}\{STRAND\} = $strand);
        ($GeneList\{$group\}\{STRAND\} ne $strand) && do \{
            &warn('NOT_EQ_STRAND',$T,$GeneList\{$group\}\{STRAND\},$strand);
            # does nothing, just warns.
        \};
        push @\{ $GeneList\{$group\}\{RECORDS\} \}, [ @f[0..7] ];
        push @\{ $GeneList\{$group\}\{MIN\} \}, $start;
        push @\{ $GeneList\{$group\}\{MAX\} \}, $end;
        defined($GeneList\{$group\}\{LENGTH\}) || ($GeneList\{$group\}\{LENGTH\} = 0);
        $GeneList\{$group\}\{LENGTH\} += ($end - $start + 1);
    \} continue \{
        &counter(++$n,$c);
    \}; # while
    &counter_end($n,$c);
    close(IFILE);
\} # parse_input_files
\nwendcode{}\nwbegindocs{102}\nwdocspar

\nwenddocs{}\nwbegincode{103}\sublabel{NWful17-GloB-4}\nwmargintag{{\nwtagstyle{}\subpageref{NWful17-GloB-4}}}\moddef{Global Vars~{\nwtagstyle{}\subpageref{NWful17-GloB-1}}}\plusendmoddef
my %GeneList = ();
\nwendcode{}\nwbegindocs{104}\nwdocspar
\nwenddocs{}\nwbegincode{105}\sublabel{NWful17-mesQ-3}\nwmargintag{{\nwtagstyle{}\subpageref{NWful17-mesQ-3}}}\moddef{messages - program running~{\nwtagstyle{}\subpageref{NWful17-mesQ-1}}}\plusendmoddef
PARSING_GFF => $sp."### PARSING GFF FILE: \\%s\\n".$sp,
\nwendcode{}\nwbegindocs{106}\nwdocspar
\nwenddocs{}\nwbegincode{107}\sublabel{NWful17-outN-2}\nwmargintag{{\nwtagstyle{}\subpageref{NWful17-outN-2}}}\moddef{warnings - input/output~{\nwtagstyle{}\subpageref{NWful17-outN-1}}}\plusendmoddef
SWAPPINGCOORDS =>
  $Warn."START greater than END (\\%s > \\%s). SWAPPING COORDS !!!\\n",
NOT_EQ_STRAND =>
  $Warn."GROUP STRAND (\\%s) DOES NOT MATCH GFF-RECORD (\\%s). GROUP RULES !!!\\n",
\nwendcode{}\nwbegindocs{108}\nwdocspar

\nwenddocs{}\nwbegincode{109}\sublabel{NWful17-MaiM-5}\nwmargintag{{\nwtagstyle{}\subpageref{NWful17-MaiM-5}}}\moddef{Main program functions~{\nwtagstyle{}\subpageref{NWful17-MaiM-1}}}\plusendmoddef
sub checkgroup() \{
    my ($gpA,$gpB);
    my ($gpstr) = @_;
    $gpstr =~ /^([^\\s]+)(?:\\s+"(.+?)")?(?:.*)?/o && do \{
        $gpB = defined($2) ? $2 : ''; # for GFF version 2
        $gpA = defined($1) ? $1 : $gpB;  # for GFF version 1
    \};
    ($gpB ne '') && ( return $gpB, ':' );
    return $gpA, '.';
\} # checkgroups
\nwendcode{}\nwbegindocs{110}\nwdocspar


\subsubsctn{Sorting GFF records}

\nwenddocs{}\nwbegincode{111}\sublabel{NWful17-MaiM-6}\nwmargintag{{\nwtagstyle{}\subpageref{NWful17-MaiM-6}}}\moddef{Main program functions~{\nwtagstyle{}\subpageref{NWful17-MaiM-1}}}\plusendmoddef
sub sort_by_acceptor() \{
    &report("SORTING_FEATURES");
    $c = 0;
    foreach my $gname (keys %GeneList) \{
        my $ref = \\@\{ $GeneList\{$gname\}\{RECORDS\} \};
        print STDERR "$gname...(".($#\{$ref\} + 1)."ft)".
                     (((++$c % 4) == 0) ? "\\n" : "\\t");
        # sorting all features by acceptor
        @\{ $ref \} = map  \{ $_->[2] \}
                    sort \{ &sort_forward \}
                    map  \{ [ $_->[3], $_->[4], $_ ] \} @\{ $ref \};
        # getting group boundaries
        $GeneList\{$gname\}\{MIN\} = min(@\{ $GeneList\{$gname\}\{MIN\} \});
        $GeneList\{$gname\}\{MAX\} = max(@\{ $GeneList\{$gname\}\{MAX\} \});
        push @SortedGenes,
             [ $gname, $GeneList\{$gname\}\{MIN\}, $GeneList\{$gname\}\{MAX\},
               $GeneList\{$gname\}\{STRAND\}, $GeneList\{$gname\}\{LENGTH\} ];
    \}; # foreach $gname
    ((++$c % 4) != 0) && print STDERR "\\n";
    &report("SORTING_GROUPS");
    @SortedGenes = map  \{ $_->[2] \}
                   sort \{ &sort_forward \}
                   map  \{ [ $_->[1], $_->[2], $_ ] \} @SortedGenes;
\} # sort_by_acceptor
\nwendcode{}\nwbegindocs{112}\nwdocspar

\nwenddocs{}\nwbegincode{113}\sublabel{NWful17-GloB-5}\nwmargintag{{\nwtagstyle{}\subpageref{NWful17-GloB-5}}}\moddef{Global Vars~{\nwtagstyle{}\subpageref{NWful17-GloB-1}}}\plusendmoddef
my @SortedGenes = ();
\nwendcode{}\nwbegindocs{114}\nwdocspar
\nwenddocs{}\nwbegincode{115}\sublabel{NWful17-mesQ-4}\nwmargintag{{\nwtagstyle{}\subpageref{NWful17-mesQ-4}}}\moddef{messages - program running~{\nwtagstyle{}\subpageref{NWful17-mesQ-1}}}\plusendmoddef
SORTING_FEATURES => $sp."### Sorting GFF Group Features\\n".$sp,
SORTING_GROUPS   => $sp."### Sorting GFF Groups\\n".$sp,
\nwendcode{}\nwbegindocs{116}\nwdocspar

\nwenddocs{}\nwbegincode{117}\sublabel{NWful17-MaiM-7}\nwmargintag{{\nwtagstyle{}\subpageref{NWful17-MaiM-7}}}\moddef{Main program functions~{\nwtagstyle{}\subpageref{NWful17-MaiM-1}}}\plusendmoddef
sub sort_forward \{
    $a->[0] <=> $b->[0]  # sorting by start
             or
    $b->[1] <=> $a->[1]; # reverse sorting by end if same start
\} # sort_forward
\nwendcode{}\nwbegindocs{118}\nwdocspar


\subsubsctn{Thowing ``expanded output''}

\nwenddocs{}\nwbegincode{119}\sublabel{NWful17-MaiM-8}\nwmargintag{{\nwtagstyle{}\subpageref{NWful17-MaiM-8}}}\moddef{Main program functions~{\nwtagstyle{}\subpageref{NWful17-MaiM-1}}}\plusendmoddef
sub output_extended_GFF() \{
    my $rpt = '%-15s%12s%12s%7s%10s'."\\n";
    &report("WRITING_FEATURES");
    printf STDERR $rpt,"# GeneName","MinCoord","MaxCoord","Strand","CDSlength";
    for (my $g = 0; $g <= $#SortedGenes; $g++) \{
        my ($ref,$gene,$ori,$end,$str,$len,$f,$do_it,$base);
        ($gene,$ori,$end,$str,$len) = @\{ $SortedGenes[$g] \};
        printf STDERR $rpt,"  $gene",$ori,$end,$str,$len;
        $ref = \\@\{ $GeneList\{$gene\}\{RECORDS\} \};
        if ($str ne '-') \{
            $do_it = \\&do_it_forward;
            $base  = 0;
        \} else \{     
            $do_it = \\&do_it_reverse;
            $base  = $len + 1;
        \};
        for ($f = 0; $f <= $#\{$ref\}; $f++) \{
            my ($o_cds,$e_cds,$o_aa,$e_aa);
            # ($o_gn,$e_gn) = ($ref->[],$ref->[]);
            ($o_cds,$e_cds) = &$do_it($base,$ref->[$f][3],$ref->[$f][4]);
            $base = $e_cds;
            ($o_aa,$e_aa) = &to_protein($o_cds,$e_cds);
            ($str eq '-') && do \{ # just to display GFF-like (start<end)
                ($o_cds,$e_cds,$o_aa,$e_aa) = ($e_cds,$o_cds,$e_aa,$o_aa);
            \};
            &$printGFF(@\{ $ref->[$f] \},$gene,$o_cds,$e_cds,$o_aa,$e_aa);
        \}; # for $f
    \}; # for $g
\} # output_extended_GFF
\nwendcode{}\nwbegindocs{120}\nwdocspar

\nwenddocs{}\nwbegincode{121}\sublabel{NWful17-mesQ-5}\nwmargintag{{\nwtagstyle{}\subpageref{NWful17-mesQ-5}}}\moddef{messages - program running~{\nwtagstyle{}\subpageref{NWful17-mesQ-1}}}\plusendmoddef
WRITING_FEATURES => $sp."### Writing NEW GFF Records to STDOUT...\\n".$sp,
\nwendcode{}\nwbegindocs{122}\nwdocspar

\nwenddocs{}\nwbegincode{123}\sublabel{NWful17-MaiM-9}\nwmargintag{{\nwtagstyle{}\subpageref{NWful17-MaiM-9}}}\moddef{Main program functions~{\nwtagstyle{}\subpageref{NWful17-MaiM-1}}}\plusendmoddef
sub do_it_forward() \{
    my ($bori,$gori,$gend) = @_;
    my ($cori,$cend);
    $cori = $bori + 1;
    $cend = $bori + ($gend - $gori + 1);
    return ($cori,$cend);
\} # do_it_forward
\nwendcode{}\nwbegindocs{124}\nwdocspar

The following function, together with {\tt{}{\char36}GeneList{\char123}{\char36}gene{\char95}name{\char125}{\char123}LENGTH{\char125}}, is only needed if we are working with all the groups sorted by acceptor, an easier solution would be sorting reverse-strand groups by donor instead by acceptor (but then we produce GFF-records that are not sorted by acceptor in genomic coords, which is the main coords system by default ---just take into account Enrique's programs\ldots).%'

\nwenddocs{}\nwbegincode{125}\sublabel{NWful17-MaiM-A}\nwmargintag{{\nwtagstyle{}\subpageref{NWful17-MaiM-A}}}\moddef{Main program functions~{\nwtagstyle{}\subpageref{NWful17-MaiM-1}}}\plusendmoddef
sub do_it_reverse() \{
    my ($bori,$gori,$gend) = @_;
    my ($cori,$cend);
    $cori = $bori - 1;
    $cend = $cori - ($gend - $gori);
    return ($cori,$cend);
\} # do_it_reverse
\nwendcode{}\nwbegindocs{126}\nwdocspar

Here we need to convert to protein coords.

\nwenddocs{}\nwbegincode{127}\sublabel{NWful17-MaiM-B}\nwmargintag{{\nwtagstyle{}\subpageref{NWful17-MaiM-B}}}\moddef{Main program functions~{\nwtagstyle{}\subpageref{NWful17-MaiM-1}}}\plusendmoddef
sub to_protein() \{
    my ($cori,$cend) = @_;
    my ($pori,$pend);
    $pori = &toprot($cori);
    $pend = &toprot($cend);
    return ($pori,$pend);
\} # to_protein
sub toprot() \{
    my ($val) = @_;
    my ($a,$b);
    $a = $val % 3;                # nucleotide order within codon
    ($a == 0) && ($a = 3);
    $b = int(($val - 1) / 3) + 1; # codon number
    return "$b.$a"; 
\} # toprot
\nwendcode{}\nwbegindocs{128}\nwdocspar

%%%%%%%%%%%%%%%%%%%% BACKMATTER

% \newpage %%%%%%%%%%%%%%%%%%%%%%%%%%%%%%%%%%%%%%%%%%%%%%%%%
% 
% \bibliographystyle{apalike}
% \bibliography{/home1/rguigo/docs/biblio/References}

\newpage %%%%%%%%%%%%%%%%%%%%%%%%%%%%%%%%%%%%%%%%%%%%%%%%%
\appendix

\sctn{Common code blocks}

\subsctn{PERL scripts}

\nwenddocs{}\nwbegincode{129}\sublabel{NWful17-PERC-1}\nwmargintag{{\nwtagstyle{}\subpageref{NWful17-PERC-1}}}\moddef{PERL shebang~{\nwtagstyle{}\subpageref{NWful17-PERC-1}}}\endmoddef
#!/usr/bin/perl -w
# This is perl, version 5.005_03 built for i386-linux
#
\LA{}Program Description~{\nwtagstyle{}\subpageref{NWful17-ProJ-1}}\RA{}
#
\LA{}GNU License~{\nwtagstyle{}\subpageref{NWful17-GNUB-1}}\RA{}
#
\LA{}Version Control Id Tag~{\nwtagstyle{}\subpageref{NWful17-VerM-1}}\RA{}
#
use strict;
#
\LA{}Program Info~{\nwtagstyle{}\subpageref{NWful17-ProC-1}}\RA{}
my $DATE = localtime;
my $USER = defined($ENV\{USER\}) ? $ENV\{USER\} : 'Child Process';
my $host = `hostname`;
chomp($host);
#
\nwused{\\{NWful17-fulD-1}}\nwendcode{}\nwbegindocs{130}\nwdocspar

\nwenddocs{}\nwbegincode{131}\sublabel{NWful17-Boo7-1}\nwmargintag{{\nwtagstyle{}\subpageref{NWful17-Boo7-1}}}\moddef{Boolean~{\nwtagstyle{}\subpageref{NWful17-Boo7-1}}}\endmoddef
my ($T,$F) = (1,0); # for 'T'rue and 'F'alse
\eatline
\nwused{\\{NWful17-GloB-1}}\nwendcode{}\nwbegindocs{132}\nwdocspar

\subsubsctn{Timing our scripts}

The '{\tt{}Benchmark}' module encapsulates a number of routines to help to figure out how long it takes to execute a piece of code and the whole script.

\nwenddocs{}\nwbegincode{133}\sublabel{NWful17-UseN-1}\nwmargintag{{\nwtagstyle{}\subpageref{NWful17-UseN-1}}}\moddef{Use Modules - Benchmark~{\nwtagstyle{}\subpageref{NWful17-UseN-1}}}\endmoddef
use Benchmark;
  \LA{}Timer ON~{\nwtagstyle{}\subpageref{NWful17-Tim8-1}}\RA{}
\nwused{\\{NWful17-UseB-1}}\nwendcode{}\nwbegindocs{134}\nwdocspar

See '{\tt{}man\ Benchmark}' for further info about this package. 
We set an array to keep record of timing for each section.

\nwenddocs{}\nwbegincode{135}\sublabel{NWful17-Tim8-1}\nwmargintag{{\nwtagstyle{}\subpageref{NWful17-Tim8-1}}}\moddef{Timer ON~{\nwtagstyle{}\subpageref{NWful17-Tim8-1}}}\endmoddef
my @Timer = (new Benchmark);
\nwused{\\{NWful17-UseN-1}}\nwendcode{}\nwbegindocs{136}\nwdocspar

\nwenddocs{}\nwbegincode{137}\sublabel{NWful17-ComS-1}\nwmargintag{{\nwtagstyle{}\subpageref{NWful17-ComS-1}}}\moddef{Common PERL subs - Benchmark~{\nwtagstyle{}\subpageref{NWful17-ComS-1}}}\endmoddef
sub timing() \{
    push @Timer, (new Benchmark);
    # partial time 
    $_[0] || 
        (return timestr(timediff($Timer[$#Timer],$Timer[($#Timer - 1)])));
    # total time
    return timestr(timediff($Timer[$#Timer],$Timer[0]));
\} # timing
\nwused{\\{NWful17-Fun9-1}}\nwendcode{}\nwbegindocs{138}\nwdocspar

\nwenddocs{}\nwbegincode{139}\sublabel{NWful17-perP-1}\nwmargintag{{\nwtagstyle{}\subpageref{NWful17-perP-1}}}\moddef{perl requires - Benchmark~{\nwtagstyle{}\subpageref{NWful17-perP-1}}}\endmoddef
"Benchmark" - checking and comparing running times of code.
\nwused{\\{NWful17-perI-1}}\nwendcode{}\nwbegindocs{140}\nwdocspar


\subsubsctn{Printing complex Data Structures}

With '{\tt{}Data::Dumper}' we are able to pretty print complex data structures for debugging them.


\nwenddocs{}\nwbegincode{141}\sublabel{NWful17-UseK.2-1}\nwmargintag{{\nwtagstyle{}\subpageref{NWful17-UseK.2-1}}}\moddef{Use Modules - Dumper~{\nwtagstyle{}\subpageref{NWful17-UseK.2-1}}}\endmoddef
use Data::Dumper;
local $Data::Dumper::Purity = 0;
local $Data::Dumper::Deepcopy = 1;
\nwnotused{Use\ Modules\ -\ Dumper}\nwendcode{}\nwbegindocs{142}\nwdocspar


\subsubsctn{Common functions}

\nwenddocs{}\nwbegincode{143}\sublabel{NWful17-SkiV-1}\nwmargintag{{\nwtagstyle{}\subpageref{NWful17-SkiV-1}}}\moddef{Skip comments and empty records~{\nwtagstyle{}\subpageref{NWful17-SkiV-1}}}\endmoddef
next if /^\\#/o;
next if /^\\s*$/o;
chomp;
\nwused{\\{NWful17-MaiM-4}}\nwendcode{}\nwbegindocs{144}\nwdocspar

\nwenddocs{}\nwbegincode{145}\sublabel{NWful17-ComQ-1}\nwmargintag{{\nwtagstyle{}\subpageref{NWful17-ComQ-1}}}\moddef{Common PERL subs - Min Max~{\nwtagstyle{}\subpageref{NWful17-ComQ-1}}}\endmoddef
#
sub max() \{
    my $z = shift @_;
    foreach my $l (@_) \{ $z = $l if $l > $z \};
    return $z;
\} # max
sub min() \{
    my $z = shift @_;
    foreach my $l (@_) \{ $z = $l if $l < $z \};
    return $z;
\} # min
\nwused{\\{NWful17-Fun9-1}}\nwendcode{}\nwbegindocs{146}\nwdocspar

\nwenddocs{}\nwbegincode{147}\sublabel{NWful17-ComS.2-1}\nwmargintag{{\nwtagstyle{}\subpageref{NWful17-ComS.2-1}}}\moddef{Common PERL subs - Text fill~{\nwtagstyle{}\subpageref{NWful17-ComS.2-1}}}\endmoddef
#
sub fill_right() \{ $_[0].($_[2] x ($_[1] - length($_[0]))) \}
sub fill_left()  \{ ($_[2] x ($_[1] - length($_[0]))).$_[0] \}
sub fill_mid()   \{ 
    my $l = length($_[0]);
    my $k = int(($_[1] - $l)/2);
    ($_[2] x $k).$_[0].($_[2] x ($_[1] - ($l+$k)));
\} # fill_mid
\nwused{\\{NWful17-Fun9-1}}\nwendcode{}\nwbegindocs{148}\nwdocspar

These functions are used to report to STDERR a single char for each record processed (useful for reporting parsed records).

\nwenddocs{}\nwbegincode{149}\sublabel{NWful17-ComQ.2-1}\nwmargintag{{\nwtagstyle{}\subpageref{NWful17-ComQ.2-1}}}\moddef{Common PERL subs - Counter~{\nwtagstyle{}\subpageref{NWful17-ComQ.2-1}}}\endmoddef
#
sub counter \{ # $_[0]~current_pos++ $_[1]~char
    print STDERR "$_[1]";
    (($_[0] % 50) == 0) && (print STDERR "[".&fill_left($_[0],6,"0")."]\\n");
\} # counter
#
sub counter_end \{ # $_[0]~current_pos   $_[1]~char
    (($_[0] % 50) != 0) && (print STDERR "[".&fill_left($_[0],6,"0")."]\\n");
\} # counter_end
\nwused{\\{NWful17-Fun9-1}}\nwendcode{}\nwbegindocs{150}\nwdocspar

\nwenddocs{}\nwbegincode{151}\sublabel{NWful17-CouC-1}\nwmargintag{{\nwtagstyle{}\subpageref{NWful17-CouC-1}}}\moddef{Counter vars~{\nwtagstyle{}\subpageref{NWful17-CouC-1}}}\endmoddef
my ($n,$c); # counter and char (for &counter function)
\eatline
\nwused{\\{NWful17-GloB-1}}\nwendcode{}\nwbegindocs{152}\nwdocspar

\subsubsctn{Common functions for reporting program processes}
\label{sec:messagerpt}

Function '{\tt{}report}' requires that a hash variable '{\tt{}{\char37}Messages}' has been set, such hash contains the strings for each report message we will need. The first parameter for '{\tt{}report}' is a key for that hash, in order to retrieve the message string, the other parameters passed are processed by the {\tt{}sprintf} function on that string.

\nwenddocs{}\nwbegincode{153}\sublabel{NWful17-ComP-1}\nwmargintag{{\nwtagstyle{}\subpageref{NWful17-ComP-1}}}\moddef{Common PERL subs - STDERR~{\nwtagstyle{}\subpageref{NWful17-ComP-1}}}\endmoddef
sub report() \{ print STDERR sprintf($Messages\{ shift @_ \},@_) \}
\nwalsodefined{\\{NWful17-ComP-2}}\nwused{\\{NWful17-Fun9-1}}\nwendcode{}\nwbegindocs{154}\nwdocspar

The same happens to '{\tt{}warn}' function which also uses the hash variable '{\tt{}{\char37}Messages}' containing the error messages.

\nwenddocs{}\nwbegincode{155}\sublabel{NWful17-ComP-2}\nwmargintag{{\nwtagstyle{}\subpageref{NWful17-ComP-2}}}\moddef{Common PERL subs - STDERR~{\nwtagstyle{}\subpageref{NWful17-ComP-1}}}\plusendmoddef
sub warn() \{ print STDERR sprintf($Messages\{ shift @_ \}, @_) \}
\nwendcode{}\nwbegindocs{156}\nwdocspar

Those are accessory variables for the messages strings:

\nwenddocs{}\nwbegincode{157}\sublabel{NWful17-StdG-1}\nwmargintag{{\nwtagstyle{}\subpageref{NWful17-StdG-1}}}\moddef{Stderr subs vars~{\nwtagstyle{}\subpageref{NWful17-StdG-1}}}\endmoddef
my $line = ('#' x 80)."\\n";
my $s = '### ';
my $sp = "###\\n";
my $Error = "\\<\\<\\<  ERROR  \\>\\>\\> ";
my $Warn  = "\\<\\<\\< WARNING \\>\\>\\> ";
my $spl   = "\\<\\<\\<\\-\\-\\-\\-\\-\\-\\-\\-\\-\\>\\>\\>\\n";
my $spw   = "\\<\\<\\<         \\>\\>\\> ";
\nwalsodefined{\\{NWful17-StdG-2}}\nwused{\\{NWful17-GloB-1}}\nwendcode{}\nwbegindocs{158}\nwdocspar

And here the main messages hash:

\nwenddocs{}\nwbegincode{159}\sublabel{NWful17-StdG-2}\nwmargintag{{\nwtagstyle{}\subpageref{NWful17-StdG-2}}}\moddef{Stderr subs vars~{\nwtagstyle{}\subpageref{NWful17-StdG-1}}}\plusendmoddef
my %Messages = (
    # ERROR MESSAGES
    \LA{}warnings - command-line~{\nwtagstyle{}\subpageref{NWful17-warN-1}}\RA{}
    \LA{}warnings - input/output~{\nwtagstyle{}\subpageref{NWful17-outN-1}}\RA{}
    # WORKING MESSAGES
    \LA{}messages - program running~{\nwtagstyle{}\subpageref{NWful17-mesQ-1}}\RA{}
    \LA{}messages - parsing command-line~{\nwtagstyle{}\subpageref{NWful17-mesV-1}}\RA{}
    \LA{}messages - input/output~{\nwtagstyle{}\subpageref{NWful17-outN.2-1}}\RA{}
   ); # %Messages 
\nwendcode{}\nwbegindocs{160}\nwdocspar

\subsctn{BASH scripts}

\nwenddocs{}\nwbegincode{161}\sublabel{NWful17-BASC-1}\nwmargintag{{\nwtagstyle{}\subpageref{NWful17-BASC-1}}}\moddef{BASH shebang~{\nwtagstyle{}\subpageref{NWful17-BASC-1}}}\endmoddef
#!/usr/bin/bash
# GNU bash, version 2.03.6(1)-release (i386-redhat-linux-gnu)
\LA{}Version Control Id Tag~{\nwtagstyle{}\subpageref{NWful17-VerM-1}}\RA{}
#
SECONDS=0 # Reset Timing
# Which script are we running...
L="####################"
\{ echo "$L$L$L$L";
  echo "### RUNNING [$0]";
  echo "### Current date:`date`";
  echo "###"; \} 1>&2;
\nwused{\\{NWful17-wea7-1}\\{NWful17-LaT8-1}}\nwendcode{}\nwbegindocs{162}\nwdocspar

\nwenddocs{}\nwbegincode{163}\sublabel{NWful17-BASF-1}\nwmargintag{{\nwtagstyle{}\subpageref{NWful17-BASF-1}}}\moddef{BASH script end~{\nwtagstyle{}\subpageref{NWful17-BASF-1}}}\endmoddef
\{ echo "###"; echo "### Execution time for [$0] : $SECONDS secs";
  echo "$L$L$L$L";
  echo ""; \} 1>&2;
#
exit 0
\nwused{\\{NWful17-wea7-1}\\{NWful17-LaT8-1}}\nwendcode{}\nwbegindocs{164}\nwdocspar

\subsctn{Version control tags}

This document is under Revision Control System (RCS). The version you are currently reading is the following:

\nwenddocs{}\nwbegincode{165}\sublabel{NWful17-VerM-1}\nwmargintag{{\nwtagstyle{}\subpageref{NWful17-VerM-1}}}\moddef{Version Control Id Tag~{\nwtagstyle{}\subpageref{NWful17-VerM-1}}}\endmoddef
# $Id: fullgffcoords.tex,v 1.1 2001-10-22 14:44:13 jabril Exp $
\nwused{\\{NWful17-PERC-1}\\{NWful17-BASC-1}}\nwendcode{}\nwbegindocs{166}\nwdocspar

\subsctn{GNU General Public License}

\nwenddocs{}\nwbegincode{167}\sublabel{NWful17-GNUB-1}\nwmargintag{{\nwtagstyle{}\subpageref{NWful17-GNUB-1}}}\moddef{GNU License~{\nwtagstyle{}\subpageref{NWful17-GNUB-1}}}\endmoddef
# This program is free software; you can redistribute it and/or modify
# it under the terms of the GNU General Public License as published by
# the Free Software Foundation; either version 2 of the License, or
# (at your option) any later version.
# 
# This program is distributed in the hope that it will be useful,
# but WITHOUT ANY WARRANTY; without even the implied warranty of
# MERCHANTABILITY or FITNESS FOR A PARTICULAR PURPOSE.  See the
# GNU General Public License for more details.
# 
# You should have received a copy of the GNU General Public License
# along with this program; if not, write to the Free Software
# Foundation, Inc., 675 Mass Ave, Cambridge, MA 02139, USA.
# 
# #----------------------------------------------------------------#
\nwused{\\{NWful17-PERC-1}}\nwendcode{}\nwbegindocs{168}\nwdocspar

\newpage %%%%%%%%%%%%%%%%%%%%%%%%%%%%%%%%%%%%%%%%%%%%%%%%%

\sctn{Extracting code blocks from this document}

From this file we can obtain both the code and the
documentation. The following instructions are needed:

\subsctn{Extracts Script code chunks from the {\noweb} file} % \\[-0.5ex]

Remember when tangling that '-L' option allows you to include program line-numbering relative to original {\noweb} file. Then the first line of the executable files is a comment, not a shebang, and must be removed to make scripts runnable.

\nwenddocs{}\nwbegincode{169}\sublabel{NWful17-tan8-1}\nwmargintag{{\nwtagstyle{}\subpageref{NWful17-tan8-1}}}\moddef{tangling~{\nwtagstyle{}\subpageref{NWful17-tan8-1}}}\endmoddef
# showing line numbering comments in program
notangle -L -R"fullgffcoords" $WORK/$nwfile.nw | \\
    perl -ne '$.>1 && print' | cpif $BIN/fullgffcoords.pl ;
chmod a+x $BIN/fullgffcoords.pl ;
\nwalsodefined{\\{NWful17-tan8-2}\\{NWful17-tan8-3}\\{NWful17-tan8-4}\\{NWful17-tan8-5}}\nwnotused{tangling}\nwendcode{}\nwbegindocs{170}\nwdocspar

\nwenddocs{}\nwbegincode{171}\sublabel{NWful17-tan8-2}\nwmargintag{{\nwtagstyle{}\subpageref{NWful17-tan8-2}}}\moddef{tangling~{\nwtagstyle{}\subpageref{NWful17-tan8-1}}}\plusendmoddef
# reformating program with perltidy
notangle -R"fullgffcoords" $WORK/$nwfile.nw | \\
    perltidy - | cpif $SRC/fullgffcoords ;
# html pretty-printing program with perltidy
notangle -R"fullgffcoords" $WORK/$nwfile.nw | \\
    perltidy -html - | cpif $DOCS/html/fullgffcoords.html ;
#
\nwendcode{}\nwbegindocs{172}\nwdocspar

\subsctn{Extracting different Config Files} % \\[-0.5ex]

\nwenddocs{}\nwbegincode{173}\sublabel{NWful17-tan8-3}\nwmargintag{{\nwtagstyle{}\subpageref{NWful17-tan8-3}}}\moddef{tangling~{\nwtagstyle{}\subpageref{NWful17-tan8-1}}}\plusendmoddef
notangle -R"root" $WORK/$nwfile.nw | \\
         cpif $DATA/root_config ;
\nwendcode{}\nwbegindocs{174}%$

\subsctn{Extracting documentation and \LaTeX{}'ing it} % \\[-0.5ex] %'

\nwenddocs{}\nwbegincode{175}\sublabel{NWful17-tan8-4}\nwmargintag{{\nwtagstyle{}\subpageref{NWful17-tan8-4}}}\moddef{tangling~{\nwtagstyle{}\subpageref{NWful17-tan8-1}}}\plusendmoddef
notangle -Rweaving  $WORK/$nwfile.nw | cpif $WORK/nw2tex ;
notangle -RLaTeXing $WORK/$nwfile.nw | cpif $WORK/ltx ;
chmod a+x $WORK/nw2tex $WORK/ltx;
\nwendcode{}\nwbegindocs{176}\nwdocspar

\nwenddocs{}\nwbegincode{177}\sublabel{NWful17-tanY-1}\nwmargintag{{\nwtagstyle{}\subpageref{NWful17-tanY-1}}}\moddef{tangling complementary LaTeX files~{\nwtagstyle{}\subpageref{NWful17-tanY-1}}}\endmoddef
notangle -R"HIDE: LaTeX new definitions" $WORK/$nwfile.nw | cpif $DOCS/defs.tex ;
notangle -R"HIDE: TODO" $WORK/$nwfile.nw | cpif $DOCS/todo.tex ; 
\nwused{\\{NWful17-wea7-1}}\nwendcode{}\nwbegindocs{178}\nwdocspar

\nwenddocs{}\nwbegincode{179}\sublabel{NWful17-wea7-1}\nwmargintag{{\nwtagstyle{}\subpageref{NWful17-wea7-1}}}\moddef{weaving~{\nwtagstyle{}\subpageref{NWful17-wea7-1}}}\endmoddef
\LA{}BASH shebang~{\nwtagstyle{}\subpageref{NWful17-BASC-1}}\RA{}
# weaving and LaTeXing
\LA{}BASH Environment Variables~{\nwtagstyle{}\subpageref{NWful17-BASQ-1}}\RA{}
\LA{}tangling complementary LaTeX files~{\nwtagstyle{}\subpageref{NWful17-tanY-1}}\RA{}
noweave -v -t4 -delay -x -filter 'elide "HIDE: *"' \\
        $WORK/$nwfile.nw | cpif $DOCS/$nwfile.tex ;
# noweave -t4 -delay -index $WORK/$nwfile.nw > $DOCS/$nwfile.tex 
pushd $DOCS/ ;
#
latex $nwfile.tex ;
dvips $nwfile.dvi -o $nwfile.ps -t a4 ;
#
popd;
\LA{}BASH script end~{\nwtagstyle{}\subpageref{NWful17-BASF-1}}\RA{}
\nwnotused{weaving}\nwendcode{}\nwbegindocs{180}\nwdocspar

\nwenddocs{}\nwbegincode{181}\sublabel{NWful17-LaT8-1}\nwmargintag{{\nwtagstyle{}\subpageref{NWful17-LaT8-1}}}\moddef{LaTeXing~{\nwtagstyle{}\subpageref{NWful17-LaT8-1}}}\endmoddef
\LA{}BASH shebang~{\nwtagstyle{}\subpageref{NWful17-BASC-1}}\RA{}
# only LaTeXing
\LA{}BASH Environment Variables~{\nwtagstyle{}\subpageref{NWful17-BASQ-1}}\RA{}
pushd $DOCS/ ;
#
echo "### RUNNING LaTeX on $nwfile.tex" 1>&2 ;
latex $nwfile.tex ; 
latex $nwfile.tex ; 
latex $nwfile.tex ;
dvips $nwfile.dvi -o $nwfile.ps -t a4 ;
#
# pdflatex $nwfile.tex ;
echo "### CONVERTING PS to PDF: $nwfile" 1>&2 ;
ps2pdf $nwfile.ps $nwfile.pdf ;
#
popd ;
\LA{}BASH script end~{\nwtagstyle{}\subpageref{NWful17-BASF-1}}\RA{}
\nwnotused{LaTeXing}\nwendcode{}\nwbegindocs{182}%$

\subsctn{Defining working shell variables for the current project} % \\[-0.5ex]

\nwenddocs{}\nwbegincode{183}\sublabel{NWful17-BASQ-1}\nwmargintag{{\nwtagstyle{}\subpageref{NWful17-BASQ-1}}}\moddef{BASH Environment Variables~{\nwtagstyle{}\subpageref{NWful17-BASQ-1}}}\endmoddef
#
# Setting Global Variables
WORK="/home/ug/jabril/development/softjabril/fullgffcoords" ;
BIN="$WORK/bin" ;
PARAM="$BIN/param" ;
SRC="$WORK/src" ; # where to put the distributable files
DOCS="$WORK/docs" ;
DATA="$WORK/data" ;
TEST="$WORK/tests" ;
nwfile="fullgffcoords" ;
export WORK BIN PARAM DOCS DATA nwfile ;
#
\nwused{\\{NWful17-wea7-1}\\{NWful17-LaT8-1}}\nwendcode{}\nwbegindocs{184}\nwdocspar

\nwenddocs{}\nwbegincode{185}\sublabel{NWful17-tan8-5}\nwmargintag{{\nwtagstyle{}\subpageref{NWful17-tan8-5}}}\moddef{tangling~{\nwtagstyle{}\subpageref{NWful17-tan8-1}}}\plusendmoddef
# 
# BASH Environment Variables
notangle -R'BASH Environment Variables' $WORK/$nwfile.nw | \\
         cpif $WORK/.bash_VARS ; 
source $WORK/.bash_VARS ;
#
\nwendcode{}

\nwixlogsorted{c}{{BASH Environment Variables}{NWful17-BASQ-1}{\nwixu{NWful17-wea7-1}\nwixu{NWful17-LaT8-1}\nwixd{NWful17-BASQ-1}}}%
\nwixlogsorted{c}{{BASH script end}{NWful17-BASF-1}{\nwixd{NWful17-BASF-1}\nwixu{NWful17-wea7-1}\nwixu{NWful17-LaT8-1}}}%
\nwixlogsorted{c}{{BASH shebang}{NWful17-BASC-1}{\nwixd{NWful17-BASC-1}\nwixu{NWful17-wea7-1}\nwixu{NWful17-LaT8-1}}}%
\nwixlogsorted{c}{{Boolean}{NWful17-Boo7-1}{\nwixu{NWful17-GloB-1}\nwixd{NWful17-Boo7-1}}}%
\nwixlogsorted{c}{{Common PERL subs - Benchmark}{NWful17-ComS-1}{\nwixu{NWful17-Fun9-1}\nwixd{NWful17-ComS-1}}}%
\nwixlogsorted{c}{{Common PERL subs - Counter}{NWful17-ComQ.2-1}{\nwixu{NWful17-Fun9-1}\nwixd{NWful17-ComQ.2-1}}}%
\nwixlogsorted{c}{{Common PERL subs - Min Max}{NWful17-ComQ-1}{\nwixu{NWful17-Fun9-1}\nwixd{NWful17-ComQ-1}}}%
\nwixlogsorted{c}{{Common PERL subs - STDERR}{NWful17-ComP-1}{\nwixu{NWful17-Fun9-1}\nwixd{NWful17-ComP-1}\nwixd{NWful17-ComP-2}}}%
\nwixlogsorted{c}{{Common PERL subs - Text fill}{NWful17-ComS.2-1}{\nwixu{NWful17-Fun9-1}\nwixd{NWful17-ComS.2-1}}}%
\nwixlogsorted{c}{{Counter vars}{NWful17-CouC-1}{\nwixu{NWful17-GloB-1}\nwixd{NWful17-CouC-1}}}%
\nwixlogsorted{c}{{Functions}{NWful17-Fun9-1}{\nwixu{NWful17-fulD-1}\nwixd{NWful17-Fun9-1}}}%
\nwixlogsorted{c}{{GNU License}{NWful17-GNUB-1}{\nwixu{NWful17-PERC-1}\nwixd{NWful17-GNUB-1}}}%
\nwixlogsorted{c}{{Global Vars}{NWful17-GloB-1}{\nwixu{NWful17-fulD-1}\nwixd{NWful17-GloB-1}\nwixd{NWful17-GloB-2}\nwixd{NWful17-GloB-3}\nwixd{NWful17-GloB-4}\nwixd{NWful17-GloB-5}}}%
\nwixlogsorted{c}{{HIDE: LaTeX new definitions}{NWful17-HIDR-1}{\nwixd{NWful17-HIDR-1}}}%
\nwixlogsorted{c}{{HIDE: TODO}{NWful17-HIDA-1}{\nwixd{NWful17-HIDA-1}\nwixd{NWful17-HIDA-2}\nwixd{NWful17-HIDA-3}}}%
\nwixlogsorted{c}{{HIDE: new LaTeX commands}{NWful17-HIDO-1}{\nwixu{NWful17-HIDR-1}\nwixd{NWful17-HIDO-1}}}%
\nwixlogsorted{c}{{HIDE: new LaTeX definitions}{NWful17-HIDR.2-1}{\nwixu{NWful17-HIDR-1}\nwixd{NWful17-HIDR.2-1}}}%
\nwixlogsorted{c}{{HIDE: new LaTeX pstricks}{NWful17-HIDO.2-1}{\nwixu{NWful17-HIDR-1}\nwixd{NWful17-HIDO.2-1}}}%
\nwixlogsorted{c}{{HIDE: new LaTeX urls}{NWful17-HIDK-1}{\nwixu{NWful17-HIDR-1}\nwixd{NWful17-HIDK-1}}}%
\nwixlogsorted{c}{{HIDE: new defs TODO}{NWful17-HIDJ-1}{\nwixu{NWful17-HIDR-1}\nwixd{NWful17-HIDJ-1}\nwixd{NWful17-HIDJ-2}\nwixd{NWful17-HIDJ-3}\nwixd{NWful17-HIDJ-4}}}%
\nwixlogsorted{c}{{LaTeXing}{NWful17-LaT8-1}{\nwixd{NWful17-LaT8-1}}}%
\nwixlogsorted{c}{{Main Loop}{NWful17-Mai9-1}{\nwixu{NWful17-fulD-1}\nwixd{NWful17-Mai9-1}}}%
\nwixlogsorted{c}{{Main program functions}{NWful17-MaiM-1}{\nwixu{NWful17-Fun9-1}\nwixd{NWful17-MaiM-1}\nwixd{NWful17-MaiM-2}\nwixd{NWful17-MaiM-3}\nwixd{NWful17-MaiM-4}\nwixd{NWful17-MaiM-5}\nwixd{NWful17-MaiM-6}\nwixd{NWful17-MaiM-7}\nwixd{NWful17-MaiM-8}\nwixd{NWful17-MaiM-9}\nwixd{NWful17-MaiM-A}\nwixd{NWful17-MaiM-B}}}%
\nwixlogsorted{c}{{PERL shebang}{NWful17-PERC-1}{\nwixu{NWful17-fulD-1}\nwixd{NWful17-PERC-1}}}%
\nwixlogsorted{c}{{Parsing command line options}{NWful17-ParS-1}{\nwixu{NWful17-Fun9-1}\nwixd{NWful17-ParS-1}\nwixd{NWful17-ParS-2}\nwixd{NWful17-ParS-3}\nwixd{NWful17-ParS-4}}}%
\nwixlogsorted{c}{{Prog DESC}{NWful17-Pro9-1}{\nwixd{NWful17-Pro9-1}\nwixu{NWful17-ParS-4}}}%
\nwixlogsorted{c}{{Prog USAGE}{NWful17-ProA-1}{\nwixd{NWful17-ProA-1}\nwixu{NWful17-ParS-4}}}%
\nwixlogsorted{c}{{Program Description}{NWful17-ProJ-1}{\nwixd{NWful17-ProJ-1}\nwixu{NWful17-PERC-1}}}%
\nwixlogsorted{c}{{Program Info}{NWful17-ProC-1}{\nwixd{NWful17-ProC-1}\nwixu{NWful17-PERC-1}}}%
\nwixlogsorted{c}{{STDIN backwards compatibility}{NWful17-STDT-1}{\nwixu{NWful17-ParS-2}\nwixd{NWful17-STDT-1}}}%
\nwixlogsorted{c}{{Skip comments and empty records}{NWful17-SkiV-1}{\nwixu{NWful17-MaiM-4}\nwixd{NWful17-SkiV-1}}}%
\nwixlogsorted{c}{{Stderr subs vars}{NWful17-StdG-1}{\nwixu{NWful17-GloB-1}\nwixd{NWful17-StdG-1}\nwixd{NWful17-StdG-2}}}%
\nwixlogsorted{c}{{Timer ON}{NWful17-Tim8-1}{\nwixu{NWful17-UseN-1}\nwixd{NWful17-Tim8-1}}}%
\nwixlogsorted{c}{{Use Modules}{NWful17-UseB-1}{\nwixu{NWful17-fulD-1}\nwixd{NWful17-UseB-1}}}%
\nwixlogsorted{c}{{Use Modules - Benchmark}{NWful17-UseN-1}{\nwixu{NWful17-UseB-1}\nwixd{NWful17-UseN-1}}}%
\nwixlogsorted{c}{{Use Modules - Dumper}{NWful17-UseK.2-1}{\nwixd{NWful17-UseK.2-1}}}%
\nwixlogsorted{c}{{Use Modules - Getopt}{NWful17-UseK-1}{\nwixu{NWful17-UseB-1}\nwixd{NWful17-UseK-1}}}%
\nwixlogsorted{c}{{Version Control Id Tag}{NWful17-VerM-1}{\nwixu{NWful17-PERC-1}\nwixu{NWful17-BASC-1}\nwixd{NWful17-VerM-1}}}%
\nwixlogsorted{c}{{cmdline defaults - format}{NWful17-cmdP-1}{\nwixu{NWful17-GloB-1}\nwixd{NWful17-cmdP-1}}}%
\nwixlogsorted{c}{{command-line defaults - GFF}{NWful17-comR-1}{\nwixu{NWful17-GloB-1}\nwixd{NWful17-comR-1}}}%
\nwixlogsorted{c}{{command-line help}{NWful17-comH-1}{\nwixu{NWful17-ParS-4}\nwixd{NWful17-comH-1}}}%
\nwixlogsorted{c}{{command-line help - GFF}{NWful17-comN-1}{\nwixu{NWful17-comH-1}\nwixd{NWful17-comN-1}}}%
\nwixlogsorted{c}{{command-line help - format}{NWful17-comQ-1}{\nwixu{NWful17-comH-1}\nwixd{NWful17-comQ-1}}}%
\nwixlogsorted{c}{{command-line help - help}{NWful17-comO-1}{\nwixd{NWful17-comO-1}\nwixu{NWful17-comH-1}}}%
\nwixlogsorted{c}{{command-line options - GFF}{NWful17-comQ.2-1}{\nwixu{NWful17-ParS-1}\nwixd{NWful17-comQ.2-1}}}%
\nwixlogsorted{c}{{command-line options - format}{NWful17-comT-1}{\nwixu{NWful17-ParS-1}\nwixd{NWful17-comT-1}}}%
\nwixlogsorted{c}{{command-line options with exit}{NWful17-comU-1}{\nwixu{NWful17-ParS-1}\nwixd{NWful17-comU-1}}}%
\nwixlogsorted{c}{{fullgffcoords}{NWful17-fulD-1}{\nwixd{NWful17-fulD-1}}}%
\nwixlogsorted{c}{{looking for STDIN}{NWful17-looH-1}{\nwixu{NWful17-ParS-1}\nwixd{NWful17-looH-1}}}%
\nwixlogsorted{c}{{messages - input/output}{NWful17-outN.2-1}{\nwixd{NWful17-outN.2-1}\nwixu{NWful17-StdG-2}}}%
\nwixlogsorted{c}{{messages - parsing command-line}{NWful17-mesV-1}{\nwixd{NWful17-mesV-1}\nwixu{NWful17-StdG-2}}}%
\nwixlogsorted{c}{{messages - program running}{NWful17-mesQ-1}{\nwixd{NWful17-mesQ-1}\nwixd{NWful17-mesQ-2}\nwixd{NWful17-mesQ-3}\nwixd{NWful17-mesQ-4}\nwixd{NWful17-mesQ-5}\nwixu{NWful17-StdG-2}}}%
\nwixlogsorted{c}{{perl requires - Benchmark}{NWful17-perP-1}{\nwixu{NWful17-perI-1}\nwixd{NWful17-perP-1}}}%
\nwixlogsorted{c}{{perl requires - Getopt}{NWful17-perM-1}{\nwixd{NWful17-perM-1}\nwixu{NWful17-perI-1}}}%
\nwixlogsorted{c}{{perl requires help}{NWful17-perI-1}{\nwixu{NWful17-ParS-4}\nwixd{NWful17-perI-1}}}%
\nwixlogsorted{c}{{tangling}{NWful17-tan8-1}{\nwixd{NWful17-tan8-1}\nwixd{NWful17-tan8-2}\nwixd{NWful17-tan8-3}\nwixd{NWful17-tan8-4}\nwixd{NWful17-tan8-5}}}%
\nwixlogsorted{c}{{tangling complementary LaTeX files}{NWful17-tanY-1}{\nwixd{NWful17-tanY-1}\nwixu{NWful17-wea7-1}}}%
\nwixlogsorted{c}{{warnings - command-line}{NWful17-warN-1}{\nwixd{NWful17-warN-1}\nwixu{NWful17-StdG-2}}}%
\nwixlogsorted{c}{{warnings - input/output}{NWful17-outN-1}{\nwixd{NWful17-outN-1}\nwixd{NWful17-outN-2}\nwixu{NWful17-StdG-2}}}%
\nwixlogsorted{c}{{weaving}{NWful17-wea7-1}{\nwixd{NWful17-wea7-1}}}%
\nwbegindocs{186}\nwdocspar

%
\end{document}
%
%%%%%%%%%%%%%%%%%%%%%%%%%%%%%%%%%%%%%%%%%%%%%%%%%%%%%%%%%%%%%%%%%%%%%%%%%%%%%%%%
\nwenddocs{}
