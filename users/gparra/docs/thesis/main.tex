%
%2345678901234567890123456789012345678901234567890123456789012345678901234567890
%        1         2         3         4         5         6         7         8
%
% $Id: main.tex,v 1.2 2004-06-29 10:40:56 gparra Exp $
%


%%%%%%%%%%%%%%%%%%%%%%%%%%%%%%%%%%%%%%%%%%%%%%%%%%%%%%%%%%%%%
%%%%%%%%%%%%%%%%%%%%%%%%%%%%%%%%%%%%%%%%%%%%%%%%%%%%%%%%%%%%% 
%%%%%%%%%%%%%%%%%%%%%%%%%%%%%%%%%%%%%%%%%%%%%%%%%%%%%%%%%%%%%
%%%         Genis Parra, gparra@imim.es, June 2004        %%%
%%%                                                       %%%
%%%          Questions and suggestions WELCOMED!          %%%
%%%                                                       %%%  
%%%       This thesis template largely derives from       %%%
%%%                                                       %%%
%%%           Robert Castelo and Sergio Mendoza           %%%
%%%                                                       %%%
%%%             Under GNU/GPL copyleft license            %%%
%%%                                                       %%%
%%%         FEEL FREE TO USE IT AND IMPROVE IT!!!         %%%
%%%%%%%%%%%%%%%%%%%%%%%%%%%%%%%%%%%%%%%%%%%%%%%%%%%%%%%%%%%%%
%%%%%%%%%%%%%%%%%%%%%%%%%%%%%%%%%%%%%%%%%%%%%%%%%%%%%%%%%%%%%
%%%%%%%%%%%%%%%%%%%%%%%%%%%%%%%%%%%%%%%%%%%%%%%%%%%%%%%%%%%%%

%%% Preamble:  LaTeX parameters and extensions %%%

%%%%%%%%%%%%%%%%%%%%%
%%% Document type %%%
%%%%%%%%%%%%%%%%%%%%%

%\documentclass[draft]{book} % While working to see output errors

\documentclass[pdftex,11pt]{book} % To produce final version

%%%%%%%%%%%%%%%%%%%%%%%%%%
%%% Our own style file %%%
%%%%%%%%%%%%%%%%%%%%%%%%%%

\usepackage{mythesis} % Style file for this template mythesis.sty

%%%%%%%%%%%%%%%%%%%%
%%% Paper layout %%%
%%%%%%%%%%%%%%%%%%%% 

\usepackage[a4paper,twoside,hmargin={3cm,2cm},vmargin={2.5cm,2.5cm}]{geometry}
   % A4 size (first submission)

%\usepackage[paperwidth=170mm,paperheight=240mm]{geometry} 
   % Book size (final submission)

%%%%%%%%%%%%%%%%
%%% Headings %%%
%%%%%%%%%%%%%%%%

\usepackage{fancyhdr} % To produce nice headings

\pagestyle{fancyplain} 

% Better than plain style 'cos we control headings in all pages
% Even (verso) page and odd (recto) page cells are defined separately

\renewcommand{\chaptermark}[1]{\markboth{\chaptername\ \thechapter. #1}{}}
%\renewcommand{\chaptermark}[1]{\markboth{\thechapter. #1}{}}
\renewcommand{\sectionmark}[1]{\markright{\thesection\ #1}}

\def\tit{\textsc{Computational identification of genes}}

\lhead[\fancyplain{}{\sffamily\thepage}]  {\fancyplain{}{\sffamily\rightmark}}
\rhead[\fancyplain{}{\sffamily\leftmark}] {\fancyplain{}{\sffamily\thepage}}
\cfoot[\fancyplain{\thepage}{}]           {\fancyplain{\thepage}{}}

%\renewcommand{\headrulewidth}{0pt}

%%%%%%%%%%%%%%%%%%%
%%% Draft marks %%%
%%%%%%%%%%%%%%%%%%%

%\usepackage[light]{draftcopy} % Works only for Postscript and not for PDF 

% Alternatively:

\newcommand{\tstamp}{\today}

% Centered head

%\chead[\fancyplain{}{}]{\fancyplain{\textcolor{blue}{\sffamily\Large DRAFT -VERSION 1.0-}}{\textcolor{blue}{\sffamily\Large DRAFT -VERSION 1.0-}}}

% Centered foot

%\cfoot[\fancyplain{\textcolor{blue}{\sffamily\Large DRAFT -\tstamp-}}{\textcolor{blue}{\sffamily\Large DRAFT -\tstamp-}}]{\fancyplain{}{}}

%%%%%%%%%%%%%%%%
%%% Language %%%
%%%%%%%%%%%%%%%%

\usepackage[english]{babel}

%%%%%%%%%%%%%%%%%%%%
%%% PDF commands %%%
%%%%%%%%%%%%%%%%%%%%

%\usepackage[draft]{pdfpages} % To include pdf documents with pdflatex. 
                              % It does NOT work with latex.
                              % Declare it before \usepackage{graphicx}

\usepackage{pdfpages} % To produce final version

%%% Global pdfpages options %%%

\includepdfset{frame=false,pagecommand={\thispagestyle{fancyplain}}} % Used in each \includepdf call unless overwritten locally

%%%%%%%%%%%%%%%%%%%%%%%%%%%
%%% To include graphics %%%
%%%%%%%%%%%%%%%%%%%%%%%%%%%

\usepackage{graphicx} % To include graphics

%%%%%%%%%%%%%%%%%%%%%%%%
%%% References style %%%
%%%%%%%%%%%%%%%%%%%%%%%%

\usepackage[round]{natbib}  % Natural Sciences Citations: author-year and numerical citations

%%%%%%%%%%%%%%%%%%%%%%%%%%
%%% Indexs preparation %%%
%%%%%%%%%%%%%%%%%%%%%%%%%%

%\usepackage[editors]{/home/ug/scaste/Tex/authorindex/authorindex} % To make an authors index
%\let\citep=\aicite

\usepackage{makeidx} % To make a subject index
\makeindex % Subject index. It is added at the very end of the book

%\usepackage{showidx} % To print index entries in margins

%%%%%%%%%%%%%%%
%%% Columns %%%
%%%%%%%%%%%%%%%

\usepackage{multicol}

%%%%%%%%%%%%%%%%%%
%%% Text fonts %%%
%%%%%%%%%%%%%%%%%%

\usepackage{palatino}

%\selectfont

%\font\avantgarnormal = pagk at 10pt

%%% to add an empty page, output pending figures and resume in an odd-numbered one %%%

\newcommand{\clearemptydoublepage}{\newpage{\pagestyle{empty}\cleardoublepage}}

%%%%%%%%%%%%%%%%%%%%%
%%% URL inclusion %%%
%%%%%%%%%%%%%%%%%%%%%

\usepackage[pdftex,a4paper,colorlinks=true,linkcolor=blue,menucolor=blue,pagecolor=blue,pdfhighlight=/P,citecolor=blue,urlcolor=blue,linktocpage=true]{hyperref} % Last package to be loaded
                                                      % Enter URLs like this: \url{http://genome.imim.es/}

%%%%%%%%%%%%%%%%
%%% PDF info %%%
%%%%%%%%%%%%%%%%

% Needs the first the \usepackage{hyperref} declaration

\hypersetup{
  pdftitle = {Computational identification of genes: ab initio and comparative approaches},
  pdfsubject = {Dissertation for the degree of Ph.D. in Biological Science at the Universitat Pompeu Fabra},
  pdfkeywords = {gene prediction geneid sgp2 comparative genomics bioinformatics},
  pdfauthor = {\textcopyright\ Gen\'{i}s Parra},
  pdfcreator = {\LaTeX\ with package \flqq hyperref\frqq},
  pdfproducer = {pdfeTeX-0.\the\pdftexversion\pdftexrevision}
  pdfpagemode=FullScreen
  pdffitwindow=true 
}

%\pdfinfo{
%  /CreationDate (D:20020125173542)
%  /ModDate (D:20020328173542)
%}

%\AtBeginDocument{
%    \pdfpagewidth=170mm
%    \pdfpageheight=240mm
%}


%%%%%%%%%%%%%%%%%%
%%% Paragraphs %%%
%%%%%%%%%%%%%%%%%%

%\usepackage{/home/ug/scaste/Tex/dropcaps/dropcaps}
%\usepackage{/home/ug/scaste/Tex/dropping/dropping} % First letter dropping

\setlength{\parskip}{1ex plus0.2ex minus0.2ex}

%%%%%%%%%%%%%%%%%
%%% Date info %%%
%%%%%%%%%%%%%%%%%

\date{} % Keep it empty

%%%%%%%%%%%%%%%%%%%%
%%% Fake chapter %%%
%%%%%%%%%%%%%%%%%%%%

% This is to create a fake chapter (like the acknwoledgments).  It doesn't
% get ennumerated, but it does get added to the Table of Contents (ToC),
% and since the \chapter* doesn't define the markright or markboth
% commands, we force it here to do so.

\newcommand{\nochapter}[1]{\chapter*{#1}%
              \addcontentsline{toc}{chapter}{#1}%
			  \markboth{{ #1 } }{{ #1 } }
%              \markboth{\MakeUppercase { #1 } }{\MakeUppercase { #1 } }%
           }

%%%%%%%%%%%%%%%%%%%
%%% Definitions %%%
%%%%%%%%%%%%%%%%%%%

% Species

\def\Dm{\textit{D. melanogaster}}
\def\DmL{\textit{Drosophila melanogaster}}
\def\Droso{\textit{Drosophila}}

\def\Tn{\textit{T. nigroviridis}}
\def\TnL{\textit{Tetraodon nigroviridis}}
\def\Tetra{\textit{Tetraodon}}

\def\Tr{\textit{T. rubripes}}
\def\TrL{\textit{T. rubripes}}
\def\Taki{\textit{Takifugu}}

\def\Hs{\textit{H. sapiens}}
\def\HsL{\textit{Homo sapiens}}

\def\Mm{\textit{M. musculus}}
\def\MmL{\textit{Mus musculus}}

\def\Rn{\textit{R. norvegicus}}
\def\RnL{\textit{Rattus norvegicus}}

\def\At{\textit{A. thaliana}}
\def\AtL{\textit{Arabidopsis thaliana}}

\def\Sc{\textit{S. cerevisae}}
\def\ScL{\textit{Saccharomices cerevisae}}

\def\Dd{\textit{D. discoideum}}
\def\DdL{\textit{Dictyostelium discoideum}}

\def\Ce{\textit{C. elegans}}
\def\CeL{\textit{Ceanorhabditis elegans}}

\def\BrL{\textit{Brachydanio rerio}}

\def\TaL{\textit{Triticum aestivum}}

\def\GgL{\textit{Gallus gallus}}

\def\PfL{\textit{Plasmodium falciparum}}

\def\TaL{\textit{Triticum aestivum}}

\def\OsL{\textit{Oryza sativa}}

% Software 

\def\geneid{\texttt{geneid}}
\def\sgp{\texttt {sgp2}}
\def\sgpo{\texttt {sgp1}}
\def\exstral{\texttt {exstral}}

\def\bln{\textsc{blastn}}
\def\tbx{\textsc{tblastx}}
\def\blp{\textsc{ blastp}}

\def\genscan{\texttt{genscan}}
\def\genomescan{\texttt{genomescan}}

\def\slam{\texttt{slam}}
\def\cem{{\sc Cem}}
\def\doublescan{{\sc doublescan}}
\def\twinscan{{\sc twinscan}}
\def\slam{{\sc Slam}}
\def\genewise{{\sc genewise}}  
\def\procrustes{{\sc procrustes}}      
\def\rosseta{{\sc rosseta}}      
\def\sim{{\sc sim96}}

% databases
     
\def\sw{{\sc swissprot}}      
\def\embl{{\sc embl}}      
\def\scm{{\sc scimog}}
\def\ensembl{{\sc ensembl}}
\def\ncbi{{\sc ncbi}}
\def\refseq{{\sc refseq}}


% others

\def\gff{{\sc gff}}
\def\hsp{{\tt HSP}}
\def\blsm{{\sc blosum62}}
\def\mg{{\sc mgsc}v3}
\def\gp{{Golden Path}}

% mail
\def\mtgparra{\htmladdnormallink{\textbf{gparra@imim.es}}{MAILTO:gparra@imim.es}}

%%%%%%%%%%%%%%%%%%%
%%% Hyphenation %%%
%%%%%%%%%%%%%%%%%%%

\hyphenation{ste-pha-nie se-le-no-pro-te-in}

%%%%%%%%%%%%%%%%%
%%% Footnotes %%%
%%%%%%%%%%%%%%%%%

\setlength{\footnotesep}{2ex} % Vertical spacing between footnotes

\renewcommand{\footnoterule}{\noindent\rule{5cm}{.1ex}\vspace{1.5ex}} % Rule height, width and space

%%%%%%%%%%%%%%%%
%%% Glossary %%%
%%%%%%%%%%%%%%%%

\makeglossary

%%%%%%%%%%%%%%%%%%%%%%%%%%%%%%%%%%%%%%%%
%%% To process only desired chapters %%%
%%%%%%%%%%%%%%%%%%%%%%%%%%%%%%%%%%%%%%%%

%\includeonly{Introduction/introduction,Objectives/objectives,Results/results,Methods/methods,Discussion/discussion,Conclusions/conclusions}

% That's all folks!



%%%%%%%%%%%%%%%%%%%%%%
%%% Begin Document %%%
%%%%%%%%%%%%%%%%%%%%%% 

\begin{document}

\thispagestyle{empty}

%Title

\title{{\Huge Computational identification of genes: \\[1ex] \textit{ab initio} and comparative approaches}\vspace{7.5cm}}

% Author

\author{
{\Large Gen\'{\i}s Parra Farr\'e}\\[1ex]
{\large Ph.D. dissertation}\\[2ex]
\textbf{Ph.D. Advisor: Roderic Guig\'o Serra}}

%\\[5ex]\today}


\maketitle                      % Print thesis title

\clearemptydoublepage


%%%%%%%%%%%%%%%%%%%% FRONTMATTER

\newpage
\pagenumbering{roman}
\setcounter{page}{1}
\newpage
\nochapter{Motivation}

It is clear to me, that we are living a really important period in the
development and the knowledge of life sciences. 50 years after the
description of the structure of the double helix, we have moved from
the analysis of one single gene to the systematic mass sequencing of
organisms. At the time of writing, whole genome sequences for more
than 800 organisms (bacteria, archea and eukaryota, as well as many
viruses and organelles) are either complete or being sequenced (a
detailed picture of the current state of the sequencing projects can
be found at the entrez genome,
\url{http://www.ncbi.nlm.nih.gov/Genomes/index.html}). All the
information we are gathering today, will probably modify the way we
will understand life, science and medicine. But, before the best use
of these data, the identification and the precise location of the
functional regions of the genomic sequences must be determined. The
most important thing to realize about the ``book of life'', is that we
barely understand the language in which it is written and that raw
genomic sequences are mainly useless for the scientific community. The
challenge ahead is to extract relevant information encoded within the
billions of nucleotides stored in our databases.

In a very simplistic description, the first step in the functional
annotation of a genome, would be to find the collection of genes
encoded in the genomic sequences. The next step would be to assign a
function to each possible protein, where the three dimensional
structure of the proteins will play a key role. And finally, establish
the network of interactions and regulations among all the proteins of
a complete genome.

The main focus research interest of our group is  the investigation
of the signals involved in gene specification in genomic sequences
(promoter elements, splice sites, translation initiation sites,
...). We are interested both in the mechanism of their recognition and
processing, and in their evolution. In addition, but related to this
basic component of our research, our group is also involved in the
development of software for gene prediction and annotation in genomic
sequences. 

The following work is focused on the first step of any genome
analysis: to find where genes are. The motivation of this thesis,
thus, is to give a little insight in how genes are encoded and
recognized by the molecular machinery of the cell. The complexity of
gene prediction differs substantially in prokaryotic than eukaryotic
genomes. While prokaryotic genes are single continuous open reading
frames, usually adjacent to each other, eukaryotic genes are separated
by stretches of intergenic regions and their coding sequences, can be
interrupted by large non coding sequences, called introns. This
thesis is focused on eukaryotic gene prediction.

This thesis addresses another significant open problem of this field,
how the sequence of related genomes can contribute to the location of
genes. The value of comparative genomics is illustrated by the
sequencing of the mouse genome for the purpose of annotating the human
genome. The availability of closely related genomes makes it possible
to carry out genome-wise comparisons and analysis of syntenic regions.
Recently comparative gene predictions programs exploit this data under
the assumption that regions conserved in the sequence will tend to
correspond to coding exons from homologous genes. Thus, the second
part of this thesis, describes a gene prediction program that combines
\textit{ab initio} gene prediction with comparative information
between two genomes to improve the accuracy of the predictions.

The work described is essentially interdisciplinary, this means that
while the basic subject of matter is biological and the obtained
results are of biological interest, techniques from other fields are
used heavily including certain statistical models and computational
programing to develop all the bioinformatic applications.

\clearemptydoublepage


\pagestyle{fancy}
% Marks redefinition must go here because pagestyle 
% resets the values to the default ones.
\renewcommand{\sectionmark}[1]{\markboth{}{\thesection.\ #1}}
\renewcommand{\subsectionmark}[1]{\markboth{}{\thesubsection.\ \textsl{#1}}}

{\setlength{\parskip}{0.4ex plus0.2ex minus0.2ex} % For a more compact index layout

\tableofcontents}

\vfill
%\begin{center}
%{\tiny$<$ \verb$Id: main.tex,v 1.2 2004-06-29 10:40:56 gparra Exp $$>$ }
%\end{center}


%%%%%%%%%%%%%%%%%%%% MAINMATTER

\clearemptydoublepage

\newpage
\pagenumbering{arabic}
\setcounter{page}{1}

\chapter{Introduction}

\section{Biological background}

\subsection[What is a gene?]{What is a gene? 
\footnote{This subsection owes its overall structure to \cite{snyder:2003a}.}}

Historically, the term gene, was coined by the Danish botanist Wilhelm
Johannsen in the early 1900s as an abstract concept to explain the
hereditary basis of traits. He also made the distinction between the
outward appearance of an individual (phenotype) and its genetic traits
(genotype).

Four years earlier, William Bateson, an early geneticist and a
proponent of Mendel's ideas, had used the word genetics in a letter;
he felt the need for a new term to describe the study of heredity and
inherited variations. But the term did not start spreading until
Wilhelm Johannsen suggested the Mendelian factors of inheritance
to be called genes.

Phenotypic traits were associated to hereditary factors even though the
physical basis of those factors were not known. Early genetic studies
by Thomas Hunt Morgan and others associated heritable traits with
specific chromosomal regions. Using fruit flies as a model organism,
Thomas Hunt Morgan and his group at Columbia University showed that
genes, located in the chromosomes, are the units of heredity.  In 1930s,
George Beadle introduced the concept of ``one gene, one enzyme'',
which later became ``one gene, one polypeptide''.

With the development of recombinant techniques and gene cloning, it
became possible to combine the assignment of a gene to a specific
segment of a chromosome and the production of a gene product. Although
it was originally presumed that the final product was a protein, the
discovery that RNA have structural, catalytic, and even regulatory
properties made it evident that the end product could be a nucleic
acid. Thus, we can define a gene in molecular terms as ``a complete
chromosomal segment responsible for making a functional
product''. This definition has several logical components: the
expression of a gene product, the requirement that it must be
functional, and the inclusion of both coding and regulatory
regions. According to this definition, it should be possible to use
straightforward criteria to identify genes in the DNA sequence of a
genome.

But the complexity of the genomic information is even beyond these
criteria, there are additional problems in gene identification:
overlapping transcripts,  alternative splicing, fusion genes or 
pseudogenes. 

There are now examples of overlapping reading frames of protein coding
genes, overlapping transcriptional units (for example where the exon
of one gene is encoded within the intron of another) and even
overlapping protein coding genes \citep{coelho:2002a,tycowski:1996a}.
In all cases of gene overlap, each gene has a unique functional
sequence that is different to the others.

The production of several isoforms from the same transcription unit by
various types of alternative splicing is a very common event. Because
a single primary transcript can have several regions that can generate
alternative splicing, the resulting combinatorial effects of selecting
different splice sites can be very pronounced, and genes that code for
tens to hundreds of different isoforms could be common
\citep{graveley:2001a}. In the human genome, more than half genes
have spliced isoforms and this is likely to be an underestimation
because not all variant have been identified \citep{modrek:2002a}.
Gene products from alternatively spliced messenger mRNAs have
functionally a unique and distinct sequence.

There are some evidences that, in some cases, two adjacent genes are
transcribed together. But, fused transcripts is not a synonym for a
fused protein. Distinct proteins in prokaryotic organisms are
organized in operons, long transcripts encoding many proteins. Thus,
an authentic gene fusion should possess a particular mechanism to
override the nonsense codon used to stop translation of the N-terminal
protein. Taking advantage of the more efficient splicing system in
eukaryotes, the observed cases, employ alternative splicing to skip
the exons containing the stop codon. Therefore, it have been described
cases where genes from two adjacent loci are transcribed together,
probably as a result of a weak terminating signal, and after splicing,
a chimeric mature transcript is build skipping the stop signal and
generating a new chimeric protein with exons from both pre-existing
and independent genes \citep{thomson:2000a,poulin:2003a}.

The definition of a gene is also linked with the definition of a
pseudogene. Pseudogenes are similar in sequence to normal genes, but
they usually contain obvious disablements such as frame shifts or stop
codons in the middle of coding domains. This prevents them from
producing a functional product or having a detectable effect on the
organism's phenotype. The boundary between living and dead genes is
often not so sharp. Pseudogenes can be transcribed; truncated proteins
could still have some functionality, and regions with stop codons
could be alternatively spliced. Conversely, there are some pseudogenes
that have entire coding regions without obvious disablements but do
not appear to be expressed; presumably they lack the regulatory
elements required for the transcription. In these cases, is difficult,
or almost impossible, to discard marginal transcription in some
isolate tissue at any developmental time.

As we have seen, gene has a broad and often diffuse definition, that
could be convenient for the cell in order to generate a huge amount of
different combinations of final mRNAs. For the rest of this thesis the
term gene will refers to protein coding genes and we will simplified
the problem not taking into account overlapped transcripts and
alternative splicing isoforms. 

%This is obviously a simplification ....

\subsection{Molecular basis of genomic information}

DNA (deoxyribonucleic acid) is a double-stranded molecule that is
twisted into a helix like a spiral staircase. Each strand is composed
of a sugar-phosphate backbone and numerous base chemicals attached in
pairs. The four bases that make up the stairs in the spiraling
staircase are adenine (A), thymine (T) (uracil, U, when we are
referring to the RNA, ribonucleic acid), cytosine (C) and guanine
(G). These stairs act as the "letters" in the genetic alphabet,
combining into complex sequences to form the words, sentences and
paragraphs that act as instructions to guide the formation and the
functional nature of the host cell. Maybe even more appropriately, the
A, T, C and G in the genetic code of the DNA molecule can be compared
to the "0" and "1" in the binary code of computer software. Like
software to a computer, the DNA code is a genetic language that
communicates information to the organic cell. It was not until early
90s, that the Human Genome Project and the scientific community did
not deeply explore the nature and complexity of the digital code
inherent in DNA.


%But what kind of information is encoded in the DNA molecules ?  A
%segment of DNA that contains the information to make a specific
%protein (or part of a protein). Like DNA, proteins are synthesized
%like "beads on a string" but with 20 different kinds of beads (amino
%acids) rather than the 4 of DNA. Chemical properties that distinguish
%different amino acids ultimately cause the protein chains to fold up
%into specific three-dimensional structures. Although DNA are
%information rich, they are chemically simple and
%homogeneous. Proteins, by contrast, are chemically complex and
%diverse, properties that enable them to do so many different
%jobs. Proteins are "where the action is" in living systems (although
%nowadays is clear that RNA molecules with catalytic activity are also
%playing and important role in cells biochemistry). Proteins are
%motors, pumps, chemical catalysts, detectors, signals and signalers,
%structural units, gateway keepers, assemblers, and garbage
%handlers. They regulate cell replication, survival, and even death.

Thus, how do we move from DNA to proteins? This process, known as the
central dogma of the biology, involves three main steps:
transcription, RNA modifications (including capping, splicing and
polyadenilation) and translation. The basic schema of the central
dogma is shown in figure \ref{central dogma}

\begin{figure}
\begin{center}
\includegraphics[width=14cm]{figures/genestruct}
\caption[Schema of the central dogma of gene expression]{Schema of the
central dogma of gene expression. In the typical process of eukaryotic
expression, a gene is transcribed from DNA to pre-mRNA. mRNA is then
produced from pre-RNA by RNA processing, which includes the capping,
splicing and polyadenilation of the transcript. It is then transported
from the nucleus to the cytoplasm for the translation. Adapted from
\cite{zhang:2002a}} \label{central dogma}
\end{center}
\end{figure}

\subsection*{Transcription}

A gene is regulated by a sequence at the promoter or enhancer regions
that are a recognized by a set of specific proteins. These proteins
function as transcriptions factor needed for RNA polymerase to
initiate transcription. Active proteins are available only under
conditions when the gene is needed to be expressed; it absents means
that the promoter is not activated by this particular case. The
control region combines several different kinds of regulatory element,
and suggests the principle that when a promoter is regulated in more
than one way, each regulatory event depends on binding of its own
protein to a particular sequence.  When the optimal combination of
transcription factors are bind to their corresponding sequence
elements, the continuous sequence of DNA corresponding to a single
gene is copied to a RNA sequence by the RNA polymerase II.

It is possible that termination by RNA polymerase II is only loosely
specified. In some transcription units termination occurs beyond 1000
bp downstream of the site corresponding to the mature 3' end of the
primary transcript (which is generated by cleavage at a specific
sequence). Instead of using specific terminator sequence, the enzyme
stops RNA synthesis within multiple sites located in rather long
``terminator regions'' \citep{lewis:1997a}. The nature of individual
termination sites is not known.


\subsection*{RNA modifications}

There are three main RNA modifications: the capping reaction, splicing
and the maturation of the 3' end by cleavage and polyadenilation.
These three processes occur while the RNA is being sintesized. There
are evidences that regulatori interactions among these processes and
transcription (through the C-terminal domain of the RNA polimerase II)
would be crucial to obtain the final mature RNA. Recent studies have
shown that the ``mRNA factory'' is a dynammic complex whose
composition changes as it moves along the transcribed sequence of genes
\ref{zorio:2004a}. 

\subsubsection{Capping}

The 5' end of the RNA (which is the end synthesized first during
transcription) is first capped by the addition of a methylated G
nucleotide. Capping occurs almost immediately, after about 30
nucleotides of RNA have been synthesized, and it involves condensation
of the triphosphate group of a molecule of GTP with a diphosphate left
at the 5' end of the initial transcript. The new G residue added to
the end of the RNA is in reverse orientation from all the other
nucleotides. This 5'cap will later play an important part in the
initiation of protein synthesis and it also seems to protect the
growing RNA transcript from degradation \citep{lewis:1997a}.

\subsubsection{Polyadenilation}

The 3' ends of mRNAs are generated by cleavage followed by
polyadenilation. RNA polymerase transcribes past the site
corresponding to the 3' end, and sequences in the RNA are recognized
as targets for and endonucleolytic cut followed by polyadenilation. A
common feature of the mature transcripts in higher eukaryotes (but not
yeast) is the presence of the sequence AAUAAA in the region from 11-30
nucleotides upstream of the site of poly(A) addition. The sequence is
highly conserved and only occasionally is even a single base
different. Deletion or mutation of the AAUAAA hexamer prevents
generation of the polyadenylated 3' end. The signal is needed for both
cleavage and polyadenylation \citep{lewis:1997a}. Generation of the
proper 3' terminal structure requires an endonuclease (consisting of
the components CFI and CFII) to cleave the RNA, a poly(A)polymerase
(PAP) to synthesize the poly(A) tail, and a specificity component
(CPSF) that recognizes the AAUAAA sequence and directs the other
activities.

The addition of poly(A) helps stabilize mRNA and seems to be related
with efficiency of translation initiation. The average size of the
poly(A) tail is over 70 adenosines in yeast, to over 240 adenosines in
mammals for newly transcribed mRNA and pre-mRNA in the
nucleus. Cytoplasmic enzymes may also cause the polyA to shorten with
age, and occasionally lengthen. Not all transcripts are
polyadenylated; some histone mRNA is poly(A) minus. For transcripts
without poly(A) tail, the 3' end seems to be protected or sequestered
by association with other factors.

\subsubsection{splicing}

The primary RNA transcript is spliced to remove intron sequences,
producing a shorter RNA molecule. Introns are removed from the
nuclear RNAs of higher eukaryotes by a system that recognizes short
consensus sequences conserved at exon-intron boundaries and within the
intron. The splicing of precursors to mRNAs occurs in two steps, both
involving single transesterification reactions. The first step
generates a 2'-5' bond at the branch site upstream of the 3' splice
site and a free 3' hydroxyl group on the 5' exon generating a lariat
RNA intermediated.  In the second step, attack of the 3' hydroxyl on
the phosphodiester bond at the 3' hydroxyl group and result in joining
of the two exons.


This reaction requires a large splicing apparatus, which takes the
form of an array of proteins and ribonucleoproteins that junctions as
a large particulate complex (the spliceosome). There are two distinct
types of spliceosome in most organisms. The major class or U2-type
(also known as canonical or GT-AG splice sites) spliceosome is
universal in eukaryotes, whereas the minor class or U12-type (also
known as AT-AC) spliceosome may not be present in some organisms. The
consensus sequences of U12-type introns are more highly conserved than
those of vertebrate U2-type introns. Figure \ref{splice signals} shows
the conservation of the U2-type involved signals.

The spliciosome is composed of five small nuclear RNAs (snRNAs) called
U1, U2, U4, U5, and U6 and numerous protein factors. Splice site
recognition and spliceosomal assembly occur simultaneously according
to a complex sequence of steps (see figure \ref{splicing process}).
The first step appears to be the recognition of the donor (5') splice
site at the exon-intron junction: a substantial amount of genetic and
biochemical evidence has established that this occurs primarily
through base pairing with the U1 snRNA over a stretch of approximately
nine nucleotides, including the last three exonic nucleotides and the
first six nucleotides of the intron. The second step in spliceosomal
assembly involves binding of U2 auxiliary factor (U2AF) and possibly
other proteins to the pyrimidine-rich region immediately upstream of
the acceptor site, which directs U2 snRNA binding to the branch point
sequence approximately 20 to 40 bp upstream of the intron-exon
junction.  The U2 snRNA sequence 3' GGTG 5' has been show to base pair
with the branch point signal, consensus 5' YYRAY 3', with the unpaired
branch point adenosine bulged out of the RNA duplex.  Mutations or
deletions of the branch site in yeast prevent splicing. In higher
eukaryotes, the relaxed constraints in its sequence result in the
ability to use related sequences in the vicinity when the authentic
branch is deleted. Subsequently, a particle containing U4, U5, and U6
is added, U5 snRNA possibly interacting with the acceptor site,
leading eventually to formation of the mature spliceosome.

%These signals, however, do not appear to carry the information
%requeired for the recognition of exons in genomic sequences
%\citep{burge:1998a}. For a more detailed review about the splicing
%process \cite{burge:1999a}.
\begin{figure}
\begin{center}
\includegraphics[width=14cm]{figures/splicing}
\caption[Classical splicing signals]{Splicing signals for U2-type. The
nearly invariant GU and AG dinucleotides at the intron ends, the
polypyrimidine tract (Y)n preceding the 3'AG, and the A residue that
serves as a branch point are shown in below. For each sequence motif,
the size of nucleotide at a given position is proportional to the
frequency of that nucleotide at that position in an alignment if
conserved sequences. Nucleotides that are part of the classical
consensus motifs are shown in blue, except for the branch point A,
which is shown in orange. Adapted from \cite{cartegni:2002}.} 
\label{splice signals}
\end{center}
\end{figure}

Several examples of intronic and exonic cis-elements that are
important for correct splice site identification and that are distinct
from the classical splicing signals have been recently described.
These elements can act stimulating (as do enhancers) or repressing (as
do silencers) splicing, and they seem to be specially relevant for
regulating alternative splicing. Exonic splicing enhancers (ESEs), in
particular, appear to be very prevalent, and might be present in most,
if not all, human exons, including constitutive ones
\citep{cartegni:2002}.  The lack of a well-defined consensus sequence
for these signals, indicates that they might consist of numerous,
functionally different classes, and that the factors involved
may recognize degenerate signal sequences \citep{cartegni:2002}.

\begin{figure}
\begin{center}
\includegraphics[width=14cm]{figures/MOOM0193-EtapesSpliceosoma}
\caption[The spliciosome cycle]{The spliciosome cycle. The processing
of the pre-mRNA containing two exons and one intron into the ligated
exon product and lariat intron is shown, emphasizing the involvement
of the small nuclear ribonucleoprotein (snRNP) particles at distinct
steps in spliceosome formation and catalysis. Adapted from
\cite{burge:1999a}.} \label{splicing process}
\end{center}
\end{figure}


\subsection*{Translation}

The mRNA sequence is translated into protein sequence, outside the
nucleus, by a sub-cellular structure known as ribosome (a compact
ribonucleo-protein consisting of two subunits). The ribosome binds to
the mRNA, and scans the sequence synthesizing the amino acid sequence
specified by consecutive non-overlapping codons. Scanning of the mRNA
proceeds until the ribosome finds one of the three codons not
specifying amino acids (the Stop Codons). At that point, elongation of
the amino acid sequence ends, and the final protein product is
released.

Selection of the start codon sets the reading frame that is maintained
normally throughout all subsequent steps in the translation
process. What makes the start different from the addition of a
methione internally in the polypeptide chain is that a special
transfer RNA (tRNA), used just to read the start codon, is present,
tRNAi. When this tRNAi is charged with Met to form Met-tRNAi, this
compound binds into the P site of ribosomes. In eukaryotes, the small
(40s) ribosomal subunit carrying met-tRNAi and other associates
proteins recognizes the 5' end of the mRNA and then migrates through
the 5' untranslated region (UTR) until it encounters the first AUG
codon which is recognized by base pairing with the anticodon in
met-tRNAi when a 60s ribosomal subunit joins the paused 40s subunit,
selection of the start codon is fixed.

Flanking sequences modulate the efficiency with which the AUG codon is
recognized as a stop signal during the scanning phase of
initiation. In vertebrates mRNAs, initiation sites usually conform to
all or part of the so called translation (or Kozak) signal:
GCCACCaugGCG \citep{kozak:1987a}.

For maximum effectiveness, the upstream GCCACC motif must directly
precede the AUG codon. If the motif is upstream or the sequence is not
optimal the effectiveness in the translation is reduced and even other
cryptic AUG can be used instead the real one \citep{kozak:1999a}. How
the consensus sequence is recognized is not yet known. One possibility
is that interaction with GCCACC might slow scanning, and thus,
facilitate the recognition of the AUG codon by met-tRNAi.

Although context effects on AUG codon recognition have been studied
primarily in mammalian systems, a strong contribution of the motif
have also been demonstrate in plants. In \Sc, however, the effects
of context are slight \citep{kozak:1999a}.

Proteins are assembled by the sequential addition of amino acids in
the direction from the N-terminus to the C-terminus as a ribosome
moves along the mRNA. Translation termination is initiated when one of
the three stop codons is present in the ribosomal A site, resulting in
binding of the Release Factor (RF). Then, the hydrolysis of the
peptide bond results in a deacylated tRNA in the ribosomal P site. RF1
is removed from the ribosome in a GTP-dependent reaction involving
RF3, resulting in the dissociation of the 60S/mRNA complex.

Chemical properties that distinguish different amino acids ultimately
cause the protein chains to fold up into specific three-dimensional
structures that enable them to do their specific function.

\section{Gene prediction background}

\subsection{Gene prediction methods}

As we have seen in the previous section, a given protein sequence is
not usually specified by a continuous DNA sequence, but genes are
often split in a number (maybe large) of (small) coding fragments
known as exons, separated by (larger) non-coding intervening fragments
known as introns. Often, intronic and intergenic DNA makes most of the
genome in high eukaryotic organisms. In the human genome, for
instance, only a very small fraction of the DNA, which can be as low
as 2\%, corresponds to protein coding exons.

The aim in any gene prediction program is, given DNA sequence, to find
the encoded gene structures and the corresponding amino acid
sequence. Therefore, what we are trying to do, is model all this
biological knowledge to be able to reproduce what the cell does in an
anonymous DNA sequence. In deed, typical computational eukaryotic gene
prediction involves the following tasks:

\begin{itemize}
\item identification of suitable signals
\item prediction of candidate exons defined by the corresponding signals
\item assembly of a subset of these exon candidates in a predicted gene structure
\end{itemize}

The particular implementation of these tasks varies considerably
between programs. For the rest of dissertation, the term gene refers to
protein coding genes and the term exons to the protein coding part of
an exon, unless stated otherwise. We will briefly review them next.

\subsubsection{Prediction of biological signals}

Sequence signals are defined as short functional DNA elements that are
recognized by the cellular machinery in the processes involved in the
pathway leading from DNA to protein sequences. As we have seen in the
previous sections there are many target signal sequences. However,
there are four basic signals that defines the boundaries of the coding
regions of the primary transcript: the translational start site (TSS), the
5' splice site (donor site), the 3' splice site (acceptor site), and
the translational stop codons.

% Sequence are not 100\% conserved ... 

The most common method to model sites has been position weight
matrices. \cite{senapathy:1990a} presented the first quantitative
frequency matrices where each matrix element $M_ij$ is the frequency
of the base $i$ in the position $j$ from a set of aligned splice sites
sequences. Figure \ref{frequency matrix} shows the frequency matrix
derived from a set of human donor and acceptor sites.

\input{figures/frequency_matrix}


If $f_i$ as the background frequency of nucleotides $i$, then the
popular log-odd scoring matrix (known as Position Weight Matrix PWM,
Weight Matrix Model WMM, or Position Specific Scoring Matrix PSSM)
$S_ij=log(f_ij/fi)$ scored the signal as the sum of the scores over
the bases within the signal. If the resulting score gives positives
score the sequence occurs in signal site more often than expected by
chance, while if the score is negative the sequence occurs in the
signal site less often than expected.

PWMs assume that bases at different positions are independent. This is
often an oversimplification.  Several authors have observed
statistically significant dependencies between position within
different signals. Certain observed dependencies between donor splice
sites position can be interpreted in terms of the thermodynamics of
RNA duplex formation between A1 snRNA and the 5' splicing region of
the pre-mRNA. Of the dependencies observed in human acceptors sites,
some appear to result simply from the compositional heterogeneity of
the human genome, whereas others probably relate to specificity of
pyrimidine tract binding proteins.

There are many ways to incorporate base dependencies. One method is to
assume that the probability of each base at a given position depends
on the base occurring at the previous position or the positions (the
so-called Markov dependence or Weight Array Model WAM,
\cite{zhang:1993a}). More dependencies, however, result in more
parameters to be estimated and it requires much more training
data. Another method is to apply a decision tree (so-called Maximal
Dependence Decomposition MDD, \cite{burge:1997a}) to partition total
training data into subsets so that splice site bases within each
subset are approximately independent and hence can be modeled by a
separate PWM.

New recent methods as multilayer neural networks \citep{reese:1997a}
or inclusion-driven learned Bayesian Networks (idlBNs)
\citep{castelo:2004a} methods have been developed. These more complex
models typically yield significant, but not dramatic, improvements in
splice sites discrimination over the simpler models which assume
independence between positions. in isolation or in the contexts of an
integrating gene finding method \citep{burge:1998a}.

\subsubsection{Prediction of putative exons}

Once all the putative signal are predicted, all the putative exons can
be built. Only the coding fraction of the coding exons is predicted by
gene finding methods. Typically, gene prediction programs redefine the
term exons to refers only to the coding fraction of the exons, and
classified them as: initial (limited by a TSS and a donor site),
internal (limited by an acceptor and a donor site), terminal (limited
by an acceptor and a stop codon) and single exon genes (limited by a
TSS and stop codon). See \cite{zhang:2002a} for a more detailed
description of all the possible types of exons.

To discriminate coding regions from non coding regions a large number
of content measures have been developed (\cite{fickett:1992a},
\cite{gelfand:1995a} and \cite{guigo:1999a}). Such content measures --
also known as coding statistics -- can be defined as functions that
compute a real number related to the likelihood that a given DNA
sequence codes for a protein. Protein coding regions exhibit
characteristic DNA sequence composition which is absent on non-coding
regions. This bias is mainly due coding restrictions: the specific amino
acids usage to build proteins and the unequal usage of the
synonymous codons.  \cite{fickett:1982a} also shows that coding
regions have asymmetries and periodicities that help to distinguish
them from non coding sequences.

Among all the different coding measures, codon position dependent 5th
order Markov model \citep{borodovsky:1993a} appear to offer the
maximum discriminative power \citep{guigo:1998a} and are at the core
of most popular gene finders today. In this model the conditional
probability of the identity of the next nucleotide depends on the
identities of previous five bases. This relatively complicated model
incorporate a combination of biases related to amino acid usage, codon
usage, di-amino acid and dicodon usage as well as other underlying
factors.

Coding content measures are usually combined with the scores of the
exon defining signals to obtain a final exon score. There is a number
of ways in which these scores can be combined. If computed as
log-odds, they can be simply summed up under the assumption of
independence(). Linear discriminant analysis \citep{solovyev:1995a}
and artificial neural networks \citep{uberbacher:1991a} have also been
used.

\subsubsection{Assembly of putative exons}

Predicted exons need to be assembled into genes. This assembly must
conform to a number of intrinsic biological constraints such as
non-overlap between assembled exons and the maintenance of an ORF
among them.

The main difficulty in exon assembly is the combinatorial explosion
problem: the number of ways $N$ candidate exons may be combined grows
exponentially with $N$. To address this problem a number of methods based
on dynamic programming techniques have been developed. In dynamic
programming \citep{bellman:1957a}, the solution to a general problem
is obtained by the recursive solution of smaller versions of the
problem. In the ``optimal exon assembly'' problem, dynamic programming
allows to find the solution efficiently, without having to enumerate
all exons assembly possibilities \citep{gelfand:1993a}. Algorithms
running in quadratic time (in time proportional to the square of the
number of predicted exons) were used in GeneParser
\citep{snyder:1993a}, GrailII \citep{xu:1994a} and fgenesh
\citep{solovyev:1995a}, among other programs. \cite{guigo:1998a}
developed a more efficient algorithm running in linear time (that is
in time proportional to the number of predicted exons). At the core of
the recently developed GAZE system \citep{howe:2002a}, there is also a
chaining algorithm that runs effectively in linear time.

A novel advance in gene prediction methodologies was the application
of generalized Hidden Markov Models (HMMs). This probability model
were first developed in the speech-recognition field and later applied
to protein and DNA sequence pattern recognition. Initially implemented
in the gene prediction field in the GENIE algorithm
\citep{kulp:1996a}.  In a HMM approach, different types of gene
structure components (such as exon or intron) are characterized with
states, and a gene model is thought to be generated by a state
machine: starting from 5' to 3', each base-pair is generated by a
``emission probability'' conditioned on the current state and
surrounding sequences and transition from one state to another is
governed by a ``transition probability'' which obeys all the
constraints (such as an intron can only follow an exon, reading frames
of two adjacent exons must be compatible, etc.). All the parameters of
the ``emission probabilities'' and ``transition probabilities'' are
learned (pre-computed) from some training data set. Since the states
are unknown (``Hidden''), an efficient dynamic programming algorithm
(called the Viterbi algorithm) may be used to select the best set of
consecutive states (called a ``parse''), which has the highest overall
probability compared with any other possible parse of the given
genomic sequence (see \cite{rabinier:1989a} for a tutorial on HMMs). It
is easy to add more states (such as intergenic regions, promoters,
UTRs, etc.) and transitions into HMM-based models to allow partial
genes, intronless genes, even multiple genes or genes on different
strands to be incorporated.

\subsection{Ab initio gene prediction}

Computational gene finding is not a brand new field and a large body
of literature has accumulated past 20 years or so. In this section we
briefly resume the initial steps of gene prediction tools. Early
studies by \cite{shepherd:1981a}, \cite{fickett:1982a} and
\cite{staden:1982a} showed that statistical measures related to
biases in amino acid and codon usage could be used to approximately
identify protein coding regions in genomic sequences.  Based on these
differences, the firsts generation of gene predictions programs,
designed to identify approximate locations of coding regions in
genomic DNA were developed. The most widely known such programs are
probably Testcode (based on \cite{fickett:1982a}) and GRAIL
\citep{uberbacher:1991a}.  These methods were able to identify coding
regions of sufficient length (100-200bp length) with fairly high
reliability, but do not accurately predict exon locations.

In order to predict exon boundaries, a new generation of algorithms
were developed. Second generation methods, such SORFIND
\citep{xu:1994a}, GRAIL II \citep{hutchinson:1992a} and XPOUNd
\citep{thomas:1994a}, use a combination of splice signal and coding region
identification techniques to predict ``spliceable open reading
frames'' (potential exons), but do not attempt to assemble predicted
exons into complete genes. Third generations methods attempt the more
difficult task of predicting complete gene structures, i.e. sets of
exons which can be assembled into translatable mRNA sequences. The
earliest examples of such integrated gene finding algorithms were
probably the GeneModeler program \citep{fields:1990a} for prediction
of genes in \Ce\ and the method of \citep{gelfand:1990a} for
mammalian sequences. Subsequently, there has been a mini-boom of
interest in development of such method, and a wide variety of programs
have appeared, including (but not limited): \geneid\
\citep{guigo:1992a}, which uses a hierarchical structure; Geneparser
\citep{snyder:1993a}, which uses a combination of neural network and dynamic
programing approaches; GeneMark \citep{borodovsky:1993a}; genlang
\citep{dong:1994a}, which treats the problem by linguistic methods;
fgenes \citep{solovyev:1994a} which uses a discriminant analysis and
other statistical techniques.


At the end of last decade, a next generation of programs appeared
simultaneously with the completion of the first genome sequencing
projects. Some of them were even used in the earlier stages of the
annotation pipelines. As a new data and more powerful computers became
more accessible, the gene fingers were able to deal with sequences
containing more than one gene. Although the gain in accuracy was
significant, it was still insufficient in general terms (see
\cite{guigo:2000a} and \cite{rogic:2001a}). Some programs of this
generation were GENIE \citep{kulp:1996a}, HMMGENE \citep{krogh:1997a}
and GENSCAN \citep{burge:1997a}.


\subsection{Genome comparison gene prediction}

With the availability of genomes from different species a number of
strategies have been developed to use genome comparisons to predict
genes. The rationale behind comparative genomics methods is that
functional regions, protein coding among them, are more conserved than
non-coding ones between genome sequences from different organisms. This
characteristic conservation can be used to identify protein coding
exons in the sequences. The approach taken by the different programs to
exploit this idea differ notably.

In one such approach \citep{bayo:2002a, pedersen:2002a}, the problem
is stated as a generalization of pairwise sequence alignment: given
two genomic sequences coding for homologous genes, the goal is to
obtain the predicted exonic structure in each sequence maximizing the
score of the alignment of the resulting amino acid sequences. Both,
\cite{bayo:2002a} and \cite{pedersen:2002a} solve the problem through
a complex extension of the classical dynamic programming algorithm for
sequence alignment. Although very appropriate for short sequences, in
the practice, the time and memory requirements of this algorithm may
limit its utility for very large genomic sequences.  Moreover,
although the approach is theoretically sound, and it is guaranteed to
produce the optimal amino acid sequence alignment, the fact that
sequence conservation may occur in regions other than protein coding,
could lead to over prediction of coding regions, in particular when
comparing large genomic sequences from homologous sequences from
closely related species.

To overcome this limitation, the programs \slam\ \citep{pachter:2003a} and
\doublescan\ \citep{meyer:2002a} rely on more sophisticated models of coding 
and non-coding DNA and splice signals, in addition of sequence
similarity. Because sequence alignment can be solved with Pair Hidden
Markov Models\citetext{PHMMs, \citealp{durbin:1998a}}, and because
Generalized HMMs (GHMMs) have been proved very useful to model the
characteristics of eukaryotic genes \citep{burge:1997a}, \slam\ and
\doublescan\ are built upon the so-called Generalized Pair HMMs. In these,
gene prediction is not the result of the sequence alignment, as in the
programs above, but both gene prediction and sequence alignment are
obtained simultaneously.

A third class of programs adopt a more heuristic approach, and
separate clearly gene prediction from sequence alignment.  The
programs \rosseta\ \citep{batzoglou:2000a}, \sgpo\ \citetext{from Syntenic
Gene Prediction, \citealp{wiehe:2001a}}, and \cem\ (from the Conserved
Exon Method, \citealp{bafna:2000a}) are representative of this
approach.  All these programs start by aligning two syntenic regions
(specifically human and mouse in \rosseta, and \cem; less species specific
in \sgpo), using some alignment tool (the {\sc glass} program,
specifically developed in the case of \rosseta\, or generic ones, such as
\tbx, or \sim\ in the case of \cem\ and \sgp), and then predict gene
structures in which the exons are compatible with the alignment. This
compatibility often requires conservation of exonic structure of the
homologous genes encoded in the anonymous syntenic regions. Although
conservation of exonic structure is an almost universal feature of
orthologous human/mouse genes \citep{mouse:2002a}, it does not
necessarily occur when comparing genome sequences of homologous genes
from other species.
 
% The programs described so far rely on the comparison of fully
%assembled (and when from different organisms, syntenic) genomic
%regions. This limits their utility when analyzing complete large
%eukariotic genomes, and in particular when the informant genome is in
%non-assembled shotgun form. To overcome this limitation, the programs
%\twinscan\ \citep{korf:2001a} and \sgp\ take still a different approach. 
%The approach is reminiscent of that used in \genscan\ to incorporate
%similarity to known proteins to modify the \geneid\ scoring schema.
%Essentially, the query sequence from the target genome is compared
%against a collection of sequences from the informant genome (which can
%be a single homologous sequence to the query sequence, a whole
%assembled genome, or a collection of shotgun reads), and the results
%of the comparison are used to modify the scores of the exons produced
%by \textit{ab initio} gene prediction programs. In \twinscan, the
%genome sequences are compared using \bln\ and the results serve to
%modify the underlying probability of the potential exons predicted by
%\geneid. In \sgp, the genome sequences are compared using \tbx, and the
%results used to modify the scores of the potential scores predicted by
%\geneid.

As the number of genomes sequenced at different evolutionary distances
increases, methods to predict genes based on the comparative analysis
of multiple species -- and not only of two species -- look
promising. For instance, \cite{dewey:2004a} combine pairwise
predictions from SLAM in the human, mouse and rat genomes to
simultaneously predict genes with conserved exonic structure in all
three species. In the so-called Phylogenetic Hidden Markov Models
(phylo-HMMs) or Evolutionary Hidden Markov Models (EHMM), a gene
prediction Hidden Markov Model is combined with a set of evolutionary
models, based on a phylogenetic tree. Phylo-HMMs take into account
that the rate (and type) of evolutionary events differs in
protein-coding and non-coding regions. Recently, Phylo-HMMs have been
applied to gene prediction with encouraging results
\citep{pedersen:2003a, siepel:2004a}.

Phylo-HMMs have been also used in the context of phylogenetic
shadowing \citep{boffelli:2003a}. Phylogenetic shadowing examines
sequences of closely related species and takes into account the
phylogenetic relationship of the set of species analyzed. This
approach enables the localization of regions of collective variation
and complementary regions of conservation, facilitating the
identification of coding as well as non-coding functional regions. The
likelihood ratio under a fast- versus slow- mutation regime can be
computed for each aligned nucleotide site across all the sequences
being analyzed. This ratio represents the relative likelihood that any
given nucleotide site was subjected to a faster or slower rate of
accumulation of variation and is related to functional constraints
imposed on each site. Exon containing sequences will display the least
amount of cross species variation, in agreement with the constraint
imposed by their functional role. Regions from different parts of the
genome, in which a functional non-coding sequence will appear, may
evolve at different rates \citep{eberseberger:2002a}, as reflected by
differences in their absolute likelihoods. Despite that, functional
non-coding regions can be retrieved from stretches of sequence having
minimal variation similar to exonic ones.

\subsection{Gene prediction accuracy}

The sheer number of such algorithms raises the obvious question of
whether the gene finding problem has perhaps already been solved by
one or more of these programs. This question was repetitively answered
in the negative by different systematic comparisons of available
integrated gene finding methods.

Of course, the answer to this question depends on a number of factors:
the species under consideration, the sequence context, the existence
of experimental evidence, etc. In general, a prediction supported by
spliced ESTs or showing strong similarity to known coding sequences is
more reliable than a prediction with no supporting evidence.
Similarly, consistent predictions on a genomic region obtained by
different programs should be more reliable than isolated predictions
by just one program. In addition of having some clue on the accuracy
of the predictions on particular cases, one would like to have an
overall measure of the accuracy of the "ab initio" gene prediction
programs. A number of attempts have been made to produce independent
comparative estimates of the performance of gene finders.

The accuracy of gene prediction programs is usually measured in
controlled data sets. To evaluate the accuracy of a gene prediction
program on a test sequence, the gene structure predicted by the
program is compared with the actual gene structure of the
sequence. The accuracy can be evaluated at different levels of
resolution. Typically, these are the nucleotide, exon, and gene
levels. These three levels offer complementary views of the accuracy
of the program. At each level, there are two basic measures:
Sensitivity and Specificity, which essentially measure prediction
errors of the first and second kind. Briefly, Sensitivity is the
proportion of real elements (coding nucleotides, exons or genes) that
have been correctly predicted, while Specificity is the proportion of
predicted elements that are correct. More specifically, if TP are the
total number of coding elements correctly predicted, TN, the number of
correctly predicted non-coding elements, FP the number of non-coding
elements predicted coding, and FN the number of coding elements
predicted non-coding, then, in the gene finding literature,
Sensitivity is defined as Sn =TP/(TP+FN) and Specificity as Sp
=TP/(TP+FP). Both, Sensitivity and Specificity, take values from 0 to
1, with perfect prediction when both measures are equal to 1. Neither
Sn nor Sp alone constitute good measures of global accuracy, since
high sensitivity can be reached with little specificity and vice
versa. It is desirable to use a single scalar value to summarize both
of them. In the gene finding literature, the preferred such measure on
the nucleotide level is the Correlation Coefficient defined as CC
ranges from -1 to 1, with 1 corresponding to a perfect prediction, and
-1 to a prediction in which each coding nucleotide is predicted as
non-coding and vice versa.

Table \ref{burset accuracy} reproduces the results from the benchmark
by \cite{burset:1996a}, one of the first systematic evaluations
of gene finders. These authors evaluated seven programs in a set of
570 vertebrate single gene genomic sequences deposited in GenBank
after January 1993. This was done to minimize the overlap between this
test set and the sets of sequences which the programs had been trained
on. Average CC for the programs analyzed ranged from 0.65 to 0.80 at
the nucleotide level, while the average exon prediction accuracy
((Sn+Sp)/2) ranged from 0.37 to 0.64.

\cite{rogic:2001a} published recently a new independent comparative
analysis of seven gene prediction programs. The programs were again
tested in a set of 195 single gene sequences from human and rodent
species. In order to avoid overlap with the training sets of the
programs, only sequences were selected that had been entered in
GenBank, after the programs were developed and trained. Table
\ref{rogic accuracy} shows the accuracy measures averaged over the set
of sequences effectively analyzed for each of the tested programs.

The programs tested by \cite{rogic:2001a} showed substantially higher
accuracy than the programs tested by \cite{burset:1996a}:
average CC at the nucleotide level ranged from 0.66 to 0.91, while
average exon prediction accuracy ranged from 0.43 to 0.76. This
illustrates the significant advances in computational gene finding that
occurred during the nineties.

The evaluations by \cite{burset:1996a}, \cite{rogic:2001a}, and others
suffered from the same limitation: gene finders were tested in
controlled data sets made of short genomic sequences encoding a single
gene with a simple gene structure. These data sets are not
representative of the genome sequences being currently produced: large
sequences of low coding density, encoding several genes and/or
incomplete genes, with complex gene structure. 
\input{tables/burset_accuracy}

\input{tables/rogic_accuracy}


\newpage
\chapter{Objectives} 

\geneid\ \citep{guigo:1992a} was one of the first programs to predict
full exonic structures of vertebrate genes in anonymous DNA
structures. However, since the original\geneid\ was released, there had
been substantial developments in the field of computational gene
identification, and it had become clearly inferior to the other
existing tools.

The goal of this Ph.D. thesis, was to develop an improve geneid
prediction accuracy, and make it useful for the new genomes that were
going to come. Therefore, the main goals were the followings:

\begin{itemize}

\item To develop and add new gene recognition techniques to the new \geneid\ 
release. The parameter file structure should be explicit and have a
simple and intuitive interpretation and be estimated from any
available set of genes.

\item To build parameter files for different species and to establish
evolutionary relations between the differences found among different
species.

\item To develop a method to incorporate genomic comparative information
to the \geneid\ prediction framework.

\item To analyze the differences between the specification of 
the splicing in different organism and to try to build a more
realistic model of splicing.

\item To provide both, predicted genes and the bioinformatic tools to
the research community.

\end{itemize}

Many of the goals listed above have been achieved by the current
implementation of \geneid\ and its extension \sgp. The parameter files
incorporate several types of information. Depending on the amount of
available data for each species and the nature of the signal every
type of site could be represented with a position weight array of
different order. As coding statistics \geneid\ allows the use of any
order of Marko chain, depending again on the amount of available
data. With the new parameter file, a complete set of initial and
transition probabilities can be incorporated for different G+C content
context.

\geneid\ and \sgp\ accuracy have been tested in different set of
sequences, showing an accuracy superior to the existing tools, being
both specially more specific than the other programs. On the other
hand, certain features remain difficult to predict including very
small exons and the boundaries of independent genes. More general
challenges in the gene prediction field are pointed out in the
$Discussion$ section.

Most of this work have been done in collaboration with international
genome sequencing consortiums. These collaborations gave me the
opportunity to meet and work with specialist of all over the wold, and
made our work very relevant. However, these collaborations had put a
lot of pressure on our group and sometimes I would have like to have
more time to focus on some aspects of my research. For instance, the
more biological analysis of the definition of the splice sites and the
building of a more realistic model has been impossible to achieve
during the realization of this thesis. On the other hand, the
annotation of recently sequenced genome has been very fruitful,
allowing us to test and adapt our gene prediction tools to the real
needs of the genomic annotation projects.

For the last objective, the dissemination of data, all the programs and
sequence data have been made available with no restriction through our
own web service.


\newpage

%%%%%%%%%%%%%%%%%%%%%%%
%%  Chapter: geneid  %%
%%%%%%%%%%%%%%%%%%%%%%%

\chapter{Ab initio gene finding method: geneid} 

In this chapter, the implementation of the parameters that specify the
information needed for \geneid\ to predict genes is described. The
first parameters obtained for the new version of \geneid\ was obtained
while the re-programing in C of the first version was still in
process. This, was motivated by the announcement of the Berkeley
Drosophila Genome Project to generate a state of the art of gene
prediction tools in which any group could participate. Our group
decided to take part in the assessment and to develop a parameter file
for \DmL. The results of the assessment are briefly commented.
After that, a short description of the the parameters obtained for
other species is presented.

\section{Genome Annotation Assessment Project - GASP1} 

The GASP (\url{http://www.fruitfly.org/GASP1/}) experiment was
organized by the Berkeley Drosophila Genome Project to formulate
guidelines and accuracy standards for evaluating computational tools
and to encourage the development of existing approaches through a
careful assessment and comparison of the predictions made by programs
at that time. The goal of the annotation process is to assign as much
information as possible to the raw target sequence with an emphasis on
the location and structure of the coding genes.

The GASP experiment consisted on the following stages:
\begin{itemize}
\item Training data set of curated sequences and the $Adh$ region,
including 2.9 Mb of \DmL\ genomic sequence was collected by the
organizers and provided to the participants.
\item A set of annotations based on experimental data was developed to
evaluate submissions while the participating groups produced and
submitted their annotations for the region.
\item The participant group\'{ }s predictions were compared to the
standards and the results were presented as a tutorial at the
Intelligent Systems for Molecular Biology (ISMB, Heidelberg 1999).
\end{itemize}

The organization chose the 2.9 Mb $Adh$ contig because it was large
enough to be challenging, contained genes with a variety of sizes and
structures, and include regions of high and low gene density.

The annotation used as standard, ideally, should contain the the
correct structure of all the genes in the region without any
error. Unfortunately, such a set was impossible to obtain because the
underlying biology was incompletely understood. The built a two-part
approximation to the perfect data set, taking advantage of data form
the cDNA sequencing project and a \Droso\ community effort to build a
set of curated annotations for this region \citep{Ashburner:1999a}.
The first standard set, known as $std1$, used high quality sequence
from a set of 80 full-length cDNA clones from the $Adh$ region to
provide a set with annotations that are very likely to be correct but
certainly are not exhaustive. The second standard set, known as
$std3$, was built from the annotations being developed for
\cite{Ashburner:1999a} to give a standard with more complete coverage
of the region, although with less confidence about the accuracy and
independence of the annotations.

The organization also provided several \Droso-specific data sets to
enable the participants to tune their tools. The curated sequences
were extracted from the Flybase. The gene curated set contained
genomic sequences from 275 multi- and 141 single exon non-redundant
genes together with their start and stop codons an the splice sites
coordinates.

Participants were given the finished sequence for the $Adh$ region the
available related training data, but they did not have access to the
full-length cDNA sequenced that were sequenced for the paper by
\cite{Ashburner:1999a} that describes the $Adh$ region in depth. The
experiment was widely announced and open to any participants.

\subsection{Training protocol for geneid}

In most gene prediction programs, there is a clear separation between
the gene model itself and the parameters of the model. Typically, the
parameters of the gene model define the characteristic of the sequence
signals involved in gene specification (i.e. PWMs for the splice
sites), the codon bias characteristic of coding exons (i.e. hexamers
counts or Markov Models for coding regions), and the relation between
the exons when assembled into gene models (i.e. intron and exons
lengths distributions, transition probabilities in HMMs, ...). These
parameters are estimated from a set of annotated genomic sequence from
the species of interest. 


\subsubsection{Site definition}

To determine which positions are relevant for the definition of a site
in the training set, the relative entropy is calculated. The positions
frequency of nucleotides in the surrounding bases of the canonical
signals in both actual exon boundaries $P$ and non functional sites
$Q$ is measured (30 base pairs upstream and downstream). Then, for
each position $j$, the relative entropy $D_j$ (also known as the
Kullback-Liebler distance) is defined as:

\begin{equation}
D^j (P,Q) =\sum_{i=A,C,G,T}\ P_{ij}\ log\ \frac{P_{ij}}{Q_{ij}}\,\,\,.
\end{equation}

The stretch of nucleotides crossing the coding exon boundary with the
relative entropy above a threshold of 0.1 is accepted. After that,
the log-likelihood ratio between the real and the non functional site 
is computed as explained in the previous chapter.

Depending on the amount of available data for each species and the
nature of the signal every type of site could be represented with a
PWA of different order. For acceptor sites in human, a first-order PWA
is constructed based on some bias detected in dinucleotides around the
canonical signal AG \cite{burge:1997a}. However, a second-order PWA is
built for start codons to capture the appearance of a second ATG
signal after the real one because a biological penalty is known to
exist in order to avoid the activation of the second ATG
\cite{kozak:1999a}. In contrast, PWAs of order zero are constructed in
species with less accurate annotations.

\subsubsection{Coding potential}

First, the sequence of coding regions (CDS) and introns from the
training set are extracted. Next, the initial and transition matrices
for the Markov model are computed as log-likelihood ratio. The optimal
order that reflects the dependencies between contiguous codons and
dependencies between contiguous codons seems to be order five.  geneid
allows the use of many order of Markov chain as a coding
statistic. Different orders could be chosen depending on the amount of
available CDS for the corresponding species.

With geneid, a complete set of initial and transition probabilities
can be incorporated for different G+C content contexts.Thus, signals
and exons can be predicted using different scoring schemas according
to their genomic context.  In human, three different initial and
transition matrices have been constructed depending on the percentage
of G+C content (from 0 to 45, from 45 to 55 and from 55 to 100).

\subsubsection{Optimization}


geneid constructs genes structures, which can contain multiple genes
in both strands. The assembly algorithm try to optimize the sum of
scores of the putative assembled exon, as is explained in more detail
in the paper ``geneid in \Droso'' (presented in the next section). As
all the exons are computed as log-likely hood ratios, this can be
approximately interpreted as the log-likelyhood ratio of the
probability of the defining sites and the hexamer composition (if a
Markov of order five is computed). However, the simple sum of
log-likelihoods does not produce genes with the correct number of
exons. 

To overcome this limitation, the score of the exons is corrected by
adding a constant $EW$. A general process of optimization is needed in
order to predict the number of real exons. Genes are predicted in the
training set using different values of the parameter $EW$. Thus, the
value that maximizes the coefficient of correlation between the actual
and the predicted coding nucleotides is selected.


\subsection{geneid in Drosophila}

Additional papers in a special issue of the {\it Genome Research}
magazine were written by the participants and describe their methods
and the results in detail. Our paper was included in this special
issue and explains how the parameters needed for \geneid\ to predict
genes in \Dm\ were computed. The final \geneid\ predictions showed
an accuracy comparable to the gene finding programs that exhibited the
highest accuracy in the GASP results published in \cite{reese:2000a}.


To evaluate the accuracy $std1$ and $std3$ sets were used. $std1$ was
a rigorous annotation set but incomplete while $std3$ is as complete
as possible but less reliable. Therefore, the organization decided to
test all the statistic related with the sensitivity to be computed in
the conservative set, and the specificity related statistics to be
computed in the extended but less reliable $std3$ set. This could
obviously bias the result to a over positive estimation of the
statistics. Otherwise, the standard sets sufficiently represent the
true nature of the region and conclusion based on them are
interesting, and more realistic than benchmarks realized on single
gene sequences.

Table \ref{accuracy gasp} showed the results of the gene finding
tools. Several gene prediction tools had a sensitivity (Sn) at
nucleotide level superior than 95\%. There was a great deal of
variability in the exon level accuracy. Several tools had Sensitivity
at exon level(Sne) over 75\%. But their specifity (Spe) were generally
much lower. The low missing exons (ME) combined with the high Sn
suggest that several tools were successful at identifying exons but
had trouble finding the correct exon boundaries. The wrong exons (WE)
scores suggest either that the tools are over-predicting. All of the
predictors had considerable difficulty in the correct assembly of
complete genes. The best tools were able to achieve sensitivities
between 0.33 and 0.44 (data not shown). The programs tent to
incorrectly predict many genes. The major problem is the initial short
coding exons that could be in some cases shorter than 10 bases, and
hardly any gene finding tool without the use of homology searches at
databases can predict them.


The GASP project gave us some insights on the performance of gene
prediction program in large scale genomic sequences. To summarize the 
conclusions reached by the GASP organizers:

\begin{itemize}
\item 95\% of the coding nucleotides of the encode genes were
correctly predicted.

\item The correct structures were predicted for about 40\% of the
genes. Base level predictions are easier, exon level predictions are
harder.

\item Major improvements in multiple gene regions.

\item Gene finding including homology not always improves predictions.

\item Programs with specific parameter files for the problem species
performed better than the others.

\item No program is perfect.

\end{itemize}

The two latest statements encourage us to continue developing \geneid\ an
to try to obtain parameter files for the incoming sequenced genomes.

The main conclusions from this experiment were that the methods of
gene prediction at that time had improved and that they were very
useful for genome scale annotations but that high quality annotations
also depend on a solid understanding of the organism in question.


Although \geneid\ was not used by the Drosophila Genome Project to
annotate the \Dm\ genome, it had some usage through our web page
and from people who had freely download the program. Some experimental
papers have based their work on \geneid\ predictions
\citep{dunlop:2000a,castellano:2001a,beltran:2003a}.

\input{tables/gasp_accuracy}


\includepdf[offset=.5cm .5cm,pages={1-5},scale=0.95]{papers/parra_GR_2000}
\newpage
\section{Training \geneid\ in other species}

As we have seen in the previous section, parameter files derived from
one species perform better than generic parameters files. It seems
that each genome have its characteristic signatures for gene
recognition. Under this assumption we decided to develop training
sequence sets and \geneid\ parameters files for the genomes that were
going to be sequenced. 

A training set is defined as a set of genomic sequences satisfying a
number of constraints:
\begin{itemize}
\item Coordinates of coding regions are exactly mapped on the
corresponding genome.

\item Annotations must be experimentally validated: the protein must 
be known or the complete mRNA must have been sequenced.

\item In every annotation, the presence of the minimal canonical signal
for each site is checked (ATG for start sites, AG for the acceptors,
GT for the donors and TGA, TAG and TAA for the stop codons).

\item The maintenance of the open reading frame through the translation 
of coding exons until the annotated stop codon.

\item Redundancy in the training set is also filtered by comparing the 
genomic sequences using blast. Those sequences sequences with 95\% of
similarity over at least 200 nucleotides were removed

\end{itemize}



Nowadays, \geneid\ have parameters for more than twenty species. So
far, we have developed parameter files for the following species:
\AtL, \CeL, \DdL, \DmL, \HsL, \OsL, \PfL, \TnL\ and
\TaL. Parameter sets for more species will be available soon. The
datasets used to train \geneid\ are publicly available at the
following web site: \url{http://genome.imim.es/datasets/geneid}.


The first step for the development of a parameter file is to obtain a
curated set of sequences, the training set. To elaborate the this
training set we mostly get in contact with consortium that organize
the sequencing project to gather in collaboration a reliable set of
sequences. Then, after the checking protocol briefly described in the
previous section, the corresponding models were generated and the
parameter file built.

As a result of these collaboration our group have participate in the
annotation of several genomes. The annexed paper section includes two
publications that have been partially based on the predicted genes
obtained with \geneid\ using parameter files developed for each
species. 

The recompilation of this sets of curated sequence for different
species is also extremely useful for the comparison of the structures
and evolution of the signals that defines the location of genes.

Unlike the process of mRNA translation by the ribosome, which seems
follows a set of rules that is essentially invariant in all known
organisms, the rules governing the RNA splicing clearly differ between
different groups of eukaryotes.  


This do not pretend to be a large scale study nor systematic, there is
only to pin point some peculiarities we have found, that must be
analyzed deeply.

The splice signals in \Dd\ show the canonical GT-AG motif. However,
in contrast with the other species, besides these common sites only
weak preferences for nucleotides adjacent to the donor site could be
detected. The further positions are slightly flavored by a (A/T)GT
motif. This may be caused by the high mean A/T content in introns of
87\%. The splice aparatus has therefore to be able to correctly detect
and process this signal in spite of the relative weakness of the
signals compared to other organisms. Possibly the difference in
information content between intron sequences and coding sequences
composition is used by the cell as additional signal.

On the other hand, \Dm, seem to have a more conserved canonical
acceptor an donor splice sites, with a clear polypyrimidine track
upstream of the canonical AG. \Tn\ splice sites seem to be less
conserved than in human but otherwise intron are also shorter.



\begin{figure}
\begin{center}
\begin{tabular}{lc}

\DdL &\\
\includegraphics[width=6.5cm, trim= 20 400 60 135 ,clip]{figures/dicty_donor} & 
  \includegraphics[width=8.5cm, trim= 110 220 40 150,clip]{figures/dicty_acceptor}\\
\DmL &\\
\includegraphics[width=6.5cm, trim= 20 400 60 135 ,clip]{figures/dros_donor}  & 
  \includegraphics[width=8.5cm,trim= 110 220 40 150,clip]{figures/dros_acceptor} \\
\TnL & \\
\includegraphics[width=6.5cm, trim= 20 400 60 135 ,clip]{figures/fugu_donor}  & 
  \includegraphics[width=8.5cm,trim= 110 220 40 150,clip]{figures/fugu_acceptor} \\
\HsL &\\
\includegraphics[width=6.5cm, trim= 20 400 60 135 ,clip]{figures/hum_donor}   &  
  \includegraphics[width=8.5cm,trim= 110 220 40 150,clip]{figures/hum_acceptor} \\

\end{tabular}
\caption{Splice signal motifs. Sequence motifs for 5' splice site (donor site) 
and 3' splice site (acceptor) using the Pictogram program. The height
of each letter is proportional to the frequency of the corresponding
base at a given position, and bases are listed in descending order from
top to bottom. The relative entropy (in bits) of the model relative to
the background transcript base composition is also shown.} 
\label{splice motifs}
\end{center}
\end{figure}



\chapter{Genome comparison method: sgp2}
\section{Syntenic Gene Prediction}
The increasing number of available genomes has lead to the development
of new computational gene finding methods that use sequence
conservation to improve the accuracy of gene prediction methods
\citep{miller:2001a,rinner:2002a}.  Anonymous genomic sequences are
compared against anonymous genomic sequences from different organism,
under the assumption that regions conserved in the sequence will tend
to correspond to coding exons from homologous genes.

As is it mentioned in the introduction there are different ways to
exploit the comparative genomic information. The first version of sgp
(Syntenic Gene Prediction) was developed mainly by Thomas Viehe and
Roderic Guig\'o \citep{wiehe:2001a}. \sgpo separate clearly gene
prediction from the alignment problem. It start by aligning two
syntenic regions using an external alignment program (sim90 or
blast), and then predict the final gene structures in which the exons
are compatible with the alignment.

A central strategy of \sgpo is to relay as little as possible on species
specific DNA characterization, such as nucleotide composition,
isochore distribution or codon bias. Therefore, the precandidates exons
do not receive scores that depend on any of such sources of
information. Rather, scoring at the initial step relies exclusively on
splice site quality.

After the alignment \sgpo generates a set of pairs of precandidates
exons between the two species. A precandidate exon is a sequence
stretch with a well defined reading frame, and junction signals. A
filtering process checks whether begin and end position of any pair of
pre-candidates are contained in the post processed alignment. If there
is any discrepancy the pair is discarded. Optionally, the filter can
be relaxed to allow for an offset between alignment and exon
precandidate. There are two parameters: $x$, the number of vase pairs
by which locally aligned segments are extended, and $d$, the maximal
distance by which the ends of two paired precandidates may be separated
(see Figure \ref{sgp1}).

\begin{figure}[h!]
\begin{center}
\includegraphics[width=14cm]{figures/sgp1}
\caption{Relaxed filtering of precandidates (a) blunt end, but complete
coverage by the alignment. (b) A blunt end and a partial coverage by
the alignment. Setting parameters d and x to a value >0 retains
precandidates with unaligned splice sites. Adapted from
\cite{wiehe:2001a}}
\end{center}

\label{sgp1}

\end{figure}


Assembly of the exons that pass the filtering is performed
independently for both species using the chaining algorithm described
by \cite{guigo:1998a}. The assembly program attempts to build complete
gene models consisting of either a single exon or one initial exon,
an arbitrary number of internal exons, and one terminal exon. Multiple
genes, on either strand, may be assembled.

Given two sequences and their alignment as input, the program calls
subroutines for the alignment post-processing, generating exons
precandidates, filtering, rescoring and gene assembly and output. The
subroutine with the highest time complexity is one that filters the
precandidates exons. A very rough bound for its run time is given by
$O(nm)$, quadratic cost, where $n$ and $m$ are the lengths of the
input query sequences. This is due to the fact that the size of the
two exon precandidates list depends linearly on $n$ and $m$,
respectively, but pair of precandidates, one from each list, have to
be processed. This is one of the major limitations of SGP1 for whole
genomic predictions. The amount of comparisons of exons structures
increases exponentially with the length of the sequences, thus, it
makes computationally very expensive to approach comparative prediction
of complete eukarytoic genomes.

Another important limitation of SGP1 is that it relays to much on a
full syntenic sequences. If any of the sequences is partially sequenced
the accuracy of the method drops substantially. This limits, again, its
utility when analyzing complete large eukaryotic genomes, and in
particular when the informant genome is in non-assembled shotgun form.

\section{sgp2: Comparative gene prediction in human and mouse}

To overcome these limitations, \sgp\ take a different approach. The
approach is reminiscent of that used in \genomescan\ to incorporate
similarity to known proteins to modify the \genscan\ scoring
schema. Essentially, the query sequence from the target genome is
compared against a collection of sequences from the informant genome
(which can be a single homologous sequence to the query sequence, a
whole assembled genome, or a collection of shotgun reads), and the
results of the comparison are used to modify the scores of the exons
produced by \textit{ab initio} gene prediction programs.


\sgp\ is a program to predict genes by comparing anonymous genomic
sequences from two different species. It combines tblastx, a sequence
similarity search program, with geneid, an "ab initio" gene prediction
program. In "asymmetric" mode, genes are predicted in one sequence
from one species (the target sequence), using a set of sequences
(maybe only one) from the other species (the reference set).
Essentially, \geneid\ is used to predict all potential exons along the
target sequence. Scores of exons are computed as log-likelihood
ratios, function of the splice sites defining the exon, the coding
bias in composition of the exon sequence as measured by a Markov Model
of order five, and of the optimal alignment at the amino acid level
between the target exon sequence and the counterpart homologous
sequence in the reference set. From the set of predicted exons, the
gene structure is assembled (eventually multiple genes in both
strands) maximizing the sum of the scores of the assembled exons.

Some of the main features of \sgp\ are the following:

\begin{itemize}

\item  \sgp\ predictions are more accurate than those obtained by 
pure "ab initio" gene finding programs.

\item Gene prediction can be done in the target and the reference 
genome using one single alignment file from tblastx.

\item \sgp\ output can be customized to different formats as gff, 
geneid format or XML are available. Allowing different amount of
information associated to each output format.

\item Shotgun data can be used at any level of coverage as the informant
genome.

\item  \sgp\ is very efficient in terms of speed and memory usage.
\end{itemize}

\sgp\ was extensively used in the annotation of the mouse and the human
genome. In the annexed paper section is attached the paper of the Mouse
Genome Consortium in which I participate obtaining and processing the
set of \sgp\ predictions. 

An important difference between sgp1 is that \sgp\ does not attempt to
generate all the compatible exons of the two orthologous
sequences. Finding compatible exons requires that the two sequences
have the same exon-intron structure, Extending this strategy to
multi-gene sequences would require the assumption that the two
sequences have the same genes in the same order and orientation. In a
large-scale comparison there are a lot of partial duplications and
rearrangements.

Using a global alignment or the compatible exons strategy requires
informant sequences to be finished. Our conservation sequence
approach, which is based on the highest scoring local alignments,
allows one to use draft and shotgun sequences. The conservation
sequence effectively rearranges the alignments into the correct order
and orientation. In addition because the HSPs can be from any region
of the informant genome it allows to take us, apart from the orthologies,
to the paralogies or domain conservation occurring in non syntenic
regions.

The following paper give a more detailed description of the algorithm,
specially to the maxim scoring projection of the HSPs obtained with
the tblastx and the rescoring of the \geneid\ exons.  It also shows the
evaluation of accuracy in different sets of sequences. Finally it
analyze the obtained predictions in the entire human and mouse genome,
and the potentiality of this results to generate a catalog of
potential novel human-mouse genes.

\includepdf[offset=.5cm .5cm,pages={1-10},scale=0.9]{papers/parra_GR_2003}

\newpage
\chapter{Toward the completion of the mammalian catalog of genes}

The completion of the mouse genome allowed for the first time a
comparative-base annotation of the human-mouse genome, and several
methods were developed to take advantage to the conservation between
genes in order to improve predictions. This section describes a
procedure that reduces the false positive rate of predictions by
exploiting the exonic conservation between human and mouse homologous
genes. Using this protocol, we build a set of human-mouse new genes
that were partially validated experimentally.

\section{Expanding Human and Mouse standard annotation pipelines}

The difficulty of accurate and precise annotations of the coding genes
at genomic level has been broadly discussed in the previous
sections. Although the improvements achived using comparative
approaches, gene prediction methods still tend to predict many false
positives. The Mouse Genome Sequencing Consortium relied mainly for
the initial annotation of the mouse genome on the \ensembl\ gene build
pipeline. The \ensembl\ automatic annotation pipeline basically
follows these steps to obtain the set of predictions:

\begin{itemize}
\item Alignment of the known transcribed transcripts from the 
corresponding species(inlcuding RefSeqs, complete cDNA datasets and
Swissprot genes).

\item Alignment of known proteins from related species.

\item \textit{Ab initio} predictions that are supported by experimental 
evidences (basically ESTs).
\end{itemize}

Therefore, \ensembl\ can not predict genes for which there is no
preexisting evidence of transcription or a similarity to a close
related protein. This limitation could lead to a bias against
predicting genes that have a restricted expression pattern. The aim of
the following work was to survey how many coding genes have been skip
by the conservative \ensembl\ gene build pipeline. The goal of this
approach is to generate a more reliable set of predicted genes and
validate them experimentally to assess the efficiency of the
enrichment protocol.


For \sgp\ and \twinscan, the experiment, performed independently
within each predictor, consist in:

\begin{itemize}

\item Prediction of the human and mouse genes using both comparative 
gene predictors: \sgp\ and \twinscan

\item Identify human mouse orthologous pairs and discard the 
ones corresponding to the \ensembl\ annotations 

\item With the previous set of pairs of orthologous predictions, 
generate a set with conserved exonic structure

\end{itemize}

Once we had the different enriched sets of gene predictions an
experimental validation of some subsamples have been obtained. The
final results of the RT-PCR experiments showed that the comparative
enrichment selection correlates with the ratio of amplification.

My contribution to this project was to develop the filtering
process that leads to the different subsets of gene predictions, and
to develop, in collaboration with Robert Castelo, the software needed
for the superimposition of the exon-intron boundaries over a protein
alignment. In this section we describe the protocol to generate a set
of \sgp\ and \twinscan mouse predictions to be tested by RT-PCR
experiments.

\section{Obtaining comparative gene predictions}

\sgp\ is a program to predict genes by comparing anonymous genomic
sequences from two different species. In this analysis, prediction have
been done on the mouse genome (MGSCv3 assembly) using comparative
information from the human genome (December,2001 GoldenPath equivalent
to NCBI Build 28). To make the predictions, \sgp\ combines TBlastX, a
sequence similarity search program, with geneid, an "ab initio" gene
prediction program. The mouse sequences was cut into 100kb fragments
to build the blast database. The masked human chromosomes were also
cut in 100kb fragments which were run against the mouse database using
TBlastX with the following parameters:

\begin{verbatim}
       B=9000   V=9000  hspmax=500  topcomboN=100 
       W=5 matrix=blosum62mod  E=0.01  E2=0.01  
       Z=3000000000  nogaps  filter=xnu+seg  S2=80
\end{verbatim}


Although these parameters increase the speed of the comparison, the
whole computation took one week of CPU time using 100 Alpha
processors. The resulting high-scoring segment pairs (HSPs) were
processed to find the maximum scoring projection.  \sgp\ predictions
were obtained in a mode in which RefSeqs coordinates (taken from
Golden Path M. musculus February 2002 freeze) are given to \sgp, and
the predictions are built on top of these RefSeqs. Chimeric
predictions including RefSeqs genes are avoided, \sgp\ only predicted
genes in the regions between known genes.  \geneid\ has essentially no
limits to the length of the input query sequence, and deals well with
chromosome sequences. Therefore, prediction were computed from the
entire chromosomic sequences (no fragmentation was needed). The
predictions were done on the unmasked sequences of the mouse genome
(WGSCv3). The computation took one day in a MOSIX cluster containing
four PCs (PentiumIII Dual 500Mhz processors).



\section{Obtaining the homologous pairs of predictions}

The enrichment procedure was applied separately to predictions of
\twinscan\ and \sgp. The protein sequences predicted by each program in
human and mouse were compared using \blp. For each predicted mouse
protein, all predicted human protein with expected values lower than
1x10-6 were called homologs. A global protein alignment was produced
for the best scoring homologs (up to five) by using T-coffe with
default parameters.

Identify human mouse orthologous pairs.
Mouse predicted amino acid sequences were compared with the human
predicted amino acids using Blastp. Orthologous pairs were assigned
where sequence pairs were aligned with Expect values lower than
1x10e-6.

To discard known genes mouse predictions overlapping \ensembl genes
coordinates were considered not novel. We use the preliminary \ensembl
annotation generated with cDNA RIKEN database. Moreover, to assure the
novelty classification, mouse predictions were compared against RefSeq
mRNA and \ensembl database using Blastn, and predictions sharing more
than 95\% nucleotide identity over at least 100bp were rejected
(considered not novel).


A subset of random predictions was extracted from each set (\sgp\ and
\twinscan), and two adjacent exons across an intron were chosen from
the selected predictions for the RT-PCR test. The experimental setup
required that the exons were at least 30bp long, and the introns were
at least 1000bp long. Pairs of exons verifying these requirements are
sorted by the sum of the scores of the exons, and the top scoring pair
was selected for the RT-PCR test.

\section{Conserved exonic structure}

A global alignment was produced for every hypothetical orthologous
pair. T-coffee was run with default parameters to align each pair of
amino acid sequences.

Exonic structure was added to the global pairwise alignments using
\exstral\ (Exon Structural Alignment). This program computes the 
relative position of the intron boundaries in aligned pairs of
sequences.

Exstral is a program that takes as input a global alignment of two
proteins an a file with the genomic coordinates of the exonic
structure of both genes and outputs the exon and intron junctions
supper imposed over the protein alignment. In addition of the
graphical output, \exstral\ also provides information of the exonic
junctions that are compatible in both proteins. 
%In figure \ref{exstral} shows the output of Exstral. 
Superimposition of the exon-intron junction takes into account the
corresponding position of the protein aligned as well as the condon
position where the exon-intron occurs. For each coincident exon-intron
junction the following information is given:

\begin{itemize}

\item the number of the aligned intron and the coordinates in
the amino acid alignment.

\item conservation (identity and  similarity)of the amino acids
 around the corresponding splice site as well as the number of gaps.

\item length of the two aligned exons in both species.

\item score of the predicted exons if it is provided in the coordinates 
file.

\end{itemize}

The main utility of this program is to compare the exonic structure of
two pre-aligned homologous proteins, to easily locate homologous
exon-intron junctions.

We considered the intron to be aligned when the boundaries of the
intron appears in the same coordinates of the alignment in both
sequences and the alignment is at least 50\% conserved residues at
both sides of the aligned intron.

When both members of an aligned pair contained an intron in
the same coordinate with at least 50\% identity over 15 amino acids
both sides the corresponding mouse prediction was assigned to the
``enriched'' pool. Predictions with homologs but no aligned intron
were assigned to the ``similar'' pool.


\begin{figure}
\begin{center}
\includegraphics[width=12cm]{figures/RT-PCRgenes.jpg}
\caption{Schema RT-PCR amplification process}
\end{center}
\end{figure}



\begin{figure}
\begin{center}
\includegraphics[width=18cm]{graphs/sgp2}
\caption{Schema of the protocol to obtain human-mouse \sgp\ prediction 
and filtering process}
\end{center}
\end{figure}

\begin{figure}
\begin{center}
\includegraphics[width=18cm,trim= 0 150 0 0 ,clip]{graphs/twinscan}
\caption{Schema of the filtering process for twinscan comparative 
human-mouse predictions}
\end{center}
\end{figure}

%\input{figures/exstral} 

\includepdf[offset=.5cm .5cm,pages={1-6},scale=0.9]{papers/Guigo_et_al_PNAS_100_3_1140_20030204}


%%%%%%%%%%%%%%%%%%%%%%%%%%%%%%%%%%%%%%%%%%%%%%%%%%%%%%%%%%%%%%%%%%%%%%%%
%%%
%%% DISCUSSION
%%%

\chapter{Discussion}

The rapid release of completed genomes, has driven significant
developments in genome annotation and gene finding tools. There is no
gene finding methods whose accuracy is clearly superior in comparison
to others. In the short term future, the combination of different
sources of predictions and evidences seems to be the most feasible
solution to genome annotation. Comparative approaches will also be
essential in order to build a more accurate catalog of genes for each
organism. Previous chapters have described the development of
different tools and protocols to obtain sets of gene finding
predictions that outperforms the currently existing methods.  In what
follows, I will discuss the utility of the \textit{ab initio} gene
finding methods versus the comparative approaches, the utility of the
generated datasets, how gene prediction can help the high throughput
experimental approaches and some open problems in the field of
computational gene prediction.

\section{Ab initio vs. Comparative gene prediction}

The exhaustive scrutiny to which the sequence of human chromosome 22
\citep{dunham:1999a} has been subjected through the Vertebrate Genome
Annotation (VEGA) database project at the Sanger Center offers, an
excellent platform to obtain a more representative estimation of the
accuracy of current gene finders. However, VEGA uses GENSCAN and
FGENES in the annotation pipeline, and may be biased toward these
programs. Table \ref{guigo accuracy} adapted from Guig\'o and Zhang
(2004), shows the accuracy of a number of \textit{ab initio} and
comparative gene finders in chromosome 22 when compared with the
curated annotations from VEGA.  As it can be seen, accuracy suffers
substantially when moving from single gene sequences (as we have seen
in the $Introduction$) to whole chromosome sequences. For instance,
GENSCAN CC drops from 0.91 in the evaluation by \cite{rogic:2001a} to
0.64 for chromosome 22. But even more sophisticated annotation
pipelines, such as \ensembl\ (based on GENEWISE) or FGENESH++, which use
known cDNAs and RefSeq genes, respectively, are far from producing
perfect predictions, with CCs around 0.75. These numbers strongly
suggest that current mammalian gene counts are still of a highly
hypothetical nature.

\input{tables/guigo_accuracy}

The analysis of the performance, also shows that comparative approaches
methods outperforms standard \textit{ab initio} approaches. Although,
sensitivity in comparative gene finders does not clearly improve, the
specificity is superior (increasing from 0.60 to 0.70 in average).
Thus, is there any reason to keep developing such \textit{ab initio}
gene finding methods?

To my point of view, the question is answered affirmative. First of
all, because the core of all comparative gene finding programs relay,
at least partially, on an \textit{ab initio} gene recognition
methods. Although comparative methods have a higher accuracy, they are
still far from being perfect. Therefore, is it clear, that if we
improve the models underlying the recognition, for instance, of splice
sites in \textit{ab initio} gene finding program, then they can be
applied in a comparative approach. Another important reason for the
developing of \textit{ab initio} gene prediction tools, is that
although we have a lot of complete sequenced genomes, for some genomes
will be difficult to obtain a reference genome at the correct
evolutionary distance to apply comparative gene prediction. This, will
specially be a problem for protozoa, parasitic metazoan and fungi,
that can have very high evolution ratios.

Even if we had had all the genes characterized and perfectly
annotated, \textit{ab initio} approaches would be useful in the sense
that they can help us to have a better understanding of how genes are
encoded. This point of view elucidate that our current knowledge of
the biological process is rather poor: we are not able to reproduce
what the cell does to obtain the proteins encoded in any genome (even
thought the unfair usage of some statistical properties, like coding
statistics, that are unlikely to be used by the cell). Therefore, I
believe that further \textit{ab initio} developments will tend to
relay more in biological information and less in pure statistical
data, trying to mimic the biolgical scenario, and helping to model
actual underlying process.


\section{Evolution of the signals that define genes}

The fact that individual parameters files for each species perform
better than general models can be explained because each genome seems
to have its proprietary signatures for gene recognition which was
shaped by environmental pressures during evolution. This implies that
gene structures and the transcription and translation machinery in
each species are adapted to each other enabling the cell appropriate
transcription and translation for each genome.

During the process of building different parameters files for geneid,
a completed database of curated genes for different species have been
generated.The study of the differences on the specification of genes
can elucidate the evolution of the mechanism and help us to better
understand the underling processes. Some of these variations seems to
be related with the general G+C content of the genome.
 
Analysis of gene signatures of a species genome provides not only
tools for the most accurate prediction of gene structures but also may
help to elucidate the common mechanisms underlying gene specification
and its evolution. In addition, they may also be useful to resolve
conflicting phylogenies and to pinpoint horizontal gene transfers,
genetic drifts and other evolutionary events.

The aim of another thesis, could be the analysis of the variation of
the properties of each species.

\section{Experimental validation of the predictions}

The emergence of high-throughput techniques, characteristic of
genomics research, has lead to the so-called data- or discovery- driven
biology, in which data is obtained without the need for a hypothesis
about the nature of any biological problem, in contra position to the
classical hypothesis-driven approach in which experiments are
performed (and data obtained) to test previously formulated hypothesis
with in the framework of pre-existing theory.

Genome projects are mostly high throughput biology, and they certainly
produce a lot of valuable data. High throughput biology alone, however
- either through indiscriminate sequencing of cDNA libraries, or
through genome expression microarrays - appears to have reached a
limit in its ability to annotate genes in the human genome. For
instance, we now start to see regions of the genome that are
transcribed but do not appear to be coding for proteins. It is
therefore time to the computational biologist to generate gene models
with the enough confidence to be worth trying to be validated with
high throughput experimental approaches.

Synergy between computational and experimental methods of gene
identification will facilitate the full analysis of the currently
sequenced genomes. As more genomes are sequenced, the need
for experimentally validated high thought put annotation will continue
to grow, as will the data available for such methods.

Confirmation by RT-PCR and direct sequencing, seems to be a cost
effective technique that will probably constitute the basis for the
final curated annotation of many available genomes. The success of the
pilot study we have shown suggest a new paradigma in high-through put
genome annotation, in which gene predictions serve as the hypothesis
that drive experimental determination of intron-exon structures.

\section{Gene finding: open problems}

Existing gene finding programs, although significantly advanced over
those available a few years ago, still have several important
limitations. Some key problems and future challenges in the
gene prediction field could be:

\begin{itemize}

\item To identify non coding exons. 

\item To achieve a better understanding of CpG islands, methilation 
patterns and G+C variations across the genome, and to use it to
improve the predictions.

\item To identify promoter regions and the corresponding transcription 
start site.

\item To characterize promoter regions to be able to elucidate 
the combination of transcription factors needed for the activation and
inhibition as well as the tissue and developmental expression pattern.

\item To predict genes that encode for functional RNAs.

\item To have a better characterization of the splicing enhancers and 
silencers that mediate alternative splicing, to allow models to
predict alternative exons or aberrant splicing events.

\item To predict insulators and boundary elements, and matrix-attachment 
and scaffold-attachment regions that could play a key role in the
accessibility of the transcription machinery to the chromatin.

\item To predict non common features as overlapping genes, genes within 
introns, genes with non canonical splice sites, mRNA editing or frame
shifting. We assume these cases to be rare, but because these
assumptions are implicit in our gene models, we may have been
seriously underestimating their incidence.

\end{itemize}

At the root of these limitations lies our still incomplete knowledge
of what defines an eukaryotic gene, and what are the mechanisms by
which the sequence signals involved in gene identification are
recognized and processed in the eukaryotic cell.


%%%%%%%%%%%%%%%%%%%%%%%%%%%%%%%%%%%%%%%%%%%%%%%%%%%%%%%%%%%%%%%%%%%%%%%%
%%%
%%% CONCLUSIONS
%%%

\chapter{Conclusions} 

This research work has mainly contribute to:

\begin{enumerate}
\item The development of a new structure of the parameter file for
\geneid. Including some of the latest measures for gene prediction.

\item The generation of a parameter files for different species. That
have been specially useful for the annotation of recently sequenced
genomes, accesible through: \url{http://genome.imim.es/software/geneid}

\item The generation of a dataset of curated genes for different 
species, that is freely accessible to the Internet through the web
page: \url{http://genome.imim.es/datasets/geneid}

\item The development and implementation of a new procedure to
combine genome comparison information with \textit{ab initio} gene finding
methods.

\item The implementation of a web server freely accessible to 
the Internet through the web page: 
\url{http://genome.imim.es/software/sgp2/sgp2.html}

\item The annotation of some genomes. The two above methods not only 
introduce new theoretical concepts for comparative predicting protein
coding genes, but have also been implemented into efficient computer
programs so that they can be applied to realistic large scale
problems.  \geneid\ has been used for the annotation of \Dd\ and
\Tn , and is currently used for the annotation of many genomes. 
SGP2 have been use for the annotation of the \Mm\ and \Rn\ genome
and is currently used for the annotation of \GgL\ using the human
genome as reference.

\item The development of a filtering protocol based on exonic
structure conservation between close related species. And the
generation of a set of new genes based on the homology between human
and mouse.

\end{enumerate}

\nochapter{Annexed Papers}

In this section are gathered the other relevant papers I have
collaborated in. In this cases my participation is less relevant than
in the ones showed in the main block. What follows is a little
description of my contribution to each work.


\subsubsection{Sequence and Analysis of Chromosome 2 of \DdL}
G. Gl\"okner, L. Eichinger, K. Szafranski, J.A. Pachebat,
A.T. Bankier, P.H. Dear, R. Lehmann, C. Baumgart, G. Parra,
J.F. Abril, R. Guig\'o, K. Kumpf, B. Tunggal, the Dictyostelium Genome
Sequencing Consortium, E. Cox, M.A. Quail, M. Platzer, A. Rosenthal
and A.A. Noegel.  ``Sequence and Analysis of Chromosome 2 of
\DdL.''  Nature 418(6893):79-85 (2002)


This paper was done in collaboration with the Dictyostelium Genome
Sequencing Consortium, which is an international collaboration between
the University of Cologne, the Institute of Molecular Biotechnology in
Jena, the Baylor College of Medicine in Houston, Institut Pasteur in
Paris, and the Sanger Center in Hinxton for the sequencing and the
analysis of the genome of \Dd. 

\Dd\ is a soil-living amoeba. The hereditary information is carried
on six chromosomes with sizes ranging from 4 to 7 Mb resulting in a
total of about 34 Mb of haploid DNA genome with a base composition of
77\% [A+T]. This extreme base compositon biased to A+T nucleotides
gave us the opportunity to discover how a high A+T content can
influence the signals and the codon usage which are the landmarks for
gene prediction.

Obviously, with such a biased base composition the performance of
current gene prediction programs were rather poor. Our group developed
a parameter file for \geneid\ based on experimental annotated
sequences from \Dd.

In this paper it is not explained the training of \geneid\ that
basically followed the same protocol that the training on \Dm\
sequences, but there is a complete analysis of the \geneid\ predicted
proteins. The annotation of the 2,799 genes of the chromosome 2 of
\Dd\ was based on \geneid. The paper is focused on the analysis of
the function and the structure of \geneid\ predicted genes. This
analysis reinforce the view that the evolutionary position of \Dd\
is located before the branching of metazoa and fungi but before the
divergence of the plant kingdom.


\subsubsection{Analysis of the draft sequence of \TnL}
Tetraodon Genome Sequencing Consortium (including G. Parra and
R. Guig\'o).  ``Analysis of the draft sequence of \TnL\ genome
provides new insights into vertebrates evolution.'' Summited

The fish \Tn\ lives in the rivers and estuaries of Indonesia,
Malaysia and India. As a vertebrate, its gene pool is very similar to
that of other vertebrates, including mammals such as humans and mice.
It has been observed that the genome of Fugu rubripes, another puffer
fish from the same family, has a remarkably low content of repetitive
DNA, and this also applies to \Tn. Therefore, for geneticist s
interested in studying genes, fishes of the Tetraodontiform family
have a huge advantage over mammals: their gene pool is contained
within approximately 8 times less DNA (i.e. the genome is 8 times Small
ler). This feature allows us to rapidly target our studies on the
interesting part: the genes.

Genoscope has set out to perform a global analysis of this genome,
by first determining the sequence of a large number of random DNA
fragments. Because of the compactness of the genome, an important
fraction of these sequences will carry a fragment of a gene. With
several hundred thousand such sequences , we should be able to build a
collection containing at least part of each gene. This should allow
us, for the first time, to get the complete picture of the gene
content of a vertebrate genome. This collection should also serve as a
reference for comparisons with the human genome.

\subsubsection{Initial sequencing and comparative analysis of the mouse genome}

Mouse Genome Sequencing Consortium (including G. Parra and R. Guig\'o).
``Initial sequencing and comparative analysis of the mouse genome.''
Nature 420(6915):520-562 (2002) 

In this paper, our group computed the predictions of the coding genes
for the mouse genome using the human genome as reference using
\sgp. We compared the \sgp\ predictions with the annotations given by
\ensembl\ in order to guess how many putative new genes have been
missed by the conservative \ensembl\ annotations.



% Dicty
\clearemptydoublepage
\addcontentsline{toc}{section}{Sequence and Analysis of Chromosome 2 of \DdL}
\includepdf[offset=.5cm .5cm,pages={1-7},scale=0.9]{papers/gernot_Nature_2002}

% Tetra
\clearemptydoublepage
\addcontentsline{toc}{section}{Analysis of the draft sequence of \TnL\ genome
provides new insights into vertebrates evolution}
\includepdf[offset=.5cm .5cm,pages={1,2},scale=0.9]{papers/tetra_genome_sub1}
\includepdf[offset=.5cm .5cm,pages={9-13,30,31},scale=0.9]{papers/SUPP_INFO_sub1.pdf}



% Mouse
\clearemptydoublepage
\addcontentsline{toc}{section}{Initial sequencing and comparative analysis of the mouse genome}
\includepdf[offset=.5cm .5cm,pages={1,20,21},scale=0.9]{papers/2002_Nature_v420_i6915_p520_MGSC}

\newpage
\addcontentsline{toc}{chapter}{Bibliography}
\bibliographystyle{apalike}
\bibliography{main}

\end{document}


% You made it!!! look for a postdoc now!!!

