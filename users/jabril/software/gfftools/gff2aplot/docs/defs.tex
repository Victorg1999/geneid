%%%%% Colors for gff2ps% ===> this file was generated automatically by noweave --- better not edit it
\input tables/AplotColorDefs.tex

%%%%% New Commands are defined here
\newcommand{\sctn}[1]{\chapter{#1}} % \section{#1}}
\newcommand{\subsctn}[1]{\section{#1}} % \subsection{#1}}
\newcommand{\subsubsctn}[1]{\subsection{#1}} % \subsubsection{#1}}
\newcommand{\subsubsubsctn}[1]{\subsubsection{#1}} % \paragraph{#1}}
\newcommand{\desc}[1]{\item[#1] \ \\}
\newcommand{\pa}[1]{{\footnotesize\textsf{$<$\textsl{#1}$>$}}}
\newcommand{\todo}[1]{
  \vskip 3ex
  \hspace{-0.75cm}
   \psframebox[framearc=0.2,linecolor=darkred,linewidth=1pt,
              fillstyle=solid,fillcolor=verylightyellow,framesep=2ex]{
     \begin{minipage}[t]{16cm}
     \vskip -4.75ex
     \hspace{-1.25cm}
       \psframebox[framearc=1,linecolor=darkred,linewidth=1.25pt,
               fillstyle=solid,fillcolor=verylightorange,framesep=5pt]{
               \textcolor{darkred}{\textbf{\hspace{2ex}TO DO\hspace{2ex}}}
         } % psframebox
      \begin{itemize}\setlength{\itemsep}{-0.5ex} #1 \end{itemize}
     \end{minipage}
     } % psframebox
  \vskip 1.5ex
} % newcommand todo
\newcommand{\todoitem}[2]{
\item[$\triangleright$] [\textit{Section}~\ref{#2}, \textit{page}~\pageref{#2}]\\
              #1
} % newcommand todoitem
\input tabledefs.tex
 
%%%%% PSTRICKs definitions
\pslongbox{ExFrame}{\psframebox}
\newcommand{\cln}[1]{\fcolorbox{black}{#1}{\textcolor{#1}{\rule[-.3ex]{1cm}{1ex}}}}
\newcommand{\clns}[1]{\fcolorbox{black}{#1}{\textcolor{#1}{\rule[-.3ex]{0.75cm}{1ex}}}}
\newpsobject{showgrid}{psgrid}{subgriddiv=0,griddots=1,gridlabels=6pt}
% \pscharpath[fillstyle=solid, fillcolor=verydarkcyan, linecolor=black, linewidth=1pt]{\sffamily\scshape\bfseries\veryHuge #1 }
\newcommand{\clrtbl}[9]{\vskip -2.95ex \noindent\hrulefill\cln{#1}\cln{#2}\cln{#3}\cln{#4}\cln{#5}\cln{#6}\cln{#7}\cln{#8}\cln{#9}}
\newcommand{\clrtblg}[9]{\vskip -2.95ex \noindent\hrulefill\clns{black}\clns{#1}\clns{#2}\clns{#3}\clns{#4}\clns{#5}\clns{#6}\clns{#7}\clns{#8}\clns{#9}\clns{white}}

%%%%% global urls
% \newcommand{\getpsf}[1]{\html{(\htmladdnormallink{Get PostScript file}{./Psfiles/#1})}}   
\def\mtjabril{\htmladdnormallink{\textbf{jabril@imim.es}}{MAILTO:jabril@imim.es}}
\def\perltidy{
 \htmladdnormallinkfoot{\texttt{perltidy}}
                   {\url|http://perltidy.sourceforge.net/|}
 } % def perltidy

%%%%% defs
\def\prog{\textsc{\textbf{\PROGname}}}
\def\gps{\textsc{\textbf{gff2ps}}}
\def\gft{\textsc{\textbf{gfftools}}}
\def\noweb{\textsc{noweb}}
\def\gv{\textsc{ghostview}}
\def\gs{\textsc{ghostscript}}
\def\ps{\textsc{PostScript}}
\def\eoline{$\backslash\backslash$}
\def\ver{\PROGversion}
\def\lastyear{2003}

%%%%% TODO defs
\def\todoAAZ{Cite here Vista, PiP-maker, Dotter and other dotplot tools.} % todoAAZ
\def\todoAAB{Flow diagram here (Graphic and Web/CGI modes) for figure~\ref{fig:othermodes}.} % todoAAB
\def\todoBBG{
Remember to check for some of the strings values such: '++none++', '++sequence++', and so on.
} % todoBBG
\def\todoDDB{
Set the final options/variables/parameters for the whole section.
} % todoDDB
\def\todoEEB{
Move this section to a perl package: \\ suggesting {\Tt{}gfftools::parser}.
} % todoEEB
\def\todoELA{
DATA STRUCTURE for {\Tt{}{\%}ALN{\_}DATA} was defined:
Include a figure if differs from {\Tt{}{\%}GFF{\_}DATA} definition (table~\ref{tbl:alndata}).
} % todoELA
\def\todoELB{
DATA STRUCTURE for vectors within {\Tt{}{\%}GFF{\_}DATA} at elements level.
} % todoELB
\def\todoIAA{
Description of Document Structuring Convention (DSC) and sketch PS file structure on figure~\ref{fig:PSdsc}.
} % todoIAA
\def\todoJAA{
Move this section to a perl package: \\ suggesting {\Tt{}gfftools::PostScript::colors}.
} % todoJAA
\def\todoJBB{
Move this section to a perl package: \\ suggesting {\Tt{}gfftools::PostScript::formats}.
} % todoJBB
\def\todoIAD{Plotting data into extra box (lower panel).} % todoIAD
\def\todoYAA{
Write tests to check functions, to build the examples. Maybe {\Tt{}configure} or {\Tt{}Makefile} scripts.
} % todoYAA
\def\todoTAA{
Implement a perl script to test {\prog} functions.
} % todoTAA

%%%%%%%%%%%%%%%%%%%%%%%%%%%%%%%%%%%%%%%%%%%%%%%%%%%%%%%%%%%%%%%%%%%%%%%%%%%
%
\def\genomelab{\textbf{Genome BioInformatics Lab}}
\def\progname{{\PROGname}.pl}
\def\tit{\textsc{\progname}}
%
\def\mtjabril{
 \htmladdnormallink{\texttt{jabril@imim.es}}
                   {MAILTO:jabril@imim.es?subject=[gff2aplot]}
 } % def mtjabril
\def\mttwiehe{
 \htmladdnormallink{\texttt{twiehe@ice.mpg.de}}
                   {MAILTO:twiehe@ice.mpg.de?subject=[gff2aplot]}
 } % def mttwiehe
\def\mtrguigo{
 \htmladdnormallink{\texttt{rguigo@imim.es}}
                   {MAILTO:rguigo@imim.es?subject=[gff2aplot]}
 } % def mtrguigo
%
\def\authorslist{
 Josep F. Abril   {\mdseries\small\dotfill \mtjabril } \\
 Thomas   Wiehe   {\mdseries\small\dotfill \mttwiehe } \\
 Roderic  Guig\'o {\mdseries\small\dotfill \mtrguigo } \\
 % Other authors here...\\
 } % def authorslist
\def\authorshort{
 Abril, JF; Wiehe, T; Guig\'o, R % Other authors here...
 } % def authorshort
%
\def\license{GNU General Public License (GNU-GPL)}
%
\def\progdesc{
{\prog} is a tool to visualize the alignment of two genomic sequences together with their annotations. Input to the program are single or multiple files in General Feature Format (GFF). Output is in PostScript format so the program serves to generate scalable print-quality images for comparative genomics sequence analysis.
 } % def progdesc 
%
\def\showaffiliation{
\scalebox{0.9 1}{\Large\textsl{\genomelab}}\\
Grup de Recerca en Infom\`atica Biom\`edica\\
Institut Municipal d'Investigaci\'o M\`edica\\ %`
Universitat Pompeu Fabra\\[2ex]
 } % def showaffiliation
%
%%%%%%%%%%%%%%%%%%%%%%%%%%%%%%%%%%%%%%%%%%%%%%%%%%%%%%%%%%%%%%%%%%%%%%%%%%%

%%%%% Setting text for footers and headers
\fancyhead{} % clear all fields
\fancyfoot{} % clear all fields
\fancyhead[RO,LE]{\thepage}
\fancyhead[LO,RE]{\rightmark}
\fancyfoot[LO,LE]{\small\textbf{\genomelab}}
\fancyfoot[CO,CE]{\small\textsl{\authorshort}}
\fancyfoot[RO,RE]{\small\textbf{\today}}
\renewcommand{\headrulewidth}{1pt}
\renewcommand{\footrulewidth}{1pt}

\nwfilename{}
