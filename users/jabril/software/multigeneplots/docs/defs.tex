%%%%% Colors for gff2ps
\input ColorDefs.tex

%%%%% New Commands are defined here
\newcommand{\sctn}[1]{\section{#1}}
\newcommand{\subsctn}[1]{\subsection{#1}}
\newcommand{\subsubsctn}[1]{\subsubsection{#1}}
\newcommand{\desc}[1]{\item[#1] \ \\}
\newcommand{\todo}[1]{
  \vskip 3ex
  \hspace{-0.75cm}
   \psframebox[framearc=0.2,linecolor=darkred,linewidth=1pt,
              fillstyle=solid,fillcolor=verylightyellow,framesep=2ex]{
     \begin{minipage}[t]{16cm}
     \vskip -4.75ex
     \hspace{-1.25cm}
       \psframebox[framearc=1,linecolor=darkred,linewidth=1.25pt,
               fillstyle=solid,fillcolor=verylightorange,framesep=5pt]{
               \textcolor{darkred}{\textbf{\hspace{2ex}TO DO\hspace{2ex}}}
         } % psframebox
      \begin{itemize}\setlength{\itemsep}{-0.5ex} #1 \end{itemize}
     \end{minipage}
     } % psframebox
  \vskip 1.5ex
} % newcommand todo
\newcommand{\todoitem}[2]{
  \item[$\triangleright$] [\textit{Section}~\ref{#2}, 
                           \textit{page}~\pageref{#2}]\\ {#1}
} % newcommand todoitem


%%%%% PSTRICKs definitions
\pslongbox{ExFrame}{\psframebox}
\newcommand{\cln}[1]{\fcolorbox{black}{#1}{\textcolor{#1}{\rule[-.3ex]{1cm}{1ex}}}}
\newpsobject{showgrid}{psgrid}{subgriddiv=0,griddots=1,gridlabels=6pt}
% \pscharpath[fillstyle=solid, fillcolor=verydarkcyan, linecolor=black, linewidth=1pt]{\sffamily\scshape\bfseries\veryHuge #1 }


%%%%% global urls
% \newcommand{\getpsf}[1]{\html{(\htmladdnormallink{Get PostScript file}{./Psfiles/#1})}}   


%%%%% defs
\def\noweb{\textsc{noweb}}
\def\ps{\textsc{PostScript}}


%%%%% TODO defs
\def\todoAAA{This is a first draft of the {\progname}.} % todoAAA
\def\todoBAA{
Using GetOPTS. New options:

\begin{minipage}[c]{\linewidth}
\begin{center}
\begin{description}
\setlength{\parsep}{0ex}
\setlength{\itemsep}{-0.5ex}
\item[ --save-gff ] to save GFF-records set for each gene.
\item[ --jpeg ] to produce jpeg from PostScript output.
\item[ --png ] to produce png from PostScript output.
\item[ --pdf ] to produce pdf from PostScript output (can it be easily done???).
\item[ --proportional-length ] 
\item[ --fixed-length ] 
\item[ --log-length ]
\item[ --nucleotides-cm ]
\end{description}
\end{center}
\end{minipage}
} % todoBAA

%%%%%%%%%%%%%%%%%%%%%%%%%%%%%%%%%%%%%%%%%%%%%%%%%%%%%%%%%%%%%%%%%%%%%%%%%%%
%
\def\genomelab{\textbf{Genome Informatics Research Lab}}
\def\progname{multigeneplots.pl}
\def\tit{\textsc{\progname}}
%
\def\mtjabril{
 \htmladdnormallink{\texttt{jabril@imim.es}}
                   {MAILTO:jabril@imim.es?subject=[maskedfastacoords]}
 } % def mtjabril
%
\def\authorslist{
 Josep F. Abril {\mdseries\small\dotfill \mtjabril } \\
 % Other authors here...\\
 } % def authorslist
\def\authorshort{
 Abril, JF % Other authors here...
 } % def authorshort
%
\def\license{GNU General Public License (GNU-GPL)}
%
\def\progdesc{
From a table containing gene identifiers, start and end coords, and the strand,the program loops through a set of GFF files, making the plots for each single dataset using \texttt{gff2ps}. Once the PostScript figures have been obtained, it can convert them to a bitmap graphical format (say here \textsc{jpeg} or \textsc{png}). It must compute the width of every plot according to gene length.
 } % def progdesc
%
\def\showaffiliation{
\scalebox{0.9 1}{\Large\textsl{\genomelab}}\\
Grup de Recerca en Infom\`atica Biom\`edica\\
Institut Municipal d'Investigaci\'o M\`edica\\
Universitat Pompeu Fabra\\[2ex]
 } % def showaffiliation
%
%%%%%%%%%%%%%%%%%%%%%%%%%%%%%%%%%%%%%%%%%%%%%%%%%%%%%%%%%%%%%%%%%%%%%%%%%%%

%%%%% Setting text for footers and headers
\fancyhead{} % clear all fields
\fancyfoot{} % clear all fields
\fancyhead[RO,LE]{\thepage}
\fancyhead[LO,RE]{\tit\quad\rightmark}
\fancyfoot[LO,LE]{\small\textbf{\genomelab}}
\fancyfoot[CO,CE]{\small\textsl{\authorshort}}
\fancyfoot[RO,RE]{\small\textbf{\today}}
\renewcommand{\headrulewidth}{1pt}
\renewcommand{\footrulewidth}{1pt}
%
